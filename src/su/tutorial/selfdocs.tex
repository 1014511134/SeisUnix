%selfdocs.tex --- complete list of CWP Free program selfdocs
% generated by --- GENDOCS
 
\documentclass[12pt]{article}
 
\textwidth 6.25in
\textheight 8.75in
\oddsidemargin .125in
\evensidemargin .125in
%\topmargin -.5in
 
\begin{document}
\input /usr/local/cwp/src/su/tutorial/titlepagesd.tex

\pagebreak
\section*{Names and Short descriptions of the Codes}
 CWPROOT = /usr/local/cwp

\begin{verbatim}

Mains: 

In CWPROOT/src/cwp/main:
* CTRLSTRIP - Strip non-graphic characters
* DOWNFORT - change Fortran programs to lower case, preserving strings
* FCAT - fast cat with 1 read per file 
* ISATTY - pass on return from isatty(2)
* MAXINTS - Compute maximum and minimum sizes for integer types 
* PAUSE - prompt and wait for user signal to continue
* T - time and date for non-military types
* UPFORT - change Fortran programs to upper case, preserving strings

In CWPROOT/src/par/main:
A2B - convert ascii floats to binary 				
A2I - convert Ascii to binary Integers			
ADDRVL3D - Add a random velocity layer (RVL) to a gridded             
B2A - convert binary floats to ascii				
CELLAUTO - Two-dimensional CELLular AUTOmata			  	
char* sdoc[] = {
CSHOTPLOT - convert CSHOT data to files for CWP graphers		
DZDV - determine depth derivative with respect to the velocity	",  
FARITH - File ARITHmetic -- perform simple arithmetic with binary files
FLOAT2IBM - convert native binary FLOATS to IBM tape FLOATS	
FTNSTRIP - convert a file of binary data plus record delimiters created
FTNUNSTRIP - convert C binary floats to Fortran style floats	
GRM - Generalized Reciprocal refraction analysis for a single layer	
H2B - convert 8 bit hexidecimal floats to binary		
HTI2STIFF - convert HTI parameters alpha, beta, d(V), e(V), gamma	
 HUDSON - compute  effective parameters of anisotropic solids	        
I2A - convert binary integers to ascii				
IBM2FLOAT - convert IBM tape FLOATS to native binary FLOATS	
KAPERTURE - generate the k domain of a line scatterer for a seismic array
LINRORT - linearized P-P, P-S1 and P-S2 reflection coefficients 	

MAKEVEL - MAKE a VELocity function v(x,y,z)				
MKPARFILE - convert ascii to par file format 				
MRAFXZWT - Multi-Resolution Analysis of a function F(X,Z) by Wavelet	
PDFHISTOGRAM - generate a HISTOGRAM of the Probability Density function
PRPLOT - PRinter PLOT of 1-D arrays f(x1) from a 2-D function f(x1,x2)
RANDVEL3D - Add a random velocity layer (RVL) to a gridded             
RAYT2DAN -- P- and SV-wave raytracing in 2D anisotropic media		
RAYT2D - traveltime Tables calculated by 2D paraxial RAY tracing	
RECAST - RECAST data type (convert from one data type to another)	
REFREALAZIHTI -  REAL AZImuthal REFL/transm coeff for HTI media 	
REFREALVTI -  REAL REFL/transm coeff for VTI media and symmetry-axis	
REGRID3 - REwrite a [ni3][ni2][ni1] GRID to a [no3][no2][no1] 3-D grid
RESAMP - RESAMPle the 1st dimension of a 2-dimensional function f(x1,x2)

SMOOTH2 --- SMOOTH a uniformly sampled 2d array of data, within a user-
SMOOTH3D - 3D grid velocity SMOOTHing by the damped least squares	
SMOOTHINT2 --- SMOOTH non-uniformly sampled INTerfaces, via the damped
STIFF2VEL - Transforms 2D elastic stiffnesses to (vp,vs,epsilon,delta) 
SUBSET - select a SUBSET of the samples from a 3-dimensional file	
SWAPBYTES - SWAP the BYTES of various  data types			
THOM2HTI - Convert Thompson parameters V_p0, V_s0, eps, gamma,	
THOM2STIFF - convert Thomsen's parameters into (density normalized)	
TRANSP3D - TRANSPose an n1 by n2 by n3 element matrix			
TRANSP - TRANSPose an n1 by n2 element matrix				
TVNMOQC - Check tnmo-vnmo pairs; create t-v column files           
UNIF2ANISO - generate a 2-D UNIFormly sampled profile of elastic	
UNIF2 - generate a 2-D UNIFormly sampled velocity profile from a layered
UNIF2TI2 - generate a 2-D UNIFormly sampled profile of stiffness 	
UNISAM2 - UNIformly SAMple a 2-D function f(x1,x2)			
UNISAM - UNIformly SAMple a function y(x) specified as x,y pairs	
UTMCONV - CONVert longitude and latitude to UTM, and vice versa       
VEL2STIFF - Transforms VELocities, densities, and Thomsen or Sayers   
VELCONV - VELocity CONVersion					
VELPERTAN - Velocity PERTerbation analysis in ANisotropic media to    
VELPERT - estimate velocity parameter perturbation from covariance 	

VTLVZ -- Velocity as function of Time for Linear V(Z);		
WKBJ - Compute WKBJ ray theoretic parameters, via finite differencing	
XY2Z - converts (X,Y)-pairs to spike Z values on a uniform grid	
Z2XYZ - convert binary floats representing Z-values to ascii	

In CWPROOT/src/psplot/main:
PSBBOX - change BoundingBOX of existing PostScript file	
PSCONTOUR - PostScript CONTOURing of a two-dimensional function f(x1,x2)
PSCUBE - PostScript image plot with Legend of a data CUBE       
PSCCONTOUR - PostScript Contour plot of a data CUBE		        
PSEPSI - add an EPSI formatted preview bitmap to an EPS file		
PSGRAPH - PostScript GRAPHer						
PSIMAGE - PostScript IMAGE plot of a uniformly-sampled function f(x1,x2)
PSLABEL - output PostScript file consisting of a single TEXT string	
PSMANAGER - printer MANAGER for HP 4MV and HP 5Si Mx Laserjet 
PSMERGE - MERGE PostScript files					
PSMOVIE - PostScript MOVIE plot of a uniformly-sampled function f(x1,x2,x3)
PSWIGB - PostScript WIGgle-trace plot of f(x1,x2) via Bitmap		
PSWIGP - PostScript WIGgle-trace plot of f(x1,x2) via Polygons	

In CWPROOT/src/xplot/main:
* LCMAP - List Color Map of root window of default screen 
* LPROP - List PROPerties of root window of default screen of display 
* SCMAP - set default standard color map (RGB_DEFAULT_MAP)
XCONTOUR - X CONTOUR plot of f(x1,x2) via vector plot call		
* XESPB - X windows display of Encapsulated PostScript as a single Bitmap
* XEPSP - X windows display of Encapsulated PostScript
XIMAGE - X IMAGE plot of a uniformly-sampled function f(x1,x2)     	
XPICKER - X wiggle-trace plot of f(x1,x2) via Bitmap with PICKing	
* XPSP - Display conforming PostScript in an X-window
XWIGB - X WIGgle-trace plot of f(x1,x2) via Bitmap			

In CWPROOT/src/Xtcwp/main:
XGRAPH - X GRAPHer							
XMOVIE - image one or more frames of a uniformly sampled function f(x1,x2)
XRECTS - plot rectangles on a two-dimensional grid			

In CWPROOT/src/Xmcwp/main:
* FFTLAB - Motif-X based graphical 1D Fourier Transform

In CWPROOT/src/su/graphics/psplot:
SUPSCONTOUR - PostScript CONTOUR plot of a segy data set		
SUPSCUBE - PostScript CUBE plot of a segy data set			
SUPSCUBECONTOUR - PostScript CUBE plot of a segy data set		
SUPSGRAPH - PostScript GRAPH plot of a segy data set			
SUPSIMAGE - PostScript IMAGE plot of a segy data set			
SUPSMAX - PostScript of the MAX, min, or absolute max value on each trace
SUPSMOVIE - PostScript MOVIE plot of a segy data set			
SUPSWIGB - PostScript Bit-mapped WIGgle plot of a segy data set	
SUPSWIGP - PostScript Polygon-filled WIGgle plot of a segy data set	

In CWPROOT/src/su/main/amplitudes:
SUCENTSAMP - CENTRoid SAMPle seismic traces			
SUDIPDIVCOR - Dip-dependent Divergence (spreading) correction	
SUDIVCOR - Divergence (spreading) correction				
SUGAIN - apply various types of gain				  	
SUNAN - remove NaNs & Infs from the input stream		
SUNORMALIZE - Trace NORMALIZation by rms, max, or median       ", 
SUPGC   -   Programmed Gain Control--apply agc like function	
SUWEIGHT - weight traces by header parameter, such as offset		
SUZERO -- zero-out (or set constant) data within a time window	

In CWPROOT/src/su/main/attributes_parameter_estimation:
SUATTRIBUTES - instantaneous trace ATTRIBUTES 			

SUHISTOGRAM - create histogram of input amplitudes		
SUMAX - get trace by trace local/global maxima, minima, or absolute maximum
SUMEAN - get the mean values of data traces				",	
SUQUANTILE - display some quantiles or ranks of a data set            

In CWPROOT/src/su/main/convolution_correlation:
SUACOR - auto-correlation						
SUACORFRAC -- general FRACtional Auto-CORrelation/convolution		
SUCONV - convolution with user-supplied filter			
SUREFCON -  Convolution of user-supplied Forward and Reverse		
SUXCOR - correlation with user-supplied filter			

In CWPROOT/src/su/main/data_compression:
DCTCOMP - Compression by Discrete Cosine Transform			
SUPACK1 - pack segy trace data into chars			
SUPACK2 - pack segy trace data into 2 byte shorts		
SUUNPACK1 - unpack segy trace data from chars to floats	
SUUNPACK2 - unpack segy trace data from shorts to floats	

In CWPROOT/src/su/main/data_conversion:
DT1TOSU - Convert ground-penetrating radar data in the	
SEGYCLEAN - zero out unassigned portion of header		
SEGYHDRMOD - replace the text header on a SEGY file		
SEGYHDRS - make SEG-Y ascii and binary headers for segywrite		
SEGYREAD - read an SEG-Y tape						
SEGYSCAN -- SCANs SEGY file trace headers for min-max in  several	
SEGYWRITE - write an SEG-Y tape					
SETBHED - SET the fields in a SEGY Binary tape HEaDer file, as would be
SUASCII - print non zero header values and data in various formats    
SUINTVEL - convert stacking velocity model to interval velocity model	
SUOLDTONEW - convert existing su data to xdr format		
SUSTKVEL - convert constant dip layer interval velocity model to the	
SUSWAPBYTES - SWAP the BYTES in SU data to convert data from big endian
SWAPBHED - SWAP the BYTES in a SEGY Binary tape HEaDer file		

In CWPROOT/src/su/main/datuming:
SUDATUMFD - 2D zero-offset Finite Difference acoustic wave-equation	
SUDATUMK2DR - Kirchhoff datuming of receivers for 2D prestack data	
SUDATUMK2DS - Kirchhoff datuming of sources for 2D prestack data	
 SUKDMDCR - 2.5D datuming of receivers for prestack, common source    
 SUKDMDCS - 2.5D datuming of sources for prestack common receiver 	

In CWPROOT/src/su/main/decon_shaping:
SUCDDECON - DECONvolution with user-supplied filter by straightforward
SUFXDECON - random noise attenuation by FX-DECONvolution              
SUPEF - Wiener (least squares) predictive error filtering		
SUPHIDECON - PHase Inversion Deconvolution				
SUSHAPE - Wiener shaping filter					

In CWPROOT/src/su/main/dip_moveout:
SUDMOFK - DMO via F-K domain (log-stretch) method for common-offset gathers
SUDMOFKCW - converted-wave DMO via F-K domain (log-stretch) method for
SUDMOTIVZ - DMO for Transeversely Isotropic V(Z) media for common-offset
SUDMOTX - DMO via T-X domain (Kirchhoff) method for common-offset gathers
SUDMOVZ - DMO for V(Z) media for common-offset gathers		
SUTIHALEDMO - TI Hale Dip MoveOut (based on Hale's PhD thesis)	

In CWPROOT/src/su/main/filters:
SUBFILT - apply Butterworth bandpass filter 			
SUCCFILT -  FX domain Correlation Coefficient FILTER			
SUDIPFILT - DIP--or better--SLOPE Filter in f-k domain	
SUFILTER - applies a zero-phase, sine-squared tapered filter		
SUFRAC -- take general (fractional) time derivative or integral of	
SUFWATRIM - FX domain Alpha TRIM					
SUK1K2FILTER - symmetric box-like K-domain filter defined by the	
SUKFILTER - radially symmetric K-domain, sin^2-tapered, polygonal	
SUKFRAC - apply FRACtional powers of i|k| to data, with phase shift 
SULFAF -  Low frequency array forming					", 
SUMEDIAN - MEDIAN filter about a user-defined polygonal curve with	
SUPHASE - PHASE manipulation by linear transformation			
SUSMGAUSS2 --- SMOOTH a uniformly sampled 2d array of velocities	
SUTVBAND - time-variant bandpass filter (sine-squared taper)  

In CWPROOT/src/su/main/headers:
BHEDTOPAR - convert a Binary tape HEaDer file to PAR file format	
SU3DCHART - plot x-midpoints vs. y-midpoints for 3-D data	
SUABSHW - replace header key word by its absolute value	
SUADDHEAD - put headers on bare traces and set the tracl and ns fields
SUAHW - Assign Header Word using another header word			
SUAZIMUTH - compute trace AZIMUTH, offset, and midpoint coordinates    
SUCDPBIN - Compute CDP bin number					
SUCHART - prepare data for x vs. y plot			
SUCHW - Change Header Word using one or two header word fields	
SUCLIPHEAD - Clip header values					
SUCOUNTKEY - COUNT the number of unique values for a given KEYword.	
SUDUMPTRACE - print selected header values and data.              
SUEDIT - examine segy diskfiles and edit headers			
SUGEOM - Fill up geometry in trace headers.                              
SUGETHW - sugethw writes the values of the selected key words		
SUHTMATH - do unary arithmetic operation on segy traces with 	
SUKEYCOUNT - sukeycount writes a count of a selected key    
SULCTHW - Linear Coordinate Transformation of Header Words		
SULHEAD - Load information from an ascii column file into HEADERS based
SUPASTE - paste existing SU headers on existing binary data	
surandhw - set header word to random variable 		
SURANGE - get max and min values for non-zero header entries	
SUSEHW - Set the value the Header Word denoting trace number within	
SUSHW - Set one or more Header Words using trace number, mod and	
SUSTRIP - remove the SEGY headers from the traces		
SUTRCOUNT - SU program to count the TRaces in infile		
SUUTM - UTM projection of longitude and latitude in SU trace headers  
SUXEDIT - examine segy diskfiles and edit headers			

In CWPROOT/src/su/main/interp_extrap:
SUINTERP - interpolate traces using automatic event picking		
SUINTERPFOWLER - interpolate output image from constant velocity panels
SUOCEXT - smaller Offset EXTrapolation via Offset Continuation        

In CWPROOT/src/su/main/migration_inversion:
SUGAZMIGQ - SU version of Jeno GAZDAG's phase-shift migration 	
SUINVXZCO - Seismic INVersion of Common Offset data for a smooth 	
SUINVZCO3D - Seismic INVersion of Common Offset data with V(Z) velocity
SUKDMIG2D - Kirchhoff Depth Migration of 2D poststack/prestack data	
SUKDMIG3D - Kirchhoff Depth Migration of 3D poststack/prestack data	
SUKTMIG2D - prestack time migration of a common-offset	
SUMIGFD - 45-90 degree Finite difference depth migration for		
SUMIGFFD - Fourier finite difference depth migration for		
SUMIGGBZOAN - MIGration via Gaussian beams ANisotropic media (P-wave)	
SUMIGGBZO - MIGration via Gaussian Beams of Zero-Offset SU data	
SUMIGPREFD --- The 2-D prestack common-shot 45-90 degree		
SUMIGPREFFD - The 2-D prestack common-shot Fourier finite-difference	
char *sdoc[] = {
SUMIGPRESP - The 2-D prestack common-shot split-step Fourier		", 
SUMIGPS - MIGration by Phase Shift with turning rays			
SUMIGPSPI - Gazdag's phase-shift plus interpolation depth migration   
SUMIGPSTI - MIGration by Phase Shift for TI media with turning rays	
SUMIGSPLIT - Split-step depth migration for zero-offset data.         
SUMIGTK - MIGration via T-K domain method for common-midpoint stacked data
SUMIGTOPO2D - Kirchhoff Depth Migration of 2D postack/prestack data	
SUSTOLT - Stolt migration for stacked data or common-offset gathers	
SUTIFOWLER   VTI constant velocity prestack time migration		

In CWPROOT/src/su/main/multicomponent:
SUALFORD - trace by trace Alford Rotation of shear wave data volumes  
SUEIPOFI - EIgenimage (SVD) based POlarization FIlter for             
SUHROT - Horizontal ROTation of three-component data			
SULTT - trace by trace, sample by sample, rotation of shear wave data 
SUPOFILT - POlarization FILTer for three-component data               
SUPOLAR - POLarization analysis of three-component data               

In CWPROOT/src/su/main/noise:
SUADDNOISE - add noise to traces					
SUHARLAN - signal-noise separation by the invertible linear		
SUJITTER - Add random time shifts to seismic traces			

In CWPROOT/src/su/main/operations:
SUFLIP - flip a data set in various ways			
SUFWMIX -  FX domain multidimensional Weighted Mix			
SUMATH - do math operation on su data 		
SUMIX - compute weighted moving average (trace MIX) on a panel	
SUOP2 - do a binary operation on two data sets			
SUOP - do unary arithmetic operation on segys 		
SUPERMUTE - permute or transpose a 3d datacube	 		
SUSIMAT - Correlation similarity matrix for two traces.		
SUVCAT -  append one data set to another, with or without an  ", 
SUVLENGTH - Adjust variable length traces to common length   	

In CWPROOT/src/su/main/picking:
SUFBPICKW - First break auto picker				
SUFNZERO - get Time of First Non-ZERO sample by trace              
SUPICKAMP - pick amplitudes within user defined and resampled window	

In CWPROOT/src/su/main/stacking:
SUCVS4FOWLER --compute constant velocity stacks for input to Fowler codes
SUDIVSTACK -  Diversity Stacking using either average power or peak   
SUPWS - Phase stack or phase-weighted stack (PWS) of adjacent traces	
SURECIP - sum opposing offsets in prepared data (see below)	
SUSTACK - stack adjacent traces having the same key header word

In CWPROOT/src/su/main/statics:
SUADDSTATICS - ADD random STATICS on seismic data			
SURANDSTAT - Add RANDom time shifts STATIC errors to seismic traces	
SURESSTAT - Surface consistent source and receiver statics calculation
SUSTATICB - Elevation static corrections, apply corrections from	
SUSTATIC - Elevation static corrections, apply corrections from	
SUSTATICRRS - Elevation STATIC corrections, apply corrections from	

In CWPROOT/src/su/main/stretching_moveout_resamp:
SUILOG -- time axis inverse log-stretch of seismic traces	
SULOG -- time axis log-stretch of seismic traces		
SUNMO_a - NMO for an arbitrary velocity function of time and CDP with	     
SUNMO - NMO for an arbitrary velocity function of time and CDP	     
SUREDUCE - convert traces to display in reduced time		", 
SURESAMP - Resample in time                                       
SUSHIFT - shifted/windowed traces in time				
SUTAUPNMO - NMO for an arbitrary velocity function of tau and CDP	
SUTSQ -- time axis time-squared stretch of seismic traces	
SUTTOZ - resample from time to depth					
SUZTOT - resample from depth to time					

In CWPROOT/src/su/main/supromax:
SUGET  - Connect SU program to file descriptor for input stream.	
SUPUT - Connect SU program to file descriptor for output stream.	

In CWPROOT/src/su/main/synthetics_waveforms_testpatterns:

SUDGWAVEFORM - make Gaussian derivative waveform in SU format		
SUEA2DF - SU version of (an)elastic anisotropic 2D finite difference 	
SUFCTANISMOD - Flux-Corrected Transport correction applied to the 2D
SUFDMOD1 - Finite difference modelling (1-D 1rst order) for the	
SUFDMOD2 - Finite-Difference MODeling (2nd order) for acoustic wave equation
SUFDMOD2_PML - Finite-Difference MODeling (2nd order) for acoustic wave
SUGOUPILLAUD - calculate 1D impulse response of	 		
SUGOUPILLAUDPO - calculate Primaries-Only impulse response of a lossless
SUIMP2D - generate shot records for a line scatterer	
SUIMP3D - generate inplane shot records for a point 	
SUIMPEDANCE - Convert reflection coefficients to impedances.  
SUKDSYN2D - Kirchhoff Depth SYNthesis of 2D seismic data from a	
SUNHMOSPIKE - generates SPIKE test data set with a choice of several   
SUNULL - create null (all zeroes) traces	 		
SUPLANE - create common offset data file with up to 3 planes	
SURANDSPIKE - make a small data set of RANDom SPIKEs 		
SUREMAC2D - Acoustic 2D Fourier method modeling with high accuracy     
SUREMEL2DAN - Elastic anisotropic 2D Fourier method modeling with high 
SUSPIKE - make a small spike data set 			
SUSYNCZ - SYNthetic seismograms for piecewise constant V(Z) function	
SUSYNLV - SYNthetic seismograms for Linear Velocity function		
SUSYNLVCW - SYNthetic seismograms for Linear Velocity function	
SUSYNLVFTI - SYNthetic seismograms for Linear Velocity function in a  
SUSYNVXZ - SYNthetic seismograms of common offset V(X,Z) media via	
SUSYNVXZCS - SYNthetic seismograms of common shot in V(X,Z) media via	
SUVIBRO - Generates a Vibroseis sweep (linear, linear-segment,
SUWAVEFORM - generate a seismic wavelet				

In CWPROOT/src/su/main/tapering:
SUGAUSSTAPER - Multiply traces with gaussian taper		
SURAMP - Linearly taper the start and/or end of traces to zero.	
SUTAPER - Taper the edge traces of a data panel to zero.	
SUTXTAPER - TAPER in (X,T) the edges of a data panel to zero.	

In CWPROOT/src/su/main/transforms:
SUAMP - output amp, phase, real or imag trace from			
SUANALYTIC - use the Hilbert transform to generate an ANALYTIC	
SUCCEPSTRUM - Compute the complex CEPSTRUM of a seismic trace 	"
SUCCWT - Complex continuous wavelet transform of seismic traces	
SUCEPSTRUM - transform to the CEPSTRal domain				
SUCLOGFFT - fft real time traces to complex log frequency domain traces
SUCWT - generates Continous Wavelet Transform amplitude, regularity	
SUFFT - fft real time traces to complex frequency traces		
SUGABOR -  Outputs a time-frequency representation of seismic data via
SUHILB - Hilbert transform					
SUICEPSTRUM - fft of complex log frequency traces to real time traces
SUICLOGFFT - fft of complex log frequency traces to real time traces
SUIFFT - fft complex frequency traces to real time traces	
SUMINPHASE - convert input to minimum phase				
SUPHASEVEL - Multi-mode PHASE VELocity dispersion map computed
SURADON - compute forward or reverse Radon transform or remove multiples
SUSLOWFT - Fourier Transforms by a (SLOW) DFT algorithm (Not an FFT)
SUSLOWIFT - Fourier Transforms by (SLOW) DFT algorithm (Not an FFT)
SUSPECFK - F-K Fourier SPECtrum of data set			
SUSPECFX - Fourier SPECtrum (T -> F) of traces 		
SUSPECK1K2 - 2D (K1,K2) Fourier SPECtrum of (x1,x2) data set		
SUTAUP - forward and inverse T-X and F-K global slant stacks		
SUWFFT - Weighted amplitude FFT with spectrum flattening 0->Nyquist	
SUZEROPHASE - convert input to zero phase equivalent			

In CWPROOT/src/su/main/velocity_analysis:
SURELANAN - REsiduaL-moveout semblance ANalysis for ANisotropic media	
SURELAN - compute residual-moveout semblance for cdp gathers based	
 SUTIVEL -  SU Transversely Isotropic velocity table builder		
SUVEL2DF - compute stacking VELocity semblance for a single time in   
SUVELAN - compute stacking velocity semblance for cdp gathers		     
SUVELAN_NCCS - compute stacking VELocity panel for cdp gathers	     
SUVELAN_NSEL - compute stacking VELocity panel for cdp gathers	     
SUVELAN_UCCS - compute stacking VELocity panel for cdp gathers	     
SUVELAN_USEL - compute stacking velocity panel for cdp gathers	     

In CWPROOT/src/su/main/well_logs:
LAS2SU - convert las2 format well log curves to su traces	
SUBACKUS - calculate Thomsen anisotropy parameters from 	
SUBACKUSH - calculate Thomsen anisotropy parameters from 	
SUGASSMAN - Model reflectivity change with rock/fluid properties	
SULPRIME - find appropriate Backus average length for  	
SUWELLRF - convert WELL log depth, velocity, density data into a	

In CWPROOT/src/su/main/windowing_sorting_muting:
SUCOMMAND - pipe traces having the same key header word to command	
SUGETGTHR - Gets su files from a directory and put them               
SUGPRFB - SU program to remove First Breaks from GPR data		
SUKILL - zero out traces					
SUMIXGATHERS - mix two gathers					
SUMUTE - MUTE above (or below) a user-defined polygonal curve with	", 
SUPAD - Pad zero traces						
SUPUTGTHR - split the stdout flow to gathers on the bases of given	
SUSORT - sort on any segy header keywords			
SUSORTY - make a small 2-D common shot off-end  		
SUSPLIT - Split traces into different output files by keyword value	
SUWIND - window traces by key word					
SUWINDPOLY - WINDow data to extract traces on or within a respective	

In CWPROOT/src/su/graphics/psplot:
SUPSCONTOUR - PostScript CONTOUR plot of a segy data set		
SUPSCUBE - PostScript CUBE plot of a segy data set			
SUPSCUBECONTOUR - PostScript CUBE plot of a segy data set		
SUPSGRAPH - PostScript GRAPH plot of a segy data set			
SUPSIMAGE - PostScript IMAGE plot of a segy data set			
SUPSMAX - PostScript of the MAX, min, or absolute max value on each trace
SUPSMOVIE - PostScript MOVIE plot of a segy data set			
SUPSWIGB - PostScript Bit-mapped WIGgle plot of a segy data set	
SUPSWIGP - PostScript Polygon-filled WIGgle plot of a segy data set	

In CWPROOT/src/su/graphics/xplot:
SUXCONTOUR - X CONTOUR plot of Seismic UNIX tracefile via vector plot call
SUXGRAPH - X-windows GRAPH plot of a segy data set			
SUXIMAGE - X-windows IMAGE plot of a segy data set	                
SUXMAX - X-windows graph of the MAX, min, or absolute max value on	
SUXMOVIE - X MOVIE plot of a 2D or 3D segy data set 			
SUXPICKER - X-windows  WIGgle plot PICKER of a segy data set		
SUXWIGB - X-windows Bit-mapped WIGgle plot of a segy data set		

In CWPROOT/src/tri/main:
GBBEAM - Gaussian beam synthetic seismograms for a sloth model 	
NORMRAY - dynamic ray tracing for normal incidence rays in a sloth model
TRI2UNI - convert a TRIangulated model to UNIformly sampled model	
TRIMODEL - make a triangulated sloth (1/velocity^2) model                  		
TRIRAY - dynamic RAY tracing for a TRIangulated sloth model		
TRISEIS - Gaussian beam synthetic seismograms for a sloth model	
UNI2TRI - convert UNIformly sampled model to a TRIangulated model	

In CWPROOT/src/xtri:
SXPLOT - X Window plot a triangulated sloth function s(x1,x2)		

In CWPROOT/src/tri/graphics/psplot:
SPSPLOT - plot a triangulated sloth function s(x,z) via PostScript	

In CWPROOT/src/comp/dct/main:
DCTCOMP - Compression by Discrete Cosine Transform			
DCTUNCOMP - Discrete Cosine Transform Uncompression 			
ENTROPY - compute the ENTROPY of a signal			
WPTCOMP - Compression by Wavelet Packet Compression 			
WPTUNCOMP - Uncompress  WPT compressed data				
WTCOMP - Compression by Wavelet Transform				
WTUNCOMP - UNCOMPression of WT compressed data			

In CWPROOT/src/comp/dwpt/1d/main:
WPC1COMP2 --- COMPress a 2D seismic section trace-by-trace using 	
WPC1UNCOMP2 --- UNCOMPRESS a 2D seismic section, which has been	

In CWPROOT/src/comp/dwpt/2d/main:
WPCCOMPRESS --- COMPRESS a 2D section using Wavelet Packets		
WPCUNCOMPRESS --- UNCOMPRESS a 2D section 				

Shells: 

In CWPROOT/src/cwp/shell:
# ARGV - give examples of dereferencing char **argv
# COPYRIGHT - insert CSM COPYRIGHT lines at top of files in working directory
# CPALL , RCPALL - for local and remote directory tree/file transfer
# CWPFIND - look for files with patterns in CWPROOT/src/cwp/lib
# Grep  - recursively call egrep in pwd
# DIRTREE - show DIRectory TREE
# FILETYPE - list all files of given type
# NEWCASE - Changes the case of all the filenames in a directory, dir
# OVERWRITE - copy stdin to stdout after EOF
# PRECEDENCE - give table of C precedences from Kernighan and Ritchie
# REPLACE - REPLACE string1 with string2  in files
# THIS_YEAR - print the current year
# TIME_NOW - prints time in ZULU format with no spaces 
# TODAYS_DATE - prints today's date in ZULU format with no spaces 
# USERNAMES - get list of all login names
# VARLIST - list variables used in a Fortran program
# WEEKDAY - prints today's WEEKDAY designation
# ZAP - kill processes by name

In CWPROOT/src/par/shell:
# GENDOCS - generate complete list of selfdocs in latex form
# STRIPTOTXT -  put files from $CWPROOT/src/doc/Stripped into a new
# UPDATEDOCALL - put self-docs in ../doc/Stripped
# UPDATEDOC - put self-docs in ../doc/Stripped and ../doc/Headers
# UPDATEHEAD - update ../doc/Headers/Headers.all

In CWPROOT/src/psplot/shell:
# MERGE2 - put 2 standard size PostScript figures on one page
# MERGE4 - put 4 standard size PostScript plots on one page

In CWPROOT/src/su/shell:
# LOOKPAR - show getpar lines in SU code with defines evaluated
# MAXDIFF - find absolute maximum difference in two segy data sets
# RECIP - sum opposing (reciprocal) offsets in cdp sorted data
# RMAXDIFF - find percentage maximum difference in two segy data sets
# SUAGC - perform agc on SU data 
# SUBAND - Trapezoid-like Sin squared tapered Bandpass filter via  SUFILTER
# SUDIFF, SUSUM, SUPROD, SUQUO, SUPTDIFF, SUPTSUM,
# SUDOC - get DOC listing for code
# SUENV - Instantaneous amplitude, frequency, and phase via: SUATTRIBUTES
# SUFIND - get info from self-docs
# SUFIND - get info from self-docs
# SUGENDOCS - generate complete list of selfdocs in latex form
# SUHELP - list the CWP/SU programs and shells
# SUKEYWORD -- guide to SU keywords in segy.h
# SUNAME - get name line from self-docs
# UNGLITCH - zonk outliers in data

Libs: 

In CWPROOT/src/cwp/lib:
ABEL - Functions to compute the discrete ABEL transform:
AIRY - Approximate the Airy functions  Ai(x), Bi(x) and their respective
ALLOC - Allocate and free multi-dimensional arrays
ANTIALIAS - Butterworth anti-aliasing filter
AXB - Functions to solve a linear system of equations Ax=b by LU
BIGMATRIX - Functions to manipulate 2-dimensional matrices that are too big 
BUTTERWORTH - Functions to design and apply Butterworth filters:
COMPLEX - Functions to manipulate complex numbers
COMPLEXD - Functions to manipulate double-precision complex numbers
COMPLEXF  - Subroutines to perform operations on complex numbers.
COMPLEXFD  - Subroutines to perform operations on double complex numbers.
Conjugate Gradient routines -
CONVOLUTION - Compute z = x convolved with y
CUBICSPLINE - Functions to compute CUBIC SPLINE interpolation coefficients
DBLAS - Double precision Basic Linear Algebra subroutines
DGE - Double precision Gaussian Elimination matrix subroutines  adapted
DIFFERENTIATE - simple DIFFERENTIATOR codes
PFAFFT - Functions to perform Prime Factor (PFA) FFT's, in place
*CWP_Exit - exit subroutine for CWP/SU codes
FRANNOR - functions to generate a pseudo-random float normally distributed
FRANUNI - Functions to generate a pseudo-random float uniformly distributed
HANKEL - Functions to compute discrete Hankel transforms
Hartley - routines for fast Hartley transform
HILBERT - Compute Hilbert transform y of x
HOLBERG1D - Compute coefficients of Holberg's 1st derivative filter
INTCUB - evaluate y(x), y'(x), y''(x), ... via piecewise cubic interpolation
INTL2B - bilinear interpolation of a 2-D array of bytes
INTLIN - evaluate y(x) via linear interpolation of y(x[0]), y(x[1]), ...
INTLINC - evaluate complex y(x) via linear interpolation of y(x[0]), y(x[1]), ...
INTLIRR2B - bilinear interpolation of a 2-D array of bytes
INTSINC8 - Functions to interpolate uniformly-sampled data via 8-coeff. sinc
INTTABLE8 -  Interpolation of a uniformly-sampled complex function y(x)
LINEAR_REGRESSION - Compute linear regression of (y1,y2,...,yn) against 
maxmin - subroutines that pertain to maximum and minimum values
MKDIFF - Make an n-th order DIFFerentiator via Taylor's series method.
MKHDIFF - Compute filter approximating the bandlimited HalF-DIFFerentiator.
MKSINC - Compute least-squares optimal sinc interpolation coefficients.
MNEWT - Solve non-linear system of equations f(x) = 0 via Newton's method
ORTHPOLYNOMIALS - compute ORTHogonal POLYNOMIALS
PFAFFT - Functions to perform Prime Factor (PFA) FFT's, in place
POLAR - Functions to map data in rectangular coordinates to polar and vise versa
PRINTERPLOT - Functions to make a printer plot of a 1-dimensional array
QUEST - Functions to ESTimate Quantiles:
RESSINC8 - Functions to resample uniformly-sampled data  via 8-coefficient sinc
RFWTVA - Rasterize a Float array as Wiggle-Trace-Variable-Area.
RFWTVAINT - Rasterize a Float array as Wiggle-Trace-Variable-Area, with
SBLAS - Single precision Basic Linear Algebra Subroutines
SCAXIS - compute a readable scale for use in plotting axes
SGA - Single precision general matrix functions adapted from LINPACK FORTRAN:
SHFS8R - Shift a uniformly-sampled real-valued function y(x) via
SINC - Return SINC(x) for as floats or as doubles
SORT - Functions to sort arrays of data or arrays of indices
SQR - Single precision QR decomposition functions adapted from LINPACK FORTRAN:
STOEP - Functions to solve a symmetric Toeplitz linear system of equations
STRSTUFF -- STRing manuplation subs
SWAPBYTE - Functions to SWAP the BYTE order of binary data 
SYMMEIGEN - Functions solving the eigenvalue problem for symmetric matrices

TRIDIAGONAL - Functions to solve tridiagonal systems of equations Tu=r for u.
UNWRAP_PHASE - routines to UNWRAP phase of fourier transformed data
VANDERMONDE - Functions to solve Vandermonde system of equations Vx=b 
WAVEFORMS   Subroutines to define some wavelets for modeling of seismic
WINDOW - windowing routines
wrapArray - wrap an array
XCOR - Compute z = x cross-correlated with y
XINDEX - determine index of x with respect to an array of x values
YCLIP - Clip a function y(x) defined by linear interpolation of the
YXTOXY - Compute a regularly-sampled, monotonically increasing function x(y)
ZASC - routine to translate ncharacters from ebcdic to ascii
ZEBC - routine to translate ncharacters from ascii to ebcdic

In CWPROOT/src/par/lib:
ATOPKGE - convert ascii to arithmetic and with error checking
DOCPKGE - Function to implement the CWP self-documentation facility
EALLOC - Allocate and free multi-dimensional arrays with error reports.
ERRPKGE - routines for reporting errors
FILESTAT - Functions to determine and output the type of a file from file
GETPARS - Functions to GET PARameterS from the command line. Numeric
LINCOEFF - subroutines to create linearized reflection coefficients
MINFUNC - routines to MINimize FUNCtions
MODELING - Seismic Modeling Subroutines for SUSYNLV and clones
REFANISO - Reflection coefficients for Anisotropic media
RKE - integrate a system of n-first order ordinary differential equations
SMOOTH - Functions to compute smoothing of 1-D or 2-D input data
SUBCALLS - routines for system functions with error checking
SYSCALLS - routines for SYSTEM CALLs with error checking
TAUP - Functions to perform forward and inverse taup transforms (radon or
UPWEIK - Upwind Finite Difference Eikonal Solver
VND - large out-of-core multidimensional block matrix transpose
WTLIB - Functions for wavelet transforms

In CWPROOT/src/su/lib:
FGETGTHR - get gathers from SU datafiles
fgethdr - get segy tape identification headers from the file by file pointer
FGETTR - Routines to get an SU trace from a file 
FPUTGTHR - put gathers to a file
FPUTTR - Routines to put an SU trace to a file 
HDRPKGE - routines to access the SEGY header via the hdr structure.
TABPLOT - TABPLOT selected sample points on selected trace
VALPKGE - routines to handle variables of type Value

In CWPROOT/src/psplot/lib:
BASIC - Basic C function interface to PostScript
PSAXESBOX3 -  Functions draw an axes box via PostScript, estimate bounding box
PSAXESBOX - Functions to draw PostScript axes and estimate bounding box
PSCAXESBOX - Draw an axes box for cube via PostScript
PSCONTOUR - draw contour of a two-dimensional array via PostScript
PSDRAWCURVE - Functions to draw a curve from a set of points
PSLEGENDBOX - Functions to draw PostScript axes and estimate bounding box
PSWIGGLE - draw wiggle-trace with (optional) area-fill via PostScript

In CWPROOT/src/xplot/lib:
AXESBOX - Functions to draw axes in X-windows graphics
COLORMAP - Functions to manipulate X colormaps:
DRAWCURVE - Functions to draw a curve from a set of points
IMAGE - Function for making the image in an X-windows image plot
LEGENDBOX - draw a labeled axes box for a legend (i.e. colorscale)
RUBBERBOX -  Function to draw a rubberband box in X-windows plots
WINDOW - Function to create a window in X-windows graphics
XCONTOUR - draw contour of a two-dimensional array via X vectorplot calls

In CWPROOT/src/Xtcwp/lib:
AXES - the Axes Widget
COLORMAP - Functions to manipulate X colormaps:
FX - Functions to support floating point coordinates in X
MISC - Miscellaneous X-Toolkit functions
RESCONV - general purpose resource type converters
RUBBERBOX -  Function to draw a rubberband box in X-windows plots

In CWPROOT/src/Xmcwp/lib:
RADIOBUTTONS -  convenience functions creating and using radio buttons
SAMPLES - Motif-based Graphics Functions

In CWPROOT/src/tri/lib:
CHECK - CHECK triangulated models
CIRCUM - define CIRCUMcircles for Delaunay triangulation
COLINEAR - determine if edges or vertecies are COLINEAR in triangulated
CREATE - create model, boundary edge triangles, edge face, edge vertex, add
DELETE - DELETE vertex, model, edge, or boundary edge from triangulated model
DTE - Distance to Edge
FIXEDGES - FIX or unFIX EDGES between verticies
INSIDE -  Is a vertex or point inside a circum circle, etc. of a triangulated
NEAREST - NEAREST edge or vertex in triangulated model
PROJECT - project to edge in triangulated model
READWRITE - READ or WRITE a triangulated model

In CWPROOT/src/cwputils:
CPUSEC - return cpu time (UNIX user time) in seconds
CPUTIME - return cpu time (UNIX user time) in seconds using ANSI C built-ins
WALLSEC - Functions to time processes
WALLTIME - Function to show time a process takes to run

In CWPROOT/src/comp/dct/lib:
BUFFALLOC - routines to ALLOCate/initialize and free BUFFers
DCT1 - 1D Discreet Cosine Transform Routines
DCT2 - 2D Discrete Cosine Transform Routines
DCTALLOC - ALLOCate space for transform tables for 1D DCT
GETFILTER - GET wavelet FILTER type
HUFFMAN - Routines for in memory Huffman coding/decoding
LCT1 - functions used to perform the 1D Local Cosine Transform (LCT)
LPRED - Lateral Prediction of Several Plane Waves
PENCODING - Routines to en/decode the quantized integers for lossless 
QUANT - QUANTization routines
RLE - routines for in memory silence en/decoding
WAVEPACK1 - 1D wavelet packet transform
WAVEPACK2 - 2D Wavelet PACKet transform 
WAVEPACK1 - 1D wavelet packet transform
WAVETRANS2 - 2D wavelet transform by tensor-product of two 1D transforms

In CWPROOT/src/comp/dct/lib:
BUFFALLOC - routines to ALLOCate/initialize and free BUFFers
DCT1 - 1D Discreet Cosine Transform Routines
DCT2 - 2D Discrete Cosine Transform Routines
DCTALLOC - ALLOCate space for transform tables for 1D DCT
GETFILTER - GET wavelet FILTER type
HUFFMAN - Routines for in memory Huffman coding/decoding
LCT1 - functions used to perform the 1D Local Cosine Transform (LCT)
LPRED - Lateral Prediction of Several Plane Waves
PENCODING - Routines to en/decode the quantized integers for lossless 
QUANT - QUANTization routines
RLE - routines for in memory silence en/decoding
WAVEPACK1 - 1D wavelet packet transform
WAVEPACK2 - 2D Wavelet PACKet transform 
WAVEPACK1 - 1D wavelet packet transform
WAVETRANS2 - 2D wavelet transform by tensor-product of two 1D transforms

In CWPROOT/src/comp/dwpt/1d/lib:
WBUFFALLOC -  routines to allocate/initialize and free buffers in wavelet
WPC1 - routines for compress a single seismic trace using wavelet packets 
WPC1CODING - routines for encoding the integer symbols in 1D WPC 
wpc1Quant - quantize
WPC1TRANS - routines to perform a 1D wavelet packet transform 

In CWPROOT/src/comp/dwpt/2d/lib:
WPCBUFFAL - routines to allocate/initialize and free buffers
WPC - routines for compress a 2D seismic section using wavelet packets 
WPCCODING - Routines for in memory coding of the quantized coeffiecients
WPCENDEC -  Wavelet Packet Coding, Encoding and Decoding routines
WPCHUFF -Routines for in memory Huffman coding/decoding
WPCPACK2 - routine to perform a 2D wavelet packet transform 
WPCQUANT - quantization routines for WPC
WPCSILENCE - routines for in memory silence en/decoding
\end{verbatim}
\begin{verbatim}

To search on a program name fragment, type:
      suname name_fragment <CR>

For more information type: program_name <CR>

  Items labeled with an asterisk (*) are C programs 				that may
  or may not have this self documentation feature. 

  Items labeled with a pound sign (#) are shell 				scripts that may,
  or may not have the self documentation feature.

\end{verbatim}
\pagebreak
\section*{Self Documentations}
Mains: 
\begin{verbatim}
______
 CTRLSTRIP - Strip non-graphic characters

 ctrlstrip <dirtyfile >cleanfile
 

\end{verbatim}
\begin{verbatim}
______
 DOWNFORT - change Fortran programs to lower case, preserving strings

 Usage:   downfort < infile.f > outfile.f




 Credits:
 	 Brian Sumner  c. 1984

\end{verbatim}
\begin{verbatim}
______
 FCAT - fast cat with 1 read per file 
	
 Usage: fcat file1 file2 ... > file3
	



 Credits:
	Shuki

 This program belongs to the Center for Wave Phenomena
 Colorado School of Mines
/
\end{verbatim}
\begin{verbatim}
______
 ISATTY - pass on return from isatty(2)

 Usage:  isatty filedes

 See:   man isatty     for further information 



 Credits:
	CWP: Shuki

 This program belongs to the Center for Wave Phenomena
 Colorado School of Mines

\end{verbatim}
\begin{verbatim}
______
 MAXINTS - Compute maximum and minimum sizes for integer types 
	(quick and dirty)

 Usage:  maxints

 Note: These results will be in limits.h on most systems




 Credits:
	CWP: Jack K. Cohen

 This program belongs to the Center for Wave Phenomena
 Colorado School of Mines

\end{verbatim}
\begin{verbatim}
______
 PAUSE - prompt and wait for user signal to continue

 Usage:   pause [optional arguments]

 Note:
 Default prompt is "press return key to continue" which is *evoked 
 by calling pause with no arguments.  The word,
 "continue", is replaced by any optional arguments handed to pause.
 Thus, the command "pause do plot" will evoke the prompt,
 "press return key to do plot".


\end{verbatim}
\begin{verbatim}
______
 T - time and date for non-military types

 Usage: t

 Credit: Jack 

\end{verbatim}
\begin{verbatim}
______
 UPFORT - change Fortran programs to upper case, preserving strings

 Usage:   upfort < infile.f > outfile.f

 Reverse of: downfort


\end{verbatim}
\pagebreak
\begin{verbatim}
 A2B - convert ascii floats to binary 				

 a2b <stdin >stdout outpar=/dev/null 				

 Required parameters:						
 	none							

 Optional parameters:						
 	n1=2		floats per line in input file		

 	outpar=/dev/null output parameter file, contains the	
			number of lines (n=)			
 			other choices for outpar are: /dev/tty,	
 			/dev/stderr, or a name of a disk file	

 Credits:
	CWP: Jack K. Cohen, Dave Hale
	Hans Ecke 2002: Replaced line-wise file reading via gets() with 
			float-wise reading via fscanf(). This makes it 
			much more robust: it does not impose a specific 
			structure on the input file.


\end{verbatim}
\pagebreak
\begin{verbatim}
 A2I - convert Ascii to binary Integers			

 a2i <stdin >stdout outpar=/dev/tty 				

 Required parameters:						
 	none							

 Optional parameters:						
 	n1=2		integers per line in input file		

 	outpar=/dev/tty	output parameter file, contains the	
			number of lines (n=)			

\end{verbatim}
\pagebreak
\begin{verbatim}
 ADDRVL3D - Add a random velocity layer (RVL) to a gridded             
            v(x,y,z) velocity model                                    

	addrvl3d <infile n1= n2= >outfile [parameters]			

 Required Parameters:							
 n1=		number of samples along 1st dimension			
 n2=		number of samples along 2nd dimension			

 Optional Parameters:							

 n3=1          number of samples along 3rd dimension			

 mode=1             add single layer populated with random vels	
                    =2 add nrvl layers of random thickness and vel     
 seed=from_clock    random number seed (integer)			

 ---->New layer geometry info						
 i1beg=1       1st dimension beginning sample 				
 i1end=n1/5    1st dimension ending sample 				
 i2beg=1       2nd dimension beginning sample 				
 i2end=n2      2nd dimension ending sample 				
 i3beg=1       3rd dimension beginning sample 				
 i3end=n3      3rd dimension ending sample 				
 ---->New layer velocity info						
 vlsd=v/3     range (std dev) of random velocity in layer, 		
               where v=v(0,0,i1) and i1=(i1beg+i1end)/2 	 	
 add=1         add random vel to original vel (v_orig) at that point 	
               =0 replace vel at that point with (v_orig+v_rand) 	
 how=0         random vels can be higher or lower than v_orig		
               =1 random vels are always lower than v_orig		
               =2 random vels are always higher than v_orig		
 cvel=2000     layer filled with constant velocity cvel 		
               (overides vlsd,add,how params)			
 ---->Smoothing parameters (0 = no smoothing)				
 r1=0.0	1st dimension operator length in samples		
 r2=0.0	2nd dimension operator length in samples		
 r3=0.0	3rd dimension operator length in samples		
 slowness=0	=1 smoothing on slowness; =0 smoothing on velocity	

 nrvl=n1/10    number of const velocity layers to add     		
 pdv=10.       percentage velocity deviation (max) from input model	

 Notes:								
 1. Smoothing radii usually fall in the range of [0,20].		
 2. Smoothing radii can be used to set aspect ratio of random velocity 
    anomalies in the new layer.  For example (r1=5,r2=0,r3=0) will     
    result in vertical vel streaks that mimick vertical fracturing.    
 3. Smoothing on slowness works better to preserve traveltimes relative
    to the unsmoothed case.						
 4. Default case is a random velocity (+/-30%) near surface layer whose
    thickness is 20% of the total 2D model thickness.			
 5. Each layer vel is a random perturbation on input model at that level.
 6. The depth dimension is assumed to be along axis 1.			

 Example:								
 1. 2D RVL with no smoothing						
   makevel nz=250 nx=200 | addrvl3d n1=250 n2=200 | ximage n1=250      
 2. 3D RVL with no smoothing						
   makevel nz=250 nx=200 ny=220 |					
   addrvl3d n1=250 n2=200 n3=220 | 					
   xmovie n1=250 n2=200					    	



 Author:  Saudi Aramco: Chris Liner Jan/Feb 2005
          Based on smooth3d (CWP: Zhenyue Liu  March 1995)


\end{verbatim}
\pagebreak
\begin{verbatim}
 B2A - convert binary floats to ascii				

 b2a <stdin >stdout 						

 Required parameters:						
 	none							

 Optional parameters:						
 	n1=2		floats per line in output file 		
       format=0	scientific notation	 		
 			=1 long decimal float form		

 	outpar=/dev/tty output parameter file, contains the	
			number of lines (n=)			
                       other choices for outpar are: /dev/tty, 
                       /dev/stderr, or a name of a disk file   

 Note: 							
 Parameter:							", 
  format=0 uses printf("%15.10e ", x[i1])			
  format=1 uses printf("%15.15f ", x[i1])			

 Credits:
	CWP: Jack K. Cohen

\end{verbatim}
\pagebreak
\begin{verbatim}
 CELLAUTO - Two-dimensional CELLular AUTOmata			  	

   cellauto > stdout [optional params]					

 Optional Parameters:							
 n1=500	output dimensions of image (n1 x n1 pixels)	 	
 rule=30	CA rule (Wolfram classification)			
 		Others: 54,60,62,90,94,102,110,122,126			
                       150,158,182,188,190,220,222,225,226,250		
 fill=0	Don't fill image (=1 fill image)			
 f0=330	fill zero values with f0				
 f1=3000	fill non-zero values with f1				
 ic=1		initial condition for centered unit value at t=0	
               = 2 for multiple random units				
 nc=20		number of random units (if ic=2)			
 tc=1		random initial units at t=0 (if ic=2)			
               = 2 for initial units at random (t,x)			
 verbose=0	silent operation					
               = 1 echos 'porosity' of the CA in bottom half of image	
 seed=from_clock    	random number seed (integer)            	

 Notes:								
 This program generates a select set of Wolframs fundamental cellular	
 automata. This may be useful for constructing rough, vuggy wavespeed	
 profiles. The numbering scheme follows Stephen Wolfram's.		

 Example: 								
  cellauto rule=110 ic=2 nc=100 fill=1 f1=3000 | ximage n1=500 nx=500 &

 Here we simulate a complex near surface with air-filled 		
 vugs in hard country rock, with smoothing applied via smooth2 	

  cellauto rule=110 ic=2 nc=100 fill=1 f1=3000 n1=500 |		
  smooth2 n1=500 n2=500 r1=5 r2=5 > vfile.bin				


 Credits:
	UHouston: Chris Liner 	

 Trace header fields accessed:  ns
 Trace header fields modified:  ns and delrt

\end{verbatim}
\pagebreak
\begin{verbatim}
char* sdoc[] = {

   CPFTREND - generate picks of the Cumulate Probability Function 	

 Required parameters:							
  ix=      - column containing X variable				
  iy=      - column containing Y variable				
  min_x=   - minimum X bin						
  max_x=   - maximum X bin						
  min_y=   - minimum Y bin 						
  max_y=   - maximum Y bin						

 Optional parameters:							
  nx=100    - number of X bins 					
  ny=100    - number of Y bins 					
  logx=0   - =1 use logarithmic scale for X axis			
  logy=0   - =1 use logarithmic scale for Y axis			
  ir=       - column containing reject variable 			
  rmin=     - reject values below rmin 				
  rmax=     - reject values above rmax 				
              NOTE: only one, rmin or rmax, may be used		
 NOTES:								
  cpftrend makes picks on the 2D cumulate representing the		
  probability density function of the input data.			

   Commandline options allow selecting any of several normalizations	
   to apply to the distributions.					


  cpftrend(1) makes picks on the 2D cumulate representing the
  probability density function of the input data.

   Commandline options allow selecting any of several normalizations 
   to apply to the distributions.

 Credits:  Reginald H. Beardsley                      rhb@acm.org
     Copyright 2006 Exploration Software Consultants Inc.
\end{verbatim}
\pagebreak
\begin{verbatim}
 CSHOTPLOT - convert CSHOT data to files for CWP graphers		

 cshotplot <cshot1plot [optional parameter file]			

 Required parameters:							
 	none 								

 Optional parameter:							
 	outpar=/dev/tty		output parameter file, contains:	
					number of plots (n2=)		
					points in each plot (n1=)	
					colors for plots (linecolor=)	

\end{verbatim}
\pagebreak
\begin{verbatim}
 DZDV - determine depth derivative with respect to the velocity	",  
  parameter, dz/dv,  by ratios of migrated data with the primary 	
  amplitude and those with the extra amplitude				

 dzdv <infile afile=afile dfile=dfile>outfile [parameters]		

 Required Parameters:							
 infile=	input of common image gathers with primary amplitude	
 afile=	input of common image gathers with extra amplitude	
 dfile=	output of imaged depths in common image gathers 	
 outfile=	output of dz/dv at the imaged points			
 nx= 	        number of migrated traces 				
 nz=	        number of points in migrated traces 			
 dx=		horizontal spacing of migrated trace 			
 dz=	        vertical spacing of output trace 			
 fx=	        x-coordinate of first migrated trace 			
 fz=	        z-coordinate of first point in migrated trace 		
 off0=         first offset in common image gathers 			
 noff=	        number of offsets in common image gathers  		
 doff=	        offset increment in common image gathers  		
 cip=x1,z1,r1,..., cip=xn,zn,rn         description of input CIGS	
	x	x-value of a common image point				
	z	z-value of a common image point	at zero offset		
	r	r-parameter in a common image gather			

 Optional Parameters:							
 nxw, nzw=0		window widths along x- and z-directions in 	
			which points are contributed in solving dz/dv. 	


 Notes:								
 This program is used as part of the velocity analysis technique developed
 by Zhenyue Liu, CWP:1995.						

 Author: CWP: Zhenyue Liu,  1995
 
 Reference: 
 Liu, Z. 1995, "Migration Velocity Analysis", Ph.D. Thesis, Colorado
      School of Mines, CWP report #168.
 

\end{verbatim}
\pagebreak
\begin{verbatim}
 FARITH - File ARITHmetic -- perform simple arithmetic with binary files

 farith <infile >outfile [optional parameters]				

 Optional Parameters:							
 in=stdin	input file						
 out=stdout	output file						
 in2=	   second input file (required for binary operations)		
		   if it can't be opened as a file, it might be a scalar
 n=size_of_in,  fastest dimension (used only for op=cartprod is set)	
 isig=		 index at which signum function acts (used only for 	
			op=signum)					
 scale=	value to scale in by, used only for op=scale)		
 bias=		value to bias in by, used only for op=bias)		

 op=noop   noop for out = in						
	   neg  for out = -in						
	   abs  for out = abs(in)					
	   scale for out = in *scale					
	   bias for out = in + bias 					
	   exp  for out = exp(in)					
	   sin  for out = sin(in)					
	   cos  for out = cos(in)					
	   log  for out = log(in)					
	   sqrt for out = (signed) sqrt(in)				
	   sqr  for out = in*in						
	   degrad  for out = in*PI/180					
	   raddeg  for out = in*180/PI					
	   pinv  for out = (punctuated) 1 / in   			
	   pinvsqr  for out = (punctuated) 1 /in*in 			
	   pinvsqrt for out = (punctuated signed) 1 /sqrt(in) 		
	   add  for out = in + in2					
	   sub  for out = in - in2					
	   mul  for out = in * in2					
	   div  for out = in / in2					
		cartprod for out = in x in2					
		requires: n=size_of_in, fastest dimension in output	
		signum for out[i] = in[i] for i< isig  and			
				= -in[i] for i>= isig			
		requires: isig=point where signum function acts		
 Seismic operations:							
	   sloth   for  out =  1/in^2		Sloth from velocity in	
	   slowp   for  out =  1/in - 1/in2	Slowness perturbation	
	   slothp  for  out =  1/in^2 - 1/in2^2   Sloth perturbation	

 Notes:								
 op=sqrt takes sqrt(x) for x>=0 and -sqrt(ABS(x)) for x<0 (signed sqrt)

 op=pinv takes y=1/x for x!=0,  if x=0 then y=0. (punctuated inverse)	

 The seismic operations assume that in and in2 are wavespeed profiles.	
 "Slowness" is 1/wavespeed and "sloth" is  1/wavespeed^2.		
 Use "suop" and "suop2" to perform unary and binary operations on	
 data in the SU (SEGY trace) format.					

 The options "pinvsq" and "pinvsqrt" are also useful for seismic	
 computations involving converting velocity to sloth and vice versa.	

 The option "cartprod" (cartesian product) requires also that the	
 parameter n=size_of_in be set. This will be the fastest dimension	
 of the rectangular array that is output.				

 The option "signum" causes a flip in sign for all values with index	
 greater than "isig"	(really -1*signum(index)).			

 For file operations on SU format files, please use:  suop, suop2	



   AUTHOR:  Dave Hale, Colorado School of Mines, 07/07/89
	Zhaobo Meng added scale and cartprod, 10/01/96
	Zhaobo Meng added signum, 9 May 1997
	Tony Kocurko added scalar operations, August 1997
      John Stockwell added bias option 4 August 2004

\end{verbatim}
\pagebreak
\begin{verbatim}
 FLOAT2IBM - convert native binary FLOATS to IBM tape FLOATS	

 float2ibm <stdin >stdout 					

 Required parameters:						
 	none							

 Optional parameters:						
 endian=	byte order of your system (autodetected)	
 outpar=/dev/tty output parameter file, contains the		
			number of values (n=)			
		       other choices for outpar are: /dev/tty, 
		       /dev/stderr, or a name of a disk file   

 Notes:							
 endian=1 (big endian) endian=0 (little endian) byte order 	
 You probably will not have to set this, as the byte order of  
 your system is autodetected by the program. 			
 This program is usable for writing SEG Y traces with the headers
 stripped off.							

 Credits:
	CWP: John Stockwell, based on code by Jack K. Cohen

\end{verbatim}
\pagebreak
\begin{verbatim}
 FTNSTRIP - convert a file of binary data plus record delimiters created
      via Fortran to a file containing only binary values (as created via C)

 ftnstrip <ftn_data >c_data 						

 Caveat: this code assumes the conventional Fortran format of header	
         and trailer integer containing the number of byte in the	
         record.  This is overwhelmingly common, but not universal.	


 Credits:
	CWP: Jack K. Cohen

\end{verbatim}
\pagebreak
\begin{verbatim}
 FTNUNSTRIP - convert C binary floats to Fortran style floats	

 ftnunstrip <stdin >stdout 					

 Required parameters:						
 	none							

 Optional parameters:						
 	n1=1		floats per line in output file 		

 	outpar=/dev/tty output parameter file, contains the	
			number of lines (n=)			
 			other choices for outpar are: /dev/tty,	
 			/dev/stderr, or a name of a disk file	

 Notes: This program assumes that the record length is constant
 throughout the input and output files. 			
 In fortran code reading these floats, the following implied	
 do loop syntax would be used: 				
        DO i=1,n2						
                 READ (10) (someARRAY(j), j=1,n1) 		
        END DO							
 Here n1 is the number of samples per record, n2 is the number 
 of records, 10 is some default file (fort.10, for example), and
 someArray(j) is an array dimensioned to size n1		


 Credits:
	CWP: John Stockwell, Feb 1998,
            based on ftnstrip by: Jack K. Cohen

\end{verbatim}
\pagebreak
\begin{verbatim}
 GRM - Generalized Reciprocal refraction analysis for a single layer	

     grm <stdin >stdout  [parameters]    		 		

 Required parameters:							
 nt=		Number of arrival time pairs				
 dx=		Geophone spacing (m)					
 v0=		Velocity in weathering layer (m/s)			
 abtime=	If set to 0, use last time as a-b, else give time (ms)  

 Optional parameters:							
 XY=      Value of XY if you want to override the optimum XY		
	  algorithm in the program. If it is not an integer multiple of 
	 dx, then it will be converted to the closest			
		 one.							
	XYmax   Maximum offset distance allowed when searching for      
		optimum XY (m)  (Default is 2*dx*10)			
	depthres  Size of increment in x during verical depth search(m) 
		  (Default is 0.5m)					
 Input file:								
	4 column ASCII - x,y, forward time, reverse time 		
 Output file:								
	1) XYoptimum  							
	2) apparent refractor velcocity					
	3) x, y, z(x,y), y-z(x,y)					
		z(x,y) = calculated (GRM) depth below (x y) 		
		y-z(x,y) = GRM depth subtracted from y - absolute depth 
      .............							
      4) x, y, d(x,y), y-d(x,y), (error)  				
		d(x,y) = dip corrected depth estimate below (x,y)       
		y-d(x,y) = dip corrected absolute depth 		
		error = estimated error in depth due only to the inexact
		      matching of tangents to arcs in dip estimate.	

      If the XY calculation is bypassed and XY specified, the values	
      used will precede 1) above.  XYoptimum will still be calculated	
      and displayed for reference.					

 Notes:							       	
      Uses average refactor velocity along interface.			

  Credits:							       

     CWP: Steven D. Sheaffer						 
								       
     D. Palmer, "The Generalized Reciprocal Method of Seismic	  
     Refraction Interpretation", SEG, 1982.				  


\end{verbatim}
\pagebreak
\begin{verbatim}
 H2B - convert 8 bit hexidecimal floats to binary		

 h2b <stdin >stdout outpar=/dev/tty 				

 Required parameters:						
 	none							

 Optional parameters:						
 	outpar=/dev/tty output parameter file, contains the	
			number of lines (n=)			
 			other choices for outpar are: /dev/tty,	
 			/dev/stderr, or a name of a disk file	

 Note: this code may be used to recover binary data from PostScript
 bitmaps. To do this, strip away all parts of the PSfile that	
 are not the actual hexidecimal bitmap and run through h2b.	

 Note: that the binary file may need to be transposed using	
 "transp" to appear to be the same as input data.		

 Note:	output will be floats with the values 0-255		
\end{verbatim}
\pagebreak
\begin{verbatim}
 HTI2STIFF - convert HTI parameters alpha, beta, d(V), e(V), gamma	
		into stiffness tensor					

    hti2stiff  [optional parameter] (output is to   outpar)		

 Optional Parameters							
 alpha=2	    isotropy-plane p-wave velocity		   	
 beta=1	     fast isotropy-plan s-wave velocity			
 ev=0		e(V) 							
 dv=0		d(V)							
 gamma=0	    shear-wave splitting parameter			
 rho=1		density							
 sign		     sign of c13+c55 ( for most materials sign=1)	
 outpar=/dev/tty    output parameter file				

 Output:								
  c_ijkl	    stiffness components for x1=symmetry axis		
		    x3= vertical					



 Credits:   Andreas Rueger, CWP Aug 01, 1996

 Reference: Andreas Rueger, P-wave reflection coefficients for
    transverse isotropy with vertical and horizontal axis
	    of symmetry,  GEOPHYSICS


\end{verbatim}
\pagebreak
\begin{verbatim}
  HUDSON - compute  effective parameters of anisotropic solids	        
	   using Hudson's crack theory.                      		

 Required paramters: <none>                                             

 Optional parameters                                                   

 vp=4.5        p-wave velocity uncracked solid                  	
 vs=2.53       s-wave velocity uncracked solid                         
 rho=2.8       density      						
 cdens=0	crack density					        
 aspect=0      aspect ratio                                            
 fill=0        gas filled cracks                                       
               =1 water filled                                         
 outpar        =/dev/tty   output file                                 
 
Notes:									
 The cracks are assumed to be vertically aligned, penny-shaped and the	
 matrix is isotropic. The resulting anisotropic solid is of HTI symmetry.

 Output:                                                               
 Computes(a) stiffness elements                                        
         (b) density normalized stiffness components                   
         (c) generic Thomsen parameters (vp0,vs0,eps,delta,gamma)      
         (d) equivalent VTI parameters (alpha,beta,ev,dv,gv)           

 


 
 AUTHOR:: Andreas Rueger, Colorado School of Mines, 10/10/96
  
 Additional notes: 
  The routine can be easily modified to allow for any 
  filling adding attenuation is not trivial

 Technical Reference:
  Hudson's theory: Hudson, 1981: Wave speed and attenuation of elastic
                                 waves in material containing cracks.
                                 Geophys. J. R. astr. Soc 64, 133-150
                  Crampin, 1984: Effective anisotropic elastic constants
                                 for waves propagating through cracked
                                 solids: 
				  Geophys. J. R. astr. Soc 76, 135-145
  Equivalent VTI : Rueger, 1996: Reflection coefficients in transversely
                                 isotropic media with vertical and 
                                 horizontal axis of symmetry: Geophysics

\end{verbatim}
\pagebreak
\begin{verbatim}
 I2A - convert binary integers to ascii				

 i2a <stdin >stdout 						

 Required parameters:						
 	none							

 Optional parameters:						
 	n1=2		floats per line in output file 		

 	outpar=/dev/tty	output parameter file, contains the	
			number of lines (n=)			


 Credits:
 Potash Corporation: c. 2008, Balazs Nemeth,  Saskatoon, Saskatchewan.
   based on b2a.c by:  CWP: Jack K. Cohen

\end{verbatim}
\pagebreak
\begin{verbatim}
 IBM2FLOAT - convert IBM tape FLOATS to native binary FLOATS	

 ibm2float <stdin >stdout 					

 Required parameters:						
 	none							

 Optional parameters:						
 endian=	byte order of your system (autodetected)	
 outpar=/dev/tty output parameter file, contains the		
			number of values (n=)			
		       other choices for outpar are: /dev/tty, 
		       /dev/stderr, or a name of a disk file   

 Notes:							
 endian=1 (big endian) endian=0 (little endian) byte order 	
 You probably will not have to set this, as the byte order of  
 your system is autodetected by the program. 			
 This program is usable for reading SEG Y files with the headers
 stripped off.							

 Credits:
	CWP: John Stockwell, based on code by Jack K. Cohen

\end{verbatim}
\pagebreak
\begin{verbatim}
 KAPERTURE - generate the k domain of a line scatterer for a seismic array

 kaperture [optional parameters] >stdout 				

 Optional parameters							
 	x0=1000		point scatterer location			
 	z0=1000		point scatterer location			
 	nshot=1		number of shots					
 	sxmin=0		first shot location				
 	szmin=0		first shot location				
 	dsx=100		x-steps in shot location			
 	dsz=0		z-steps in shot location			
 	ngeo=1		number of receivers				
 	gxmin=0		first receiver location				
 	gzmin=0		first receiver location				
 	dgx=100		x-steps in receiver location			
 	dgz=0		z-steps in receiver location			
 	fnyq=125	Nyquist frequency  (Hz)				
 	fmax=125	maximum frequency  (Hz)				
 	fmin=5		minimum frequency  (Hz)				
 	nfreq=2		number of frequencies   			
 	both=0		= 1 gives negative freqs too			
 	nstep=60	points on Nyquist circle			
 	c=5000		speed						
 	outpar=/dev/tty output parameter file, contains:		
 				xmin, xmax, ymin, ymax 			
 				and npairs (needed for psgraph or xgraph)
 			other choices for outpar are: /dev/tty,		
 			/dev/stderr, or a name of a disk file		
 Notes:								
       nfreq=1 produces fmin						
       nstep=0 suppresses the Nyquist circle				
 				and npairs				
 Examples:								

 Default case: both=0 nfreq=2					

 	kaperture nshot=NSHOT ngeo=NGEO nstep=NSTEP |			
 	psgraph	n=NPAIRS,NSTEP mark=1,0 marksize=1,0 linewidth=0,1 |...
 		WHERE: NPAIRS=NSHOT*NGEO				

 Other cases: 								

 both=0 nfreq=NFREQ > 2						
 	kaperture both=0 nfreq=NFREQ nshot=NSHOT ngeo=NGEO nstep=NSTEP |
 	psgraph	n=NPAIRS,NSTEP mark=1,0 marksize=1,0 linewidth=0,1 |...	
 		WHERE: NPAIRS=NFREQ*NSHOT*NGEO				

 both=1 nfreq=NFREQ > 2						
 	kaperture both=1 nfreq=NFREQ nshot=NSHOT ngeo=NGEO nstep=NSTEP |
 	psgraph	n=NPAIRS,NSTEP mark=1,0 marksize=1,0 linewidth=0,1 |...	
 		 WHERE: NPAIRS=NFREQ*NSHOT*NGEO*2			

 When in doubt to the size of NPAIRS, redirect output of kaperture to	
 /dev/tty the first time to get npairs=:				
		 kaperture [optional parameters] > /dev/tty		
\end{verbatim}
\pagebreak
\begin{verbatim}
 LINRORT - linearized P-P, P-S1 and P-S2 reflection coefficients 	
		for a horizontal interface separating two of any of the	
		following halfspaces: ISOTROPIC, VTI, HTI and ORTHORHOMBIC. 

   linrort [optional parameters]					

 hspace1=ISO	medium type of the incidence halfspace:		 	
		=ISO ... isotropic					
		=VTI ... VTI anisotropy				 	
		=HTI ... HTI anisotropy				 	
		=ORT ... ORTHORHOMBIC anisotropy			
 for ISO:								
 vp1=2	 	P-wave velocity, halfspace1				
 vs1=1		S-wave velocity, halfspace1				
 rho1=2.7	density, halfspace1					

 for VTI:								
 vp1=2		P-wave vertical velocity (V33), halfspace1		
 vs1=1		S-wave vertical velocity (V44=V55), halfspace1		
 rho1=2.7	density, halfspace1					
 eps1=0	Thomsen's generic epsilon, halfspace1			
 delta1=0	Thomsen's generic delta, halfspace1			
 gamma1=0	Thomsen's generic gamma, halfspace1			", 

 for HTI:								
 vp1=2	 P-wave vertical velocity (V33), halfspace1			
 vs1=1	 "fast" S-wave vertical velocity (V44), halfspace1		
 rho1=2.7	density, halfspace1					
 eps1_v=0	Tsvankin's "vertical" epsilon, halfspace1		
 delta1_v=0	Tsvankin's "vertical" delta, halfspace1		
 gamma1_v=0	Tsvankin's "vertical" gamma, halfspace1		",	

 for ORT:								
 vp1=2		P-wave vertical velocity (V33), halfspace1		
 vs1=1	 x2-polarized S-wave vertical velocity (V44), halfspace1 	
 rho1=2.7	density, halfspace1					
 eps1_1=0	Tsvankin's epsilon in [x2,x3] plane, halfspace1	 	
 delta1_1=0	Tsvankin's delta in [x2,x3] plane, halfspace1		
 gamma1_1=0	Tsvankin's gamma in [x2,x3] plane, halfspace1	  	
 eps1_2=0	Tsvankin's epsilon in [x1,x3] plane, halfspace1		
 delta1_2=0	Tsvankin's delta in [x1,x3] plane, halfspace1		
 gamma1_2=0	Tsvankin's gamma in [x1,x3] plane, halfspace1	  	
 delta1_3=0	Tsvankin's delta in [x1,x2] plane, halfspace1		

 hspace2=ISO	medium type of the reflecting halfspace (the same	
		convention as above)					

 medium parameters of the 2nd halfspace follow the same convention	
 as above:								

 vp2=2.5		 vs2=1.2		rho2=3.0		
 eps2=0		  delta2=0					
 eps2_v=0		delta2_v=0		gamma2_v=0		
 eps2_1=0		delta2_1=0		gamma2_1=0		
 eps2_2=0		delta2_2=0		gamma2_2=0		
 delta2_3=0								

	(note you do not need "gamma2" parameter for evaluation	
	of weak-anisotropy reflection coefficients)			

 a_file=-1	the string '-1' ... incidence and azimuth angles are	
		generated automatically using the setup values below	
		a_file=file_name ... incidence and azimuth angles are	
		read from a file "file_name"; the program expects a	
		file of two columns [inc. angle, azimuth]		

 in the case of a_file=-1:						
 fangle=0	first incidence phase angle				
 langle=30	last incidence angle					
 dangle=1	incidence angle increment				
 fazim=0	first azimuth (in deg)				  	
 lazim=0	last azimuth  (in deg)				  	
 dazim=1	azimuth increment (in deg)				

 kappa=0.	azimuthal rotation of the lower halfspace2 (e.t. a	
		symmetry axis plane for HTI, or a symmetry plane for	
		ORTHORHOMBIC) with respect to the x1-axis		

 out_inf=info.out	information output file				
 out_P=Rpp.out	file with Rpp reflection coefficients			
 out_S=Rps.out	file with Rps reflection coefficients			
 out_SVSH=Rsvsh.out  file with SV and SH projections of reflection	
			coefficients					
 out_Error=error.out file containing error estimates evaluated during  
			the computation of the reflection coefficients;	


 Output:								
 out_P:								
 inc. phase angle, azimuth, reflection coefficient; for a_file=-1, the 
 inc. angle is the fast dimension					
 out_S:								
 inc. phase angle, azimuth, Rps1, Rps2, cos(PHI), sin(PHI); for	
 a_file=-1, the inc. angle is the fast dimension			", 
 out_SVSH:								
 inc. phase angle, azimuth, Rsv, Rsh, cos(PHI), sin(PHI); for	  
 a_file=-1, the inc. angle is the fast dimension			
 out_Error:								
 error estimates of Rpp, Rpsv and Rpsh approximations; global error is 
 analysed as well as partial contributions to the error due to the	
 isotropic velocity contrasts, and due to anisotropic  upper and lower 
 halfspaces. The error file is self-explanatory, see also descriptions 
 of subroutines P_err_2nd_order, SV_err_2nd_order and SH_err_2nd_order.


 Adopted Convention:							

 The right-hand Cartesian coordinate system with the x3-axis pointing  
 upward has been chosen. The upper halfspace (halfspace1)		
 contains the incident P-wave. Incidence angles can vary from <0,PI/2),
 azimuths are unlimited, +azimuth sense counted from x1->x2 axes	
 (azimuth=0 corresponds to the direction of x1-axis). In the current	
 version, the coordinate system is attached to the halfspace1 (e.t.	
 the symmetry axis plane of HTI halfspace1, or one of symmetry planes  
 of ORTHORHOMBIC halfspace1, is aligned with the x1-axis), however, the
 halfspace2 can be arbitrarily rotated along the x3-axis with respect  
 to the halfspace1. The positive weak-anisotropy polarization of the	
 reflected P-P wave (e.t. positive P-P reflection coefficient) is close
 to the direction of isotropic slowness vector of the wave (pointing	
 outward the interface). Similarly, weak-anisotropy S-wave reflection  
 coefficients are described in terms of "SV" and "SH" isotropic	
 polarizations, "SV" and "SH" being unit vectors in the plane	
 perpendicular to the isotropic slowness vector. Then, the positive	
 "SV" polarization vector lies in the incidence plane and points	
 towards the interface, and positive "SH" polarization vector is	
 perpendicular to the incidence plane, aligned with the positive	
 x2-axis, if azimuth=0. Rotation angle "PHI", characterizing a	
 rotation of "the best projection" of the S1-wave polarization	
 vector in the isotropic SV-SH plane in the incidence halfspace1, is	
 counted in the positive sense from "SV" axis (PHI=0) towards the	
 "SH" axis (PHI=PI/2). Of course, S2 is perpendicular to S1, and	
 the projection of S1 and S2 polarizations onto the SV-SH plane	
 coincides with SV and SH directions, respectively, for PHI=0.		

 The units for velocities are km/s, angles I/O are in degrees		

 Additional Notes:							
	The coefficients are computed as functions of phase incidence	
	angle and azimuth (determined by the incidence slowness vector).
	Vertical symmetry planes of the HTI and				
	ORTHORHOMBIC halfspaces can be arbitrarily rotated along the	
	x3-axis. The linearization is based on the assumption of weak	", 
	contrast in elastic medium parameters across the interface,	
	and the assumption of weak anisotropy in both halfspaces.	
	See the "Adopted Convention" paragraph below for a proper	
	input.								


 
  Author: Petr Jilek, CSM-CWP, December 1999.


\end{verbatim}
\pagebreak
\begin{verbatim}

 LORENZ - compute the LORENZ attractor				

  lorenz > [stdout]						

 Required Parameters: none					
 Optional Parameters:						
 rho=28.0		parameter for lorenz equations		
 sigma=10.0		parameter for lorenz equations		
 eta=1.6666667		parameter for lorenz equations		
 y0=1.0		initial value of y[0]			
 y1=-1.0		initial value of y[1]			
 y2=1.0		initial value of y[2]			
 h=.01			increment in time			
 tol=1.e-08		error tolerance				
 stepmax=500		maximum number of steps to compute	
 mode=xy		xy-pairs, =yz yz-pairs, =xz xz-pairs,	
			=xyz xyz-triplet, =x only, =y only, =z only
 Notes:							
 This program is really just a demo showing how to use the 	
 differential equation solver rke_solve written by Francois 	
 Pinard, based on a modified form of the 4th order Runge-Kutta 
 method, which employs the error checking method of R. England 
 1969.								

 The output consists of unformated C-style binary floats, of	
 either pairs or triplets as specified by the "mode" paramerter.

 Examples:							
 lorenz stepmax=1000 mode=xy | xgraph n=1000	&		
 lorenz stepmax=1000 mode=yz | xgraph n=1000	&		
 lorenz stepmax=1000 mode=xz | xgraph n=1000	&		

 lorenz stepmax=1000 mode=x | suaddhead ns=1000 | suxwigb &	
 lorenz stepmax=1000 mode=y | suaddhead ns=1000 | suxwigb &	
 lorenz stepmax=1000 mode=z | suaddhead ns=1000 | suxwigb &	

 GNUPLOT 3D plot example:						
 lorenz stepmax=2000 mode=xyz > lorenz.bin			
 ...when you run gnuplot type the following command ...	
 splot "lorenz.bin" binary record=2000:2000:2000 with points pointsize .1 



 The lorenz equations describe a simplified model of a convection
 cell, and are given by the autonomous system of ODE's	

	x'(t) = sigma * ( y - x )			
	y'(t) = x * ( rho - z ) - y		
	z'(t) = x * y - eta * z		

 Author: CWP: Aug 2004: John Stockwell


\end{verbatim}
\pagebreak
\begin{verbatim}
 MAKEVEL - MAKE a VELocity function v(x,y,z)				

 makevel > outfile nx= nz= [optional parameters]			

 Required Parameters:							
 nx=                    number of x samples (3rd dimension)		
 nz=                    number of z samples (1st dimension)		

 Optional Parameters:							
 ny=1                   number of y samples (2nd dimension)		
 dx=1.0                 x sampling interval				
 fx=0.0                 first x sample					
 dy=1.0                 y sampling interval				
 fy=0.0                 first y sample					
 dz=1.0                 z sampling interval				
 fz=0.0                 first z sample					
 v000=2.0               velocity at (x=0,y=0,z=0)			
 dvdx=0.0               velocity gradient with respect to x		
 dvdy=0.0               velocity gradient with respect to y		
 dvdz=0.0               velocity gradient with respect to z		
 vlens=0.0              velocity perturbation in parabolic lens	
 tlens=0.0              thickness of parabolic lens			
 dlens=0.0              diameter of parabolic lens			
 xlens=                 x coordinate of center of parabolic lens	
 ylens=                 y coordinate of center of parabolic lens	
 zlens=                 z coordinate of center of parabolic lens	
 lambda=1.0             make lambda larger to sharpen edge of lens 	
 vran=0.0		standard deviation of random perturbation	
 vzfile=                file containing v(z) profile			
 vzran=0.0              standard deviation of random perturbation to v(z)
 vzc=0.0                v(z) chirp amplitude				
 z1c=fz                 z at which to begin chirp			
 z2c=fz+(nz-1)*dz       z at which to end chirp			
 l1c=dz                 wavelength at beginning of chirp		
 l2c=dz                 wavelength at end of chirp			
 exc=1.0                exponent of chirp				
\end{verbatim}
\pagebreak
\begin{verbatim}
 MKPARFILE - convert ascii to par file format 				

 mkparfile <stdin >stdout 						

 Optional parameters:							
 	string1="par1"	first par string			
 	string2="par2"	second par string			

 This is a tool to convert values written line by line to parameter 	
 vectors in the form expected by getpar.  For example, if the input	
 file looks like:							
 	t0 v0								
 	t1 v1								
	...								
 then									
	mkparfile <input >output string1=tnmo string2=vnmo		
 yields:								
	tnmo=t0,t1,...							
	vnmo=v0,v1,...							

\end{verbatim}
\pagebreak
\begin{verbatim}
 MRAFXZWT - Multi-Resolution Analysis of a function F(X,Z) by Wavelet	
	 Transform. Modified to perform different levels of resolution  
        analysis for each dimension and also to allow to transform     
        back only the lower level of resolution.  		      	

    mrafxzwt [parameters] < infile > mrafile 			 	

 Required Parameters:							
 n1=		size of first (fast) dimension				
 n2=		size of second (slow) dimension 			

 Optional Parameters:							
 p1=		maximum integer such that 2^p1 <= n1			
 p2=		maximum integer such that 2^p2 <= n2			
 order=6	order of Daubechies wavelet used (even, 4<=order<=20)	
 mralevel1=3   maximum multi-resolution analysis level in dimension 1	
 mralevel2=3   maximum multi-resolution analysis level in dimension 2	
 trunc=0.0	truncation level (percentage) of the reconstruction	
 verbose=0	=1 to print some useful information			
 reconfile=    reconstructed data file to write			
 reconmrafile= reconstructed data file in MRA domain to write		
 dfile=	difference between infile and reconfile to write        
 dmrafile=	difference between mrafile and reconmrafile to write    
 dconly=0      =1 keep only dc	component of MRA			
 verbose=0     =1 to print some useful information                     
 if (n1 or n2 is not integer powers of 2) specify the following:	
 	nc1=n1/2 center of trimmed image in the 1st dimension           
 	nc2=n2/2 center of trimmed image in the 2nd dimension           
	trimfile= if given, output the trimmed file			

 Notes:								
 This program performs multi-resolution analysis of an input function	
 f(x,z) via the wavelet transform method. Daubechies's least asymmetric
 wavelets are used. The smallest wavelet coefficient retained is given	
 by trunc times the absolute maximum size coefficient in the MRA.	
 
 The input dimensions of the data must be expressed by (p1,p2) which   

  Author: Zhaobo Meng, 11/25/95, Colorado School of Mines             *
  Modified:  Carlos E. Theodoro, 06/25/97, Colorado School of Mines   *
	Included options for:                           	        *
	- different level of resolutionf or each dimension;   	        *
	- transform back the lower level of resolution, only.		*
									*
 Reference:								*
 Daubechies, I., 1988, Orthonormal Bases of Compactly Supported	* 
 Wavelets, Communications on Pure and Applied Mathematics, Vol. XLI,  *
 909-996.				 				* 
\end{verbatim}
\pagebreak
\begin{verbatim}
 PDFHISTOGRAM - generate a HISTOGRAM of the Probability Density function

  pdfhistogram < stdin > sdtout [Required params] (Optional params)	

 Required parameters: 							
 ix=		column containing X variable				
 iy=		column containing Y variable 				
 min_x=	minimum X bin						
 max_x=	maximum X bin						
 min_y=	minimum Y bin						
 max_y=	maximum Y bin						
 logx=0	=1 use logarithmic scale for X axis			
 logy=0	=1 use logarithmic scale for Y axis			
 norm=	selected normalization type 					
		sqrt	- bin / sqrt( xnct*ycnt) 			
		avg_cnt	- 0.5* bin / (xcnt + ycnt) 			
		avg_sum	- (bin / xcnt + bin / ycnt ) / 2 		
		xcnt	- bin / xcnt 					
		ycnt	- bin / ycnt 					
		log	- log(bin) 					
		total	- bin / total 					
 Optional parameters: 							
  nx=100	- number of X bins					
  ny=100	- number of Y bins 					
  ir=	- column containing reject variable 				
  rmin=	- reject values below rmin 				
  rmax=	- reject values above rmax 				
		NOTE: only one, rmin or rmax, may be used 		
 Notes:								
 PDFHISTOGRAM creates a 2D histogram representing the probability density
 function of the input data. The output is in the form of a binary array
 that can then be plotted via ximage.					
 Commandline options allow selecting any of several normalizations	
 to apply to the distributions.					


 Credits:
  Reginald H. Beardsley	rhb@acm.org
	Copyright 2006 Exploration Software Consultants Inc.

\end{verbatim}
\pagebreak
\begin{verbatim}
 PRPLOT - PRinter PLOT of 1-D arrays f(x1) from a 2-D function f(x1,x2)

 prplot <infile >outfile [optional parameters]				

 Optional Parameters:							
 n1=all                 number of samples in 1st dimension		
 d1=1.0                 sampling interval in 1st dimension		
 f1=d1                  first sample in 1st dimension			
 n2=all                 number of samples in 2nd dimension		
 d2=1.0                 sampling interval in 2nd dimension		
 f2=d2                  first sample in 2nd dimension			
 label2=Trace           label for 2nd dimension			

\end{verbatim}
\pagebreak
\begin{verbatim}
 RANDVEL3D - Add a random velocity layer (RVL) to a gridded             
            v(x,y,z) velocity model                                    

	randvel3d <infile n1= n2= >outfile [parameters]			

 Required Parameters:							
 n1=		number of samples along 1st dimension			
 n2=		number of samples along 2nd dimension			

 Optional Parameters:							

 n3=1          number of samples along 3rd dimension			

 mode=1             add single layer populated with random vels	
                    =2 add nrvl layers of random thickness and vel     
 seed=from_clock    random number seed (integer)			

 ---->New layer geometry info						
 i1beg=1       1st dimension beginning sample 				
 i1end=n1/5    1st dimension ending sample 				
 i2beg=1       2nd dimension beginning sample 				
 i2end=n2      2nd dimension ending sample 				
 i3beg=1       3rd dimension beginning sample 				
 i3end=n3      3rd dimension ending sample 				
 ---->New layer velocity info						
 vlsd=v/3     range (std dev) of random velocity in layer, 		
               where v=v(0,0,i1) and i1=(i1beg+i1end)/2 	 	
 add=1         add random vel to original vel (v_orig) at that point 	
               =0 replace vel at that point with (v_orig+v_rand) 	
 how=0         random vels can be higher or lower than v_orig		
               =1 random vels are always lower than v_orig		
               =2 random vels are always higher than v_orig		
 cvel=2000     layer filled with constant velocity cvel 		
               (overides vlsd,add,how params)			
 ---->Smoothing parameters (0 = no smoothing)				
 r1=0.0	1st dimension operator length in samples		
 r2=0.0	2nd dimension operator length in samples		
 r3=0.0	3rd dimension operator length in samples		
 slowness=0	=1 smoothing on slowness; =0 smoothing on velocity	

 nrvl=n1/10    number of const velocity layers to add     		
 pdv=10.       percentage velocity deviation (max) from input model	

 Notes:								
 1. Smoothing radii usually fall in the range of [0,20].		
 2. Smoothing radii can be used to set aspect ratio of random velocity 
    anomalies in the new layer.  For example (r1=5,r2=0,r3=0) will     
    result in vertical vel streaks that mimick vertical fracturing.    
 3. Smoothing on slowness works better to preserve traveltimes relative
    to the unsmoothed case.						
 4. Default case is a random velocity (+/-30%) near surface layer whose
    thickness is 20% of the total 2D model thickness.			
 5. Each layer vel is a random perturbation on input model at that level.
 6. The depth dimension is assumed to be along axis 1.			

 Example:								
 1. 2D RVL with no smoothing						
   makevel nz=250 nx=200 | randvel3d n1=250 n2=200 | ximage n1=250      
 2. 3D RVL with no smoothing						
   makevel nz=250 nx=200 ny=220 |					
   randvel3d n1=250 n2=200 n3=220 | 					
   xmovie n1=250 n2=200					    	



 Author:  U Houston: Chris Liner c. 2008
          Based on smooth3d (CWP: Zhenyue Liu  March 1995)


\end{verbatim}
\pagebreak
\begin{verbatim}
 RAYT2DAN -- P- and SV-wave raytracing in 2D anisotropic media		

 rayt2dan > ttime parameterfiles= nt= nx= nz= [optional parameters]	

 Required Parameters:							
 VP0file=		 name of file containing VP0(x,z)		
 nt=                    number of time samples	for each ray		
 nx=                    number of samples (x) for the parameter fields	
 nz=                    number of samples (z) for the parameter fields	

 Optional Parameters:							
 SV=0			 for P-waves, SV=1 for Shear waves		

 Parameters defining the velocity field				
 dt=0.008                 time sampling interval			",	
 fx=0                     first lateral sample (x) in parameter field  
 fz=0                     first lateral sample (z) in parameter field  
 dx=100.0                 sample spacing (x) for the parameter fields	
 dz=100.0                 sample spacing (z) for the parameter fields	
	
 Parameters defining the takeoff angle of a ray at a source position	",	
 fa=-60                 first take-off angle of rays (degrees)         
 na=61                  number of rays					",      
 da=2                   increment of take-off angle			
 amin=0                 minimum emergence angle; must be > -90 degrees	
 amax=90                maximum emergence angle; must be < 90 degrees	

 Parameters defining the output traveltime table			
 fxo=fx                 first lateral sample in traveltime table	
 nxo=nx                 number of later samples in traveltime table	
 dxo=dx                 lateral interval in traveltime table		
 fzo=fz                 first depth sample in traveltime table		
 nzo=nz                 number of depth samples in traveltime table	
 dzo=dz                 depth interval in traveltime table		
 fac=0.01               factor to determine the radius of extrap. 	

 Parameters defining the source positions			        
 fsx=fx                 x-coordinate of first source                   
 nsx=1                  number of sources                              
 dsx=2*dxo              x-coordinate increment of sources              
 aperx=0.5*nx*dx        ray tracing aperature in x-direction           

 Files for general anisotropic parameters confined to a vertical plane:
 a1111file		name of file containing a1111(x,z)		
 a1133file          	name of file containing a1133(x,z)		
 VS0file	       	name of file containing VS0(x,z)		
 a1113file          	name of file containing a1113(x,z)		
 a3313file          	name of file containing a3313(x,z)		

 For transversely isotropic media Thomsen's parameters could be used:	
 deltafile		name of file containing delta(x,z)		
 epsilonfile		name of file containing epsilon(x,z)		

 if anisotropy parameters are not given the program considers		", 
 isotropic media.							", 


 Credits:
      Debashish Sarkar                  
		 
		
   Technical Reference:

	Cerveny, V., 1972, Seismic rays and ray intensities 
	in inhomogeneous anisotropic media: 
	Geophys. J. R. Astr. Soc., 29, 1--13.

 	Hangya, A., 1986, Gaussian beams in anisotropic elastic media:
      Geophys. J. R. Astr. Soc., 85, 473--563.
	
	Gajewski, D. and Psencik, I., 1987, Computation of high frequency 
	seismic wavefields in 3-D  laterally inhomogeneous anisotropic 
 	media: Geophys. J. R. Astr. Soc., 91, 383-411.



\end{verbatim}
\pagebreak
\begin{verbatim}
 RAYT2D - traveltime Tables calculated by 2D paraxial RAY tracing	

     rayt2d vfile= tfile= [optional parameters]			

 Required parameters:							
 vfile=stdin		file containning velocity v[nx][nz]		
 tfile=stdout		file containning traveltime tables		
			t[nxs][nxo][nzo]				

 Optional parameters							
 dt=0.008		time sample interval in ray tracing		
 nt=401		number of time samples in ray tracing		

 fz=0			first depth sample in velocity			
 nz=101		number of depth samples in velocity		
 dz=100		depth interval in velocity			
 fx=0			first lateral sample in velocity		
 nx=101		number of lateral samples in velocity		
 dx=100		lateral interval in velocity			

 fzo=fz		first depth sample in traveltime table		
 nzo=nz		number of depth samples in traveltime table	
 dzo=dz		depth interval in traveltime table		
 fxo=fx		first lateral sample in traveltime table	
 nxo=nx		number of lateral samples in traveltime table	
 dxo=dx		lateral interval in traveltime table		

 surf="0,0;99999,0"  Recording surface "x1,z1;x2,z2;x3,z3;...
 fxs=fx		x-coordinate of first source			
 nxs=1			number of sources				
 dxs=2*dxo		x-coordinate increment of sources		
 aperx=0.5*nx*dx  	ray tracing aperature in x-direction		

 fa=-60		first take-off angle of rays (degrees)		
 na=61			number of rays  				
 da=2			increment of take-off angle  			
 amin=0		minimum emergence angle 			
 amax=90		maximum emergence angle 			

 fac=0.01		factor to determine radius for extrapolation	
 ek=1			flag of implementing eikonal in shadow zones 	
 ms=10			print verbal information at every ms sources	
 restart=n		job is restarted (=y yes; =n no)		
 npv=0			flag of computing quantities for velocity analysis
 if npv>0 specify the following three files				
 pvfile=pvfile		input file of velocity variation pv[nxo][nzo]	
 tvfile=tvfile		output file of traveltime variation tables  	
			tv[nxs][nxo][nzo]				
 csfile=csfile		output file of cosine tables cs[nxs][nxo][nzo]	

 Notes:								
 1. Each traveltime table is calculated by paraxial ray tracing; then 	
    traveltimes in shadow zones are compensated by solving eikonal	
    equation.								
 2. Input velocity is uniformly sampled and smooth one preferred.	
 3. Traveltime table and source ranges must be within velocity model.	
 4. Ray tracing aperature can be chosen as sum of migration aperature	
    plus half of maximum offset.					
 5. Memory requirement for this program is about			
      [nx*nz+4*mx*nz+3*nxo*nzo+na*(nx*nz+mx*nz+3*nxo*nzo)]*4 bytes	
    where mx = min(nx,2*(1+aperx/dx)).					

 Note: spatial units of v(z,x) must be the same as those of dx. 	
 v(z,x) is represented numerically in C-style binary floats v[xn][zn],	
 where the depth direction is the fast direction in the data. Such     
 models can be created with unif2 or makevel.    			",  


 Author:  Zhenyue Liu, 10/11/94,  Colorado School of Mines 

          Trino Salinas, 01/01/96 included the option to handle nonflat
          reference surfaces.
          Subroutines from Dave Hale's modeling library were adapted in
          this code to define topography using cubic splines.

 References:

 Beydoun, W. B., and Keho, T. H., 1987, The paraxial ray method:
   Geophysics, vol. 52, 1639-1653.

 Cerveny, V., 1985, The application of ray tracing to the numerical
   modeling of seismic wavefields in complex structures, in Dohr, G.,
   ED., Seismic shear waves (part A: Theory): Geophysical Press,
   Number 15 in Handbook of Geophysical Exploration, 1-124.
 

\end{verbatim}
\pagebreak
\begin{verbatim}
 RECAST - RECAST data type (convert from one data type to another)	

 recast <stdin [optional parameters]  >stdout 				

 Required parameters:							
 	none								

 Optional parameters:							
 in=float	input type	(float)					
 		=double		(double)				
 		=int		(int)					
 		=char		(char)					
		=uchar		(unsigned char)				
 		=short		(short)					
 		=long		(long)					
 		=ulong		(unsigned long)				

 out=double	output type	(double)				
 		=float		(float)					
 		=int		(int)					
 		=char		(char)					
 		=uchar		(unsigned char)				
 		=short		(short)					
 		=long		(long)					
 		=ulong		(unsigned long)				

 outpar=/dev/tty	output parameter file, contains the		
				number of values (n1=)			
 			other choices for outpar are: /dev/tty,		
 			/dev/stderr, or a name of a disk file		

 Notes: Converting bigger types to smaller is hazardous. For float	
	 or double conversions to integer types, the results are 	
	 rounded to the nearest integer.				


 Credits:

	CWP: John Stockwell, Jack K. Cohen


\end{verbatim}
\pagebreak
\begin{verbatim}
 REFREALAZIHTI -  REAL AZImuthal REFL/transm coeff for HTI media 	

 refRealAziHTI  [optional parameters]	>coeff.data 			

 Optional parameters: 							
 vp1=2         p-wave velocity medium 1 (with respect to symm.axes)	
 vs1=1         s-wave velocity medium 1 (with respect to symm.axes)	
 eps1=0        epsilon medium1						
 delta1=0	delta medium 1						
 gamma1=0	gamma medium 1						
 rho1=2.7	density medium 1 					
 vp2=2         p-wave velocity medium 2 (with respect to symm.axes)	
 vs2=1         s-wave velocity medium 2 (with respect to symm.axes)	
 eps2=0        epsilon medium 2					
 delta2=0	delta medium 2						
 gamma2=0	gamma medium 2						
 rho2=2.7	density medium 2 					
 modei=0 	incident mode is qP					
		=1 incident mode is qSV					
		=2 incident mode is SP					
 modet=0 	scattered mode						
 rort=1 	reflection(1)	or transmission (0)			
 azimuth=0	azimuth with respect to x1-axis (clockwise)		
 fangle=0	first incidence angle					
 langle=45	last incidence angle					
 dangle=1	angle increment						
 iscale=0      default: angle in degrees				
		=1 angle-axis in rad                                    
               =2 axis  horizontal slowness                            
               =3 sin^2 of incidence angle                             
 ibin=1 	binary output 						
		=0 Ascci output						
 outparfile	=outpar parameter file for plotting			
 coeffile	=coeff.data coefficient-output file			
 test=1 	activate testing routines in code			
 info=0 	output intermediate results				

 Notes:								
 Axes of symmetry have to coincide in both media.  This code computes	
 all 6 REAL reflection/transmissions coefficients on the fly. However,	
 the set-up is such Real reflection/transmission coefficients in 	
 HTI-media with coinciding symmetry axes.				
 However, the set-up is such that currently only one coefficient is	
 dumped into the output. This is easily changed.  The solution of the	
 scattering problem is obtained numerically and involves the Gaussian	
 elimination of a 6X6 matrix.						", 



 AUTHOR:: Andreas Rueger, Colorado School of Mines, 02/10/95
                original name of code <graebnerTIH.c>
           modified, extended version of this code <refTIH3D>
 
  Technical references:

 	Sebastian Geoltrain: Asymptotic solutions to direct
		and inverse scattering in anisotropic elastic media;
		CWP 082.
	Graebner, M.; Geophysics, Vol 57, No 11:
		Plane-wave reflection and transmission coefficients
		for a transversely isotropic solid.
	Cerveny, V., 1972, Seismic rays and ray intensities in inhomogeneous 	
		anisotropic media: Geophys. J. R. astr. Soc., 29, 1-13.

	.. and some derivations by Andreas Rueger.

 If propagation is perpendicular or 
 parallel to the symmetry axis, the solution is analytic (see 		
 graebner2D.c and rtRealIso.c		

\end{verbatim}
\pagebreak
\begin{verbatim}
 REFREALVTI -  REAL REFL/transm coeff for VTI media and symmetry-axis	
                 planes of HTI media 					

 refRealVTI  [Optional parameters]	   			        

 Optional parameters:							
 vp1=2          p-wave velocity medium 1 (along symm.axes)	        
 vs1=1          s-wave velocity medium 1 (along symm.axes)	        
 eps1=0         Thomsen's epsilon medium 1				
 delta1=0	 Thomsen's delta medium 1	         		
 rho1=2.7	 density medium 1 					
 axis1=0	 medium 1 is VTI					
		 =1 medium 1 is HTI					
 vp2=2.5         p-wave velocity medium 2 (along symm.axes)	        
 vs2=1.2          s-wave velocity medium 2 (along symm.axes)	        
 eps2=0	 epsilon medium2					
 delta2=0	delta medium 2						
 rho2=3.0	density medium 2 					
 axis2=0	medium 2 is VTI						
		=2 medium 2 is HTI					
 modei=0 	incident mode is qP					
		=1 incident mode is qSV					
 modet=0 	scattered mode						
 rort=1 	reflection(1)	or transmission (0)			
 fangle=0	first angle						
 langle=45	last angle						
 dangle=1	angle increment						
 iscale=0       =1 angle-axis in rad					
                =2 axis  horizontal slowness                           
                =3 sin^2 of incidence angle                            
 ibin=1 	binary output 						
		=0 Ascci output						
 outparfile	=outpar parameter file for plotting			
 coeffile	=coeff.data coefficient-output file			

 Notes:								
 Coefficients are based on Graebner's 1992 Geophysics paper. Note the	
 mistype in the equation for K1. The algorithm can be used for VTI	
 and HTI media on the incidence and scattering side.			



 AUTHOR:: Andreas Rueger, Colorado School of Mines, 01/20/95
       original name of algorithm: graebner1.c

  Technical reference: Graebner, M.; Geophysics, Vol 57, No 11:
		Plane-wave reflection and transmission coefficients
		for a transversely isotropic solid.
		Rueger, A.; Geophysics 1996 (accepted):
		P-wave reflection coefficients ...


\end{verbatim}
\pagebreak
\begin{verbatim}
 REGRID3 - REwrite a [ni3][ni2][ni1] GRID to a [no3][no2][no1] 3-D grid

 regrid3 < oldgrid > newgrid [parameters]				

 Optional parameters:							
 ni1=1   fastest (3rd) dimension in input grid                         
 ni2=1   second fastest (2nd) dimension in input grid			
 ni3=1   slowest (1st) dimension in input grid                       	
 no1=1   fastest (3rd) dimension in output grid                        
 no2=1   second fastest (2nd) dimension in output grid                 ",                
 no3=1   slowest (1st) dimension in output grid                        
 Optional Parameters:                                                  
 verbose=0	=1 print some useful information			

 Notes:								
 REGRID3 can be used to span a 1-D grid to a 2-D grid,  or a 2-D grid  
 to a 3-D grid; or to change grid parameters within the dimensions.	
 Together with MUL and UNIF3, most 3-D velocity model can be 		


 Credits:
  	CWP: Zhaobo Meng, 1996, Colorado School of Mines

\end{verbatim}
\pagebreak
\begin{verbatim}
 RESAMP - RESAMPle the 1st dimension of a 2-dimensional function f(x1,x2)

 resamp <infile >outfile [optional parameters]				

 Required Parameters:							

 Optional Parameters:							
 n1=all                 number of samples in 1st (fast) dimension	
 n2=all                 number of samples in 2nd (slow) dimension	
 d1=1.0                 sampling interval in 1st dimension		
 f1=d1                  first sample in 1st dimension			
 n1r=n1                 number of samples in 1st dimension after resampling
 d1r=d1                 sampling interval in 1st dimension after resampling
 f1r=f1                 first sample in 1st dimension after resampling	

 NOTE:  resamp currently performs NO ANTI-ALIAS FILTERING before resampling!
 Caveat: this program resamples data that are oscillatory in the fast	
    dimension only, such as seismic data with no SU headers. To resample
    other 2d data, such as velocity profiles, use "unisam" or "unisam2

\end{verbatim}
\pagebreak
\begin{verbatim}

 ROSSLER - compute the ROSSLER attractor				

  rossler > [stdout]						

 Required Parameters: none					
 Optional Parameters:						
 a=.2			parameter for rossler equations		
 b=.2			parameter for rossler equations		
 c=5.7			parameter for rossler equations		
 y0=1.0		initial value of y[0]			
 y1=-1.0		initial value of y[1]			
 y2=1.0		initial value of y[2]			
 h=.01			increment in time			
 tol=1.e-08		error tolerance				
 stepmax=500		maximum number of steps to compute	
 mode=xy		xy-pairs, =yz yz-pairs, =xz xz-pairs,	
			=xyz xyz-triplet, =x only, =y only, =z only
 Notes:							
 This program is really just a demo showing how to use the 	
 differential equation solver rke_solve written by Francois 	
 Pinard, based on a modified form of the 4th order Runge-Kutta 
 method, which employs the error checking method of R. England 
 1969.								

 The output consists of unformated C-style binary floats, of	
 either pairs or triplets as specified by the "mode" paramerter.

 Examples:							
 rossler stepmax=1000 mode=xy | xgraph n=1000	&		
 rossler stepmax=1000 mode=yz | xgraph n=1000	&		
 rossler stepmax=1000 mode=xz | xgraph n=1000	&		

 rossler stepmax=1000 mode=x | suaddhead ns=1000 | suxwigb &	
 rossler stepmax=1000 mode=y | suaddhead ns=1000 | suxwigb &	
 rossler stepmax=1000 mode=z | suaddhead ns=1000 | suxwigb &	

 GNUPLOT 3D plot example:                                      
 rossler stepmax=2000 mode=xyz > rossler.bin                   
 ...when you run gnuplot type the following command ...        
 splot "rossler.bin" binary record=2000:2000:2000 with points pointsize .1 



 The rossler equations describe a simple example of a chaotic system
 and are given by the autonomous system of ODE's	

	x'(t) = - y - z
	y'(t) = x + a y
	z'(t) = b + z(x - c)

 Author: CWP: Aug 2013: John Stockwell


\end{verbatim}
\pagebreak
\begin{verbatim}
 SMOOTH2 --- SMOOTH a uniformly sampled 2d array of data, within a user-
		defined window, via a damped least squares technique	

 smooth2 < stdin n1= n2= [optional parameters ] > stdout		

 Required Parameters:							
 n1=			number of samples in the 1st (fast) dimension	
 n2=			number of samples in the 2nd (slow) dimension	

 Optional Parameters:							
 r1=0			smoothing parameter in the 1 direction		
 r2=0			smoothing parameter in the 2 direction		
 win=0,n1,0,n2		array for window range				
 rw=0			smoothing parameter for window function		
 efile=                 =efilename if set write relative error(x1) to	
				efilename				

 Notes:								
 Larger r1 and r2 result in a smoother data. Recommended ranges of r1 	", 
 and r2 are from 1 to 20.						

 The file verror gives the relative error between the original velocity 
 and the smoothed one, as a function of depth. If the error is		
 between 0.01 and 0.1, the smoothing parameters are suitable. Otherwise,
 consider increasing or decreasing the smoothing parameter values.	

 Smoothing can be implemented in a selected window. The range of 1st   
 dimension for window is from win[0] to win[1]; the range of 2nd   	
 dimension is from win[2] to win[3]. 					

 Smoothing the window function (i.e. blurring the edges of the window)	
 may be done by setting a nonzero value for rw, otherwise the edges	
 of the window will be sharp.						



	Credits: 
		CWP: Zhen-yue Liu,
			adapted for par/main by John Stockwell 1 Oct 92
		Windowing feature added by Zliu on 16 Nov 1992

\end{verbatim}
\pagebreak
\begin{verbatim}
 SMOOTH3D - 3D grid velocity SMOOTHing by the damped least squares	

	smooth3d <infile >outfile [parameters]				

 Required Parameters:							
 n1=		number of samples along 1st dimension			
 n2=		number of samples along 2nd dimension			
 n3=		number of samples along 3rd dimension			

 Optional Parameters:							

 Smoothing parameters (0 = no smoothing)				
 r1=0.0	operator length in 1st dimension			
 r2=0.0	operator length in 2nd dimension			
 r3=0.0	operator length in 3rd dimension			

 Sample intervals:							
 d1=1.0	1st dimension						
 d2=1.0	2nd dimension						
 d3=1.0	3rd dimension						
 iter=2	number of iteration used				
 time=0	which dimension the time axis is (0 = no time axis)	
 depth=1	which dimension the depth axis is (ignored when time>0)	
 mu=1		the relative weight at maximum depth (or time)		
 verbose=0	=1 for printing minimum wavelengths			
 slowness=0	=1 smoothing on slowness; =0 smoothing on velocity	
 vminc=0	velocity values below it are cliped before smoothing	
 vmaxc=99999	velocity values above it are cliped before smoothing	

 Notes:								
 1. The larger the smoothing parameters, the smoother the output velocity.
    These parameters are lengths of smoothing operators in each dimension.
 2. iter controls the orders of derivatives to be smoothed in the output
    velocity. e.g., iter=2 means derivatives until 2nd order smoothed.	
 3. mu is the multipler of smoothing parameters at the bottom compared to
    those at the surface.						
 4. Minimum wavelengths of each dimension and the total may be printed	
   for the resulting output velocity is. To compute these parameters for
   the input velocity, use r1=r2=r3=0.					
 5. Smoothing directly on slowness works better to preserve traveltime.
   So the program optionally converts the input velocity into slowness	", 
   and smooths the slowness, then converts back into velocity.		



 Author:  CWP: Zhenyue Liu  March 1995

 Reference:
 Liu, Z., 1994,"A velocity smoothing technique based on damped least squares
		in Project Reveiew, May 10, 1994, Consortium Project on
		Seismic Inverse Methods for Complex Stuctures.

\end{verbatim}
\pagebreak
\begin{verbatim}
 SMOOTHINT2 --- SMOOTH non-uniformly sampled INTerfaces, via the damped
  		least-squares technique					

  smoothint2 <input ninf= >output [optional parameters]		

 Required Parameters:							
 <input                 file containing original interfaces		
 >output                file containing smoothed interfaces	 	

 Optional Parameters:							
 ninf=5                number of interfaces  				
 r=100			smoothing parameter 				
 npmax=101		maximum number of points in interfaces		

 Notes:								
 The input file is an ASCII file. Each interface is represented by pairs
 (non-uniform sampling) of x and z values, with one pair of values on	
 each line, separated by spaces or tabs. Each interface is separated with
 an entry with a large negative z value for example: 1.0     -9999.	
 There is no entry for the surface. The surface is assumed to be flat  
 with z=0.								
 This is similar to a CSHOT model file without a surface entry and	
 without comments.							

 The smoothing method is analogous to a moving window averaging process
 (but not the same) with the parameter "r" being analogous to the "width
 of the window. Thus, the size of "r" must be chosen to by compatible
 with the scale (wavelengths) of the variations of the interfaces in the
 model being smoothed.							

 Example using the test data set generated by unif2: 			
 unif2 tfile=tfilename							
 Compare the unsmoothed interface model:				
 unif2 < tfilename method=interpolation_method |			
	 			psimage n1=100 n2=100 d1=10 d2=10 | ...	
 To the smoothed interface model:					
 smoothint2 r=100 < tfilename | unif2 method=interpolation_method |	", 
	psimage n1=100 n2=100 d1=10 d2=10 | ...				



 Credits:
  CWP: Zhenyue Liu, Jan 1994
 Reference:
    Liu, Zhenyue, 1994, Velocity smoothing: theory and implementation, 
    Project Review, 1994, Consortium Project on Seismic Inverse Methods
    for Complex Stuctures (in review)


\end{verbatim}
\pagebreak
\begin{verbatim}
 STIFF2VEL - Transforms 2D elastic stiffnesses to (vp,vs,epsilon,delta) 

 stiff2vel  nx=  nz=  [optional parameters]				

 Required parameters:							
 nx=		 	 number of x samples (2nd dimension)		
 nz=		 	 number of z samples (1st dimension)		
 rho_file='rho.bin'     input file containing rho(x,z)			
 c11_file='c11.bin'     input file containing c11(x,z)			
 c13_file='c13.bin'     input file containing c13(x,z)			
 c33_file='c33.bin'     input file containing c33(x,z)			
 c44_file='c44.bin'     input file containing c44(x,z)			

 Optional Parameters:							
 vp_file='vp.bin'	 output file containing P-wave velocities	
 vs_file='vs.bin'	 output file containing S-wave velocities	
 rho_file='rho.bin'	 output file containing densities		
 eps_file='eps.bin'	 output file containing Thomsen epsilon	       	
 delta_file='delta.bin' output file containing Thomsen delta	       	

 Notes: 								
 1. All quantities in MKS units					
 2. Isotropy implied by c11(x,z)=c33(x,z)=0 			
 3. Vertical symmetry axis is assumed.					



  Coded:
  Aramco: Chris Liner 9/25/2005 
          (based on vel2stiff.c)

\end{verbatim}
\pagebreak
\begin{verbatim}
 SUBSET - select a SUBSET of the samples from a 3-dimensional file	

 subset <infile >outfile [optional parameters]				

 Optional Parameters:							
 n1=nfloats             number of samples in 1st dimension		
 n2=nfloats/n1          number of samples in 2nd dimension		
 n3=nfloats/(n1*n2)     number of samples in 3rd dimension		
 id1s=1                 increment in samples selected in 1st dimension	
 if1s=0                 index of first sample selected in 1st dimension
 n1s=1+(n1-if1s-1)/id1s number of samples selected in 1st dimension	
 ix1s=if1s,if1s+id1s,...indices of samples selected in 1st dimension	
 id2s=1                 increment in samples selected in 2nd dimension	
 if2s=0                 index of first sample selected in 2nd dimension
 n2s=1+(n2-if2s-1)/id2s number of samples selected in 2nd dimension	
 ix2s=if2s,if2s+id2s,...indices of samples selected in 2nd dimension	
 id3s=1                 increment in samples selected in 3rd dimension	
 if3s=0                 index of first sample selected in 3rd dimension
 n3s=1+(n3-if3s-1)/id3s number of samples selected in 3rd dimension	
 ix3s=if3s,if3s+id3s,...indices of samples selected in 3rd dimension	

 For the 1st dimension, output is selected from input as follows:	
       output[i1s] = input[ix1s[i1s]], for i1s = 0 to n1s-1		
 Likewise for the 2nd and 3rd dimensions.				



 AUTHOR:  Dave Hale, Colorado School of Mines, 07/07/89

\end{verbatim}
\pagebreak
\begin{verbatim}
 SWAPBYTES - SWAP the BYTES of various  data types			

 swapbytes <stdin [optional parameters]  >stdout 			

 Required parameters:							
 	none								

 Optional parameters:							
 in=float	input type	(float)					
 		=double		(double)				
 		=short		(short)					
 		=ushort		(unsigned short)			
 		=long		(long)					
 		=ulong		(unsigned long)				
 		=int		(int)					

 outpar=/dev/tty		output parameter file, contains the	
				number of values (n1=)			
 			other choices for outpar are: /dev/tty,		
 			/dev/stderr, or a name of a disk file		

 Notes:								
 The byte order of the mantissa of binary data values on PC's and DEC's
 is the reverse of so called "big endian" machines (IBM RS6000, SUN,etc.)
 hence the need for byte-swapping capability. The subroutines in this code
 have been tested for swapping between PCs and	big endian machines, but
 have not been tested for DEC products.				

 Caveat:								
 2 byte short, 4 byte long, 4 byte float, 4 byte int,			
 and 8 bit double assumed.						


 Credits:
	CWP: John Stockwell (Jan 1994)
 Institut fur Geophysik, Hamburg: Jens Hartmann supplied byte swapping
					subroutines



\end{verbatim}
\pagebreak
\begin{verbatim}
 THOM2HTI - Convert Thompson parameters V_p0, V_s0, eps, gamma,	
              to the HTI parameters alpha, beta, epsilon(V), delta(V), 
	       gamma							

 thom2hti  [optional parameter]                           		


 vp=2         symm.axis p-wave velocity                        	
 vs=1         symm.axis s-wave velocity                        	
 eps=0        Thomsen's (generic) epsilon 	         		
 gamma=0      Thomsen's generic gamma                                  
 weak=1       compute weak approximation                               
 outpar=/dev/tty	output parameter file				

 Outputs:								
   alpha, beta, e(V), d(V), gamma					

 Notes:								
 Output is dumped to the screen and, if selected to outpar		

 Code can be used to find models that satisfy the constraints		
 that are imposed on HTI models caused by vertically fractured		
 layers. For definition and use of the HTI parameter set see CWP-235.	

 
  Credits: Andreas Rueger, CWP
  For definition and use of the HTI parameter set see CWP-235.


\end{verbatim}
\pagebreak
\begin{verbatim}
 THOM2STIFF - convert Thomsen's parameters into (density normalized)	
                 stiffness components for transversely isotropic 	
	 	   material with in-plane-tilted axis of symmetry	

 thom2stiff [optional parameter] 		                        

 vp=2         symm.axis p-wave velocity                        	
 vs=1         symm.axis s-wave velocity                        	
 eps=0        Thomsen's (generic) epsilon 	         		
 delta=0      Thomsen's (generic) delta         			
 gamma=0      Thomsen's (generic) gamma                                
 rho=1        density                					
 phi=0        angle(DEG) vertical --> symmetry axes (clockwise)        
 sign=1       sign of c11+c44 (generally sign=1)                       
 outpar=/dev/tty	output parameter file				

 Output:								
 aijkl,cijkl	(density normalized) stiffness components               

 
 Author: CWP: Andreas Rueger  1995


\end{verbatim}
\pagebreak
\begin{verbatim}
 TRANSP3D - TRANSPose an n1 by n2 by n3 element matrix			

 transp3d <infile >outfile n1= n2= [optional parameters]		

 Required Parameters:							
 n1		number of elements in 1st (fast) dimension of matrix	
 n2		number of elements in 2nd (middle) dimension of matrix	

 Optional Parameters:							
 n3=all	number of elements in 3rd (slow) dimension of matrix	
 perm=231		desired output axis ordering		   	
 nbpe=sizeof(float)	number of bytes per matrix element		
 scratchdir=/tmp	directory for scratch files			
 scratchstem=foo	stem prefix for scratch file names		
 verbose=0		=1 for diagnostic information			

\end{verbatim}
\pagebreak
\begin{verbatim}
 TRANSP - TRANSPose an n1 by n2 element matrix				

 transp <infile >outfile n1= [optional parameters]			

 Required Parameters:							
 n1                     number of elements in 1st (fast) dimension of matrix

 Optional Parameters:							
 n2=all                 number of elements in 2nd (slow) dimension of matrix
 nbpe=sizeof(float)     number of bytes per matrix element		
 verbose=0              =1 for diagnostic information			

\end{verbatim}
\pagebreak
\begin{verbatim}
 TVNMOQC - Check tnmo-vnmo pairs; create t-v column files           

 tvnmoqc [parameters] cdp=... tnmo=... vnmo=...                     

   Example:                                                         
 tvnmoqc mode=1 \\                                                  
 cdp=15,35 \\                                                       
 tnmo=0.0091,0.2501,0.5001,0.7501,0.9941 \\                         
 vnmo=1497.0,2000.0,2500.0,3000.0,3500.0 \\                         
 tnmo=0.0082,0.2402,0.4902,0.7402,0.9842 \\                         
 vnmo=1495.0,1900.0,2400.0,2900.0,3400.0                            

 Required Parameter:                                                
   prefix=        Prefix of output t-v file(s)                      
                  Required only for mode=2                          

 Optional Parameter:                                                
   mode=1         1=qc: check that tnmo values increase             
                  2=qc and output t-v files                         

 mode=1                                                             
   TVNMOQC checks that there is a tnmo and vnmo series for each CDP 
     and it checks that each tnmo series increases in time.         

 mode=2                                                             
   TVNMOQC does mode=1 checking, plus ...                           

   TVNMOQC converts par (MKPARFILE) values written as:              

          cdp=15,35,...,95 \\                                       
          tnmo=t151,t152,...,t15n \\                                
          vnmo=v151,v152,...,v15n \\                                
          tnmo=t351,t352,...,t35n \\                                
          vnmo=v351,v352,...,v35n \\                                
          tnmo=... \\                                               
          vnmo=... \\                                               
          tnmo=t951,t952,...,t95n \\                                
          vnmo=v951,v952,...,v95n \\                                

   to column format. The format of each output file is:             

          t1 v1                                                     
          t2 v2                                                     
           ...                                                      
          tn vn                                                     

   One file is output for each input pair of tnmo-vnmo series.      

   A CDP VALUE MUST BE SUPPLIED FOR EACH TNMO-VNMO ROW PAIR.        

   Prefix of each output file is the user-supplied value of         
     parameter PREFIX.                                              
   Suffix of each output file is the cdp value.                     
   For the example above, output files names are:                   
     PREFIX.15  PREFIX.35  ...  PREFIX.95                           


 Credits:
      MTU: David Forel (adapted from SUNMO)

\end{verbatim}
\pagebreak
\begin{verbatim}
 UNIF2ANISO - generate a 2-D UNIFormly sampled profile of elastic	
	constants from a layered model.					

  unif2aniso < infile [Parameters]					

 Required Parameters:							
 none 									

 Optional Parameters:							
 ninf=5	number of interfaces					
 nx=100	number of x samples (2nd dimension)			
 nz=100	number of z samples (1st dimension)			
 dx=10		x sampling interval					
 dz=10		z sampling interval					

 npmax=201	maximum number of points on interfaces			

 fx=0.0	first x sample						
 fz=0.0	first z sample						


 x0=0.0,0.0,..., 	distance x at which vp00 and vs00 are specified	
 z0=0.0,0.0,..., 	depth z at which vp00 and vs00 are specified	

 vp00=1500,2000,...,	P-velocity at each x0,z0 (m/sec)		
 vs00=866,1155...,	S-velocity at each x0,z0 (m/sec)		
 rho00=1000,1100,...,	density at each x0,z0 (kg/m^3)			
 q00=110,120,130,..,		attenuation Q, at each x0,z0 (kg/m^3)	

 eps00=0,0,0...,	Thomsen or Sayers epsilon			
 delta00=0,0,0...,	Thomsen or Sayers delta				
 gamma00=0,0,0...,	Thomsen or Sayers gamma				

 dqdx=0.0,0.0,...,	x-derivative of Q (d q/dx)			
 dqdz=0.0,0.0,...,	z-derivative of Q (d q/dz)			

 drdx=0.0,0.0,...,	x-derivative of density (d rho/dx)		
 drdz=0.0,0.0,...,	z-derivative of density (d rho/dz)		

 dvpdx=0.0,0.0,...,	x-derivative of P-velocity (dvp/dx)		
 dvpdz=0.0,0.0,...,	z-derivative of P-velocity (dvs/dz)		

 dvsdx=0.0,0.0,...,	x-derivative of S-velocity (dvs/dx)		
 dvsdz=0.0,0.0,...,	z-derivative of S-velocity (dvs/dz)		

 dedx=0.0,0.0,...,	x-derivative of epsilon (de/dx)			
 dedz=0.0,0.0,...,	z-derivative of epsilon with depth z (de/dz)	

 dddx=0.0,0.0,...,	x-derivative of delta (dd/dx)			
 dddz=0.0,0.0,...,	z-derivative of delta (dd/dz)			

 dgdz=0.0,0.0,...,	x-derivative of gamma (dg/dz)			
 dgdx=0.0,0.0,...,	z-derivative of gamma (dg/dx)			

 phi00=0,0,..., 	rotation angle(s) in each layer			

 ...output filenames 							
 c11_file=c11_file	output filename for c11 values	 		
 c13_file=c13_file	output filename for c13 values	 		
 c15_file=c15_file	output filename for c15 values	 		
 c33_file=c33_file	output filename for c33 values	 		
 c35_file=c35_file	output filename for c35 values	 		
 c44_file=c44_file	output filename for c44 values	 		
 c55_file=c55_file	output filename for c55 values	 		
 c66_file=c66_file	output filename for c66 values	 		

 rho_file=rho_file	output filename for density values 		
 q_file=q_file		output filename for Q values	 		

 paramtype=1   =1 Thomsen parameters, =0 Sayers parameters(see below)	
 method=linear		for linear interpolation of interface		
 			=mono for monotonic cubic interpolation of interface
			=akima for Akima's cubic interpolation of interface
			=spline for cubic spline interpolation of interface

 tfile=		=testfilename  if set, a sample input dataset is
 			 output to "testfilename".			

 Notes:								
 The input file is an ASCII file containing x z values representing a	
 piecewise continuous velocity model with a flat surface on top.	

 The surface and each successive boundary between media is represented 
 by a list of selected x z pairs written column form. The first and	
 last x values must be the same for all boundaries. Use the entry	
 1.0  -99999  to separate the entries for successive boundaries. No	
 boundary may cross another. Note that the choice of the method of	
 interpolation may cause boundaries to cross that do not appear to	
 cross in the input data file.						

 The number of interfaces is specified by the parameter "ninf". This 
 number does not include the top surface of the model. The input data	
 format is the same as a CSHOT model file with all comments removed.	

 The algorithm works by transforming the P-wavespeed , S-wavespeed,	
 density and the Thomsen or Sayers parameters epsilon, delta, and gamma
 into elastic stiffness coefficients. Furthermore, the	user can specify
 rotations, phi, to the elasticity tensor in each layer.		

 Common ranges of Thomsen parameters are				
  epsilon:  0.0 -> 0.5							
  delta:   -0.2 -> 0.4							
  gamma:	0.0 -> 0.4							

 If only P-wave, S-wave velocities and density is given as input,	
 the model is, by definition,  isotropic.				

 If files containing Thomsen/Sayers parameters are given, the model	
 will be assumed to have VTI symmetry.		 			

 Example using test input file generating feature:			
 unif2aniso tfile=testfilename  produces a 5 interface demonstration model
 unif2aniso < testfilename 						
 ximage < c11_file n1=100 n2=100					
 ximage < c13_file n1=100 n2=100					
 ximage < c15_file n1=100 n2=100					
 ximage < c33_file n1=100 n2=100					
 ximage < c35_file n1=100 n2=100					
 ximage < c44_file n1=100 n2=100					
 ximage < c55_file n1=100 n2=100					
 ximage < c66_file n1=100 n2=100					
 ximage < rho_file n1=100 n2=100					
 ximage < q_file   n1=100 n2=100					



 Credits:
	CWP: John Stockwell, April 2005. 
 	CWP: based on program unif2 by Zhenyue Liu, 1994 


\end{verbatim}
\pagebreak
\begin{verbatim}
 UNIF2 - generate a 2-D UNIFormly sampled velocity profile from a layered
  	 model. In each layer, velocity is a linear function of position.

  unif2 < infile > outfile [parameters]				

 Required parameters:							
 none									

 Optional Parameters:							
 ninf=5	number of interfaces					
 nx=100	number of x samples (2nd dimension)			
 nz=100	number of z samples (1st dimension)			
 dx=10		x sampling interval					
 dz=10		z sampling interval					

 npmax=201	maximum number of points on interfaces			

 fx=0.0	first x sample						
 fz=0.0	first z sample						

 x0=0.0,0.0,..., 	distance x at which v00 is specified		
 z0=0.0,0.0,..., 	depth z at which v00 is specified		
 v00=1500,2000,2500...,	velocity at each x0,z0 (m/sec)		
 dvdx=0.0,0.0,...,	derivative of velocity with distance x (dv/dx)	
 dvdz=0.0,0.0,...,	derivative of velocity with depth z (dv/dz)	

 method=linear		for linear interpolation of interface		
 			=mono for monotonic cubic interpolation of interface
			=akima for Akima's cubic interpolation of interface
			=spline for cubic spline interpolation of interface

 tfile=		=testfilename  if set, a sample input dataset is
 			 output to "testfilename".			

 Notes:								
 The input file is an ASCII file containing x z values representing a	
 piecewise continuous velocity model with a flat surface on top. The surface
 and each successive boundary between media are represented by a list of
 selected x z pairs written column form. The first and last x values must
 be the same for all boundaries. Use the entry   1.0  -99999  to separate
 entries for successive boundaries. No boundary may cross another. Note
 that the choice of the method of interpolation may cause boundaries 	
 to cross that do not appear to cross in the input data file.		
 The number of interfaces is specified by the parameter "ninf". This 
 number does not include the top surface of the model. The input data	
 format is the same as a CSHOT model file with all comments removed.	

 Example using test input file generating feature:			
 unif2 tfile=testfilename    produces a 5 interface demonstration model
 unif2 < testfilename | psimage n1=100 n2=100 d1=10 d2=10 | ...	



 Credits:
 	CWP: Zhenyue Liu, 1994 
      CWP: John Stockwell, 1994, added demonstration model stuff. 


\end{verbatim}
\pagebreak
\begin{verbatim}
 UNIF2TI2 - generate a 2-D UNIFormly sampled profile of stiffness 	
 	coefficients of a layered medium (only for P waves). 		

  unif2ti2 < infile [Parameters]					

 Required Parameters:							
 none 									

 Optional Parameters:							
 ninf=5	number of interfaces					
 nx=100	number of x samples (2nd dimension)			
 nz=100	number of z samples (1st dimension)			
 dx=10		x sampling interval					
 dz=10		z sampling interval					
 ns=0          number of samples in vertical direction for smoothing   
		parameters across boundary				

 npmax=201	maximum number of points on interfaces			

 fx=0.0	first x sample						
 fz=0.0	first z sample						


 x0=0.0,0.0,..., 	distance x at which vp00 and vs00 are specified	
 z0=0.0,0.0,..., 	depth z at which vp00 and vs00 are specified	

 vp00=1500,2000,...,	P-velocity at each x0,z0 (m/sec)		
 eps00=0,0,0...,	Thomsen parameter epsilon at each x0,z0		
 delta00=0,0,0...,	Thomsen	parameter delta at each x0,z0		
 rho00=1000,1100,...,	density at each x0,z0 (kg/m^3)			

 dvpdx=0.0,0.0,...,	x-derivative of P-velocity (dvp/dx)		
 dvpdz=0.0,0.0,...,	z-derivative of P-velocity (dvs/dz)		

 dedx=0.0,0.0,...,	x-derivative of epsilon (de/dx)			
 dedz=0.0,0.0,...,	z-derivative of epsilon with depth z (de/dz)	

 dddx=0.0,0.0,...,	x-derivative of delta (dd/dx)			
 dddz=0.0,0.0,...,	z-derivative of delta (dd/dz)			

 drdx=0.0,0.0,...,	x-derivative of density (d rho/dx)		
 drdz=0.0,0.0,...,	z-derivative of density (d rho/dz)		

 nufile= 		binary file containning tilt value at each grid point


 ...output filenames 							
 c11_file=c11_file	output filename for c11 values	 		
 c13_file=c13_file	output filename for c13 values	 		
 c15_file=c15_file	output filename for c15 values	 		
 c33_file=c33_file	output filename for c33 values	 		
 c35_file=c35_file	output filename for c35 values	 		
 c55_file=c55_file	output filename for c55 values	 		


 method=linear		for linear interpolation of interface		
 			=mono for monotonic cubic interpolation of interface
			=akima for Akima's cubic interpolation of interface
			=spline for cubic spline interpolation of interface

 tfile=		=testfilename  if set, a sample input dataset is
 			 output to "testfilename".			

 Notes:								
 The input file is an ASCII file containing x z values representing a	
 piecewise continuous velocity model with a flat surface on top.	

 The surface and each successive boundary between media is represented 
 by a list of selected x z pairs written column form. The first and	
 last x values must be the same for all boundaries. Use the entry	
 1.0  -99999  to separate the entries for successive boundaries. No	
 boundary may cross another. Note that the choice of the method of	
 interpolation may cause boundaries to cross that do not appear to	
 cross in the input data file.						

 The number of interfaces is specified by the parameter "ninf". This 
 number does not include the top surface of the model. The input data	
 format is the same as a CSHOT model file with all comments removed.	

 The algorithm works by transforming the P-wavespeed, Thomsen parameters 
 epsilon and delta, and the tilt of the symmetry axis into density-normalized", 
 stiffness coefficients. 						

 At this stage, the tilt-field file can be prepared using the 		
 Matlab M-file nu_mod.m based on 2D interpolation between interfaces.  ", 
 The binary file contains nu values at each grid point.		
 The interfaces are obtained by interpolation on the picked ones 	
 stored in the infile, and the symmetry axis at each point of interface
 is assumed to be parallel to the normal direction.			

 Common ranges of Thomsen parameters are				
  epsilon:  0.0 -> 0.5							
  delta:   -0.2 -> 0.4							


 If the tilt-field file is not given, the model will be assumed to 	
 have VTI symmetry. 				 			

 Example using test input file generating feature:			
 unif2aniso tfile=testfilename  produces a 5 interface demonstration model
 unif2aniso < testfilename 						
 ximage < c11_file n1=100 n2=100					
 ximage < c13_file n1=100 n2=100					
 ximage < c15_file n1=100 n2=100					
 ximage < c33_file n1=100 n2=100					
 ximage < c35_file n1=100 n2=100					
 ximage < c55_file n1=100 n2=100					
 ximage < rho_file n1=100 n2=100					



 Credits:
      Modified by Pengfei Cai (CWP), Dec 2011
      Modified by Xiaoxiang Wang (CWP), Aug 2010
 	Based on program unif2aniso by John Stockwell, 2005 


\end{verbatim}
\pagebreak
\begin{verbatim}
 UNISAM2 - UNIformly SAMple a 2-D function f(x1,x2)			

 unisam2 [optional parameters] <inputfile >outputfile			

 Required Parameters:							
 none									
 Optional Parameters:							
 x1=             array of x1 values at which input f(x1,x2) is sampled	
 ... Or specify a unform linear set of values for x1 via:		
 nx1=1           number of input samples in 1st dimension		
 dx1=1           input sampling interval in 1st dimension		
 fx1=0           first input sample in 1st dimension			
 ...									
 n1=1            number of output samples in 1st dimension		
 d1=             output sampling interval in 1st dimension		
 f1=             first output sample in 1st dimension			
 x2=             array of x2 values at which input f(x1,x2) is sampled	
 ... Or specify a unform linear set of values for x2 via:		
 nx2=1           number of input samples in 2nd dimension		
 dx2=1           input sampling interval in 2nd dimension		
 fx2=0           first input sample in 2nd dimension			
 ...									
 n2=1            number of output samples in 2nd dimension		
 d2=             output sampling interval in 2nd dimension		
 f2=             first output sample in 2nd dimension			
 ... 									
 method1=linear  =linear for linear interpolation			
                 =mono for monotonic bicubic interpolation		
                 =akima for Akima bicubic interpolation		
                 =spline for bicubic spline interpolation		
 method2=linear  =linear for linear interpolation			
                 =mono for monotonic bicubic interpolation		
                 =akima for Akima bicubic interpolation		
                 =spline for bicubic spline interpolation		

 NOTES:								
 The number of input samples is the number of x1 values times the	
 number of x2 values.  The number of output samples is n1 times n2.	
 The output sampling intervals (d1 and d2) and first samples (f1 and f2)
 default to span the range of input x1 and x2 values.  In other words,	
 d1=(x1max-x1min)/(n1-1) and f1=x1min; likewise for d2 and f2.		

 Interpolation is first performed along the 2nd dimension for each	
 value of x1 specified.  Interpolation is then performed along the	
 1st dimension.							



 AUTHOR:  Dave Hale, Colorado School of Mines, 01/12/91\n"

\end{verbatim}
\pagebreak
\begin{verbatim}
 UNISAM - UNIformly SAMple a function y(x) specified as x,y pairs	

   unisam xin= yin= nout= [optional parameters] >binaryfile		
    ... or ...								
   unisam xfile= yfile= npairs= nout= [optional parameters] >binaryfile
    ... or ...								
   unisam xyfile= npairs= nout= [optional parameters] >binaryfile	

 Required Parameters:							
 xin=,,,	array of x values (number of xin = number of yin)	
 yin=,,,	array of y values (number of yin = number of xin)	
  ... or								
 xfile=	binary file of x values					
 yfile=	binary file of y values					
  ... or								
 xyfile=	binary file of x,y pairs				
 npairs=	number of pairs input (active only if xfile= and yfile=	
 		or xyfile= are set)					

 nout=		 number of y values output to binary file		

 Optional Parameters:							
 dxout=1.0	 output x sampling interval				
 fxout=0.0	 output first x						
 method=linear  =linear for linear interpolation (continuous y)	
		 =mono for monotonic cubic interpolation (continuous y')
		 =akima for Akima's cubic interpolation (continuous y') 
		 =spline for cubic spline interpolation (continuous y'')
 isint=,,,	 where these sine interpolations to apply		
 amp=,,,	 amplitude of sine interpolations			
 phase0=,,,	 starting phase (defaults: 0,0,0,...,0)			
 totalphase=,,, total phase (default pi,pi,pi,...,pi.)			
 nwidth=0       apply window smoothing if nwidth>0                     
 sloth=0	 apply interpolation in input (velocities)		
		 =1 apply interpolation to 1/input (slowness),		
 		 =2 apply interpolation to 1/input (sloth), and write	
 		 out velocities in each case.				
 smooth=0	 apply damped least squares smoothing to output		
 r=10		  ... damping coefficient, only active when smooth=1	


 AUTHOR:  Dave Hale, Colorado School of Mines, 07/07/89
          Zhaobo Meng, Colorado School of Mines, 
 	    added sine interpolation and window smoothing, 09/16/96 
          CWP: John Stockwell,  added file input options, 24 Nov 1997

 Remarks: In interpolation, suppose you need 2 pieces of 
 	    sine interpolation before index 3 to 4, and index 20 to 21
	    then set: isint=3,20. The sine interpolations use a sine
	    function with starting phase being phase0, total phase 
	    being totalphase (i.e. ending phase being phase0+totalphase
	    for each interpolation).
 	    

\end{verbatim}
\pagebreak
\begin{verbatim}
 UTMCONV - CONVert longitude and latitude to UTM, and vice versa       

 utmconv <stdin >stdout [optional parameters]                          

 Optional parameters:                                                  
    idx=23          reference ellipsoid index (default is WGS 1984)    
    format=%.3f     output number format (printf style for one float)  
    a=(from idx)    user-specified semimajor axis of ellipsoid         
    f=(from idx)    user-specified flattening of ellipsoid             
    letter=0        =1: use UTM letter designator for latitude/Northing
    invert=0        =0: convert latitude and longitude to UTM          
                    =1: convert UTM to latitude and longitude          
    verbose=0       =1: echo parameters and number of converted coords 

    lon0=           central meridian for TM projection in degrees      
                    (default uses the 60 standard UTM longitude zones) 
    xoff=500000     false Easting (default: UTM)                       
    ysoff=10000000  false Northing, southern hemisphere (default: UTM) 
    ynoff=0         false Northing, northern hemisphere (default: UTM) 

 Notes:                                                                
    Universal Transverse Mercator (UTM) coordinates are defined between
    latitudes 80S (-80) and 84N (84). Longitude values must be between 
    -180 degrees (west) and 179.999... degrees (east).                 

    Input and output is in ASCII format. For a conversion from lon/lat 
    to UTM (invert=0), input is a two-column table of longitude and    
    latitude in decimal degrees. Output is a three-column table of     
    UTM Easting, UTM Northing, and UTM zone. The zone is given either  
    by the zone number only (default, negative on southern hemisphere) 
    or by the positive zone number plus a letter designator (letter=1).

 Example:                                                              
    Convert 40.822N, 14.125E to UTM with zone number and letter,       
    output values rounded to nearest integer:                          
        echo 14.125 40.822 | utmconv letter=1 format=%.0f              
    The output is "426213 4519366 33T" (Easting, Northing, UTM zone).

 Reference ellipsoids:                                                 
    An ellipsoid may be specified by its semimajor axis a and its      
    flattening f, or one of the following ellipsoids may be selected   
    by its index idx (semimajor axes in meters):                       
     0  Sphere with radius of 6371000 m                                
     1  Airy 1830                                                      
     2  Australian National 1965                                       
     3  Bessel 1841 (Ethiopia, Indonesia, Japan, Korea)                
     4  Bessel 1841 (Namibia)                                          
     5  Clarke 1866                                                    
     6  Clarke 1880                                                    
     7  Everest (Brunei, E. Malaysia)                                  
     8  Everest (India 1830)                                           
     9  Everest (India 1956)                                           
    10  Everest (Pakistan)                                             
    11  Everest (W. Malaysia, Singapore 1948)                          
    12  Everest (W. Malaysia 1969)                                     
    13  Geodetic Reference System 1980 (GRS 1980)                      
    14  Helmert 1906                                                   
    15  Hough 1960                                                     
    16  Indonesian 1974                                                
    17  International 1924 / Hayford 1909                              
    18  Krassovsky 1940                                                
    19  Modified Airy                                                  
    20  Modified Fischer 1960                                          
    21  South American 1969                                            
    22  World Geodetic System 1972 (WGS 1972)                          
    23  World Geodetic System 1984 (WGS 1984) / NAD 1983               


 UTM grid:
 The Universal Transverse Mercator (UTM) system is a world wide
 coordinate system defined between 80S and 84N. It divides the
 Earth into 60 six-degree zones. Zone number 1 has its central
 meridian at 177W (-177 degrees), and numbers increase eastward.

 Within each zone, an Easting of 500,000 m is assigned to its 
 central meridian to avoid negative coordinates. On the northern
 hemisphere, Northings start at 0 m at the equator and increase 
 northward. On the southern hemisphere a false Northing of 
 10,000,000 m is applied, i.e. Northings start at 10,000,000 m at 
 the equator and decrease southward. Letters are sometimes used
 to identify different zones of latitude. The letters C-M 
 indicate zones on the southern and the letters N-X zones on 
 the northern hemisphere.

 Author: 
    Nils Maercklin, RISSC, University of Naples, Italy, March 2007

 References:
 NIMA (2000). Department of Defense World Geodetic System 1984 - 
    its definition and relationships with local geodetic systems.
    Technical Report TR8350.2. National Imagery and Mapping Agency, 
    Geodesy and Geophysics Department, St. Louis, MO. 3rd edition.
 J. P. Snyder (1987). Map Projections - A Working Manual. 
    U.S. Geological Survey Professional Paper 1395, 383 pages.
    U.S. Government Printing Office.

\end{verbatim}
\pagebreak
\begin{verbatim}
 VEL2STIFF - Transforms VELocities, densities, and Thomsen or Sayers   
		parameters to elastic STIFFnesses 			

 vel2stiff  [Required parameters] [Optional Parameters] > stdout	

 Required parameters:							
 vpfile=	file with P-wave velocities				
 vsfile=	file with S-wave velocities				
 rhofile=	file with densities					

 Optional Parameters:							
 epsfile=	file with Thomsen/Sayers epsilon			
 deltafile=	file with Thomsen/Sayers delta			 	
 gammafile=	file with Thomsen/Sayers gamma			 	
 phi_file=	angle of axis of symmetry from vertical (radians)	

 c11_file=c11_file     output filename for c11 values                  
 c13_file=c13_file     output filename for c13 values                  
 c15_file=c15_file     output filename for c15 values                  
 c33_file=c33_file     output filename for c33 values                  
 c35_file=c35_file     output filename for c35 values                  
 c44_file=c44_file     output filename for c44 values                  
 c55_file=c55_file     output filename for c55 values                  
 c66_file=c66_file     output filename for c66 values                  

 paramtype=1  (1) Thomsen parameters, (0) Sayers parameters(see below) 

 nx=101	number of x samples 2nd (slow) dimension		
 nz=101	number of z samples 1st (fast) dimension		

 Notes: 								
 Transforms velocities, density and Thomsen/Sayers parameters		
 epsilon, delta, and gamma into elastic stiffness coefficients.	

 If only P-wave, S-wave velocities and density is given as input,	
 the model is assumed to be isotropic.					

 If files containing Thomsen/Sayers parameters are given, the model	
 will be assumed to have VTI symmetry.		 			

 All input files  vpfile, vsfile, rhofile etc. are assumed to consist  
 only of C style binary floating point numbers representing the        
 corresponding  material values of vp, vs, rho etc. Similarly, the output
 files consist of the coresponding stiffnesses as C style binary floats. 
 If the output files are to be used as input for a modeling program,   
 such as suea2df, then further, the contents are assumed be arrays of  
 floating point numbers of the form of   Array[n2][n1], where the fast 
 dimension, dimension 1, represents depth.                             



  Author:
  CWP: Sverre Brandsberg-Dahl 1999

  Extended:
  CWP: Stig-Kyrre Foss 2001
  - to include the option to use the parameters by Sayers (1995) 
  instead of the Thomsen parameters

 Technical reference:
 Sayers, C. M.: Simplified anisotropy parameters for transversely 
 isotropic sedimentary rocks. Geophysics 1995, pages 1933-1935.


\end{verbatim}
\pagebreak
\begin{verbatim}
 VELCONV - VELocity CONVersion					

 velconv <infile >outfile intype= outtype= [optional parameters]

 Required Parameters:						
 intype=                input data type (see valid types below)
 outtype=               output data type (see valid types below)

 Valid types for input and output data are:			
 vintt          interval velocity as a function of time	
 vrmst          RMS velocity as a function of time		
 vintz          velocity as a function of depth		
 zt             depth as a function of time			
 tz             time as a function of depth			

 Optional Parameters:						
 nt=all                 number of time samples			
 dt=1.0                 time sampling interval			
 ft=0.0                 first time				
 nz=all                 number of depth samples		
 dz=1.0                 depth sampling interval		
 fz=0.0                 first depth				
 nx=all                 number of traces			

 Example:  "intype=vintz outtype=vrmst" converts an interval velocity
           function of depth to an RMS velocity function of time.

 Notes:  nt, dt, and ft are used only for input and output functions
         of time; you need specify these only for vintt, vrmst, orzt.
         Likewise, nz, dz, and fz are used only for input and output
         functions of depth.					

 The input and output data formats are C-style binary floats.	


  AUTHOR:  Dave Hale, Colorado School of Mines, 07/07/89

\end{verbatim}
\pagebreak
\begin{verbatim}
 VELPERTAN - Velocity PERTerbation analysis in ANisotropic media to    
             determine the model update required to flatten image gathers",  

 velpertan boundary= par=cig.par refl1= refl2= npicks1= 		
	npicks2= cdp1= cdp2= vfile= efile= dfile= nx= dx= fx= 		
	ncdp= dcdp= fcdp= off0= noff= doff= >outfile [parameters]	

 Required Parameters:							
 refl1=	file with picks on the 1st reflector	  		
 refl2=	file with picks on the 2nd reflector  			
 vfile=	file defining VP0 at all grid points from prev. iter.	", 
 efile=	file defining eps at all grid points from prev. iter.	", 
 dfile=	file defining del at all grid points from prev. iter.	", 
 boundary=	file defining the boundary above which 			
        	parameters are known; update is done below this		", 
		boundary						", 
 npicks1=	number of picks on the 1st reflector			
 npicks2=	number of picks on the 2nd reflector			
 ncdp=		number of cdp's		 				
 dcdp=		cdp spacing			 			
 fcdp=		first cdp			 			
 off0=		first offset in common image gathers 			
 noff=		number of offsets in common image gathers  		
 doff=		offset increment in common image gathers  		
 cip1=x1,r1,r2,..., cip=xn,r1n,r2n         description of input CIGS	
 cip2=x2,r1,r2,..., cip=xn,r1n,r2n         description of input CIGS	
	x	x-value of a common image point				
	r1	hyperbolic component of the residual moveout		
	r2	non-hyperbolic component of residual moveout		
 Optional Parameters:							
 method=akima         for linear interpolation of the interface       
                       =mono for monotonic cubic interpolation of interface
                       =akima for Akima's cubic interpolation of interface 
                       =spline for cubic spline interpolation of interface 
 VP0=2000	Starting value for vertical velocity at a point in the   
							target layer	
 x00=0.0	x-coordinate at which VP0 is defined			
 z00=0.0	z-coordinate at which VP0 is defined			
 eps=0.0	Starting value for Thomsen's parameter epsilon		
 del=0.0	Starting value for Thomsen's parameter delta		
 kz=0.0	Starting value for the vertical gradient in VP0		
 kx=0.0	Starting value for the lateral gradient in VP0		
 nx=100	number of nodes in the horizontal direction for the     
							velocity grid 	
 nz=100	number of nodes in the vertical direction for the	
							velocity grid	
 dx=10	horizontal grid increment				
 dz=10	vertical grid increment					
 fx=0		first horizontal grid point				
 fz=0		first vertical grid point				
 dt=0.008	traveltime increment					
 nt=500	no. of points on the ray				
 amax=360	max. angle of emergence					
 amin=0	min. angle of emergence					

 Smoothing parameters:							
 r1=0                  smoothing parameter in the 1 direction          
 r2=0                  smoothing parameter in the 2 direction          
 win=0,n1,0,n2         array for window range                          
 rw=0                  smoothing parameter for window function         
 nbound=2	number of points picked on the boundary			
 tol=0.1	tolerance in computing the offset (m)			
 Notes:								
 This program is used as part of the velocity analysis technique developed
 by Debashish Sarkar, CWP:2003.					

 Notes:								
 The output par file contains the coefficients describing the residual 
 moveout. This program is used in conjunction with surelanan.		


 Author: CSM: Debashish Sarkar, December 2003 
 based on program: velpert.c written by Zhenuye Liu.


\end{verbatim}
\pagebreak
\begin{verbatim}
 VELPERT - estimate velocity parameter perturbation from covariance 	
	 of imaged depths in common image gathers (CIGs)		

 verpert <dfile dzfile=dzfile >outfile [parameters]			

 Required Parameters:							
 dfile			input of imaged depths in CIGs			
 dzfile=dzfile		input of dz/dv at the imaged depths in CIGs	
 outfile		output of the estimated parameter 		
 noff			number of offsets 				
 ncip			number of common image gathers 			

 Optional Parameters:							
 moff=noff	number of first offsets used in velocity estimation  	

 Notes:								
 1. This program is part of Zhenyue Liu's velocity analysis technique.	
    The input dzdv values are computed using the program dzdv.		
 2. For given depths, using moff smaller than noff may avoid poor 	
    values of dz/dv at far offsets. However, a too small moff used	
    will the sensitivity of velocity error to the imaged depth.	
 3. Outfile contains three parts:					
    dlambda	correction of the velocity paramter. dlambda plus	
    		the initial parameter (used in migration) will	be	
		the updated one.					
    deviation	to measure how close imaged depths are to each other	
    		in CIGs. Old deviation corresponds to the initial	
		parameter; new deviation corresponds to the updated one.
    sensitivity  to predict how sensitive the error in the estimated	
		parameter is to an error in the measurement of imaged	
		depths.							

       error of parameter <= sensitivity * error of depth.		


 
 Author:  Zhenyue Liu, 12/29/93,  Colorado School of Mines

 Reference: 
 Liu, Z. 1995, "Migration Velocity Analysis", Ph.D. Thesis, Colorado
      School of Mines, CWP report #168.

\end{verbatim}
\pagebreak
\begin{verbatim}

 VERHULST - solve the VERHULST logistic equation		

  verhulst > [stdout]						

 Required Parameters: none					
 Optional Parameters:						
 a1=1.0		parameter for verhulst equation		
 a2=2000		parameter for verhulst equation		
 y0=10			initial value of y[0]			
 h=.01			increment in time			
 tol=1.e-08		error tolerance				
 stepmax=2000		maximum number of steps to compute	
 mode=x		xy-pairs, =yz yz-pairs, =xz xz-pairs,	
			=xyz xyz-triplet, =x only, =y only, =z only
 Notes:							
 This program is really just a demo showing how to use the 	
 differential equation solver rke_solve written by Francois 	
 Pinard, based on a modified form of the 4th order Runge-Kutta 
 method, which employs the error checking method of R. England 
 1969.								

 The output consists of unformated C-style binary floats, of	
 either pairs or triplets as specified by the "mode" paramerter.

 Examples:							
 x is the population						
 verhulst stepmax=2000 mode=x | suaddhead ns=2000 | suxwigb &	
 y is dx/dt, the rate of growth of the population		
 verhulst stepmax=2000 mode=y | suaddhead ns=2000 | suxwigb &	

 In the Verhulst equation, a1 is the reproduction rate and	
 a2 is the carrying capacity					
 	x'(t) = a1 * x * ( 1 - x/a2 )			 	


 The verhulst equation describes a simplified model of a population
 reproducing in an environment with limited resources,
 and are given by the autonomous system of ODE's	
	y'(t) = a1 * y ( 1 - y/a2 )			

 Author: CWP: Aug 2009: John Stockwell


\end{verbatim}
\pagebreak
\begin{verbatim}
 VTLVZ -- Velocity as function of Time for Linear V(Z);		
          writes out a vector of velocity = v0 exp(a t/2)		

 vtlvz > velfile nt= dt= v0= a= 					

 Required parameters							
 nt=	number of time samples						
 dt=	time sampling interval						
 v0=	velocity at the surface						
 a=	velocity gradient						

\end{verbatim}
\pagebreak
\begin{verbatim}
 WKBJ - Compute WKBJ ray theoretic parameters, via finite differencing	

 wkbj <vfile >tfile nx= nz= xs= zs= [optional parameters]		

 Required Parameters:							
 <vfile	file containing velocities v[nx][nz]			
 nx=		number of x samples (2nd dimension)			
 nz=		number of z samples (1st dimension)			
 xs=		x coordinate of source					
 zs=		z coordinate of source					

 Optional Parameters:							
 dx=1.0		 x sampling interval				
 fx=0.0		 first x sample					
 dz=1.0		 z sampling interval				
 fz=0.0		 first z sample					
 sfile=sfile	file containing sigmas sg[nx][nz]			
 bfile=bfile	file containing incident angles bet[nx][nz]		
 afile=afile	file containing propagation angles a[nx][nz]		

 Notes:								
 Traveltimes, propagation angles, sigmas, and incident angles in WKBJ	
 by finite differences  in polar coordinates. Traveltimes are calculated
 by upwind scheme; sigmas and incident angles by a Crank-Nicolson scheme.

 Credits:
	CWP: Zhenyue Liu, Dave Hale, pre 1992. 

\end{verbatim}
\pagebreak
\begin{verbatim}
 XY2Z - converts (X,Y)-pairs to spike Z values on a uniform grid	

    xy2z < stdin npairs= [optional parameters] >stdout 		

 Required parameter:							
 npairs= 	number of pairs input					

 Optional parameter:							
 scale=1.0	value to scale spikes by				
 nx1=100 	dimension of first (fast) dimension of output array	
 nx2=100 	dimension of second (slow) dimension of output array	
 x1pad=2	zero padding in x1 dimension				
 x2pad=2	zero padding in x2 dimension				
 yx=0		assume (x,y) pairs 					
 		=1	assume (y,x) pairs 				

 Notes: 								
 Converts ordered (x,y) pairs to spike x1values, of height=scale on a 	
 uniform grid.								


 Credits:
	CWP: John Stockwell, Nov 1995

\end{verbatim}
\pagebreak
\begin{verbatim}
 Z2XYZ - convert binary floats representing Z-values to ascii	
 	   form in X Y Z ordered triples			

    z2xyz <stdin >stdout 					

 Required parameters:						
 n1=		number of floats in 1st (fast) dimension	

 Optional parameters:						

 outpar=/dev/tty		 output parameter file		

 Notes: This program is useful for converting panels of float	
 data (representing evenly spaced z values) to the x y z	
 ordered triples required for certain 3D plotting packages.	

 Example of NXplot3d usage on a NeXT:				
 suplane | sufilter | z2xyz n1=64 > junk.ascii			

 Now open junk.ascii as a mesh data file with NXplot3d.	
 (NXplot3d is a NeXTStep-only utility for viewing 3d data sets	


 Credits:
	CWP: John Stockwell based on "b2a" by Jack Cohen


\end{verbatim}
\pagebreak
\begin{verbatim}
 SUCENTSAMP - CENTRoid SAMPle seismic traces			

  sucentsamp <stdin [optional parameters] >sdout  		

 Required parameters:						
 	none							

 Optional parameters:						
 	dt=from header		sampling interval		
	verbose=1		=0 to stop advisory messages	

 Notes: 							
 This program takes seismic traces as input, and returns traces
 consisting of spikes of height equal to the area of each lobe 
 of each oscillation, located at the centroid of the lobe in	
 question. The height of each spike equal to the area of the   
 corresponding lobe.						

 Caveat: No check is made that the data are real time traces!	


 Credits:

	Providence Technologies: Tom Morgan 

 Trace header fields accessed: ns, dt

\end{verbatim}
\pagebreak
\begin{verbatim}
 SUDIPDIVCOR - Dip-dependent Divergence (spreading) correction	

	sudipdivcor <stdin >stdout  [optional parms]		

 Required Parameters:						
	dxcdp	distance between sucessive cdps	in meters	

 Optional Parameters:						
	np=50		number of slopes			
	tmig=0.0	times corresponding to rms velocities in vmig
	vmig=1500.0	rms velocities corresponding to times in tmig
	vfile=binary	(non-ascii) file containing velocities vmig(t) 
	conv=0		=1 to apply the conventional divergence correction
	trans=0		=1 to include transmission factors 	
	verbose=0	=1 for diagnostic print			

 Notes:								
 The tmig, vmig arrays specify an rms velocity function of time.
 Linear interpolation and constant extrapolation is used to determine
 rms velocities at times not specified.  Values specified in tmig
 must increase monotonically.					

 Alternatively, rms velocities may be stored in a binary file
 containing one velocity for every time sample.  If vfile is	
 specified, then the tmig and vmig arrays are ignored.		
 The time of the first sample is assumed to be constant, and is
 taken as the value of the first trace header field delrt. 	

 Whereas the conventional divergence correction (sudivcor) is	
 valid only for horizontal reflectors, which have zero reflection
 slope, the dip-dependent divergence correction is valid for any
 reflector dip or reflection slope.  Only the conventional	
 correction will be applied to the data if conv=1 is specified. 
 Note that the conventional correction over-amplifies		
 reflections from dipping beds					

 The transmission factor should be applied when the divergence 
 corrected data is to be migrated with a reverse time migration 
 based on the constant density wave equation.			

 Trace header fields accessed:  ns, dt, delrt			

\end{verbatim}
\pagebreak
\begin{verbatim}
 SUDIVCOR - Divergence (spreading) correction				

 sudivcor <stdin >stdout  [optional parms]				

 Required Parameters:							
 none									

 Optional Parameters:							
 trms=0.0	times corresponding to rms velocities in vrms		
 vrms=1500.0	rms velocities corresponding to times in trms	
 vfile=	binary (non-ascii) file containing velocities vrms(t)	

 Notes:								
 The trms, vrms arrays specify an rms velocity function of time.	
 Linear interpolation and constant extrapolation is used to determine	
 rms velocities at times not specified.  Values specified in trms	
 must increase monotonically.						

 Alternatively, rms velocities may be stored in a binary file		
 containing one velocity for every time sample.  If vfile is specified,
 then the trms and vrms arrays are ignored.				

 The time of the first sample is assumed to be constant, and is taken	
 as the value of the first trace header field delrt. 			

 Credits:
	CWP: Jack K. Cohen, Francesca Fazarri

 Trace header fields accessed:  ns, dt, delrt

\end{verbatim}
\pagebreak
\begin{verbatim}
 SUGAIN - apply various types of gain				  	

 sugain <stdin >stdout [optional parameters]			   	

 Required parameters:						  	
	none (no-op)						    	

 Optional parameters:						  	
	panel=0	        =1  gain whole data set (vs. trace by trace)	
	tpow=0.0	multiply data by t^tpow			 	
	epow=0.0	multiply data by exp(epow*t)		    	
	etpow=1.0	multiply data by exp(epow*t^etpow)	    	
	gpow=1.0	take signed gpowth power of scaled data	 	
	agc=0	   flag; 1 = do automatic gain control	     		
	gagc=0	  flag; 1 = ... with gaussian taper			
	wagc=0.5	agc window in seconds (use if agc=1 or gagc=1)  
	trap=none	zero any value whose magnitude exceeds trapval  
	clip=none	clip any value whose magnitude exceeds clipval  
	pclip=none	clip any value greater than clipval  		
	nclip=none	clip any value less than  clipval 		
	qclip=1.0	clip by quantile on absolute values on trace    
	qbal=0	  flag; 1 = balance traces by qclip and scale     	
	pbal=0	  flag; 1 = bal traces by dividing by rms value   	
	mbal=0	  flag; 1 = bal traces by subtracting the mean    	
	maxbal=0	flag; 1 = balance traces by subtracting the max 
	scale=1.0	multiply data by overall scale factor	   	
	norm=0.0	divide data by overall scale factor	     	
	bias=0.0	bias data by adding an overall bias value	
	jon=0	   	flag; 1 means tpow=2, gpow=.5, qclip=.95	
	verbose=0	verbose = 1 echoes info				
	mark=0		apply gain only to traces with tr.mark=0	
			=1 apply gain only to traces with tr.mark!=0    
	vred=0	  reducing velocity of data to use with tpow		

 	tmpdir=		if non-empty, use the value as a directory path	
			prefix for storing temporary files; else if the 
			the CWP_TMPDIR environment variable is set use  
			its value for the path; else use tmpfile()	

 Operation order:							
 if (norm) scale/norm						  	

 out(t) = scale * BAL{CLIP[AGC{[t^tpow * exp(epow * t^tpow) * ( in(t)-bias )]^gpow}]}

 Notes:								
	The jon flag selects the parameter choices discussed in		
	Claerbout's Imaging the Earth, pp 233-236.			

	Extremely large/small values may be lost during agc. Windowing  
	these off and applying a scale in a preliminary pass through	
	sugain may help.						

	Sugain only applies gain to traces with tr.mark=0. Use sushw,	
	suchw, suedit, or suxedit to mark traces you do not want gained.
	See the selfdocs of sushw, suchw, suedit, and suxedit for more	
	information about setting header fields. Use "sukeyword mark
	for more information about the mark header field.		

      debias data by using mbal=1					

      option etpow only becomes active if epow is nonzero		

 Credits:
	SEP: Jon Claerbout
	CWP: Jack K. Cohen, Brian Sumner, Dave Hale

 Note: Have assumed tr.deltr >= 0 in tpow routine.

 Technical Reference:
	Jon's second book, pages 233-236.

 Trace header fields accessed: ns, dt, delrt, mark, offset

\end{verbatim}
\pagebreak
\begin{verbatim}
 SUNAN - remove NaNs & Infs from the input stream		

    sunan < in.su >out.su					

 Optional parameters:						
 verbose=1	echo locations of NaNs or Infs to stderr	
	        =0 silent					
 ...user defined ... 						

 value=0.0	NaNs and Inf replacement value			
 ... and/or....						
 interp=0	=1 replace NaNs and Infs by interpolating	
                   neighboring finite values			

 Notes:							
 A simple program to remove NaNs and Infs from an input stream.
 The program sets NaNs and Infs to "value" if interp=0. When	
 interp=1 NaNs are replaced with the average of neighboring values
 provided that the neighboring values are finite, otherwise	
 NaNs and Infs are replaced by "value".			


 Author: Reginald H. Beardsley  2003   rhb@acm.org

  A simple program to remove NaNs & Infs from an input stream. They
  shouldn't be there, but it can be hard to find the cause and fix
  the problem if you can't look at the data.

  Interpolation idea comes from a version of sunan modified by
  Balasz Nemeth while at Potash Corporation in Saskatchewan.



\end{verbatim}
\pagebreak
\begin{verbatim}
 SUNORMALIZE - Trace NORMALIZation by rms, max, or median       ", 
               or median balancing                              

   sunormalize <stdin >stdout t0=0 t1=TMAX norm=rms             

 Required parameters:                                           
    dt=tr.dt    if not set in header, dt is mandatory           
    ns=tr.ns    if not set in header, ns is mandatory           

 Optional parameters:                                           
    norm=rms    type of norm rms, max, med , balmed             
    t0=0.0      startimg time for window                        
    t1=TMAX     ending time for window                          

 Notes:                                                         
 Traces are divided by either the root mean squared amplitude,  
 trace maximum, or the median value. The option "balmed" is   
 median balancing which is a shift of the amplitudes by the	 
 median value of the amplitudes.				 



 Author: Ramone Carbonell, 
         Inst. Earth Sciences-CSIC Barcelona, Spain, April 1998.
 Modifications: Nils Maercklin,
         RISSC, University of Naples, Italy, September 2006
         (fixed user input of ns, dt, if values are not set in header).

 Trace header fields accessed: ns, dt
 Trace header fields modified: none

\end{verbatim}
\pagebreak
\begin{verbatim}
 SUPGC   -   Programmed Gain Control--apply agc like function	
              but the same function to all traces preserving	
              relative amplitudes spatially.			
 Required parameter:						
 file=             name of input file				

 Optional parameters:						
 ntrscan=200       number of traces to scan for gain function	
 lwindow=1.0       length of time window in seconds		



 Author: John Anderson (visitor to CWP from Mobil)

 Trace header fields accessed: ns, dt


\end{verbatim}
\pagebreak
\begin{verbatim}
 SUWEIGHT - weight traces by header parameter, such as offset		

   suweight < stdin > stdout [optional parameters]			

 Required Parameters:					   		
   <none>								

 Optional parameters:					   		
 key=offset	keyword of header field to weight traces by 		
 a=1.0		constant weighting parameter (see notes below)		
 b=.0005	variable weighting parameter (see notes below)		

... or use values of a header field for the weighting ...		

 key2=		keyword of header field to draw weights from		
 scale=.0001	scale factor to apply to header field values		

 inv=0		weight by header value			 		
 		=1 weight by inverse of header value	 		

 Notes:							 	
 This code is initially written with offset weighting in mind, but may	
 be used for other, user-specified schemes.				

 The rationale for this program is to correct for unwanted linear	
 amplitude trends with offset prior to either CMP stacking or AVO work.
 The code has to be edited should other functions of a keyword be required.

 The default form of the weighting is to multiply the amplitudes of the
 traces by a factor of:    ( a + b*keyword).				

 If key2=  header field is  set then this program uses the weighting	
 values read from that header field, instead. Note, that because most	
 header fields are integers, the scale=.0001 permits 10001 in the header
 to represent 1.0001.							

 To see the list of available keywords, type:    sukeyword  -o  <CR>	


 Credits:
 Author: CWP: John Stockwell  February 1999.
 Written for Chris Walker of UniqueStep Ltd., Bedford, U.K.
 inv option added by Garry Perratt (Geocon).

 header fields accessed: ns, keyword


\end{verbatim}
\pagebreak
\begin{verbatim}
 SUZERO -- zero-out (or set constant) data within a time window	

 suzero itmax= < indata > outdata				

 Required parameters						
 	itmax=		last time sample to zero out		

 Optional parameters						
 	itmin=0		first time sample to zero out		
 	value=0		value to set				

 See also: sukill, sumute					


 Credits:
	CWP: Chris
	Geocon: Garry Perratt (added value= option)

 Trace header fields accessed: ns

\end{verbatim}
\pagebreak
\begin{verbatim}
 SUATTRIBUTES - instantaneous trace ATTRIBUTES 			

 suattributes <stdin >stdout mode=amp					

 Required parameters:							
 	none								

 Optional parameter:							
 	mode=amp	output flag 					
 	       		=amp envelope traces				
 	       		=phase phase traces				
 	       		=freq frequency traces				
			=bandwith Instantaneous bandwidth		
			=normamp Normalized Phase (Cosine Phase)	
 	       		=fdenv 1st envelope traces derivative		
 	       		=sdenv 2nd envelope traces derivative		
 	       		=q Ins. Q Factor				
 ... unwrapping related options ....					
	unwrap=		default unwrap=0 for mode=phase			
 			default unwrap=1 for freq, uphase, freqw, Q	
 			dphase_min=PI/unwrap				
       trend=0		=1 remove the linear trend of the inst. phase	
 	zeromean=0	=1 assume instantaneous phase is zero mean	

			=freqw Frequency Weighted Envelope		
			=thin  Thin-Bed (inst. freq - average freq)	
	wint=		windowing for freqw				
			windowing for thin				
			default=1 					
 			o--------o--------o				
 			data-1	data	data+1				

 Notes:								
 This program performs complex trace attribute analysis. The first three
 attributes, amp,phase,freq are the classical Taner, Kohler, and	
 Sheriff, 1979.							

 The unwrapping algorithm is the "simple" unwrapping algorithm that	
 searches for jumps in phase.						

 The quantity dphase_min is the minimum change in the phase angle taken
 to be the result of phase wrapping, rather than natural phase	 
 variation in the data. Setting unwrap=0 turns off phase-unwrapping	
 alltogether. Choosing  unwrap > 1 makes the unwrapping function more	
 sensitive to instantaneous phase changes.				
 Setting unwrap > 1 may be necessary to resolve higher frequencies in	
 data (or sample data more finely).					

 Examples:								
 suvibro f1=10 f2=50 t1=0 t2=0 tv=1 | suattributes2 mode=amp | ...	
 suvibro f1=10 f2=50 t1=0 t2=0 tv=1 | suattributes2 mode=phase | ...	
 suvibro f1=10 f2=50 t1=0 t2=0 tv=1 | suattributes2 mode=freq | ...	
 suplane | suattributes mode=... | supswigb |...       		


 Credits:
	CWP: Jack K. Cohen
      CWP: John Stockwell (added freq and unwrap features)
	UGM (Geophysics Students): Agung Wiyono
	   email:aakanjas@gmail.com (others) added more attributes
					

 Algorithm:
	c(t) = hilbert_tranform_kernel(t) convolved with data(t)  

  amp(t) = sqrt( c.re^2(t) + c.im^2(t))
  phase(t) = arctan( c.im(t)/c.re(t))
  freq(t) = d(phase)/dt

 Reference: Taner, M. T., Koehler, A. F., and  Sheriff R. E.
 "Complex seismic trace analysis", Geophysics,  vol.44, p. 1041-1063, 1979

 Trace header fields accessed: ns, trid
 Trace header fields modified: d1, trid


\end{verbatim}
\pagebreak
\begin{verbatim}

 SUCMP   - CoMPare two seismic data sets, returns 0 to the shell	", 
             if the same and 1 if different				

  sucmp file_A file_B							

 Required parameters:							
      none								

   Optional parameters:						
      limit=1.0e-4    normalized difference threshold value		

 Notes:								
 This program is the seismic equivalent of the Unix cmp(1)		
 command.  However, unlike cmp(1), it understands seismic data		
 and will consider files which have only small numerical		
 differences to be the same.						

 Sucmp first checks that the number of traces and number of samples	
 are the same. It then compares the trace headers bit for bit.		
 Finally it checks that the fractional difference of A & B is		
 less than limit.							

 This program is intended as an aid in regression testing changes to	
 seismic processing programs.						

 Expected usage is in shell scripts or Makefiles, e.g.			
   #!/bin/sh								
    #-------------------------------------------------------		
    # Run a test data set and verify the result is correct		
    # If the data doesn't match show the data on the screen.		
   #-------------------------------------------------------		

  ./fubar par=tst1.par							
   sucmp tst1.su ref/tst1.su						
   if [ $? ]								
      then								
      suxwigb <tst1.su &						
      suxwigb <ref/tst1.su & "
   fi									


 Author:  Reginald H. Beardsley 
		rhb@acm.org

  sucmp - compare two seismic files in CWP/SU format to see if they
         are the same within the user specified limit.

  Algorithm:

  Loop over both input files comparing data values.  To be 
  considered the same files must have:

    - same number of traces
    - same number of samples per trace
    - trace values within limits of each other

 Note that the program exits as soon as the files fail to match.

 For readability, explicit temporary variables are used which the
 compiler will optimize away. Braces are used on all conditionals
 so that a breakpoint can be set to stop the debugger only if the 
 condition is true.

 Because of the overloading of trace header fields in CWP/SU, the
 headers are compared bit for bit.
\end{verbatim}
\pagebreak
\begin{verbatim}
 SUHISTOGRAM - create histogram of input amplitudes		

    suhistogram <in.su >out.dat				

 Required parameters:						
 min=		minimum bin 					
 max=		maximum bin 					
 bins=		number of bins					

 Optional parameters						
 trend=0	=0 1-D histogram				
	   =1 2-D histogram picks on cumulate			
	   =2 2-D histogram in trace format			

 clip=     threshold value to drop outliers			

 dt=	sample rate in feet or milliseconds.  Defaults  to	
    	tr.dt*1e-3					  	
 datum=  header key to get datum shift if desired (e.g. to	
	 hang from water bottom)			    	

 Notes:							
 trend=0 produces a two column ASCII output for use w/ gnuplot.
 Extreme values are counted in the end bins.			

 trend=1 produces a 6 column ASCII output for use w/ gnuplot   
 The columns are time/depth and picks on the cumulate		
 at 2.28%, 15.87%, 50%, 84.13% & 97.72% of the total points    
 corresponding to the median and +- 1 or 2 standard deviations 
 for a Gaussian distribution.					

 trend=2 produces an SU trace panel w/ one trace per bin that  
 can be displayed w/ suximage, etc.				

 Example for plotting with xgraph:				
 suhistogram < data.su min=MIN max=MAX bins=BINS |		
 a2b n1=2 | xgraph n=BINS nplot=1			 	


 Author: Reginald H. Beardsley  2006   rhb@acm.org
 


\end{verbatim}
\pagebreak
\begin{verbatim}
 SUMAX - get trace by trace local/global maxima, minima, or absolute maximum

 sumax <stdin >stdout [optional parameters] 			

 Required parameters:						
	none								

 Optional parameters: 						
	output=ascii 		write ascii data to outpar		
				=binary for binary floats to stdout	
				=segy for SEGY traces to stdout		

	mode=maxmin		output both minima and maxima		
				=max maxima only			
				=min minima only			
				=abs absolute maxima only      		
				=rms RMS 		      		
				=thd search first max above threshold	

	threshamp=0		threshold amplitude value		
	threshtime=0		tmin to start search for threshold 	

	verbose=0 		writes global quantities to outpar	
				=1 trace number, values, sample location
				=2 key1 & key2 instead of trace number  
	key1=fldr		key for verbose=2                       
	key2=ep			key for verbose=2                       

	outpar=/dev/tty		output parameter file; contains output	
					from verbose			

 Examples: 								
 For global max and min values:  sumax < segy_data			
 For local and global max and min values:  sumax < segy_data verbose=1	
 To plot values specified by mode:					
    sumax < segy_data output=binary mode=modeval | xgraph n=npairs	
 To plot seismic data with the only values nonzero being those specified
 by mode=modeval:							
    sumax < segy_data output=segy mode=modeval | suxwigb		

 Note:	while traces are counted from 1, sample values are counted from 0.
	Also, if multiple min, max, or abs max values exist on a trace,	
       only the first one is captured.					

 See also: suxmax, supsmax						

 Credits:
	CWP : John Stockwell (total rewrite)
	Geocon : Garry Perratt (all ASCII output changed from %e to %e)
	                       (added mode=rms).
      ESCI: Reginald Beardsley (added header key option)
	based on an original program by:
	SEP: Shuki Ronen
	CWP: Jack K. Cohen
      IFM-GEOMAR: Gerald Klein (added threshold option) 

 Trace header fields accessed: ns dt & user specified keys


\end{verbatim}
\pagebreak
\begin{verbatim}
 SUMEAN - get the mean values of data traces				",	

 sumean < stdin > stdout [optional parameters] 			

 Required parameters:							
   power = 2.0		mean to the power				
			(e.g. = 1.0 mean amplitude, = 2.0 mean energy)	

 Optional parameters: 							
   verbose = 0		writes mean value of section to outpar	   	
			= 1 writes mean value of each trace / section to
				outpar					
   outpar=/dev/tty   output parameter file				
   abs = 1             average absolute value 
                       = 0 preserve sign if power=1.0

 Notes:			 					
 Each sample is raised to the requested power, and the sum of all those
 values is averaged for each trace (verbose=1) and the section.	
 The values power=1.0 and power=2.0 are physical, however other powers	
 represent other mathematical L-p norms and may be of use, as well.	


 Credits:
  Bjoern E. Rommel, IKU, Petroleumsforskning / October 1997
		    bjorn.rommel@iku.sintef.no


\end{verbatim}
\pagebreak
\begin{verbatim}
 SUQUANTILE - display some quantiles or ranks of a data set            

 suquantile <stdin >stdout [optional parameters]			

 Required parameters:                                                  
       none (no-op)                                                    

 Optional parameters:                                                  
	panel=1		flag; 0 = do trace by trace (vs. whole data set)
	quantiles=1	flag; 0 = give ranks instead of quantiles	
 	verbose=0	verbose = 1 echoes information			

 	tmpdir= 	 if non-empty, use the value as a directory path
			 prefix for storing temporary files; else if the
		         the CWP_TMPDIR environment variable is set use	
		         its value for the path; else use tmpfile()	


 Credits:
      CWP: Jack K. Cohen


 Trace header fields accessed: ns, tracl, mark

\end{verbatim}
\pagebreak
\begin{verbatim}
 SUACORFRAC -- general FRACtional Auto-CORrelation/convolution		

 suacorfrac power= [optional parameters] <indata >outdata 		

 Optional parameters:							
 a=0			exponent of complex amplitude	 		
 b=0			multiplier of complex phase	 		
 dt=(from header)	time sample interval (in seconds)		
 verbose=0		=1 for advisory messages			
 ntout=tr.ns		number of time samples output			
 sym=0			if non-zero, produce a symmetric output from	
			lag -(ntout-1)/2 to lag +(ntout-1)/2		
 Notes:								
 The calculation is performed in the frequency domain.			
 The fractional autocorrelation/convolution is obtained by raising	
 Fourier coefficients to seperate real powers 				
		(a,b) for amp and phase:				
		     Aout exp[-i Pout] = Ain Ain^a exp[-i (1+b) Pin] 	
		where A=amplitude  P=phase.				
 Some special cases:							
		(a,b)=(1,1)	-->	auto-correlation		
		(a,b)=(0.5,0.5)	-->	half-auto-correlation		
		(a,b)=(0,0)	-->	no change to data		
		(a,b)=(0.5,-0.5)-->	half-auto-convolution		
		(a,b)=(1,-1)	-->	auto-convolution		


 Credits:
	UHouston: Chris Liner, Sept 2009
	CWP: Based on Hale's crpow

 Trace header fields accessed: ns, dt, trid, d1
/
\end{verbatim}
\pagebreak
\begin{verbatim}
 SUACOR - auto-correlation						

 suacor <stdin >stdout [optional parms]				

 Optional Parameters:							
 ntout=101	odd number of time samples output			
 norm=1	if non-zero, normalize maximum absolute output to 1	
 sym=1		if non-zero, produce a symmetric output from		
			lag -(ntout-1)/2 to lag +(ntout-1)/2		

 Credits:
	CWP: Dave Hale

 Trace header fields accessed:  ns
 Trace header fields modified:  ns and delrt

\end{verbatim}
\pagebreak
\begin{verbatim}
 SUCONV - convolution with user-supplied filter			

 suconv <stdin >stdout  filter= [optional parameters]			

 Required parameters: ONE of						
 sufile=		file containing SU trace to use as filter	
 filter=		user-supplied convolution filter (ascii)	

 Optional parameters:							
 panel=0		use only the first trace of sufile		
 			=1 convolve corresponding trace in sufile with	
 			trace in input data				

 Trace header fields accessed: ns					
 Trace header fields modified: ns					

 Notes: It is quietly assumed that the time sampling interval on the	
 single trace and the output traces is the same as that on the traces	
 in the input file.  The sufile may actually have more than one trace,	
 but only the first trace is used in panel=0. In panel=1 the corresponding
 trace from the sufile are convolved with its counterpart in the data.	
 Caveat, in panel=1 there have to be at least as many traces in sufile	
 as in the input data. If not, a warning is returned, and later traces	
 in the dataset are returned unchanged.				

 Examples:								
	suplane | suwind min=12 max=12 >TRACE				
	suconv<DATA sufile=TRACE | ...					
 Here, the su data file, "DATA", is convolved trace by trace with the
 the single su trace, "TRACE".					

	suconv<DATA filter=1,2,1 | ...					
 Here, the su data file, "DATA", is convolved trace by trace with the
 the filter shown.							


 Credits:
	CWP: Jack K. Cohen, Michel Dietrich

  CAVEATS: no space-variable or time-variable capacity.
     The more than one trace allowed in sufile is the
     beginning of a hook to handle the spatially variant case.

 Trace header fields accessed: ns
 Trace header fields modified: ns

\end{verbatim}
\pagebreak
\begin{verbatim}
 SUREFCON -  Convolution of user-supplied Forward and Reverse		
		refraction shots using XY trace offset in reverse shot	

	surefcon <forshot sufile=revshot  xy=trace offseted  >stdout	

 Required parameters:						 	
 sufile=	file containing SU trace to use as reverse shot		
 xy=		Number of traces offseted from the 1st trace in sufile	

 Optional parameters:						 	
 none								 	

 Trace header fields accessed: ns					
 Trace header fields modified: ns					

 Notes:								
 This code implements the Refraction Convolution Section (RCS)	method	
 of generalized reciprocal refraction traveltime analysis developed by 
 Derecke Palmer and Leoni Jones.					

 The time sampling interval on the output traces is half of that on the
 traces in the input files.		  	

 Example:								

	 surefcon <DATA sufile=DATA xy=1 | ...				

 Here, the su data file, "DATA", convolved the nth trace by		
 (n+xy)th trace in the same file					



 Credits: (based on suconv)
	CWP: Jack K. Cohen, Michel Dietrich
	UNSW: D. Palmer, K.T. LEE
  CAVEATS: no space-variable or time-variable capacity.
	The more than one trace allowed in sufile is the
	beginning of a hook to handle the spatially variant case.

 Trace header fields accessed: ns
 Trace header fields modified: ns
 Notes:
 This code implements the refraction convolution 
 section (RCS) method
 method described in:

 Palmer, D, 2001a, Imaging refractors with the convolution section,
           Geophysics 66, 1582-1589.
 Palmer, D, 2001b, Resolving refractor ambiguities with amplitudes,
           Geophysics 66, 1590-1593.

 Exploration Geophysics (2005) 36, 18�25
 Butsuri-Tansa (Vol. 58, No.1)
 Mulli-Tamsa (Vol. 8,
    A simple approach to refraction statics with the 
 Generalized Main Reciprocal Method and the Refraction 
 Convolution Section Heading
        by Derecke Palmer  Leonie Jones


\end{verbatim}
\pagebreak
\begin{verbatim}
 SUXCOR - correlation with user-supplied filter			

 suxcor <stdin >stdout  filter= [optional parameters]			

 Required parameters: ONE of						
 sufile=		file containing SU traces to use as filter	
 filter=		user-supplied correlation filter (ascii)	

 Optional parameters:							
 vibroseis=0		=nsout for correlating vibroseis data		
 first=1		supplied trace is default first element of	
 			correlation.  =0 for it to be second.		
 panel=0		use only the first trace of sufile as filter 	
 			=1 xcor trace by trace an entire gather		
 ftwin=0		first sample on the first trace of the window 	
 				(only with panel=1)		 	
 ltwin=0		first sample on the last trace of the window 	
 				(only with panel=1)		 	
 ntwin=nt		number of samples in the correlation window	
 				(only with panel=1)		 	
 ntrc=48		number of traces on a gather 			
 				(only with panel=1)		 	

 Trace header fields accessed: ns					
 Trace header fields modified: ns					

 Notes: It is quietly assumed that the time sampling interval on the	
 single trace and the output traces is the same as that on the traces	
 in the input file.  The sufile may actually have more than one trace,	
 but only the first trace is used when panel=0. When panel=1 the number
 of traces in the sufile MUST be the same as the number of traces in 	
 the input.								

 Examples:								
	suplane | suwind min=12 max=12 >TRACE				
	suxcor<DATA sufile=TRACE |...					
 Here, the su data file, "DATA", is correlated trace by trace with the
 the single su trace, "TRACE".					

	suxcor<DATA filter=1,2,1 | ...					
 Here, the su data file, "DATA", is correlated trace by trace with the
 the filter shown.							

 Correlating vibroseis data with a vibroseis sweep:			
 suxcor < data sufile=sweep vibroseis=nsout  |...			

 is equivalent to, but more efficient than:				

 suxcor < data sufile=sweep |						
 suwind itmin=nsweep itmax=nsweep+nsout | sushw key=delrt a=0.0 |...   

 sweep=vibroseis sweep in SU format, nsweep=number of samples on	
 the vibroseis sweep, nsout=desired number of samples on output	

 or									
 suxcor < data sufile=sweep |						
 suwind itmin=nsweep itmax=nsweep+nsout | sushw key=delrt a=0.0 |...   

 tsweep=sweep length in seconds, tout=desired output trace length in seconds

 In the spatially variant case (panel=1), a window with linear slope 	
 can be defined:						 	
 	ftwin is the first sample of the first trace in the gather,  	
 	ltwin is the first sample of the last trace in the gather,	
 	ntwin is the lengthe of the window, 				
 	ntrc is the the number of traces in a gather. 			

 	If the data consists of a number gathers which need to be 	
	correlated with the same number gathers in the sufile, ntrc	
	assures that the correlating window re-starts for each gather.	

	The default window is non-sloping and takes the entire trace	
	into account (ftwin=ltwin=0, ntwin=nt).				


 Credits:
	CWP: Jack K. Cohen, Michel Dietrich
      CWP: modified by Ttjan to include cross correlation of panels
	   permitting spatially and temporally varying cross correlation.
      UTK: modified by Rick Williams for vibroseis correlation option.

  CAVEATS: 
     In the option, panel=1 the number of traces in the sufile must be 
     the same as the number of traces on the input.

 Trace header fields accessed: ns
 Trace header fields modified: ns

\end{verbatim}
\pagebreak
\begin{verbatim}
 DCTCOMP - Compression by Discrete Cosine Transform			

   dctcomp < stdin n1= n2=   [optional parameter] > sdtout		

 Required Parameters:							
 n1=			number of samples in the fast (first) dimension	
 n2=			number of samples in the slow (second) dimension
 Optional Parameters:							
 blocksize1=16		blocksize in direction 1			
 blocksize2=16		blocksize in direction 2			
 error=0.01		acceptable error				



 Author:  CWP: Tong Chen   Dec 1995
          fixed by Graham Ganssle, Sandstone Oil & Gas, Sept 2015


\end{verbatim}
\pagebreak
\begin{verbatim}
 SUPACK1 - pack segy trace data into chars			

 supack1 <segy_file >packed_file	gpow=0.5 		

 Required parameters:						
	none							

 Optional parameter: 						
	gpow=0.5	exponent used to compress the dynamic	
			range of the traces			


 Credits:
	CWP: Jack K. Cohen, Shuki Ronen, Brian Sumner

 Caveats:
	This program is for single site use.  Use segywrite to make
	a portable tape.

	We are storing the local header words, ungpow and unscale,
	required by suunpack1 as floats.  Although not essential
	(compare the handling of such fields as dt), it allows us
	to demonstrate the convenience of using the natural data type.
	In any case, the data itself is non-portable floats in general,
	so we aren't giving up any intrinsic portability.
	
 Notes:
	ungpow and unscale are defined in segy.h
	trid = CHARPACK is defined in su.h and segy.h

 Trace header fields accessed: ns
 Trace header fields modified: ungpow, unscale, trid

\end{verbatim}
\pagebreak
\begin{verbatim}
 SUPACK2 - pack segy trace data into 2 byte shorts		

 supack2 <segy_file >packed_file	gpow=0.5 		

 Required parameters:						
	none							

 Optional parameter: 						
	gpow=0.5	exponent used to compress the dynamic	
			range of the traces			


 Credits:
	CWP: Jack K. Cohen, Shuki Ronen, Brian Sumner

 Revised: 7/4/95  Stewart A. Levin  Mobil
          Changed encoding to ensure 2 byte length (short is
	    8 bytes on Cray).

 Caveats:
	This program is for single site use.  Use segywrite to make
	a portable tape.

	We are storing the local header words, ungpow and unscale,
	required by suunpack2 as floats.
	
 Notes:
	ungpow and unscale are defined in segy.h
	trid = SHORTPACK is defined in su.h and segy.h

 Trace header fields accessed: ns
 Trace header fields modified: ungpow, unscale, trid

\end{verbatim}
\pagebreak
\begin{verbatim}
SUUNPACK1 - unpack segy trace data from chars to floats	

    suunpack1 <packed_file >unpacked_file			

suunpack1 is the approximate inverse of supack1		


 Credits:
	CWP: Jack K. Cohen, Shuki Ronen, Brian Sumner

 Caveats:
	This program is for single site use with supack1.  See the
	supack1 header comments.

 Notes:
	ungpow and unscale are defined in segy.h
	trid = CHARPACK is defined in su.h and segy.h


 Trace header fields accessed: ns, trid, ungpow, unscale
 Trace header fields modified:     trid, ungpow, unscale

\end{verbatim}
\pagebreak
\begin{verbatim}
 SUUNPACK2 - unpack segy trace data from shorts to floats	

    suunpack2 <packed_file >unpacked_file			

 suunpack2 is the approximate inverse of supack2		


 Credits:
	CWP: Jack K. Cohen, Shuki Ronen, Brian Sumner

 Revised:  7/4/95 Stewart A. Levin  Mobil
          Changed decoding to parallel 2 byte encoding of supack2

 Caveats:
	This program is for single site use with supack2.  See the
	supack2 header comments.

 Notes:
	ungpow and unscale are defined in segy.h
	trid = SHORTPACK is defined in su.h and segy.h

 Trace header fields accessed: ns, trid, ungpow, unscale
 Trace header fields modified:     trid, ungpow, unscale

\end{verbatim}
\pagebreak
\begin{verbatim}
 DT1TOSU - Convert ground-penetrating radar data in the	
	Sensors & Software X.DT1 GPR format to SU format.	

 dt1tosu < gpr_data_in_dt1_format  > stdout			

 Optional parameters:						
 ns=from header	number of samples per trace		
 dt=.8		time sample interval (see below)		
 swap=endian	endian is auto-determined =1 (big endian) swap	
		=0 don't swap bytes (little endian machines)	
 verbose=0	silent						
		=1 S & S header values from first trace		
			sent to outpar				
		=2 S & S header values from all traces		
			sent to outpar				
 outpar=/dev/tty	output parameter file			
 list=0	silent						
		=1 list explaining labels used in verbose	
		     is printed to stderr			

 Caution: An incorrect ns field will munge subsequent processing.

 Notes:							
 For compatiblity with SEGY header, apparent dt is set to	
 .8 ms (800 microsecs).  Actual dt is .8 nanosecs.		
 Using TRUE DISTANCES, this scales velocity			
 and frequency by a factor of 1 million.			
	Example: v_air = 9.83X10^8 ft/s	 (real)			
		 v_air = 983 ft/s	(apparent for su)	
	Example: fnyquist = 625 MHz	(real)			
		fnyquist = 625 Hz	(apparent for su)	

 IBM RS6000, NeXT, SUN are examples of big endian machines	
 PC's and DEC are examples of little endian machines		

 Caveat:							
 This program has not been tested on DEC, some modification of the
 byte swapping routines may be required.			


 Credits:
	CWP: John Stockwell, Jan 1994   Based on a code "sugpr" by
	UTULSA: Chris Liner & Bill Underwood  (Dec93)
 modifications permit S & S dt1 header information to be transferred
 directly to SU header

 March 2012: CWP John Stockwell  updated for the revised
 S&S DT1, which they still call "DT1" though it is different.

 Trace header fields set: ns, tracl, tracr, dt, delrt, trid,
			    hour, minute, second

 Reference: Sensors & Software pulseEKKO and Noggin^plus Data File
	     Formats
 Publication of:
 Sensors & Software: suburface imaging solutions
 1091 Brevik Place
 Mississauga, ON L4W 3R7 Canada
 Sensors & Software In
 Tel: (905) 624-8909
 Fax (905) 624-9365
 E-mail: sales@sensoft.ca
 Website: www.sensoft.ca

\end{verbatim}
\pagebreak
\begin{verbatim}
 SEGYCLEAN - zero out unassigned portion of header		

 segyclean <stdin >stdout 					

 Since "foreign" SEG-Y tapes may use the unassigned portion	
 of the trace headers and since SU now uses it too, this	
 program zeros out the fields meaningful to SU.		

  Example:							
  	segyread trmax=200 | segyclean | suximage		



 Credits:
	CWP: Jack Cohen


\end{verbatim}
\pagebreak
\begin{verbatim}
 SEGYHDRMOD - replace the text header on a SEGY file		

   segyhdrmod text=file data=file				

   Required parameters:					

   text=      name of file containing new 3200 byte text header
   data=      name of file containing SEGY data set		

 Notes:							
 This program simply does a replacement of the content of the first
 3200 bytes of the SEGY file with the contents of the file specified
 by the text= parameter. If the text header in the SEGY standard
 ebcdic format, the user will need to supply an ebcdic format file
 as the text=  as input file. A text file may be converted from
 ascii to ebcdic via:						
   dd if=ascii_filename of=ebcdic_filename conv=ebcdic ibs=3200 count=1
 or from ebcdic to ascii via:					
   dd if=ebcdic_filename of=ascii_filename ibs=3200 conv=ascii count=1



====================================================================*\

   sgyhdrmod - replace the text header on a SEGY data file in place

   This program only reads and writes 3200 bytes

   Reginald H. Beardsley                            rhb@acm.org

\*====================================================================*/
\end{verbatim}
\pagebreak
\begin{verbatim}
 SEGYHDRS - make SEG-Y ascii and binary headers for segywrite		

 segyhdrs [ < sudata ] [optional parameters] [ > copy of sudata ]      

 Required parameters:							
	ns=  if no input trace header					
	dt=  if no input trace header					
 Optional parameters:							
 	ns=tr.ns from header    number of samples on input traces	
 	dt=tr.dt from header	sample rate (microseconds) from traces	
 	bfile=binary		name of file containing binary block	
 	hfile=header		name of file containing ascii block	
   Some binary header fields are set:					
 	jobid=1			job id field				
 	lino=1			line number (only one line per reel)	
 	reno=1			reel number				
 	format=1		data format				

 All other fields are set to 0, by default.				
 To set any binary header field, use sukeyword to find out		
 the appropriate keyword, then use the getpar form:			
 	keyword=value	to set keyword to value				

 The header file is created as ascii and is translated to ebcdic	
 by segywrite before being written to tape.  Its contents are		
 formal but can be edited after creation as long as the forty		
 line format is maintained.						

 Caveat: This program has not been tested under XDR for machines       
	 not having a 2 byte unsigned short integral data type.	


 Credits:

	CWP: Jack K. Cohen,  John Stockwell 
      MOBIL: Stew Levin

\end{verbatim}
\pagebreak
\begin{verbatim}
 SEGYREAD - read an SEG-Y tape						

   segyread > stdout tape=						

   or									

   SEG-Y data stream ... | segyread tape=-  > stdout			

 Required parameter:							
 tape=		input tape device or seg-y filename (see notes)		

 Optional parameters:							
 buff=1	for buffered device (9-track reel tape drive)		
		=0 possibly useful for 8mm EXABYTE drives		
 verbose=0	silent operation					
		=1 ; echo every 'vblock' traces				
 vblock=50	echo every 'vblock' traces under verbose option		
 hfile=header	file to store ebcdic block (as ascii)			
 bfile=binary	file to store binary block				
 xfile=xhdrs	file to store extended text block			
 over=0	quit if bhed format not equal 1, 2, 3, 5, or 8		
		= 1 ; override and attempt conversion			
 format=bh.format	if over=1 try to convert assuming format value  
 conv=1	convert data to native format				
			= 0 ; assume data is in native format		
 ebcdic=1	perform ebcdic to ascii conversion on 3200 byte textural
               header. =0 do not perform conversion			
 ns=bh.hns	number of samples (use if bhed ns wrong)		
 trcwt=1	apply trace weighting factor (bytes 169-170)		
		=0, do not apply.  (Default is 0 for formats 1 and 5)	
 trmin=1		first trace to read				
 trmax=INT_MAX	last trace to read					
 endian=(autodetected) =1 for big-endian,  =0 for little-endian byte order
 swapbhed=endian	swap binary reel header?			
 swaphdrs=endian	swap trace headers?				
 swapdata=endian	swap data?					
 errmax=0	allowable number of consecutive tape IO errors		
 remap=...,...	remap key(s) 						
 byte=...,...	formats to use for header remapping 			

 Notes:								
 Traditionally tape=/dev/rmt0.	 However, in the modern world tape device
 names are much less uniform.  The magic name can often be deduced by	
 "ls /dev".  Likely man pages with the names of the tape devices are:
 "mt", "sd" "st".  Also try "man -k scsi", " man mt", etc.	
 Sometimes "mt status" will tell the device name.			

 For a SEG-Y diskfile use tape=filename.				

 The xfile argument will only be used if the file contains extended	
 text headers.								

 Remark: a SEG-Y file is not the same as an su file. A SEG-Y file	
 consists of three parts: an ebcdic header, a binary reel header, and	
 the traces.  The traces are (usually) in 32 bit IBM floating point	
 format.  An SU file consists only of the trace portion written in the 
 native binary floats.							

 Formats supported:							
 1: IBM floating point, 4 byte (32 bits)				
 2: two's complement integer, 4 byte (32 bits)				
 3: two's complement integer, 2 byte (16 bits)				
 5: IEEE floating point, 4 byte (32 bits)				
 8: two's complement integer, 1 byte (8 bits)				

 tape=-   read from standard input. Caveat, under Solaris, you will	
 need to use the buff=1 option, as well.				

 Header remap:								
 The value of header word remap is mapped from the values of byte	

 Map a float at location 221 to sample spacing d1:			
	segyread <data >outdata remap=d1 byte=221f			

 Map a long at location 225 to source location sx:			
	segyread <data >outdata remap=sx byte=225l			

 Map a short at location 229 to gain constant igc:			
	segyread <data >outdata remap=igc byte=229s			

 Or all combined: 							
	segyread <data >outdata remap=d1,sx,igc byte=221f,225l,229s	

 Segy header words are accessed as Xt where X denotes the byte number	
 starting at 1 in correspondance with the SEGY standard (1975)		
 Known types include:	f	float (4 bytes)				
 			l	long int (4 bytes)			
 			s	short int (2 bytes)			
 			b	byte (1 bytes)				

	  type:	  sudoc segyread   for further information		



 Note:
      If you have a tape with multiple sequences of ebcdic header,
	binary header,traces, use the device that
	invokes the no-rewind option and issue multiple segyread
	commands (making an appropriate shell script if you
	want to save all the headers).	Consider using >> if
	you want a single trace file in the end.  Similar
	considerations apply for multiple reels of tapes,
	but use the standard rewind on end of file.

 Note: For buff=1 (default) tape is accessed with 'read', for buff=0
	tape is accessed with fread. We suggest that you try buff=1
	even with EXABYTE tapes.
 Caveat: may be slow on an 8mm streaming (EXABYTE) tapedrive
 Warning: segyread or segywrite to 8mm tape is fragile. Allow sufficient
	time between successive reads and writes.
 Warning: may return the error message "efclose: fclose failed"
	intermittently when segyreading/segywriting to 8mm (EXABYTE) tape
	even if actual segyread/segywrite is successful. However, this
	error message may be returned if your tape drive has a fixed
	block size set.
 Caution: When reading or writing SEG-Y tapes, the tape
	drive should be set to be able to read variable block length
	tape files.


 Credits:
	SEP: Einar Kjartansson
	CWP: Jack K. Cohen, Brian Sumner, Chris Liner
	   : John Stockwell (added 8mm tape stuff)
 conv parameter added by:
	Tony Kocurko
	Department of Earth Sciences
	Memorial University of Newfoundland
	St. John's, Newfoundland
 read from stdin via tape=-  added by	Tony Kocurko
 bhed format = 2,3 conversion by:
	Remco Romijn (Applied Geophysics, TU Delft)
	J.W. de Bruijn (Applied Geophysics, TU Delft)
 bhed format = 8 conversion by: John Stockwell
 header remap feature added by:
 	Matthias Imhof, Virginia Tech
--------------------------
 Additional Notes:
	Brian's subroutine, ibm_to_float, which converts IBM floating
	point to IEEE floating point is NOT portable and must be
	altered for non-IEEE machines.	See the subroutine notes below.

	A direct read by dd would suck up the entire tape; hence the
	dancing around with buffers and files.


\end{verbatim}
\pagebreak
\begin{verbatim}
 SEGYSCAN -- SCANs SEGY file trace headers for min-max in  several	
     possible formats.							

   segyscan < segyfile							

 Notes:								
 The SEGY file trace headers are scanned assuming short, ushort, int,  
 uint, float, and double and the results are printed as tables.	



 Credits: Stew Levin, June 2013 

\end{verbatim}
\pagebreak
\begin{verbatim}
 SEGYWRITE - write an SEG-Y tape					

 segywrite <stdin tape=						

 Required parameters:							
	tape=		tape device to use (see sudoc segyread)		

 Optional parameter:							
 verbose=0	silent operation				
		=1 ; echo every 'vblock' traces			
 vblock=50	echo every 'vblock' traces under verbose option 
 buff=1		for buffered device (9-track reel tape drive)	
		=0 possibly useful for 8mm EXABYTE drive	
 conv=1		=0 don't convert to IBM format			
 ebcdic=1	convert text header to ebcdic, =0 leave as ascii	
 hfile=header	ebcdic card image header file			
 bfile=binary	binary header file				
 trmin=1 first trace to write					
 trmax=INT_MAX  last trace to write			       
 endian=(autodetected)	=1 for big-endian and =0 for little-endian byte order
 errmax=0	allowable number of consecutive tape IO errors	
 format=		override value of format in binary header file	

 Note: The header files may be created with  'segyhdrs'.		


 Note: For buff=1 (default) tape is accessed with 'write', for buff=0	
	tape is accessed with fwrite. Try the default setting of buff=1 
	for all tape types.						
 Caveat: may be slow on an 8mm streaming (EXABYTE) tapedrive		
 Warning: segyread or segywrite to 8mm tape is fragile. Allow time	
	   between successive reads and writes.				
 Precaution: make sure tapedrive is set to read/write variable blocksize
	   tapefiles.							

 For more information, type:	sudoc <segywrite>			



 Warning: may return the error message "efclose: fclose failed"
	 intermittently when segyreading/segywriting to 8mm EXABYTE tape,
	 even if actual segyread/segywrite is successful. However, this
	 may indicate that your tape drive has been set to a fixed block
	 size. Tape drives should be set to variable block size before reading
	 or writing tapes in the SEG-Y format.

 Credits:
	SEP: Einar Kjartansson
	CWP: Jack, Brian, Chris
	   : John Stockwell (added EXABYTE functionality)
 Notes:
	Brian's subroutine, float_to_ibm, for converting IEEE floating
	point to IBM floating point is NOT portable and must be
	altered for non-IEEE machines.	See the subroutine notes below.

	On machines where shorts are not 2 bytes and/or ints are not 
	4 bytes, routines to convert SEGY 16 bit and 32 bit integers 
	will be required.

	The program, segyhdrs, can be used to make the ascii and binary
	files required by this code.


\end{verbatim}
\pagebreak
\begin{verbatim}
 SETBHED - SET the fields in a SEGY Binary tape HEaDer file, as would be
 	    produced by segyread and segyhdrs				

 setbhed par= [optional parameters]					

 Required parameter:							
 	none								
 Optional parameters:							
	bfile=binary		output binary tape header file		
	par=			=parfile				
 Set field by field, if desired:					
 	jobid=			job id field				
 	lino=			line number (only one line per reel)	
 	reno=			reel number				
 	format=			data format				
 ... etc....								
 To set any binary header field, use sukeyword to find out		
 the appropriate keyword, then use the getpar form:			
 	keyword=value	to set keyword to value				
 Notes:								
 As with all other programs in the CWP/SU package that use getpars, 	
 (GET PARameters from the command line) a file filled with such	
 statments may be included via option par=parfile. In particular, a	
 parfile created by   "bhedtopar"  may be used as input for the program
 "setbhed".								

 The binary header file that results from running segyread may have the
 wrong byte order. You will need to use "swapbhed" to change the byte,"
 order before applying this program. 					

 Example:								
   segyread tape=yourdata.segy bfile=yourdata.b > yourdata.su		
 If  									
   bhedtopar < yourdata.b | more 					
 shows impossible values, then apply 					
   swapbhed < yourdata.b > swapped.b					
 then apply 								
   bhedtopar < swapped.b | more 					
   bhedtopar < swapped.b outpar=parfile				
 hand edit parfile, and then apply 					
  setbhed par=parfile bfile=swapped.b > new.b				
 then apply 								
   segywrite tape=fixeddata.segy bfile=new.b < yourdata.su		

 Caveat: This program breaks if a "short" isn't 2 bytes since	
         the SEG-Y standard demands a 2 byte integer for ns.		

 Credits:

	CWP: John Stockwell  11 Nov 1994

\end{verbatim}
\pagebreak
\begin{verbatim}
 SUASCII - print non zero header values and data in various formats    

 suascii <stdin >ascii_file                                            

 Optional parameter:                                                   
    bare=0     print headers and data                                  
        =1     print only data                                         
        =2     print headers only                                      
        =3     print data in print data in .csv format, e.g. for Excel 
        =4     print data as tab delimited .txt file, e.g. for GnuPlot 
        =5     print data as .xyz file, e.g. for plotting with GMT     

    ntr=50     maximum number of output traces (bare=3 or bare=4 only) 
    index=0    don't include time/depth index in ascii file (bare=4)   
         =1    include time/depth index in ascii file                  

    key=       if set, name of keyword containing x-value              
               in .xyz output (bare=5 only)                            
    sep=       if set, string separating traces in .xyz output         
               (bare=5; default is no separation)                      

    verbose=0  =1 for detailed information                             

 Notes:                                                                
    The programs suwind and suresamp provide trace selection and       
    subsampling, respectively.                                         
    With bare=0 and bare=1 traces are separated by a blank line.       

    With bare=3 a maximum of ntr traces are output in .csv format      
    ("comma-separated value"), e.g. for import into spreadsheet      
    applications like Excel.                                           

    With bare=4 a maximum of ntr traces are output in as tab delimited 
    columns. Use bare=4 for plotting in GnuPlot.                       

    With bare=5 traces are written as "x y z" triples as required    
    by certain plotting programs such as the Generic Mapping Tools     
    (GMT). If sep= is set, traces are separated by a line containing   
    the string provided, e.g. sep=">" for GMT multisegment files.    

    "option=" is an acceptable alias for "bare=".                  

 Related programs: sugethw, sudumptrace                                


 Credits:
    CWP: Jack K. Cohen  c. 1989
    CENPET: Werner M. Heigl 2006 - bug fixes & extensions
    RISSC:  Nils Maercklin 2006

 Trace header field accessed: ns, dt, delrt, d1, f1, trid

\end{verbatim}
\pagebreak
\begin{verbatim}
 SUINTVEL - convert stacking velocity model to interval velocity model	

 suintvel vs= t0= outpar=/dev/tty					

 Required parameters:					        	
	vs=	stacking velocities 					
	t0=	normal incidence times		 			

 Optional parameters:							
	mode=0			output h= v= ; =1 output v=  t= 	
	outpar=/dev/tty		output parameter file in the form:	
				h=layer thicknesses vector		
				v=interval velocities vector		
				....or ...				
				t=vector of times from t0		
				v=interval velocities vector		

 Examples:								
    suintvel vs=5000,5523,6339,7264 t0=.4,.8,1.125,1.425 outpar=intpar	

    suintvel par=stkpar outpar=intpar					

 If the file, stkpar, contains:					
    vs=5000,5523,6339,7264						
    t0=.4,.8,1.125,1.425						
 then the two examples are equivalent.					

 Note: suintvel does not have standard su syntax since it does not	
      operate on seismic data.  Hence stdin and stdout are not used.	

 Note: may go away in favor of par program, velconv, by Dave		


 Credits:
	CWP: Jack 

 Technical Reference:
	The Common Depth Point Stack
	William A. Schneider
	Proc. IEEE, v. 72, n. 10, p. 1238-1254
	1984

 Formulas:
    	Note: All sums on i are from 1 to k

	From Schneider:
	Let h[i] be the ith layer thickness measured at the cmp and
	v[i] the ith interval velocity.
	Set:
		t[i] = h[i]/v[i]
	Define:
		t0by2[k] = 0.5 * t0[k] = Sum h[i]/v[i]
		vh[k] = vs[k]*vs[k]*t0by2[k] = Sum v[i]*h[i]
	Then:
		dt[i] = h[i]/v[i] = t0by2[i] - t0by2[i-1]
		dvh[i] = h[i]*v[i] = vh[i] - vh[i-1]
		h[i] = sqrt(dvh[i] * dt[i])
		v[i] = sqrt(dvh[i] / dt[i])



\end{verbatim}
\pagebreak
\begin{verbatim}
 SUOLDTONEW - convert existing su data to xdr format		

 suoldtonew <oldsu >newsu  					

 Required parameters:						
	none							

 Optional parameters:						
	none							

 Notes:							
 This program is used to convert native machine datasets to	
 xdr-based, system-independent format.				



 Author: Stewart A. Levin, Mobil, 1966
  

\end{verbatim}
\pagebreak
\begin{verbatim}
 SUSTKVEL - convert constant dip layer interval velocity model to the	
	   stacking velocity model required by sunmo			

 sustkvel v= h= dip=0.0 outpar=/dev/tty				

 Required parameters:					        	
	v=	interval velocities 					
	h=	layer thicknesses at the cmp	 			

 Optional parameters:							
	dip=0.0			(constant) dip of the layers (degrees)	
	outpar=/dev/tty		output parameter file in the form	
				required by sunmo:			
				tv=zero incidence time pick vector	
				v=stacking velocities vector		

 Examples:								
    sustkvel v=5000,6000,8000,10000 h=1000,1200,1300,1500 outpar=stkpar
    sunmo <data.cdp par=stkpar >data.nmo				

    sustkvel par=intpar outpar=stkpar					
    sunmo <data.cdp par=stkpar >data.nmo				

 If the file, intpar, contains:					
    v=5000,6000,8000,10000						
    h=1000,1200,1300,1500						
 then the two examples are equivalent.  The created parameter file,	
 stkpar, is in the form of the velocity model required by sunmo.	

 Note: sustkvel does not have standard su syntax since it does not	
      operate on seismic data.  Hence stdin and stdout are not used.	

 Caveat: Does not accept a series of interval velocity models to	
	produce a variable velocity file for sunmo.			


 Credits:
	CWP: Jack 

 Technical Reference:
	The Common Depth Point Stack
	William A. Schneider
	Proc. IEEE, v. 72, n. 10, p. 1238-1254
	1984

 Formulas:
    	Note: All sums on i are from 1 to k

	From Schneider:
	Let h[i] be the ith layer thickness measured at the cmp and
	v[i] the ith interval velocity.
	Set:
		t[i] = h[i]/v[i]
		t0[k] = 2 Sum t[i] * cos(dip)
		vs[k] = (1.0/cos(dip)) sqrt(Sum v[i]*v[i]*t[i] / Sum t[i])
	Define:
		t0by2[k] = Sum h[i]/v[i]
		vh[k]    = Sum v[i]*h[i]
	Then:
		t0[k] = 2 * t0by2[k] * cos(dip)
		vs[k] = sqrt(vh[k] / t0by2[k]) / cos(dip)



\end{verbatim}
\pagebreak
\begin{verbatim}
 SUSWAPBYTES - SWAP the BYTES in SU data to convert data from big endian
               to little endian byte order, and vice versa		

 suswapbytes < stdin [optional parameter] > sdtout			

 	format=0		foreign to native			
 				=1 native to foreign			
	swaphdr=1		swap the header byte order		
 				=0 do not change the header byte order	
	swapdata=1		swap the data byte order		
 				=0 do not change the data byte order	
 	ns=from header		if ns not set in header, must be set by hand
 Notes:								
  The 'native'	endian is the endian (byte order) of the machine you are
  running this program on. The 'foreign' endian is the opposite byte order.

 Examples of big endian machines are: IBM RS6000, SUN, NeXT		
 Examples of little endian machines are: PCs, DEC			

 Caveat: this code has not been tested on DEC				


 Credits: 
	CWP: adapted for SU by John Stockwell 
		based on a code supplied by:
	Institute fur Geophysik, Hamburg: Jens Hartmann (June 1993)

 Trace header fields accessed: ns

\end{verbatim}
\pagebreak
\begin{verbatim}
 SWAPBHED - SWAP the BYTES in a SEGY Binary tape HEaDer file		

 swapbhed < binary_in > binary out					

 Required parameter:							
 	none								
 Optional parameters:							
	none 								


 Credits:

	CWP: John Stockwell  13 May 2011

\end{verbatim}
\pagebreak
\begin{verbatim}
 SUDATUMFD - 2D zero-offset Finite Difference acoustic wave-equation	
		 DATUMing    						

 sudatumfd <stdin > stdout [optional parameters]			

 Required parameters:						   	

 nt=	   number of time samples on each trace	       			
 nx=	   number of receivers per shot gather				
 nsx=	  number of shot gathers				    	
 nz=	   number of downward continuation depth steps			
 dz=	   depth sampling interval (in meters)				
 mx=	   number of horizontal samples in the velocity model		
 mz=	   number of vertical samples in the velocity model		
 vfile1=       velocity file used for thin-lens term	    		
 vfile2=       velocity file used for diffraction term			
 dx=           horizontal sampling interval (in meters)                

 Optional parameters:						   	

 dt=.004       time sampling interval (in seconds)			
 buff=5	number of zero traces added to each side of each   	
	     shot gather as a pad			       		
 tap_len=5     taper length (in number of traces)			
 x_0=0.0       x coordinate of leftmost position in velocity model     

 Notes:								
 The algorithm is a 45-degree implicit-finite-difference scheme based  
 on the one-way wave equation.  It works on poststack (zero-offset)    
 data only.  The two velocity files, vfile1 and vfile2, are binary     
 files containing floats with the format v[ix][iz].  There are two     
 potentially different velocity files for the thin-lens and            
 diffraction terms to allow for the use of a zero-velocity layer       
 which allows for datuming from an irregular surface.                  

 Source and receiver locations must be set in the header values in     
 order for the datuming to work properly.  The leftmost position of    
 of the velocity models given in vfile1 and vfile2 must also be given. 



 
 Author:  Chris Robinson, 10/16/00, CWP, Colorado School of Mines


 References:
  Beasley, C., and Lynn, W., 1992, The zero-velocity layer: migration
    from irregular surfaces: Geophysics, 57, 1435-1443.

  Claerbout, J. F., 1985, Imaging the earth's interior:  Blackwell
    Scientific Publications.



\end{verbatim}
\pagebreak
\begin{verbatim}
SUDATUMK2DR - Kirchhoff datuming of receivers for 2D prestack data	
		(shot gathers are the input)				

    sudatumk2dr  infile=  outfile=  [parameters] 			

 Required parameters:							
 infile=stdin		file for input seismic traces			
 outfile=stdout	file for common offset migration output  	
 ttfile=		file for input traveltime tables		
   The following 9 parameters describe traveltime tables:		
 fzt= 			first depth sample in traveltime table		
 nzt= 			number of depth samples in traveltime table	
 dzt=			depth interval in traveltime table		
 fxt=			first lateral sample in traveltime table	
 nxt=			number of lateral samples in traveltime table	
 dxt=			lateral interval in traveltime table		
 fs= 			x-coordinate of first source			
 ns= 			number of sources				
 ds= 			x-coordinate increment of sources		

 fxi=                  x-coordinate of the first surface location      
 dxi=                  horizontal spacing on surface                   
 nxi=                  number of input surface locations               
 sgn=			Sign of the datuming process (up=-1 or down=1)  

 Optional Parameters:							
 dt= or from header (dt) 	time sampling interval of input data	
 ft= or from header (ft) 	first time sample of input data		
 surf="0,0;99999,0"  The first surface defined the recording surface 
 surf="0,0;99999,0"  and the second one, the new datum.              
                       "x1,z1;x2,z2;x3,z3;...
 fzo=fzt		z-coordinate of first point in output trace 	
 dzo=0.2*dzt		vertical spacing of output trace 		
 nzo=5*(nzt-1)+1 	number of points in output trace		",	
 fxso=fxt		x-coordinate of first shot	 		
 dxso=0.5*dxt		shot horizontal spacing		 		
 nxso=2*(nxt-1)+1  	number of shots 				
 fxgo=fxt		x-coordinate of first receiver			
 dxgo=0.5*dxt		receiver horizontal spacing			
 nxgo=nxso		number of receivers per shot			
 fmax=0.25/dt		frequency-highcut for input traces		
 offmax=99999		maximum absolute offset allowed in migration 	
 aperx=nxt*dxt/2  	migration lateral aperature 			
 angmax=60		migration angle aperature from vertical 	
 v0=1500(m/s)		reference velocity value at surface		
 dvz=0.0  		reference velocity vertical gradient		
 antiali=1             Antialiase filter (no-filter = 0)               
 jpfile=stderr		job print file name 				
 mtr=100  		print verbal information at every mtr traces	
 ntr=100000		maximum number of input traces to be migrated	

 verbose=0		silent, =1 chatty				

 Notes:								
 1. Traveltime tables were generated by program rayt2d (or other ones)	
    on relatively coarse grids, with dimension ns*nxt*nzt. In the	
    datuming process, traveltimes are interpolated into shot/gephone 	
    positions and output grids.					
 2. Input traces must be SU format and organized in common rec. gathers
 3. If the offset value of an input trace is not in the offset array 	
    of output, the nearest one in the array is chosen. 		
 4. Amplitudes are computed using the reference velocity profile, v(z),
    specified by the parameters v0= and dvz=.				
 5. Input traces must specify source and receiver positions via the header
    fields tr.sx and tr.gx. Offset is computed automatically.		


 Author:  Trino Salinas, 05/01/96,  Colorado School of Mines

 This code is based on sukzmig2d.c written by Zhenyue Liu, 03/01/95.
 Subroutines from Dave Hale's modeling library were adapted in
 this code to define topography using cubic splines.

 This code implements a Kirchhoff extraplolation operator that allows to
 transfer data from one reference surface to another.  The formula used in
 this application is a far field approximation of the Berryhill's original
 formula (Berryhill, 1979).  This equation is the result of a stationary
 phase analysis to get an analog asymptotic expansion for the two-and-one
 half dimensional extrapolation formula (Bleistein, 1984).

 The extrapolation formula permits the downward continuation of upgoing
 waves  and  upward  continuation  of  downgoing waves.  For upward conti-
 nuation of upgoing waves and downward continuation of downgoing waves,
 the conjugate transpose of the equation is used (Bevc, 1993).

 References :

 Berryhill, J.R., 1979, Wave equation datuming: Geophysics,
   44, 1329--1344.

 _______________, 1984, Wave equation datuming before stack
   (short note) : Geophysics, 49, 2064--2067.

 Bevc, D., 1993, Data parallel wave equation datuming with
   irregular acquisition topography :  63rd Ann. Internat.
   Mtg., SEG, Expanded Abstracts, 197--200.

 Bleistein, N., 1984, Mathematical methods for wave phenomena,
   Academic Press Inc. (Harcourt Brace Jovanovich Publishers),
   New York.

\end{verbatim}
\pagebreak
\begin{verbatim}
SUDATUMK2DS - Kirchhoff datuming of sources for 2D prestack data	
 		(input data are receiver gathers) 			

    sudatumk2ds  infile=  outfile=  [parameters] 			

 Required parameters:							
 infile=stdin		file for input seismic traces			
 outfile=stdout	file for common offset migration output  	
 ttfile=		file for input traveltime tables		
   The following 9 parameters describe traveltime tables:		
 fzt=			first depth sample in traveltime table		
 nzt= 			number of depth samples in traveltime table	
 dzt=			depth interval in traveltime table		
 fxt=			first lateral sample in traveltime table	
 nxt=			number of lateral samples in traveltime table	
 dxt=			lateral interval in traveltime table		
 fs= 			x-coordinate of first source			
 ns= 			number of sources				
 ds= 			x-coordinate increment of sources		

 fxi=                   x-coordinate of the first surface location      
 dxi=                   horizontal spacing on surface                   
 nxi=                   number of input surface locations               
 sgn=                   Sign of the datuming process (up=-1 or down=1)  

 Optional Parameters:							
 dt= or from header (dt) 	time sampling interval of input data	
 ft= or from header (ft) 	first time sample of input data		
 surf="0,0;99999,0"  The first surface defined the recording surface 
 surf="0,0;99999,0"  and the second one, the new datum.              
                       "x1,z1;x2,z2;x3,z3;...
 fzo=fzt		z-coordinate of first point in output trace 	
 dzo=0.2*dzt		vertical spacing of output trace 		
 nzo=5*(nzt-1)+1 	number of points in output trace		",	
 fxso=fxt		x-coordinate of first shot	 		
 dxso=0.5*dxt		shot horizontal spacing		 		
 nxso=2*(nxt-1)+1  	number of shots 				
 fxgo=fxt		x-coordinate of first receiver			
 exgo=fxgo+(nxgo-1)*dxgo	x-coordinate of the last receiver	
 dxgo=0.5*dxt		receiver horizontal spacing			
 nxgo=nxso		number of receivers per shot			
 fmax=0.25/dt		frequency-highcut for input traces		
 offmax=99999		maximum absolute offset allowed in migration 	
 aperx=nxt*dxt/2  	migration lateral aperature 			
 angmax=60		migration angle aperature from vertical 	
 v0=1500(m/s)		reference velocity value at surface		
 dvz=0.0  		reference velocity vertical gradient		
 antiali=1             Antialiase filter (no-filter = 0)               
 jpfile=stderr		job print file name 				
 mtr=100  		print verbal information at every mtr traces	
 ntr=100000		maximum number of input traces to be migrated	

 Notes:								
 1. Traveltime tables were generated by program rayt2d (or other ones)	
    on relatively coarse grids, with dimension ns*nxt*nzt. In the	
    migration process, traveltimes are interpolated into shot/gephone 	
    positions and output grids.					
 2. Input traces must be SU format and organized in common shot gathers
 3. If the offset value of an input trace is not in the offset array 	
    of output, the nearest one in the array is chosen. 		
 4. Amplitudes are computed using the reference velocity profile, v(z),
    specified by the parameters v0= and dvz=.				
 5. Input traces must specify source and receiver positions via the header
    fields tr.sx and tr.gx. Offset is computed automatically.		


 Author:  Trino Salinas, 05/01/96,  Colorado School of Mines

 This code is based on sukzmig2d.c written by Zhenyue Liu, 03/01/95.
 Subroutines from Dave Hale's modeling library were adapted in
 this code to define topography using cubic splines.

 This code implements a Kirchhoff extraplolation operator that allows to
 transfer data from one reference surface to another.  The formula used in
 this application is a far field approximation of the Berryhill's original
 formula (Berryhill, 1979).  This equation is the result of a stationary
 phase analysis to get an analog asymptotic expansion for the two-and-one
 half dimensional extrapolation formula (Bleistein, 1984).

 The extrapolation formula permits the downward continuation of upgoing
 waves  and  upward  continuation  of  downgoing waves.  For upward conti-
 nuation of upgoing waves and downward continuation of downgoing waves,
 the conjugate transpose of the equation is used (Bevc, 1993).

 References :

 Berryhill, J.R., 1979, Wave equation datuming: Geophysics,
   44, 1329--1344.

 _______________, 1984, Wave equation datuming before stack
   (short note) : Geophysics, 49, 2064--2067.

 Bevc, D., 1993, Data parallel wave equation datuming with
   irregular acquisition topography :  63rd Ann. Internat.
   Mtg., SEG, Expanded Abstracts, 197--200.

 Bleistein, N., 1984, Mathematical methods for wave phenomena,
   Academic Press Inc. (Harcourt Brace Jovanovich Publishers),
   New York.

\end{verbatim}
\pagebreak
\begin{verbatim}
  SUKDMDCR - 2.5D datuming of receivers for prestack, common source    
            data using constant-background data mapping formula.       
            (See selfdoc for specific survey requirements.)            

    sukdmdcr  infile=  outfile=  [parameters] 	         		

 Required file parameters:						
 infile=stdin		file for input seismic traces			
 outfile=stdout	file for output          			
 ttfile		file for input traveltime tables		

 Required parameters describing the traveltime tables:		        
 fzt 			first depth sample in traveltime table		
 nzt 			number of depth samples in traveltime table	
 dzt			depth interval in traveltime table		
 fxt			first lateral sample in traveltime table	
 nxt			number of lateral samples in traveltime table	
 dxt			lateral interval in traveltime table		
 fs 			x-coordinate of first source in table		
 ns 			number of sources in table			
 ds 			x-coordinate increment of sources in table	

 Parameters describing the input data:                                 
 nxso                  number of shots                                 
 dxso                  shot interval                                   
 fxso=0                x-coordinate of first shot                      
 nxgo                  number of receiver offsets per shot             
 dxgo                  receiver offset interval                        
 fxgo=0                first receiver offset                           
 dt= or from header (dt)       time sampling interval of input data    
 ft= or from header (ft)       first time sample of input data         

 Parameters describing the domain of the problem:             		
 dzo=0.2*dzt		vertical spacing in surface determination       
 offmax=99999		maximum absolute offset allowed          	

 Parameters describing the recording and datuming surfaces:            
 recsurf=0             recording surface (horizontal=0, topographic=1) 
 zrec                  defines recording surface when recsurf=0        
 recfile=              defines recording surface when recsurf=1        
 datsurf=0             datuming surface (horizontal=0, irregular=1)    
 zdat                  defines datuming surface when datsurf=0         
 datfile=              defines datuming surface when datsurf=1         

 Optional parameters describing the extrapolation:                     
 aperx=nxt*dxt/2  	lateral half-aperture 				
 v0=1500(m/s)		reference wavespeed               		
 freq=50               dominant frequency in data, used to determine   
                       the minimum distance below the datum that       
                       the stationary phase calculation is valid.      
 scale=1.0             user defined scale factor for output            
 jpfile=stderr		job print file name 				
 mtr=100  		print verbal information at every mtr traces	
 ntr=100000		maximum number of input traces to be datumed	



 Computational Notes:                                                
   
 1. Input traces must be SU format and organized in common shot gathers.

 2. Traveltime tables were generated by program rayt2d (or equivalent)     
    on any grid, with dimension ns*nxt*nzt. In the extrapolation process,       
    traveltimes are interpolated into shot/geophone locations and     
    output grids.                                          

 3. If the offset value of an input trace is not in the offset array     
    of output, the nearest one in the array is chosen.                   

 4. Amplitudes are computed using the constant reference wavepeed v0.
                                
 5. Input traces must specify source and receiver positions via the header  
    fields tr.sx and tr.gx.             

 6. Recording and datuming surfaces are defined as follows:  If recording
    surface is horizontal, set recsurf=0 (default).  Then, zrec will be a
    required parameter, set to the depth of surface.  If the recording  
    surface is topographic, then set recsurf=1.  Then, recfile is a required
    input file.  The input file recfile should be a single column ascii file
    with the depth of the recording surface at every surface location (first 
    source to last offset), with spacing equal to dxgo. 
 
    The same holds for the datuming surface, using datsurf, zdat, and datfile.


 Assumptions and limitations:

 1. This code implements a 2.5D extraplolation operator that allows to
    transfer data from one reference surface to another.  The formula used in
    this application is an adaptation of Bleistein & Jaramillo's 2.5D data
    mapping formula for receiver extrapolation.  This is the result of a
    stationary phase analysis of the data mapping equation in the case of
    a constant source location (shot gather). 
 

 Credits:
 
 Authors:  Steven D. Sheaffer (CWP), 11/97 


 References:  Sheaffer, S., 1999, "2.5D Downward Continuation of the Seismic
              Wavefield Using Kirchhoff Data Mapping."  M.Sc. Thesis, 
              Dept. of Mathematical & Computer Sciences, 
              Colorado School of Mines.



\end{verbatim}
\pagebreak
\begin{verbatim}
  SUKDMDCS - 2.5D datuming of sources for prestack common receiver 	
 	     data, using constant-background data-mapping formula.      
            (See selfdoc for specific survey geometry requirements.)   

    sukdmdcs  infile=  outfile=  [parameters] 		         	

 Required parameters:							
 infile=stdin		file for input seismic traces			
 outfile=stdout	file for output  	                        
 ttfile		file for input traveltime tables		

 Required parameters describing the traveltime tables:	         	
 fzt 			first depth sample in traveltime table		
 nzt 			number of depth samples in traveltime table	
 dzt			depth interval in traveltime table		
 fxt			first lateral sample in traveltime table	
 nxt			number of lateral samples in traveltime table	
 dxt			lateral interval in traveltime table		
 fs 			x-coordinate of first source in table		
 ns 			number of sources in table			
 ds 			x-coordinate increment of sources in table	

 Parameters describing the input data:                                 
 nxso                  number of shots                                 
 dxso                  shot interval                                   
 fxso=0                x-coordinate of first shot                      
 nxgo                  number of receiver offsets per shot             
 dxgo                  receiver offset interval                        
 fxgo=0                first receiver offset                           
 dt= or from header (dt)       time sampling interval of input data    
 ft= or from header (ft)       first time sample of input data         
 dc=0                  flag for previously datumed receivers:          
                          dc=0 receivers on recording surface          
                          dc=1 receivers on datum                      ", 

 Parameters descrbing the domain of the problem:	                
 dzo=0.2*dzt		vertical spacing in surface determination	
 offmax=99999		maximum absolute offset allowed             	

 Parameters describing the recording and datuming surfaces:            
 recsurf=0             recording surface (horizontal=0, topographic=1) 
 zrec                  defines recording surface when recsurf=0        
 recfile=              defines recording surface when recsurf=1        
 datsurf=0             datuming surface (horizontal=0, irregular=1)    
 zdat                  defines datuming surface when datsurf=0         
 datfile=              defines datuming surface when datsurf=1         

 Parameters describing the extrapolation:                              
 aperx=nxt*dxt/2  	lateral aperture         			
 v0=1500(m/s)		reference wavespeed             		
 freq=50               dominant frequency in data, used to determine   
                       the minimum distance below the datum that       
                       the stationary phase calculation is valid.      
 scale=1.0             user defined scale factor for output            
 jpfile=stderr		job print file name 				
 mtr=100  		print verbal information at every mtr traces	
 ntr=100000		maximum number of input traces to be datumed	



 Computational Notes:                                                
   
 1. Input traces must be SU format and organized in common receiver gathers.
    
 2. Traveltime tables were generated by program rayt2d (or equivalent)     
    on any grid, with dimension ns*nxt*nzt. In the extrapolation process,       
    traveltimes are interpolated into shot/geophone locations and     
    output grids.                                          

 3. If the offset value of an input trace is not in the offset array     
    of output, the nearest one in the array is chosen.                   

 4. Amplitudes are computed using the constant reference wavespeed v0.  
                                
 5. Input traces must specify source and receiver positions via the header  
    fields tr.sx and tr.gx.             

 6. Recording and datuming surfaces are defined as follows:  If recording
    surface is horizontal, set recsurf=0 (default).  Then, zrec will be a
    required parameter, set to the depth of surface.  If the recording  
    surface is topographic, then set recsurf=1.  Then, recfile is a required
    input file.  The input file recfile should be a single column ascii file
    with the depth of the recording surface at every surface location (first 
    source to last offset), with spacing equal to dxgo. 
 
    The same holds for the datuming surface, using datsurf, zdat, and datfile.


 Assumptions and limitations:

 1. This code implements a 2.5D extraplolation operator that allows to
    transfer data from one reference surface to another.  The formula used in
    this application is an adaptation of Bleistein & Jaramillo's 2.5D data
    mapping formula for receiver extrapolation.  This is the result of a
    stationary phase analysis of the data mapping equation in the case of
    a constant input receiver location (receiver gather). 
 

 Credits:
 
 Authors:  Steven D. Sheaffer (CWP), 11/97 


 References:  Sheaffer, S., 1999, "2.5D Downward Continuation of the Seismic
              Wavefield Using Kirchhoff Data Mapping."  M.Sc. Thesis, 
              Dept. of Mathematical & Computer Sciences, 
              Colorado School of Mines.



\end{verbatim}
\pagebreak
\begin{verbatim}
 SUCDDECON - DECONvolution with user-supplied filter by straightforward
 	      Complex Division in the frequency domain			

 sucddecon <stdin >stdout [optional parameters]			

 Required parameters:							
 filter= 		ascii filter values separated by commas		
 		...or...						
 sufile=		file containing SU traces to use as filter	
                       (must have same number of traces as input data	
 			 for panel=1)					
 Optional parameters:							
 panel=0		use only the first trace of sufile as filter	
 			=1 decon trace by trace an entire gather	
 pnoise=0.001		white noise factor for stabilizing results	
	 				(see below)		 	
 sym=0		not centered, =1 center the output on each trace
 verbose=0		silent, =1 chatty				

 Notes:								
 For given time-domain input data I(t) (stdin) and deconvolution	
 filter F(t), the frequency-domain deconvolved trace can be written as:

	 I(f)		I(f) * complex_conjugate[F(f)]			
 D(f) = ----- ===> D(f) = ------------------------ 			
	 F(f)		|F(f)|^2 + delta				

 The real scalar delta is introduced to prevent the resulting deconvolved
 trace to be dominated by frequencies at which the filter power is close
 to zero. As described above, delta is set to some fraction (pnoise) of 
 the mean of the filter power spectra. Time sampling rate must be the 	
 same in the input data and filter traces. If panel=1 the two input files
 must have the same number of traces. Data and filter traces don't need to
 necessarily have the same number of samples, but the filter trace length
 length be always equal or shorter than the data traces. 		

 Caveat: 								
 You may need to apply frequency filtering to get acceptable output	
   sucddecon  ...| sufilter f=f1,f2,f3,f4 				
 where f1,f2,f3,f4 are an acceptable frequency range, and you may need 
 to mute artifacts that appear at the beginning of the output, as well.

 Trace header fields accessed: ns, dt					
 Trace header fields modified: none					


 Credits:
	CWP: Ivan Vasconcelos
              some changes by John Stockwell
  CAVEATS: 
	In the option, panel=1 the number of traces in the sufile must be 
	the same as the number of traces on the input.

 Trace header fields accessed: ns,dt
 Trace header fields modified: none

\end{verbatim}
\pagebreak
\begin{verbatim}
 SUFXDECON - random noise attenuation by FX-DECONvolution              

 sufxdecon <stdin >stdout [...]	                                

 Required Parameters:							

 Optional Parameters:							
 taper=.1	length of taper                                         
 fmin=6.       minimum frequency to process in Hz  (accord to twlen)   
 fmax=.6/(2*dt)  maximum frequency to process in Hz                    
 twlen=entire trace  time window length (minimum .3 for lower freqs)   
 ntrw=10       number of traces in window                              
 ntrf=4        number of traces for filter (smaller than ntrw)         
 verbose=0	=1 for diagnostic print					
 tmpdir=	if non-empty, use the value as a directory path	prefix	
		for storing temporary files; else, if the CWP_TMPDIR	
		environment variable is set, use its value for the path;
		else use tmpfile()					

 Notes: Each trace is transformed to the frequency domain.             
        For each frequency, Wiener filtering, with unity prediction in 
        space, is used to predict the next sample.                     
        At the end of the process, data is mapped back to t-x domain.  ", 



 Credits:			

	CWP: Carlos E. Theodoro (10/07/97)

 References:      							
		Canales(1984):'Random noise reduction' 54th. SEGM	
		Gulunay(1986):'FXDECON and complex Wiener Predicition   
                             filter' 56th. SEGM	                
		Galbraith(1991):'Random noise attenuation by F-X        
                             prediction: a tutorial' 61th. SEGM	

 Algorithm:
	- read data
	- loop over time windows
		- select data
		- FFT (t -> f)
		- loop over space windows
			- select data
			- loop over frequencies
				- autocorelation
				- matrix problem
				- construct filter
				- filter data
			- loop along space window
				- FFT (f -> t)
				- reconstruct data
 	- output data

 Trace header fields accessed: ns, dt, d1
 Trace header fields modified: 


\end{verbatim}
\pagebreak
\begin{verbatim}
 SUPEF - Wiener (least squares) predictive error filtering		

 supef <stdin >stdout  [optional parameters]				

 Required parameters:							
 dt is mandatory if not set in header			 		

 Optional parameters:							
 cdp= 			CDPs for which minlag, maxlag, pnoise, mincorr, 
			maxcorr are set	(see Notes)			
 minlag=dt		first lag of prediction filter (sec)		
 maxlag=last		lag default is (tmax-tmin)/20			
 pnoise=0.001		relative additive noise level			
 mincorr=tmin		start of autocorrelation window (sec)		
 maxcorr=tmax		end of autocorrelation window (sec)		
 wienerout=0		=1 to show Wiener filter on each trace		
 mix=1,...	 	array of weights (floats) for moving		
				average of the autocorrelations		
 outpar=/dev/null	output parameter file, contains the Wiener filter
 			if wienerout=1 is set				
 method=linear	 for linear interpolation of cdp values			
		       =mono for monotonic cubic interpolation of cdps	
		       =akima for Akima's cubic interpolation of cdps	
		       =spline for cubic spline interpolation of cdps	

 Trace header fields accessed: ns, dt					
 Trace header fields modified: none					

 Notes:								

 1) To apply spiking decon (Wiener filtering with no gap):		

 Run the following command						

    suacor < data.su | suximage perc=95				

 You will see horizontal stripe running across the center of your plot.
 This is the autocorrelation wavelet for each trace. The idea of spiking
 decon is to apply a Wiener filter with no gap to the data to collapse	
 the waveform into a spike. The idea is to pick the width of the	
 autocorrelation waveform _from beginning to end_ (not trough to trough)
 and use this time for MAXLAG_SPIKING:					

  supef < data.su maxlag=MAXLAG_SPIKING  > dataspiked.su		

 2) Prediction Error Filter (i.e. gapped Wiener filtering)		
 The purpose of gapped decon is to suppress repetitions in the data	
 such as those caused by water bottom multiples.			

 To look for the period of the repetitions				

    suacor ntout=1000 < dataspiked.su | suximage perc=95		
 or 									
    suacor ntout=1000 < dataspiked.su | sustack key=dt |suxwigb	

 The value of ntout must be larger than the default 100. The idea is	
 to look for repetitions in the autocorrelation. These repetitions will
 appear as a family of parallel stripes above and below the main	
 autocorrelation waveform. Or, if you stack the data, these will be	
 repetitive spikes.  This repetition time is the GAP. We set 		
 MINLAG_PEF to the value of this repetition time.			

 We set the minlag to MINLAG_PEF = GAP					

 We set the maxlag to MAXLAG_PEF = GAP + MAXLAG_SPIKING		

  supef < dataspiked.su minlag=MINLAG_PEF maxlag=MAXLAG_PEF > datapef.su

 Some experimentation may be required to get a satisfactory result.	
 In particular you may find that you need to reduce the value of the   
 minlag 								

 3) It may be effective to sort your data into cdp gathers with susort,
 and perform sunmo correction to the water speed with sunmo, prior to 	
 attempts to suppress water bottom multiples. After applying supef, the
 user should apply inverse nmo to undo the nmo to water speed prior to	
 further processing. Or, do the predictive decon on fully nmo-corrected
 gathers.								

 If you flatten your data with sunmo, then make sure that you turn off 
 the stretch mute by using smute=20					

  | sunmo vnmo=v1,v2,... tnmo=t1,t2,... smute=20 | supef ...		

 For a filter expressed as a function of cdp, specify the array	
     cdp=cdp1,cdp2,...							
 and for each cdp specified, specify the minlag and maxlag arrays as	
      minlag=min1,min2,...     maxlag=max1,max2,...   			

 It is required that the number of minlag and maxlag values be equal to
 the number of cdp's specified.  If the number of			
 values in these arrays does not equal the number of cdp's, only the first
 value will be used.							

 Caveat:								
 The wienerout=1 option writes out the wiener filter to outpar, and   
 the prediction error filter to stdout, which is			", 
     1,0,0,...,-wiener[0],...,-wiener[imaxlag-1] 			
 where the sample value of -wiener[0], is  iminlag in the pe-filter.	
 The pe-filter is output as a SU format datafile, one pe-filter for each
 trace input.								
	...| supef ... wienerout | suxwigb				
 shows the prediction error filters					

 Credits:
	CWP: Shuki Ronen, Jack K. Cohen, Ken Larner
      CWP: John Stockwell, added mixing feature (April 1998)
      CSM: Tanya Slota (September 2005) added cdp feature

      Technical Reference:
	A. Ziolkowski, "Deconvolution", for value of maxlag default:
		page 91: imaxlag < nt/10.  I took nt/20.

 Notes:
	The prediction error filter is 1,0,0...,0,-wiener[0], ...,
	so no point in explicitly forming it.

	If imaxlag < 2*iminlag - 1, then we don't need to compute the
	autocorrelation for lags:
		imaxlag-iminlag+1, ..., iminlag-1
	It doesn't seem worth the duplicated code to implement this.

 Trace header fields accessed: ns

\end{verbatim}
\pagebreak
\begin{verbatim}
 SUPHIDECON - PHase Inversion Deconvolution				

    suphidecon < stdin > stdout					

 Required parameters:						  	
	none							   	
 Optional parameters:							
 ... time range used for wavelet extraction:			   	
 tm=-0.1	Pre zero time (maximum phase component )		
 tp=+0.4	Post zero time (minimum phase component + multiples)    
 pad=.1	percentage padding for nt prior to cepstrum calculation	

 pnoise=0.001	Pre-withening (assumed noise to prevent division by zero)

 Notes:								
 The wavelet is separated from the reflectivity and noise based on	
 their different 'smoothness' in the pseudo cepstrum domain.		
 The extracted wavelet also includes multiples. 			
 The wavelet is reconstructed in frequency domain, end removed		", 
 from the trace. (Method by Lichman and Northwood, 1996.)		



 Credits: Potash Corporation, Saskatechwan  Balasz Nemeth 
 given to CWP by Potash Corporation 2008 (originally as supid.c)

 Reference:
 Lichman,and Northwood, 1996; Phase Inversion deconvolution for
 long and short period multiples attenuation, in
 SEG Deconvolution 2, Geophysics reprint Series No. 17
 p. 701-718, originally presented at the 54th EAGE meeting, Paris,
 June 1992, revised March 1993, revision accepted September 1994.
 



\end{verbatim}
\pagebreak
\begin{verbatim}
 SUSHAPE - Wiener shaping filter					

  sushape <stdin >stdout  [optional parameters]			

 Required parameters:							
 w=		vector of input wavelet to be shaped or ...		
 ...or ... 								
 wfile=        ... file containing input wavelet in SU (SEGY trace) format
 d=		vector of desired output wavelet or ...			
 ...or ... 								
 dfile=        ... file containing desired output wavelet in SU format	
 dt=tr.dt		if tr.dt is not set in header, then dt is mandatory

 Optional parameters:							
 nshape=trace		length of shaping filter			
 pnoise=0.001		relative additive noise level			
 showshaper=0		=1 to show shaping filter 			

 verbose=0		silent; =1 chatty				

Notes:									

 Example of commandline input wavelets: 				
sushape < indata  w=0,-.1,.1,... d=0,-.1,1,.1,... > shaped_data	

sushape < indata  wfile=inputwavelet.su dfile=desire.su > shaped_data	

 To get the shaping filters into an ascii file:			
 ... | sushape ... showwshaper=1 2>file | ...   (sh or ksh)		
 (... | sushape ... showshaper=1 | ...) >&file  (csh)			


 Credits:
	CWP: Jack Cohen
	CWP: John Stockwell, added wfile and dfile  options

 Trace header fields accessed: ns, dt
 Trace header fields modified: none


\end{verbatim}
\pagebreak
\begin{verbatim}
 SUDMOFKCW - converted-wave DMO via F-K domain (log-stretch) method for
 		common-offset gathers					

 sudmofkcw <stdin >stdout cdpmin= cdpmax= dxcdp= noffmix= [...]	

 Required Parameters:							
 cdpmin		  minimum cdp (integer number) for which to apply DMO
 cdpmax		  maximum cdp (integer number) for which to apply DMO
 dxcdp		   distance between adjacent cdp bins (m)		
 noffmix		 number of offsets to mix (see notes)		

 Optional Parameters:							
 tdmo=0.0		times corresponding to rms velocities in vdmo (s)
 vdmo=1500.0		rms velocities corresponding to times in tdmo (m/s)
 gamma=0.5		 velocity ratio, upgoing/downgoing		
 ntable=1000		 number of tabulated z/h and b/h (see notes)	
 sdmo=1.0		DMO stretch factor; try 0.6 for typical v(z)	
 flip=0		 =1 for negative shifts and exchanging s1 and s2
 			 (see notes below)				
 fmax=0.5/dt		maximum frequency in input traces (Hz)		
 verbose=0		=1 for diagnostic print				

 Notes:								
 Input traces should be sorted into common-offset gathers.  One common-
 offset gather ends and another begins when the offset field of the trace
 headers changes.							

 The cdp field of the input trace headers must be the cdp bin NUMBER, NOT
 the cdp location expressed in units of meters or feet.		

 The number of offsets to mix (noffmix) should typically equal the ratio of
 the shotpoint spacing to the cdp spacing.  This choice ensures that every
 cdp will be represented in each offset mix.  Traces in each mix will	
 contribute through DMO to other traces in adjacent cdps within that mix.

 The tdmo and vdmo arrays specify a velocity function of time that is	
 used to implement a first-order correction for depth-variable velocity.
 The times in tdmo must be monotonically increasing. The velocity function
 is assumed to have been gotten by traditional NMO. 			

 For each offset, the minimum time at which a non-zero sample exists is
 used to determine a mute time.  Output samples for times earlier than this
 mute time will be zeroed.  Computation time may be significantly reduced
 if the input traces are zeroed (muted) for early times at large offsets.

 z/h is horizontal-reflector depth normalized to half source-reciver offset
 h.  Normalized shift of conversion point is b/h.  The code now does not
 support signed offsets, therefore it is recommended that only end-on data,
 not split-spread, be used as input (of course after being sorted into	
 common-offset gathers).  z/h vs b/h depends on gamma (see Alfaraj's Ph.D.
 thesis, 1993).							

 Flip factor = 1 implies positive shift of traces (in the increasing CDP
 bin number direction).  When processing split-spread data, for example,
 if one side of the spread is processed with flip=0, then the other side
 of the spread should be processed with flip=1.  The flip factor also	
 determines the actions of the factors s1 and s2, i.e., stretching or	
 squeezing.								

 Trace header fields accessed:  nt, dt, delrt, offset, cdp.		


 Credits:
	CWP: Mohamed Alfaraj
		Dave Hale

 Technical Reference:
	Transformation to zero offset for mode-converted waves
	Mohammed Alfaraj, Ph.D. thesis, 1993, Colorado School of Mines

	Dip-Moveout Processing - SEG Course Notes
	Dave Hale, 1988

\end{verbatim}
\pagebreak
\begin{verbatim}
 SUDMOFK - DMO via F-K domain (log-stretch) method for common-offset gathers

 sudmofk <stdin >stdout cdpmin= cdpmax= dxcdp= noffmix= [...]		

 Required Parameters:							
 cdpmin	minimum cdp (integer number) for which to apply DMO	
 cdpmax	maximum cdp (integer number) for which to apply DMO	
 dxcdp		distance between adjacent cdp bins (m)			
 noffmix	number of offsets to mix (see notes)			

 Optional Parameters:							
 tdmo=0.0	times corresponding to rms velocities in vdmo (s)	
 vdmo=1500.0	rms velocities corresponding to times in tdmo (m/s)	
 sdmo=1.0	DMO stretch factor; try 0.6 for typical v(z)		
 fmax=0.5/dt	maximum frequency in input traces (Hz)			
 verbose=0	=1 for diagnostic print					
 tmpdir=	if non-empty, use the value as a directory path	prefix	
		for storing temporary files; else if the CWP_TMPDIR	
		environment variable is set use	its value for the path;	
		else use tmpfile()					

 Notes:								
 Input traces should be sorted into common-offset gathers.  One common- 
 offset gather ends and another begins when the offset field of the trace
 headers changes.							

 The cdp field of the input trace headers must be the cdp bin NUMBER, NOT
 the cdp location expressed in units of meters or feet.		

 The number of offsets to mix (noffmix) should typically no smaller than
 the ratio of the shotpoint spacing to the cdp spacing.  This choice	
 ensures that every cdp will be represented in each offset mix.  Traces 
 in each mix will contribute through DMO to other traces in adjacent cdps
 within that mix. (Values of noffmix 2 or 3 times the ratio of shotpoint
 spacing to the cdp spacing often yield better results.)		

 The tdmo and vdmo arrays specify a velocity function of time that is	
 used to implement a first-order correction for depth-variable velocity.
 The times in tdmo must be monotonically increasing.			

 For each offset, the minimum time at which a non-zero sample exists is 
 used to determine a mute time.  Output samples for times earlier than this", 
 mute time will be zeroed.  Computation time may be significantly reduced
 if the input traces are zeroed (muted) for early times at large offsets.

 Credits:
	CWP: Dave

 Technical Reference:
	Dip-Moveout Processing - SEG Course Notes
	Dave Hale, 1988

 Trace header fields accessed:  ns, dt, delrt, offset, cdp.

\end{verbatim}
\pagebreak
\begin{verbatim}
 SUDMOTIVZ - DMO for Transeversely Isotropic V(Z) media for common-offset
            gathers							

 sudmotivz <stdin >stdout cdpmin= cdpmax= dxcdp= noffmix= [...]	

 Required Parameters:							
 cdpmin         minimum cdp (integer number) for which to apply DMO	
 cdpmax         maximum cdp (integer number) for which to apply DMO	
 dxcdp          distance between adjacent cdp bins (m)			
 noffmix        number of offsets to mix (see notes)			

 Optional Parameters:							
 vnfile=        binary (non-ascii) file containing NMO interval	
                  velocities (m/s)					
 vfile=         binary (non-ascii) file containing interval velocities	(m/s)
 etafile=       binary (non-ascii) file containing eta interval values (m/s)
 tdmo=0.0       times corresponding to interval velocities in vdmo (s)	
 vndmo=1500.0   NMO interval velocities corresponding to times in tdmo (m/s)
 vdmo=vndmo    interval velocities corresponding to times in tdmo (m/s)
 etadmo=1500.0  eta interval values corresponding to times in tdmo (m/s)
 fmax=0.5/dt    maximum frequency in input traces (Hz)			
 smute=1.5      stretch mute used for NMO correction			
 speed=1.0      twist this knob for speed (and aliasing)		
 verbose=0      =1 for diagnostic print				

 Notes:								
 Input traces should be sorted into common-offset gathers.  One common-
 offset gather ends and another begins when the offset field of the trace
 headers changes.							

 The cdp field of the input trace headers must be the cdp bin NUMBER, NOT
 the cdp location expressed in units of meters or feet.		

 The number of offsets to mix (noffmix) should typically equal the ratio of
 the shotpoint spacing to the cdp spacing.  This choice ensures that every
 cdp will be represented in each offset mix.  Traces in each mix will	
 contribute through DMO to other traces in adjacent cdps within that mix.

 vnfile, vfile and etafile should contain the regularly sampled interval
 values of NMO velocity, velocity and eta respectivily as a		
 function of time.  If, for example, vfile is not supplied, the interval
 velocity function is defined by linear interpolation of the values in the
 tdmo and vdmo arrays.  The times in tdmo must be monotonically increasing.
 If vfile or vdmo are not given it will be equated to vnfile or vndmo. 

 For each offset, the minimum time to process is determined using the	
 smute parameter.  The DMO correction is not computed for samples that	
 have experienced greater stretch during NMO.				

 Trace header fields accessed:  nt, dt, delrt, offset, cdp.		
\end{verbatim}
\pagebreak
\begin{verbatim}
 SUDMOTX - DMO via T-X domain (Kirchhoff) method for common-offset gathers

 sudmotx <stdin >stdout cdpmin= cdpmax= dxcdp= noffmix= [optional parms]

 Required Parameters:							
 cdpmin                  minimum cdp (integer number) for which to apply DMO
 cdpmax                  maximum cdp (integer number) for which to apply DMO
 dxcdp                   distance between successive cdps		
 noffmix                 number of offsets to mix (see notes)		

 Optional Parameters:							
 offmax=3000.0           maximum offset				
 tmute=2.0               mute time at maximum offset offmax		
 vrms=1500.0             RMS velocity at mute time tmute		
 verbose=0               =1 for diagnostic print			
 tmpdir=	if non-empty, use the value as a directory path	prefix	
		for storing temporary files; else if the CWP_TMPDIR	
		environment variable is set use	its value for the path;	
		else use tmpfile()					


 Notes:								
 Input traces should be sorted into common-offset gathers.  One common-
 offset gather ends and another begins when the offset field of the trace
 headers changes.							

 The cdp field of the input trace headers must be the cdp bin NUMBER, NOT
 the cdp location expressed in units of meters or feet.		

 The number of offsets to mix (noffmix) should typically equal the ratio of
 the shotpoint spacing to the cdp spacing.  This choice ensures that every
 cdp will be represented in each offset mix.  Traces in each mix will	
 contribute through DMO to other traces in adjacent cdps within that mix.

 The defaults for offmax and vrms are appropriate only for metric units.
 If distances are measured in feet, then these parameters should be	
 specified explicitly.							

 offmax, tmute, and vrms need not be specified precisely.		
 If these values are unknown, then one should overestimate offmax	
 and underestimate tmute and vrms.					

 No muting is actually performed.  The tmute parameter is used only to	
 determine parameters required to perform DMO.				

 Credits:
	CWP: Dave Hale

 Technical Reference:
      A non-aliased integral method for dip-moveout
      Dave Hale
      submitted to Geophysics, June, 1990

 Trace header fields accessed:  ns, dt, delrt, offset, cdp.

\end{verbatim}
\pagebreak
\begin{verbatim}
 SUDMOVZ - DMO for V(Z) media for common-offset gathers		

 sudmovz <stdin >stdout cdpmin= cdpmax= dxcdp= noffmix= [...]		

 Required Parameters:							
 cdpmin         minimum cdp (integer number) for which to apply DMO	
 cdpmax         maximum cdp (integer number) for which to apply DMO	
 dxcdp          distance between adjacent cdp bins (m)			
 noffmix        number of offsets to mix (see notes)			

 Optional Parameters:							
 vfile=         binary (non-ascii) file containing interval velocities (m/s)
 tdmo=0.0       times corresponding to interval velocities in vdmo (s)	
 vdmo=1500.0    interval velocities corresponding to times in tdmo (m/s)
 fmax=0.5/dt    maximum frequency in input traces (Hz)			
 smute=1.5      stretch mute used for NMO correction			
 speed=1.0      twist this knob for speed (and aliasing)		
 verbose=0      =1 for diagnostic print				

 Notes:								
 Input traces should be sorted into common-offset gathers.  One common-
 offset gather ends and another begins when the offset field of the trace
 headers changes.							

 The cdp field of the input trace headers must be the cdp bin NUMBER, NOT
 the cdp location expressed in units of meters or feet.		

 The number of offsets to mix (noffmix) should typically equal the ratio of
 the shotpoint spacing to the cdp spacing.  This choice ensures that every
 cdp will be represented in each offset mix.  Traces in each mix will	
 contribute through DMO to other traces in adjacent cdps within that mix.

 vfile should contain the regularly sampled interval velocities as a	
 function of time.  If vfile is not supplied, the interval velocity	
 function is defined by linear interpolation of the values in the tdmo	
 and vdmo arrays.  The times in tdmo must be monotonically increasing.	

 For each offset, the minimum time to process is determined using the	
 smute parameter.  The DMO correction is not computed for samples that	
 have experienced greater stretch during NMO.				

 Trace header fields accessed:  nt, dt, delrt, offset, cdp.		
\end{verbatim}
\pagebreak
\begin{verbatim}
 SUTIHALEDMO - TI Hale Dip MoveOut (based on Hale's PhD thesis)	

  sutihaledmo <infile >outfile [optional parameters]			


 Required Parameters:							
 nxmax		  maximum number of midpoints in common offset gather

 Optional Parameters:							
 option=1		1 = traditional Hale DMO (from PhD thesis)	
			2 = Bleistein's true amplitude DMO		
			3 = Bleistein's cos*cos weighted DMO		
			4 = Zhang's DMO					
			5 = Tsvankin's anisotropic DMO			
			6 = Tsvankin's VTI DMO weak anisotropy approximation
 dx=50.		 midpoint sampling interval between traces	
			in a common offset gather.  (usually shot	
			interval in meters)				
 v=1500.0		velocity (in meters/sec)			
			(must enter a positive value for option=3)	
			(for excluding evanescent energy)		
 h=200.0		source-receiver half-offset (in meters)		
 ntpad=0		number of time samples to pad			
 nxpad=h/dx		number of midpoints to pad			
 file=vnmo		name of file with vnmo as a function of p	
			used for option=5--otherwise not used		
			(Generate this file by running program		
			sutivel with appropriate list of Thomsen's	
			parameters.)					
 e=0.			Thompsen's epsilon				
 d=0.			Thompsen's delta				

Note:									

 This module assumes a single common offset gather after NMO is	
 to be input, DMO corrected, and output.  It is useful for computing	
 theoretical DMO impulse responses.  The Hale algorithm is		
 computationally intensive and not commonly used for bulk processing	
 of all of the offsets on a 2-D line as there are cheaper alternative	
 algorithms.  The Hale algorithm is commonly used in theoretical studies.
 Bulk processing for multiple common offset gathers is typically done	
 using other modules.							

 Test run:   suspike | sutihaledmo nxmax=32 option=1 v=1500 | suxwigb & 


 Author:  (Visitor to CSM from Mobil) John E. Anderson Spring 1994
 References: Anderson, J.E., and Tsvankin, I., 1994, Dip-moveout by
	Fourier transform in anisotropic media, CWP-146

\end{verbatim}
\pagebreak
\begin{verbatim}
 SUBFILT - apply Butterworth bandpass filter 			

 subfilt <stdin >stdout [optional parameters]			

 Required parameters:						
 	if dt is not set in header, then dt is mandatory	

 Optional parameters: (nyquist calculated internally)		
 	zerophase=1		=0 for minimum phase filter 	
 	locut=1			=0 for no low cut filter 	
 	hicut=1			=0 for no high cut filter 	
 	fstoplo=0.10*(nyq)	freq(Hz) in low cut stop band	
 	astoplo=0.05		upper bound on amp at fstoplo 	
 	fpasslo=0.15*(nyq)	freq(Hz) in low cut pass band	
 	apasslo=0.95		lower bound on amp at fpasslo 	
 	fpasshi=0.40*(nyq)	freq(Hz) in high cut pass band	
 	apasshi=0.95		lower bound on amp at fpasshi 	
 	fstophi=0.55*(nyq)	freq(Hz) in high cut stop band	
 	astophi=0.05		upper bound on amp at fstophi 	
 	verbose=0		=1 for filter design info 	
 	dt = (from header)	time sampling interval (sec)	

 ... or  set filter by defining  poles and 3db cutoff frequencies
	npoleselo=calculated     number of poles of the lo pass band
	npolesehi=calculated     number of poles of the lo pass band
	f3dblo=calculated	frequency of 3db cutoff frequency
	f3dbhi=calculated	frequency of 3db cutoff frequency

 Notes:						        
 Butterworth filters were originally of interest because they  
 can be implemented in hardware form through the combination of
 inductors, capacitors, and an amplifier. Such a filter can be 
 constructed in such a way as to have very small oscillations	
 in the flat portion of the bandpass---a desireable attribute.	
 Because the filters are composed of LC circuits, the impulse  
 response is an ordinary differential equation, which translates
 into a polynomial in the transform domain. The filter is expressed
 as the division by this polynomial. Hence the poles of the filter
 are of interest.					        

 The user may define low pass, high pass, and band pass filters
 that are either minimum phase or are zero phase.  The default	
 is to let the program calculate the optimal number of poles in
 low and high cut bands. 					

 Alternately the user may manually define the filter by the 3db
 frequency and by the number of poles in the low and or high	
 cut region. 							

 The advantage of using the alternate method is that the user  
 can control the smoothness of the filter. Greater smoothness  
 through a larger pole number results in a more bell shaped    
 amplitude spectrum.						

 For simple zero phase filtering with sin squared tapering use 
 "sufilter".						        

 Credits:
	CWP: Dave Hale c. 1993 for bf.c subs and test drivers
	CWP: Jack K. Cohen for su wrapper c. 1993
      SEAM Project: Bruce Verwest 2009 added explicit pole option
                    in a program called "subfiltpole"
      CWP: John Stockwell (2012) combined Bruce Verwests changes
           into the original subfilt.

 Caveat: zerophase will not do good if trace has a spike near
	   the end.  One could make a try at getting the "effective"
	   length of the causal filter, but padding the traces seems
	   painful in an already expensive algorithm.


 Theory:
 The 

 Trace header fields accessed: ns, dt, trid

\end{verbatim}
\pagebreak
\begin{verbatim}
 SUCCFILT -  FX domain Correlation Coefficient FILTER			

   sucff < stdin > stdout [optional parameters]			

 Optional parameters:							
 cch=1.0		Correlation coefficient high pass value		
 ccl=0.3		Correlation coefficient low pass value		
 key=ep		ensemble identifier				
 padd=25		FFT padding in percentage			

 Notes:                       						
 This program uses "get_gather" and "put_gather" so requires that	
 the  data be sorted into ensembles designated by "key", with the ntr
 field set to the number of traces in each respective ensemble.  	

 Example:                     						
 susort ep offset < data.su > datasorted.su				
 suputgthr dir=Data verbose=1 < datasorted.su				
 sugetgthr dir=Data verbose=1 > dataupdated.su				
 succfilt  < dataupdated.su > ccfiltdata.su				

 (Work in progress, editing required)                 			

 
 Credits:
  Potash Corporation: Balazs Nemeth, Saskatoon Canada. c. 2008



\end{verbatim}
\pagebreak
\begin{verbatim}
 SUDIPFILT - DIP--or better--SLOPE Filter in f-k domain	

 sudipfilt <infile >outfile [optional parameters]		

 Required Parameters:						
 dt=(from header)	if not set in header then mandatory	
 dx=(from header, d1)	if not set in header then mandatory	

 Optional parameters:						
 slopes=0.0		monotonically increasing slopes		
 amps=1.0		amplitudes corresponding to slopes	
 bias=0.0		slope made horizontal before filtering	

 verbose=0	verbose = 1 echoes information			

 tmpdir= 	 if non-empty, use the value as a directory path
		 prefix for storing temporary files; else if the
	         the CWP_TMPDIR environment variable is set use	
	         its value for the path; else use tmpfile()	

 Notes:							
 d2 is an acceptable alias for dx in the getpar		

 Slopes are defined by delta_t/delta_x, where delta		
 means change. Units of delta_t and delta_x are the same	
 as dt and dx. It is sometimes useful to fool the program	
 with dx=1 dt=1, thus avoiding units and small slope values.	

 Linear interpolation and constant extrapolation are used to	
 determine amplitudes for slopes that are not specified.	
 Linear moveout is removed before slope filtering to make	
 slopes equal to bias appear horizontal.  This linear moveout	
 is restored after filtering.  The bias parameter may be	
 useful for spatially aliased data.  The slopes parameter is	
 compensated for bias, so you need not change slopes when you	
 change bias.							


 Credits:

	CWP: Dave (algorithm--originally called slopef)
	     Jack (reformatting for SU)

 Trace header fields accessed: ns, dt, d2

\end{verbatim}
\pagebreak
\begin{verbatim}
 SUFILTER - applies a zero-phase, sine-squared tapered filter		

 sufilter <stdin >stdout [optional parameters]         		

 Required parameters:                                         		
       if dt is not set in header, then dt is mandatory        	

 Optional parameters:							
       f=f1,f2,...             array of filter frequencies(HZ) 	
       amps=a1,a2,...          array of filter amplitudes		
       dt = (from header)      time sampling interval (sec)        	
	verbose=0		=1 for advisory messages		

 Defaults:f=.10*(nyquist),.15*(nyquist),.45*(nyquist),.50*(nyquist)	
                        (nyquist calculated internally)		
          amps=0.,1.,...,1.,0.  trapezoid-like bandpass filter		

 Examples of filters:							
 Bandpass:   sufilter <data f=10,20,40,50 | ...			
 Bandreject: sufilter <data f=10,20,30,40 amps=1.,0.,0.,1. | ..	
 Lowpass:    sufilter <data f=10,20,40,50 amps=1.,1.,0.,0. | ...	
 Highpass:   sufilter <data f=10,20,40,50 amps=0.,0.,1.,1. | ...	
 Notch:      sufilter <data f=10,12.5,35,50,60 amps=1.,.5,0.,.5,1. |..	

 Credits:
      CWP: John Stockwell, Jack Cohen
	CENPET: Werner M. Heigl - added well log support

 Possible optimization: Do assignments instead of crmuls where
 filter is 0.0.

 Trace header fields accessed: ns, dt, d1

\end{verbatim}
\pagebreak
\begin{verbatim}
 SUFRAC -- take general (fractional) time derivative or integral of	
	    data, plus a phase shift.  Input is CAUSAL time-indexed	
	    or depth-indexed data.					

 sufrac power= [optional parameters] <indata >outdata 			

 Optional parameters:							
	power=0		exponent of (-i*omega)	 			
			=0  ==> phase shift only			
			>0  ==> differentiation				
			<0  ==> integration				

	sign=-1			sign in front of i * omega		
	dt=(from header)	time sample interval (in seconds)	
	phasefac=0		phase shift by phase=phasefac*PI	
	verbose=0		=1 for advisory messages		

 Examples:								
  preprocess to correct 3D data for 2.5D migration			
         sufrac < sudata power=.5 sign=1 | ...				
  preprocess to correct susynlv, susynvxz, etc. (2D data) for 2D migration
         sufrac < sudata phasefac=.25 | ...				
 The filter is applied in frequency domain.				
 if dt is not set in header, then dt is mandatory			

 Algorithm:								
		g(t) = Re[INVFTT{ ( (sign) iw)^power FFT(f)}]		
 Caveat:								
 Large amplitude errors will result if the data set has too few points.

 Good numerical integration routine needed!				
 For example, see Gnu Scientific Library.				


 Credits:
	CWP: Chris Liner, Jack K. Cohen, Dave Hale (pfas)
      CWP: Zhenyue Liu and John Stockwell added phase shift option
	CENPET: Werner M. Heigl - added well log support

 Trace header fields accessed: ns, dt, trid, d1
/
\end{verbatim}
\pagebreak
\begin{verbatim}
 SUFWATRIM - FX domain Alpha TRIM					

  sufwatrim  <stdin > stdout [optional parameters]			

 Required parameters:							
 key=key1,key2,..	Header words defining mixing dimesnion		
 dx=d1,d2,..		Distance units for each header word		
 Optional parameters:							
 keyg=ep		Header word indicating the start of gather	
 vf=0			=1 Do a frequency dependent mix			
 vmin=5000		Velocity of the reflection slope		
			than should not be attenuated			
 Notes:		 						
 Trace with the header word mark set to one will be 			
 the output trace 							


 Credits: Potash Corporation of Saskatchewan, Balasz Nemeth
 Code given to CWP in 2008



\end{verbatim}
\pagebreak
\begin{verbatim}
 SUK1K2FILTER - symmetric box-like K-domain filter defined by the	
		  cartesian product of two sin^2-tapered polygonal	
		  filters defined in k1 and k2				

     suk1k2filter <infile >outfile [optional parameters]		

 Optional parameters:							
 k1=val1,val2,...	array of K1 filter wavenumbers			
 k2=val1,val2,...	array of K2 filter wavenumbers			
 amps1=a1,a2,...	array of K1 filter amplitudes			
 amps2=a1,a2,...	array of K2 filter amplitudes			
 d1=tr.d1 or 1.0	sampling interval in first (fast) dimension	
 d2=tr.d1 or 1.0	sampling interval in second (slow) dimension	
 quad=0		=0 all four quandrants				
			=1 (quadrants 1 and 4) 				
			=2 (quadrants 2 and 3) 				

 Defaults:								
 k1=.10*(nyq1),.15*(nyq1),.45*(nyq1),.50*(nyq1)			
 k2=.10*(nyq2),.15*(nyq2),.45*(nyq2),.50*(nyq2)			
 amps1=0.,1.,...,1.,0.  trapezoid-like bandpass filter			
 amps2=0.,1.,...,1.,0.  trapezoid-like bandpass filter			

 The nyquist wavenumbers, nyq1 and nyq2, are computed internally.	

 verbose=0	verbose = 1 echoes information				

 tmpdir= 	 if non-empty, use the value as a directory path	
		 prefix for storing temporary files; else if the	
	         the CWP_TMPDIR environment variable is set use		
	         its value for the path; else use tmpfile()		

 Notes:								
 The filter is assumed to be symmetric, to yield real output		

 Because the data are assumed to be purely spatial (i.e. non-seismic), 
 the data are assumed to have trace id (30), corresponding to (z,x) data

 The relation: w = 2 pi F is well known for frequency, but there	
 doesn't seem to be a commonly used letter corresponding to F for the	
 spatial conjugate transform variables.  We use K1 and K2 for this.	
 More specifically we assume a phase:					
		-i(k1 x1 + k2 x2) = -2 pi i(K1 x1 + K2 x2).		
 and K1, K2 define our respective wavenumbers.				


 Credits:
     CWP: John Stockwell, November 1995.

 Trace header fields accessed: ns, d1, d2

\end{verbatim}
\pagebreak
\begin{verbatim}
 SUKFILTER - radially symmetric K-domain, sin^2-tapered, polygonal	
		  filter						

     sukfilter <infile >outfile [optional parameters]			

 Optional parameters:							
 k=val1,val2,...	array of K filter wavenumbers			
 amps=a1,a2,...	array of K filter amplitudes			
 d1=tr.d1 or 1.0	sampling interval in first (fast) dimension	
 d2=tr.d1 or 1.0	sampling interval in second (slow) dimension	

 Defaults:								
 k=.10*(nyq),.15*(nyq),.45*(nyq),.50*(nyq)				
 amps=0.,1.,...,1.,0.  trapezoid-like bandpass filter			

 The nyquist wavenumbers, nyq=sqrt(nyq1^2 + nyq2^2) is  computed	
 internally.								

 Notes:								
 The filter is assumed to be symmetric, to yield real output.		

 Because the data are assumed to be purely spatial (i.e. non-seismic), 
 the data are assumed to have trace id (30), corresponding to (z,x) data

 The relation: w = 2 pi F is well known for frequency, but there	
 doesn't seem to be a commonly used letter corresponding to F for the	
 spatial conjugate transform variables.  We use K1 and K2 for this.	
 More specifically we assume a phase:					
		-i(k1 x1 + k2 x2) = -2 pi i(K1 x1 + K2 x2).		
 and K1, K2 define our respective wavenumbers.				


 Credits:
     CWP: John Stockwell, June 1997.

 Trace header fields accessed: ns, d1, d2

\end{verbatim}
\pagebreak
\begin{verbatim}
 SUKFRAC - apply FRACtional powers of i|k| to data, with phase shift 

     sukfrac <infile >outfile [optional parameters]			

 Optional parameters:							
  power=0		exponent of (i*sqrt(k1^2 + k2^2))^power		
			=0 ===> phase shift only			
			>0 ===> differentiation				
			<0 ===> integration				
  sign=1		sign on transform exponent       		
  d1=1.0		x1 sampling interval				
  d2=1.0		x2 sampling interval				
  phasefac=0		phase shift by phase=phasefac*PI		
  ...directional derivative, active only if angle= is set ....		
  angle=		if set applies operator directionally in the	
			direction specified by the value of angle,	
			-360.0 <= angle < 359.99999 degress		

 Notes:								
 The relation: w = 2 pi F is well known for frequency, but there	
 doesn't seem to be a commonly used letter corresponding to F for the	
 spatial conjugate transform variables.  We use K1 and K2 for this.	
 More specifically we assume a phase:					
		-i(k1 x1 + k2 x2) = -2 pi i(K1 x1 + K2 x2).		
 and K1, K2 define our respective wavenumbers.				

 Algorithms 								
 	g(x1,x2)=Re[2DINVFFT{ ( (sign) i |k|)^power 2DFFT(f)}e^i(phase)]
 where: 								
       |k| = sqrt[ (k1)^2 + (k2)^2 ] 					
 or when angle= is set 						
       |k| = sqrt[ (k1  cos(angle) )^2 + (k2 sin(angle) )^2 ] 		

 In the default mode a factor of (i|k|)^(power) is applied in the 	
 transform domain. For time data the time axis direction is taken to 	
 be the k1-direction.  The effect of this filter is to differentiate   
 the input in the normal direction to any curvilinear features in	
 in the data, and thus be a non-directional-specific edge enhancer.	

 If angle= is set, then the intended effect is a derivative in the 	
 direction specified by the angle, with the k1-direction being angle=0,
 corresponding to curves whose normals lie in the x1-direction. 	

 Caveat:								
 Large amplitude errors will result of the data set has too few points.

 Examples:								
 Edge sharpening: 							
 Laplacean : 								
    sukfrac < image_data  power=2 phasefac=-1 | ... 			

 Image enhancement:							
   Derivative filter:							
    sukfrac < image_data  power=1 phasefac=-.5 | ... 			

 Image enhancement:							
   Half derivative (this one is the best for photographs): 		
    sukfrac < image_data  power=.5 phasefac=-.25 | ... 		


 Credits:
     CWP: John Stockwell, June 1997, based on sufrac.

 Trace header fields accessed: ns, d1, d2

\end{verbatim}
\pagebreak
\begin{verbatim}
 SULFAF -  Low frequency array forming					", 

  sulfaf < stdin > stdout [optional parameters]			

 Optional Parameters:	  						
 key=ep	header keyword describing the gathers			
 f1=3		lower frequency	cutof					
 f2=20		high frequency cutof					
 fr=5		frequency ramp						

 vel=1000	surface wave velocity					
 dx=10		trace spacing						
 maxmix=tr.ntr	default is the entire gather				
 adb=1.0	add back ratio 1.0=pure filtered 0.0=origibal		

 tri=0		1 Triangular weight in mixing window			

 Notes:		  						
 The traces transformed into the freqiency domain			
 where a trace mix is performed in the specifed frequency range	
 as Mix=vel/(freq*dx)							

 This program uses "get_gather" and "put_gather" so requires that  
 the  data be sorted into ensembles designated by "key", with the ntr
 field set to the number of traces in each respective ensemble.	

 Example:								 
 susort ep offset < data.su > datasorted.su				
 suputgthr dir=Data verbose=1 < datasorted.su			  
 sugetgthr dir=Data verbose=1 > dataupdated.su			 
 sulfaf  < dataupdated.su > ccfiltdata.su				

 (Work in progress, editing required)                 			

define LOOKFAC 1	/* Look ahead factor for npfaro
define PFA_MAX 720720  /* Largest allowed nfft		*/
define PIP2 PI/2.0	/* IP/2				*/

int 
main( int argc, char *argv[] )
{
	cwp_String key;		/* header key word from segy.h		*/
	cwp_String type;	/* ... its type				*/
	Value val;		/* ... its value			*/
	segy **rec_o;		/* trace header+data matrix		*/
	int first=0;		/* true when we passed the first gather
	int ng=0;
	float dt;		/* time sampling interval	*/
	int nt;			/* num time samples per trace	*/
	int ntr;		/* num of traces per ensemble	*/
	
	int nfft=0;		/* lenghth of padded array	*/
	float snfft;		/* scale factor for inverse fft
	int nf=0;		/* number of frequencies	*/
	float d1;		/* frequency sampling interval.	*/
	float *rt=NULL;		/* real trace			*/
	complex *ct=NULL;	/* complex trace		*/
	float **ffdr=NULL;	/* frequency domain data	*/
	float **ffdi=NULL;	/* frequency domain data	*/
	float **ffdrm=NULL;	/* frequency domain mixed data	*/
	float **ffdim=NULL;	/* frequency domain mixed data	*/
	
	int verbose;		/* flag: =0 silent; =1 chatty	*/
	
	
	float f1;		/* minimum frequency		*/
	int if1;		/* ...    ...  integerized	*/
	float f2;		/* maximum frequency		*/
	int if2;		/* ...    ...  integerized	*/
	float fr;		/* slope of frequency ramp	*/
	int ifr;		/* ...    ...  integerized	*/
	float vel;		/* velocity of guided waves	*/
	float dx;		/* spatial sampling intervall	*/
	int maxmix;		/* size of mix			*/
	int tri;		/* flag: =1 trianglular window	*/
	float adb;		/* add back ratio		*/
		
	/* Initialize
	initargs(argc, argv);
	requestdoc(1);
	
	if (!getparstring("key", &key))		key = "ep";
	if (!getparfloat("f1", &f1))		f1 = 3.0;
	if (!getparfloat("f2", &f2))		f2 = 20.0;
	if (!getparfloat("dx", &dx))		dx = 10;
	if (!getparfloat("vel", &vel))		vel = 1000;
	if (!getparfloat("fr", &fr))		fr = 5;
	if (!getparint("maxmix", &maxmix))	maxmix = -1;
	if (!getparint("tri", &tri))		tri = 0;
	if (!getparfloat("adb", &adb))		adb = 1.0;
	
	if (!getparint("verbose", &verbose)) verbose = 0;

	/* get the first record
	rec_o = get_gather(&key,&type,&val,&nt,&ntr,&dt,&first);
	if(ntr==0) err("Can't get first record\n");
		
	/* set up the fft
	nfft = npfaro(nt, LOOKFAC * nt);
	 if (nfft >= SU_NFLTS || nfft >= PFA_MAX)
		 	err("Padded nt=%d--too big", nfft);
	 nf = nfft/2 + 1;
	 snfft=1.0/nfft;
	d1=1.0/(nfft*dt);
	
	ct=ealloc1complex(nf);
	rt=ealloc1float(nfft);
	
	if1=NINT(f1/d1);
	if2=NINT(f2/d1);
	ifr=NINT(fr/d1);
	
	do {
		if(maxmix==-1) maxmix=ntr;
		ng++;
		
		/* Allocate arrays for fft
		ffdr = ealloc2float(nf,ntr);
		ffdi = ealloc2float(nf,ntr);
		ffdrm = ealloc2float(if2+ifr,ntr);
		ffdim = ealloc2float(if2+ifr,ntr);
		{ int itr,iw;
			for(itr=0;itr<ntr;itr++) {
				
				memcpy( (void *) rt, (const void *) (*rec_o[itr]).data, nt*FSIZE);

				memset( (void *) &rt[nt], 0,(nfft-nt)*FSIZE);
				
				pfarc(1,nfft,rt,ct);
				
				for(iw=0;iw<nf;iw++) {
					ffdr[itr][iw] = ct[iw].r;
					ffdi[itr][iw] = ct[iw].i;
				}
			}
		}
		
		/* Mixing
		{ int mix,iw,nmix;
		  int ims,ime;
		  int itr,itrm,iww,ws;
		  float tmpr,tmpi,wh;
		
			for(iw=if1;iw<if2+ifr;iw++) {
				
				mix=MIN(NINT(vel/iw*d1*dx),maxmix);
				if(!ISODD(mix)) mix -=1;
				if (verbose) warn(" %f %d",iw*d1,mix);
				
				for(itr=0;itr<ntr;itr++) {
						
					ims=MAX(itr-mix/2,0);
					ime=MIN(ims+mix,ntr-1);

					tmpr=0.0; tmpi=0.0;
					wh=1.0; ws=mix/2;
					nmix=0;
					for(itrm=ims,iww=-mix/2;itrm<ime;++itrm,++iww) {
						++nmix;
						if(tri) wh = (float)(ws-abs(iww));
						tmpr+=ffdr[itrm][iw]*wh;
						tmpi+=ffdi[itrm][iw]*wh;
					}
					ffdrm[itr][iw]=tmpr/nmix;
					ffdim[itr][iw]=tmpi/nmix;
				}
			}
			
			for(iw=if1;iw<if2;iw++) {
				for(itr=0;itr<ntr;itr++) {
					ffdr[itr][iw]=ffdrm[itr][iw]*adb+ffdr[itr][iw]*(1.0-adb);
					ffdi[itr][iw]=ffdim[itr][iw]*adb+ffdi[itr][iw]*(1.0-adb);
				}
			}
			
			for(iw=if2,iww=0;iw<if2+ifr;iw++,iww++) {
				
				wh=(float)(1.0-(float)iww/(float)ifr);
				
				for(itr=0;itr<ntr;itr++) {
					ffdr[itr][iw] = (wh*ffdrm[itr][iw]+ffdr[itr][iw]*(1.0-wh))*adb+ffdr[itr][iw]*(1.0-adb);
					ffdi[itr][iw] = (wh*ffdim[itr][iw]+ffdi[itr][iw]*(1.0-wh))*adb+ffdi[itr][iw]*(1.0-adb);
				}
			}
		
		}
		  
		
		{ int itr,iw;
			for(itr=0;itr<ntr;itr++) {
				
				for(iw=0;iw<nf;iw++) {
					ct[iw].r = ffdr[itr][iw]*snfft;
					ct[iw].i = ffdi[itr][iw]*snfft;
				} 
				
				pfacr(-1,nfft,ct,rt);
				memcpy( (void *) (*rec_o[itr]).data, (const void *) rt, nt*FSIZE);
				
			}
		}
		
			rec_o = put_gather(rec_o,&nt,&ntr);

		free2float(ffdr);
		free2float(ffdi);
		free2float(ffdrm);
		free2float(ffdim);
		rec_o = get_gather(&key,&type,&val,&nt,&ntr,&dt,&first);
		
	} while(ntr);

	warn("Number of gathers %10d\n",ng);
	 
	return EXIT_SUCCESS;
}

\end{verbatim}
\pagebreak
\begin{verbatim}
 SUMEDIAN - MEDIAN filter about a user-defined polygonal curve with	
	   the distance along the curve specified by key header word 	

 sumedian <stdin >stdout xshift= tshift= [optional parameters]		

 Required parameters:							
 xshift=               array of position values as specified by	
                       the `key' parameter				
 tshift=               array of corresponding time values (sec)	
  ... or input via files:						
 nshift=               number of x,t values defining median times	
 xfile=                file containing position values as specified by	
                       the `key' parameter				
 tfile=                file containing corresponding time values (sec)	

 Optional parameters:							
 key=tracl             Key header word specifying trace number 	
                       =offset  use trace offset instead		

 mix=.6,1,1,1,.6       array of weights for mix (weighted moving average)
 median=0              =0  for mix					
                       =1  for median filter				
 nmed=5                odd no. of traces to median filter		
 sign=-1               =-1  for upward shift				
                       =+1  for downward shift				
 subtract=1            =1  subtract filtered data from input		
                       =0  don't subtract				
 verbose=0             =1  echoes information				

 tmpdir= 	 if non-empty, use the value as a directory path	
		 prefix for storing temporary files; else if the	
	         the CWP_TMPDIR environment variable is set use		
	         its value for the path; else use tmpfile()		

 Notes: 								
 ------								
 Median filtering is a process for suppressing a particular moveout on 
 seismic sections. Its advantage over traditional dip filtering is that
 events with arbitrary moveout may be suppressed. Median filtering is	
 commonly used in up/down wavefield separation of VSP data.		

 The process generally consists of 3 steps:				
 1. A copy of the data panel is shifted such that the polygon in x,t	
    specifying moveout is flattened to horizontal. The x,t pairs are 	
    specified either by the vector xshift,tshift or by the values in	
    the datafiles xfile,tfile.	For fractional shift, the shifted data	
    is interpolated.							
 2. Then a mix (weighted moving average) is performed over the shifted	
    panel to emphasize events with the specified moveout and suppress	
    events with other moveouts.					
 3. The panel is then shifted back (and interpolated) to its original	
    moveout, and subtracted from the original data. Thus all events	
    with the user-specified moveout are removed.			

 For VSP data the following modifications are provided:		
 1. The moveout polygon in x,t is usually the first break times for	
    each trace. The parameter sign allows for downward shift in order	
    align upgoing events.						
 2. Alternative to a mix, a median filter can be applied by setting	
    the parameter median=1 and nmed= to the number of traces filtered.	
 3. By setting subtract=0 the filtered panel is only shifted back but	
    not subtracted from the original data.				

 The values of tshift are linearly interpolated for traces falling	
 between given xshift values. The tshift interpolant is extrapolated	
 to the left by the smallest time sample on the trace and to the right	
 by the last value given in the tshift array. 				

 The files tfile and xfile are files of binary (C-style) floats.	

 The number of values defined by mix=val1,val2,... determines the	
 number of traces to be averaged, the values determine the weights.	

 Caveat:								
 The median filter may perform poorly on the edges of a section.	
 Choosing larger beginning and ending mix values may help, but may	
 also introduce additional artifacts.					

 Examples:								



 Credits:

 CWP: John Stockwell, based in part on sumute, sureduce, sumix
 CENPET: Werner M. Heigl - fixed various errors, added VSP functionality

 U of Durham, UK: Richard Hobbs - fixed the program so it applies the
                                   median filter
 ideas for improvement:
	a versatile median filter needs to do:
	shift traces by fractional amounts -> needs sinc interpolation
	positive and negative shifts similar to SUSTATIC
	make subtraction of filtered events a user choice
	provide a median stack as well as a weighted average stack
 Trace header fields accessed: ns, dt, delrt, key=keyword


\end{verbatim}
\pagebreak
\begin{verbatim}
 SUPHASE - PHASE manipulation by linear transformation			

  suphase  <stdin >sdout      						

 Required parameters:							
 none									
 Optional parameters:							
 a=90			constant phase shift              		
 b=180/PI              linear phase shift				
 c=0.0			phase = a +b*(old_phase)+c*f;			

 Notes: 								
 A program that allows the user to experiment with changes in the phase
 spectrum of a signal.							

\end{verbatim}
\pagebreak
\begin{verbatim}
 SUSMGAUSS2 --- SMOOTH a uniformly sampled 2d array of velocities	
		using a Gaussian filter specified with correlation 	
	lengths a1 and a2.	

 susmgauss2 < stdin [optional parameters ] > stdout			

 Optional Parameters:							
 a1=0			smoothing parameter in the 1 direction		
 a2=0			smoothing parameter in the 2 direction		

 Notes:								
 Larger a1 and a2 result in a smoother velocity. The velocities are	
 first transformed to slowness and then a Gaussian filter is applied	
 in the wavenumber domain.						

 Input file must be in SU format. The output file is smoothed velocity 



	Credits: 
		CWP: Carlos Pacheco, 2005

\end{verbatim}
\pagebreak
\begin{verbatim}
 SUTVBAND - time-variant bandpass filter (sine-squared taper)  

 sutvband <stdin >stdout tf= f=			        

 Required parameters:                                          
       dt = (from header)      time sampling interval (sec)    
       tf=             times for which f-vector is specified   
       f=f1,f2,f3,f4   Corner frequencies corresponding to the 
                       times in tf. Specify as many f= as      
                       there are entries in tf.                

 The filters are applied in frequency domain.                  

 Example:                                                      
 sutvband <data tf=.2,1.5 f=10,12.5,40,50 f=10,12.5,30,40 | ...

 Credits:
      CWP: Jack, Ken

 Trace header fields accessed:  ns, dt, delrt

\end{verbatim}
\pagebreak
\begin{verbatim}
\end{verbatim}
\pagebreak
\begin{verbatim}
 BHEDTOPAR - convert a Binary tape HEaDer file to PAR file format	

     bhedtopar < stdin outpar=parfile					

 Required parameter:							
 	none								
 Optional parameters:							
	swap=0 			=1 to swap bytes			
 	outpar=/dev/tty		=parfile  name of output param file	

 Notes: 								
 This program dumps the contents of a SEGY binary tape header file, as 
 would be produced by segyread and segyhdrs to a file in "parfile" format.
 A "parfile" is an ASCII file containing entries of the form param=value.
 Here "param" is the keyword for the binary tape header field and	
 "value" is the value of that field. The parfile may be edited as	
 any ASCII file. The edited parfile may then be made into a new binary tape 
 header file via the program    setbhed.				

 See    sudoc  setbhed   for examples					


 Credits:

	CWP: John Stockwell  11 Nov 1994

\end{verbatim}
\pagebreak
\begin{verbatim}
 SU3DCHART - plot x-midpoints vs. y-midpoints for 3-D data	

 su3dchart <stdin >stdout					

 Optional parameters:						
 outpar=null	name of parameter file				
 degree=0	=1 convert seconds of arc to degrees		

 The output is the (x, y) pairs of binary floats		

 Example:							
 su3dchart <segy_data outpar=pfile >plot_data			
 psgraph <plot_data par=pfile \\				
	linewidth=0 marksize=2 mark=8 | ...			
 rm plot_data 							

 su3dchart <segy_data | psgraph n=1024 d1=.004 \\		
	linewidth=0 marksize=2 mark=8 | ...			

 Note:  sx, etc., are declared double because float has only 7
 significant numbers, that's not enough, for example,    
 when tr.scalco=100 and coordinates are in second of arc    
 and located near 30 degree latitude and 59 degree longitude           
                                                            


 Credits:
	CWP: Shuki Ronen
	Toralf Foerster

 Trace header fields accessed: sx, sy, gx, gy, counit, scalco.


\end{verbatim}
\pagebreak
\begin{verbatim}
 SUABSHW - replace header key word by its absolute value	

 suabshw <stdin >stdout key=offset				

 Required parameters:						
 	none							

 Optional parameter:						
 	key=offset		header key word			


 Credits:
	CWP: Jack K. Cohen

\end{verbatim}
\pagebreak
\begin{verbatim}
 SUADDHEAD - put headers on bare traces and set the tracl and ns fields

 suaddhead <stdin >stdout ns= ftn=0					

 Required parameter:							
 	ns=the number of samples per trace				

 Optional parameter:							
ifdef SU_LINE_HEADER
	head=           file to read headers in				
                       not supplied --  will generate headers 		
                       given        --  will read in headers and attach
                                        floating point arrays to form 	
                                        traces 			", 
                       (head can be created via sustrip program)	
endif
 	ftn=0		Fortran flag					
 			0 = data written unformatted from C		
 			1 = data written unformatted from Fortran	
       tsort=3         trace sorting code:				
                                1 = as recorded (no sorting)		
                                2 = CDP ensemble			
                                3 = single fold continuous profile	
                                4 = horizontally stacked		", 
       ntrpr=1         number of data traces per record		
                       if tsort=2, this is the number of traces per cdp", 

 Trace header fields set: ns, tracl					
 Use sushw/suchw to set other needed fields.				

 Caution: An incorrect ns field will munge subsequent processing.	
 Note:    n1 and nt are acceptable aliases for ns.			

 Example:								
 suaddhead ns=1024 <bare_traces | sushw key=dt a=4000 >segy_traces	

 This command line adds headers with ns=1024 samples.  The second part	
 of the pipe sets the trace header field dt to 4 ms.	See also the	
 selfdocs of related programs  sustrip and supaste.			
 See:   sudoc supaste							
 Related Programs:  supaste, sustrip 					
\end{verbatim}
\pagebreak
\begin{verbatim}
 SUAHW - Assign Header Word using another header word			

  suahw <stdin >stdout [optional parameters]				

 Required parameters:							
  key1=ep		output key 					
  key2=fldr		input key 					
  a=			array of key1 output values			
  b=			array of key2 input values			
  mode=extrapolate	how to assign a key1-value, when the key2-value	
			is not found in b:				
			=interpolate	interpolate			
			=extrapolate	interpolate and extrapolate	
			=zero		zero key1-values		
			=preserve	preserve key1-values		
			=transfer	transfer key2-values to key1	

 Optional parameters:							
  key3=tracf		input key 					
  c=			array of key3 input values			

 The key1-value is assigned based on the key2-value and the arrays a,b.
 If the header value of key2= equals the n'th element in b=, then the	
 header value key1= is set to the n'th element in a=.			
 The arrays a= and b= must have the same size, and the elements of b=	
 must be in ascending order.						

 The mode-switch decides what to do when a trace header has a key2-value
 that is not an element of the b-array:				
    zero - the key1-value will be set to zero				
    preserve - the key1-value will not be modified			
    transfer - the key2-value will be assigned to key1			
    interpolate - if the key2-value is greater than the n'th element	
	and less than the (n+1)'th element of b=, then the key1-value	
	will be	interpolated accordingly from the n'th and (n+1)'th	
	element of a=. Otherwise, key1 will not be changed.		
    extrapolate - same as interpolate, plus, if the key2-value is	
	smaller/greater than the first/last element of b=, then the	
	key1-value will be set to the first/last element of a=		

 The array c= can be used to prevent the modification of trace headers	
 with certain key3-values. The number of elements in c= is independent	
 of the other arrays.							
 The key1-value will not be modified, if the mode-switch is set to	
    zero, preserve, transfer - and the key3-value is an element of c=	
    interpolate, extrapolate - and the key3-value is outside of c=	
				(smaller than the first or greater than	
				the last element of c=)			

 Examples:								
  Assign shot numbers 1-3 to field file ID 1009,1011,1015 and 0 to the	
  remaining FFID (fldr):						
    suahw <data a=1,2,3 b=1009,1011,1015 mode=zero			

  Use channel numbers (tracf) to assign stations numbers (tracr) for a	
  split spread with a gap:						
    suahw <data key1=tracr a=151,128,124,101 key2=tracf b=1,24,25,48	

  Assign shot-statics:							
    suahw <data key1=sstat key2=ep a=-32,13,-4 b=1,2,3			

  Set trid to 0 for channel 1-24, but only for the record 1016:	
    suahw <data key1=trid key2=tracf key3=fldr a=0,0 b=1,24 c=1016	

 Credits:
	Florian Bleibinhaus, U Salzburg, Austria
	cloned from suchw of Einar Kajartansson, SEP

\end{verbatim}
\pagebreak
\begin{verbatim}
 SUAZIMUTH - compute trace AZIMUTH, offset, and midpoint coordinates    
             and set user-specified header fields to these values       

  suazimuth <stdin >stdout [optional parameters]                        

 Required parameters:                                                   
     none                                                               

 Optional parameters:                                                   
   key=otrav      header field to store computed azimuths in            
   scale=1.0      value(key) = scale * azimuth                          
   az=0           azimuth convention flag                               
                   0: 0-179.999 deg, reciprocity assumed                
                   1: 0-359.999 deg, points from receiver to source     
                  -1: 0-359.999 deg, points from source to receiver     
   sector=1.0     if set, defines output in sectors of size             
                  sector=degrees_per_sector, the default mode is        
                  the full range of angles specified by az              

   offset=0       if offset=1 then set offset header field              
   signedflag=0   is offset signed? if =1 signedflag=1			 
   offkey=offset  header field to store computed offsets in             
   offkey=offset  header field to store computed offsets in             

   cmp=0          if cmp=1, then compute midpoint coordinates and       
                  set header fields for (cmpx, cmpy)                    
   mxkey=ep       header field to store computed cmpx in                
   mykey=cdp      header field to store computed cmpy in                

 Notes:                                                                 
   All values are computed from the values in the coordinate fields     
   sx,sy (source) and gx,gy (receiver).                                 
   The output field "otrav" for the azimuth was chosen arbitrarily as 
   an example of a little-used header field, however, the user may      
   choose any field that is convenient for his or her application.      

   Setting the sector=number_of_degrees_per_sector sets key field to    
   sector number rather than an angle in degrees.                       

   For az=0, azimuths are measured from the North, however, reciprocity 
   is assumed, so azimuths go from 0 to 179.9999 degrees. If sector     
   option is set, then the range is from 0 to 180/sector.               

   For az=1, azimuths are measured from the North, with the assumption  
   that the direction vector points from the receiver to the source.    
   For az=-1, the direction vector points from the source to the        
   receiver. No reciprocity is assumed in these cases, so the angles go 
   from 0 to 359.999 degrees.                                           
   If the sector option is set, then the range is from 0 to 360/sector. 

 Caveat:                                                                
   This program honors the value of scalco in scaling the values of     
   sx,sy,gx,gy. Type "sukeyword scalco" for more information.         

   Type "sukeyword -o" to see the keywords and descriptions of all    
   header fields.                                                       

   To plot midpoints, use: su3dchart                                    


 Credits:
  based on suchw, su3dchart
      CWP: John Stockwell and  UTulsa: Chris Liner, Oct. 1998
      UTulsa: Chris Liner added offset option, Feb. 2002
         cll: fixed offset option and added cmp option, May 2003
      RISSC: Nils Maercklin added key options for offset and 
             midpoints, and added azimuth direction option, Sep. 2006

  Algorithms:
      offset = osign * sqrt( (gx-sx)*(gx-sx) + (gy-sy)*(gy-sy) )
               with osign = sgn( min((sx-gx),(sy-gy)) )

      midpoint x  value  xm = (sx + gx)/2
      midpoint y  value  ym = (sy + gy)/2
 
  Azimuth will be defined as the angle, measured in degrees,
  turned from North, of a vector pointing to the source from the midpoint, 
  or from the midpoint to the source. Azimuths go from 0-179.000 degrees
  or from 0-180.0 degrees.
   
  value(key) = scale*[90.0 - (180.0/PI)*(atan((sy - ym)/(sx - xm))) ]
      or
  value(key) = scale*[180.0 - (180.0/PI)*(atan2((ym - sy),(xm - sx)) ]
 
  Trace header fields accessed: sx, sy, gx, gy, scalco. 
  Trace header fields modified: (user-specified keys)


\end{verbatim}
\pagebreak
\begin{verbatim}
 SUCDPBIN - Compute CDP bin number					

 sucdpbin <stdin >stdout xline= yline= dcdp=				

 Required parameters:							
 xline=		array of X defining the CDP line		
 yline=		array of Y defining the CDP line		
 dcdp=			distance between bin centers			

 Optional parameters							
 verbose=0		<>0 output informations				
 cdpmin=1001		min cdp bin number				
 distmax=dcdp		search radius					

 xline,yline defines the CDP line made of continuous straight lines. 	
 If a smoother line is required, use unisam to interpolate.		
 Bin centers are located at dcdp constant interval on this line. 	
 Each trace will be numbered with the number of the closest bin. If no  
 bin center is found within the search radius. cdp is set to 0		


 Credits:
 2009 Dominique Rousset - Mohamed Hamza 
      Université de Pau et des Pays de l'Adour (France)


\end{verbatim}
\pagebreak
\begin{verbatim}
 SUCHART - prepare data for x vs. y plot			

 suchart <stdin >stdout key1=sx key2=gx			

 Required parameters:						
 	none							

 Optional parameters:						
 	key1=sx  	abscissa 				
 	key2=gx		ordinate				
	outpar=null	name of parameter file			

 The output is the (x, y) pairs of binary floats		

 Examples:							
 suchart < sudata outpar=pfile >plot_data			
 psgraph <plot_data par=pfile title="CMG" \\			
	linewidth=0 marksize=2 mark=8 | ...			
 rm plot_data 							

 suchart < sudata | psgraph n=1024 d1=.004 \\			
	linewidth=0 marksize=2 mark=8 | ...			

 fold chart: 							
 suchart < stacked_data key1=cdp key2=nhs |			
            psgraph n=NUMBER_OF_TRACES d1=.004 \\		
	linewidth=0 marksize=2 mark=8 > chart.ps		



 Credits:
	SEP: Einar Kjartansson
	CWP: Jack K. Cohen

 Notes:
	The vtof routine from valpkge converts values to floats.

\end{verbatim}
\pagebreak
\begin{verbatim}
 SUCHW - Change Header Word using one or two header word fields	

  suchw <stdin >stdout [optional parameters]				

 Required parameters:							
 none									

 Optional parameters:							
 key1=cdp,...	output key(s) 						
 key2=cdp,...	input key(s) 						
 key3=cdp,...	input key(s)  						
 a=0,...		overall shift(s)				
 b=1,...		scale(s) on first input key(s) 			
 c=0,...		scale on second input key(s) 			
 d=1,...		overall scale(s)				
 e=1,...		exponent on first input key(s)
 f=1,...		exponent on second input key(s)

 The value of header word key1 is computed from the values of		
 key2 and key3 by:							

	val(key1) = (a + b * val(key2)^e + c * val(key3)^f) / d		

 Examples:								
 Shift cdp numbers by -1:						
	suchw <data >outdata a=-1					

 Add 1000 to tracr value:						
 	suchw key1=tracr key2=tracr a=1000 <infile >outfile		

 We set the receiver point (gx) field by summing the offset		
 and shot point (sx) fields and then we set the cdp field by		
 averaging the sx and gx fields (we choose to use the actual		
 locations for the cdp fields instead of the conventional		
 1, 2, 3, ... enumeration):						

   suchw <indata key1=gx key2=offset key3=sx b=1 c=1 |			
   suchw key1=cdp key2=gx key3=sx b=1 c=1 d=2 >outdata			

 Do both operations in one call:					

 suchw<indata key1=gx,cdp key2=offset,gx key3=sx,sx b=1,1 c=1,1 d=1,2 >outdata


 Credits:
	SEP: Einar Kjartansson
	CWP: Jack K. Cohen
      CWP: John Stockwell, 7 July 1995, added array of keys feature
      Delphi: Alexander Koek, 6 November 1995, changed calculation so
              headers of different types can be expressed in each other

\end{verbatim}
\pagebreak
\begin{verbatim}
 SUCLIPHEAD - Clip header values					

 sucliphead <stdin >stdout [optional parameters]			

 Required parameters:							
	none								

 Optional parameters:							
	key=cdp,...			header key word(s) to clip	
	min=0,...			minimum value to clip		
	max=ULONG_MAX,ULONG_MAX,...	maximum value to clip		



 Credits:
	Geocon: Garry Perratt


\end{verbatim}
\pagebreak
\begin{verbatim}
 SUCOUNTKEY - COUNT the number of unique values for a given KEYword.	

 sucountkey < input.su key=[sx,gx,cdp,...]				
 Required parameter:							
 key=			array of SU header keywords being counted	
 Optional parameters:							
 verbose=1		chatty, =0 just print keyword number		
 Example:								
	  suplane | sucountkey key=tracl,tracr,offset			



 Credits: Baoniu Han, bhan@mines.edu, Nov, 2000

\end{verbatim}
\pagebreak
\begin{verbatim}
 SUDUMPTRACE - print selected header values and data.              
               Print first num traces.                             
               Use SUWIND to skip traces.                          

 sudumptrace < stdin [> ascii_file]                                

 Optional parameters:                                              
     num=4                    number of traces to dump             
     key=key1,key2,...        key(s) to print above trace values   
     hpf=0                    header print format is float         
                              =1 print format is exponential       

 Examples:                                                         
   sudumptrace < inseis.su            PRINTS: 4 traces, no headers 
   sudumptrace < inseis.su key=tracf,offset                        
   sudumptrace < inseis.su num=7 key=tracf,offset > info.txt       
   sudumptrace < inseis.su num=7 key=tracf,offset hpf=1 > info.txt 

 Related programs: suascii, sugethw                                


 Credits:
   MTU: David Forel, Jan 2005

 Trace header field accessed: nt, dt, delrt

\end{verbatim}
\pagebreak
\begin{verbatim}
 SUEDIT - examine segy diskfiles and edit headers			

 suedit diskfile  (open for possible header modification if writable)	
 suedit <diskfile  (open read only)					

 The following commands are recognized:				
 number	read in that trace and print nonzero header words	
 <CR>		go to trace one step away (step is initially -1)	
 +		read in next trace (step is set to +1)			
 -		read in previous trace (step is set to -1)		
 dN		advance N traces (step is set to N)			
 %		print some percentiles of the trace data		
 r		print some ranks (rank[j] = jth smallest datum) 	
 p [n1 [n2]]  	tab plot sample n1 to n2 on current trace		
 g [tr1 tr2]  	ximage plot the trace [traces tr1 to tr2]	
 w [tr1 tr2]  	xwigb plot the trace [traces tr1 to tr2]	
 f [tr1 tr2]   ximage plot the amplitude spectra of the trace		
 u [tr1 tr2]   apply user pipeline to specified traces 
 ! key=val  	change a value in a field (e.g. ! tracr=101)		
 ?		print help file						
 q		quit							

 NOTE: sample numbers are 1-based (first sample is 1).			

 'u 1000000  1000100 suwind >subset.su' will quickly extract a few     
 traces from the middle of a large dataset                             


 Credits:
 SEP: Einar Kjartansson, Shuki Ronen, Stew Levin
 CWP: Jack K. Cohen
 Unocal: Reg Beardsley
 Trace header fields accessed: ns
 Trace header fields modified: ntr (only for internal plotting)

\end{verbatim}
\pagebreak
\begin{verbatim}
 SUGEOM - Fill up geometry in trace headers.                              

 sugeom rps= sps= xps=  <stdin >stdout [optional parameters]              

 Required parameters:                                                     
       rps=  filename for the receiver points definitions. (AKA R file)   
       sps=  filename for the source   points definitions. (AKA S file)   
       xps=  filename for the relation specification.      (AKA X file)   

 OBS:  if 'prefix' (see bellow) is given the last three parameters become 
       optional, but any one can be used as an override for the name      
       built using prefix.                                                
          Ex: sugeom prefix=blah  xps=bleh.xxx                            
              Used filenames:  blah.s blah.r and bleh.xxx                 

 Optional parameters:                                                     
       prefix=  prefix name for the SPS files.  If given the filename will
                be constructed as '<prefix>.s', '<prefix>.r' and          
                '<prefix>.x'.                                             
       rev=0             for SPS revision 0 format.                       
          =2 (Default)   for SPS revision 2.1 format.                     
   verbose=0 (Default)   silent run.                                      
          =1             verbose stats.                                   
     ibin=0  (default)   inline binsize (cdp interval)                    
                         0 -> do not compute cdp number                   

  For SPS format specification consult:                                   
    http://www.seg.org/resources/publications/misc/technical-standards    


  WARNING: This is not a full fledged geometry program.                   

  The SPS format is fully described in two documents available at the     
  above SEG address.                                                      

  To make short what can be a loooong discussion, the SPS file format is  
  intended to describe completely a land (or tranzition zone) survey. The 
  main difference between land and marine survey is that in land all      
  points are meant to be previously located and remain fixed along        
  acquisition operation.  In a marine survey the cables and boat can be   
  dragged by current and wind, making the prosition of source and receivers
  unpredictable.  The planning and survey description of a marine program 
  must take into account all moving points, being necessary a full        
  positioning description of source and receivers at every single shot.   

  The SPS format standard is composed of three text files with 80 columns 
  each.  Two of those files are Point Record Specification and are used to
  describe all survey points, that is the receiver stations and source    
  points.  The remaining file is the Relation Record Specification and is 
  used to describe how each source point is related to a set of           
  corresponding receiver points during the registration of each record    
  (fldr).                                                                 

  These files usually have a number of lines at the start describing the  
  survey and the projections used for coordinates.  This program just     
  skip them.  Those are header entries and have an 'H' in first column.   

  Each line (entry) of the  Point Record Specification contain the        
  information of a single point.  The only difference of a source point   
  specification entry and a receiver point specification entry is the     
  first  column of the file, it has an 'S' for a source point description 
  and an 'R' for a receiver point description.  For each point entry there
  is an identification with a pair of informations, the Line it belongs   
  and the Point Number, a set of coordinates (X, Y, surface Elevation,    
  depth of element), and the static correction.  All source points        
  description are in a single file (known as S-File), and all receiver    
  station informations are in another file (R-File).                      

  The Relation Record Specification (X-File) is a file with as many       
  entries (text lines) as necessary to completely describe each record    
  (fldr) acquired.   Each entry containing a record information starts    
  with an 'X' in first column.  The informations in the entry starts with 
  the tape number in which the register was recorded (not used for this   
  program) and the Field Record Number (fldr).  Next comes the source     
  point description (the same Line and Point(ep) from the S-File). Then a 
  sequence of channels (tracf) numbers (just first and last) and a channel
  increment.  Next comes the receiver description using the same Line and 
  Point/station from the R-File).  The Receiver Points are specified as the
  first and last station used at that line.  Finally comes the recording  
  time.   If the spread has a gap, very common for older lines, it will   
  require at least two entries (text lines) to describe this record, one  
  describing the channels and receiver stations before the gap, and another
  for the channels and stations after the gap.   The initial informations 
  (tape, record, and source) are repeated for all entries.   In a 3D there
  is one entry for each line of the patch.                                

  To use this program it is necessary to use all three SPS files.  The    
  S-File and the R-File can describe more points than will be used for the
  processing.  For example, if one is processing just a section of a 2D   
  line the Point Record Specification files can describe all points of the
  complete line, the information in excess will be disregarded.           

  The X-File must have the information in the same order of the input SU  
  file, if not so all files (fldr) that do not match the order will be    
  skipped until the program find the fldr corresponding to the sequence of
  the X-File.                                                             

  Although this program read all of the SPS files, it is not ready for 3D 
  geometry processing.   It is just a basic 2D straight line geometry     
  processing program that (hopefully) will fill correctly the informations
  in header.  It can be used for a crooked line for the coordinates (X, Y 
  Elevation, and element depth), the static correction, and offset.  The  
  cdp numbering will need an specialized program for computation.         

  All coordinates are expected to be an UTM projection in meters or feet. ", 

  The X file order must be the same as the input file.  Upon reading      
  a trace whose trcf is not the next record to be processed in X file,    
  this record, and the following, will be skipped until a match is found  
  for the current entry in X file.   This is a way to, e.g., skip a noisy 
  record, just remove its entry from X file.                              

  If a trace (understand fldr/tracf pair) is not represented in the X file
  it will be skipped until a trace in current fldr matches the current    
  set of X entry for that fldr.  If it is desirable to process just a     
  subset of channels, just keep in the X-File information about those     
  channels.                                                               

  Updated trace header positions:                                         

  tracl - it keeps counting to overcome possible trace skipping.          
  ep    - If zeroed out in header it uses the relation (X) file info.     

  sx, sy, gx, gy -  will be updated with the coordinates in point files   
                    R and S. The scalco value used is the already         
                    stored in header, almost ever equal zero.             
  gelev, selev, sdepth - will be updated with values in point files       
                    R and S. The scalel value used is the already         
                    stored in header, almost ever equal zero.             
  sstat, gstatt - these values are filled with the information of the     
           Point Record Specification files.  If not available will be zero.
  offset - is computed from source and receiver coordinates.  The offset  
           signal is made negative if source station number is greater    
           than the receiver station number.  It consider that source and 
           receiver numbering are the same.   Fractionary stations, e.g.  
           source point halfway between stations are ok.                  
  cdp    - if parameter 'ibin' is passed it is considered the cdp spacing 
           and the cdp number will be computed as follow:                 
           The cdp position is computed as the midpoint between source    
           and receiver.  The distance from this point to the first       
           station of R point file is divided by 'ibin' and added to 100. 
           This 100 is absolutely artitrary.  ;)                          


 Credits:
 Fernando Roxo <petro@roxo.org>

 Trace header fields accessed:             
 fldr, tracf, scalco, scalel, ep, sx, sy, gx, gy, gelev, selev, sdepth, cdp

\end{verbatim}
\pagebreak
\begin{verbatim}
 SUGETHW - sugethw writes the values of the selected key words		

   sugethw key=key1,... [output=] <infile [>outfile]			

 Required parameters:							
 key=key1,...		At least one key word.				

 Optional parameters:							
 output=ascii		output written as ascii for display		
 			=binary for output as binary floats		
 			=geom   ascii output for geometry setting	
 verbose=0 		quiet						
 			=1 chatty					

 Output is written in the order of the keys on the command		
 line for each trace in the data set.					

 Example:								
 	sugethw <stdin key=sx,gx					
 writes sx, gx values as ascii trace by trace to the terminal.		

 Comment: 								
 Users wishing to edit one or more header field (as in geometry setting)
 may do this via the following sequence:				
     sugethw < sudata output=geom key=key1,key2,... > hdrfile 		
 Now edit the ASCII file hdrfile with any editor, setting the fields	
 appropriately. Convert hdrfile to a binary format via:		
     a2b < hdrfile n1=nfields > binary_file				
 Then set the header fields via:					
     sushw < sudata infile=binary_file key=key1,key2,... > sudata.edited


 Credits:

	SEP: Shuki Ronen
	CWP: Jack K. Cohen
      CWP: John Stockwell, added geom stuff, and getparstringarray

\end{verbatim}
\pagebreak
\begin{verbatim}
 SUHTMATH - do unary arithmetic operation on segy traces with 	
	     headers values					

 suhtmath <stdin >stdout				

 Required parameters:						
	none							

 Optional parameter:						
	key=tracl	header word to use			
	op=nop		operation flag				
			nop   : no operation			
			add   : add header to trace		
			mult  : multiply trace with header	
			div   : divide trace by header		

	scale=1.0	scalar multiplier for header value	
	const=0.0	additive constant for header value	

 Operation order:						",      

 op=add:	out(t) = in(t) + (scale * key + const)		
 op=mult:	out(t) = in(t) * (scale * key + const)		
 op=div:	out(t) = in(t) / (scale * key + const)		

 Credits:
	Matthias Imhof, Virginia Tech, Fri Dec 27 09:17:29 EST 2002

\end{verbatim}
\pagebreak
\begin{verbatim}
 SUKEYCOUNT - sukeycount writes a count of a selected key    

   sukeycount key=keyword < infile [> outfile]                  

 Required parameters:                                        
 key=keyword      One key word.                                 

 Optional parameters:                                        
 verbose=0  quiet                                            
        =1  chatty                                           

 Writes the key and the count to the terminal or a text      
   file when a change of key occurs. This does not provide   
   a unique key count (see SUCOUNTKEY for that).             
 Note that for key values  1 2 3 4 2 5                       
   value 2 is counted once per occurrence since this program 
   only recognizes a change of key, not total occurrence.    

 Examples:                                                   
    sukeycount < stdin key=fldr                              
    sukeycount < stdin key=fldr > out.txt                    


 Credits:

   MTU: David Forel, Jan 2005

\end{verbatim}
\pagebreak
\begin{verbatim}
 SULCTHW - Linear Coordinate Transformation of Header Words		

   sulcthw <infile >outfile						

 xt=0.0	Translation of X					
 yt=0.0	Translation of Y					
 zt=0.0	Translation of Z					
 xr=0.0	Rotation around X in degrees	 			
 yr=0.0	Rotation aroun Y  in degrees	 			
 zr=0.0	Rotation around Z in degrees 				

 Notes:								
 Translation:								
 x = x'+ xt;y = y'+ yt;z = z' + zt;					

 Rotations:					  			
 Around Z axis								
 X = x*cos(zr)+y*sin(zr);			  			
 Y = y*cos(zr)-x*sin(zr);			  			
 Around Y axis								
 Z = z*cos(yr)+x*sin(yr);			  			
 X = x*cos(yr)-z*sin(yr);			  			
 Around X axis								
 Y = y*cos(xr)+z*sin(xr);			  			
 Z = Z*cos(xr)-y*sin(xr);			  			

 Header words triplets that are transformed				
 sx,sy,selev								
 gx,gy,gelev								

 The header words restored as 32 bit integers using SEG-Y		
 convention (with coordinate scalers scalco and scalel).		

 After transformation they are converted back to integers and stored.	



  Credits: Potash Corporation of Saskatchewan: Balasz Nemeth   c. 2008



\end{verbatim}
\pagebreak
\begin{verbatim}
SULHEAD - Load information from an ascii column file into HEADERS based
	   on the value of the user specified header field		
  sulhead < inflie > outfile cf=Column_file key=..  [ optional parameters]

 Required parameters:							
 cf=Name of column file						
 key=key1,key2,...Number of column entires				
 Optional parameters:							
 mc=1		Column number to use to match rows to traces		

Notes:									
 Caveat: This is not simple trace header setting, but conditional	
 setting.								

 This utility reads the column file and loads the values into the	
 specified header locations. Each column represents one set of header  
 words, one of them (#mc) is used to match the rows to the traces	
 using header tr.key[mc].						

 Example:								
 key=cdp,ep,sx   mc=1	cf=file						
 file contains:							
	1  2  3								
	2  3  4								

 if tr.cdp = 1 then tr.ep and tr.sx will be set to 2 and 3		
 if tr.cdp = 2 then tr.ep and tr.sx will be set to 3 and 4		
 if tr.cdp=other than tr.trid=3					

 Caveat: the user has to make it sure that number of entires in key=	
	 is equal the number of columns stored in the file.		

 For simple mass setting of header words, see selfdoc of:  sushw	



 Credits: Balasz Nemeth, Potash Corporation, Saskatoon Saskatchewan
 Given to CWP in 2008 


\end{verbatim}
\pagebreak
\begin{verbatim}
 SUPASTE - paste existing SU headers on existing binary data	

 supaste <bare_data >segys  ns= head=headers ftn=0		

 Required parameter:						
	ns=the number of samples per trace			

 Optional parameters:						
 	head=headers	file with segy headers			
	ftn=0		Fortran flag				
			0 = unformatted data from C		
			1 = ... from Fortran			
	verbose=0	1= echo number of traces pasted		
 Caution:							
	An incorrect ns field will munge subsequent processing.	

 Notes:							
 This program is used when the option head=headers is used in	
 sustrip. See:   sudoc sustrip    for more details. 		

 Related programs:  sustrip, suaddhead				

 Credits:
	CWP:  Jack K. Cohen, November 1990

\end{verbatim}
\pagebreak
\begin{verbatim}
 surandhw - set header word to random variable 		

 surandhw <stdin >stdout key=tstat a=0 min=0 max=1		

 Required parameters:						
 	none (no op)						

 Optional parameters:						
 	key=tstat	header key word to set			
 	a=0		=1 flag to add original value to final key
 	noise=gauss	noise probability distribution		
 			=flat for uniform; default Gaussian	
 	seed=from_clock	random number seed (integer)		
 	min=0		minimum random number			
 	max=1		maximum radnom number		 	

 NOTES:							
 The value of header word key is computed using the formula:	
 	val(key) = a * val(key) + rand				

 Example:							
  	surandhw <indata key=tstat a=0 min=0 max=10  > outdata	

\end{verbatim}
\pagebreak
\begin{verbatim}
 SURANGE - get max and min values for non-zero header entries	

 surange <stdin	 					

 Optional parameters:						
	key=		Header key(s) to range (default=all)	

 Note: Gives partial results if interrupted			

 Output is: 							
 number of traces 						
 keyword min max (first - last) 				
 north-south-east-west limits of shot, receiver and midpoint   


 Credits:
      Stanford: Stewart A. Levin
              Added print of eastmost, northmost, westmost,
              southmost coordinates of shots, receivers, and 
              midpoints.  These coordinates have had any
              nonzero coscal header value applied.
	Geocon: Garry Perratt (output one header per line;
		option to specify headers to range;
		added first & last values where min<max)
	Based upon original by:
		SEP: Stew Levin
		CWP: Jack K. Cohen

 Note: the use of "signal" is inherited from BSD days and may
       break on some UNIXs.  It is dicey in that the responsibility
	 for program termination is lateraled back to the main.


\end{verbatim}
\pagebreak
\begin{verbatim}
 SUSEHW - Set the value the Header Word denoting trace number within	
	     an Ensemble defined by the value of another header word	

     susehw <stdin >stdout [options]					

 Required Parameters:							
	none								

 Optional Parameters:							
 key1=cdp	Key header word defining the ensemble			
 key2=cdpt	Key header word defining the count within the ensemble	
 a=1		starting value of the count in the ensemble		
 b=1		increment or decrement within the ensemble		

 Notes:								
 This code was written because suresstat requires cdpt to be set.	
 The computation is 							
 	val(key2) = a + b*i						

 The input data must first be sorted into constant key1 gathers.	
 Example: setting the cdpt field					", 
        susetehw < cdpgathers.su a=1 b=1 key1=cdp key2=cdpt > new.su	


 Credits:
  CWP: John Stockwell (Feb 2008) in answer to a question by Warren Franz
        based on various codes, including susplit, susshw, suchw

\end{verbatim}
\pagebreak
\begin{verbatim}
 SUSHW - Set one or more Header Words using trace number, mod and	
	 integer divide to compute the header word values or input	
	 the header word values from a file				

 ... compute header fields						
   sushw <stdin >stdout key=cdp,.. a=0,..  b=0,.. c=0,.. d=0,.. j=..,..

 ... or read headers from a binary file				
   sushw <stdin > stdout  key=key1,..    infile=binary_file		


 Required Parameters for setting headers from infile:			
 key=key1,key2 ... is the list of header fields as they appear in infile
 infile= 	binary file of values for field specified by		
 		key1,key2,...						

 Optional parameters ():						
 key=cdp,...			header key word(s) to set 		
 a=0,...			value(s) on first trace			
 b=0,...			increment(s) within group		
 c=0,...			group increment(s)	 		
 d=0,...			trace number shift(s)			
 j=ULONG_MAX,ULONG_MAX,...	number of elements in group		

 Notes:								
 Fields that are getparred must have the same number of entries as key	
 words being set. Any field that is not getparred is set to the default
 value(s) above. Explicitly setting j=0 will set j to ULONG_MAX.	

 The value of each header word key is computed using the formula:	
 	i = itr + d							
 	val(key) = a + b * (i % j) + c * (int(i / j))			
 where itr is the trace number (first trace has itr=0, NOT 1)		

 Examples:								
 1. set every dt field to 4ms						
 	sushw <indata key=dt a=4000 |...				
 2. set the sx field of the first 32 traces to 6400, the second 32 traces
    to 6300, decrementing by -100 for each 32 trace groups		
   ...| sushw key=sx a=6400 c=-100 j=32 |...				
 3. set the offset fields of each group of 32 traces to 200,400,...,6400
   ...| sushw key=offset a=200 b=200 j=32 |...				
 4. perform operations 1., 2., and 3. in one call			
  ..| sushw key=dt,sx,offset a=4000,6400,200 b=0,0,200 c=0,-100,0 j=0,32,32 |

 In this example, we set every dt field to 4ms.  Then we set the first	
 32 shotpoint fields to 6400, the second 32 shotpoint fields to 6300 and
 so forth.  Next we set each group of 32 offset fields to 200, 400, ...,
 6400.									

 Example of a typical processing sequence using suchw:			
  sushw <indata key=dt a=4000 |					
  sushw key=sx a=6400 c=-100 j=32 |					
  sushw key=offset a=200 b=200 j=32 |			     		
  suchw key1=gx key2=offset key3=sx b=1 c=1 |		     		
  suchw key1=cdp key2=gx key3=sx b=1 c=1 d=2 >outdata	     		

 Again, it is possible to eliminate the multiple calls to both sushw and
 sushw, as in Example 4.						

 Reading header values from a binary file:				
 If the parameter infile=binary_file is set, then the values that are to
 be set for the fields specified by key=key1,key2,... are read from that
 file. The values are read sequentially from the file and assigned trace
 by trace to the input SU data. The infile consists of C (unformated)	
 binary floats in the form of an array of size (nkeys)*(ntraces) where	
 nkeys is the number of floats in the first (fast) dimension and ntraces
 is the number of traces.						

 Comment: 								
 Users wishing to edit one or more header fields (as in geometry setting)
 may do this via the following sequence:				
     sugethw < sudata output=geom  key=key1,key2 ... > hdrfile 	
 Now edit the ASCII file hdrfile with any editor, setting the fields	
 appropriately. Convert hdrfile to a binary format via:		
     a2b < hdrfile n1=nfields > binary_file				
 Then set the header fields via:					
     sushw < sudata infile=binary_file key=key1,key2,... > sudata.edited

 Caveat: 								
 If the (number of traces)*(number of key words) exceeds the number of	
 values in the infile then the user may still set a single header field
 on the remaining traces via the parameters key=keyword a,b,c,d,j.	

 Example:								
    sushw < sudata=key1,key2 ... infile=binary_file [Optional Parameters]

 Credits:
	SEP: Einar Kajartansson
	CWP: Jack K. Cohen
      CWP: John Stockwell, added multiple fields and infile= options

 Caveat:
	All constants are cast to doubles.

\end{verbatim}
\pagebreak
\begin{verbatim}
 SUSTRIP - remove the SEGY headers from the traces		

 sustrip <stdin >stdout head=/dev/null outpar=/dev/tty ftn=0	

 Required parameters:						
 	none							

 Optional parameters:						
 	head=/dev/null		file to save headers in		

 	outpar=/dev/tty		output parameter file, contains:
 				number of samples (n1=)		
 				number of traces (n2=)		
 				sample rate in seconds (d1=)	

 	ftn=0			Fortran flag			
 				0 = write unformatted for C	
 				1 = ... for Fortran		

 Notes:							
 Invoking head=filename will write the trace headers into filename.
 You may paste the headers back onto the traces with supaste	
 See:  sudoc  supaste 	 for more information 			
 Related programs: supaste, suaddhead				

 Credits:
	SEP: Einar Kjartansson  c. 1985
	CWP: Jack K. Cohen        April 1990

 Trace header fields accessed: ns, dt

\end{verbatim}
\pagebreak
\begin{verbatim}
 SUTRCOUNT - SU program to count the TRaces in infile		

   sutrcount < infile					     	
 Required parameters:						
       none							
 Optional parameter:						
    outpar=stdout						
 Notes:       							
 Once you have the value of ntr, you may set the ntr header field
 via:      							
       sushw key=ntr a=NTR < datain.su  > dataout.su 		
 Where NTR is the value of the count obtained with sutrcount 	


 Credits:  B.Nemeth, Potash Corporation, Saskatchewan 
 		given to CWP in 2008 with permission of Potash Corporation


\end{verbatim}
\pagebreak
\begin{verbatim}
 SUUTM - UTM projection of longitude and latitude in SU trace headers  

 suutm <stdin >stdout [optional parameters]                            

 Optional parameters:                                                  
    counit=(from header) input coordinate units code:                  
                    =1: length (meters or feet; no UTM projection)     
                    =2: seconds of arc                                 
                    =3: decimal degrees                                
                    =4: degrees, minutes, seconds                      
    idx=23          reference ellipsoid index (default is WGS 1984)    
    a=(from idx)    user-specified semimajor axis of ellipsoid         
    f=(from idx)    user-specified flattening of ellipsoid             
    zkey=           if set, header key to store UTM zone number        
    verbose=0       =1: echo ellipsoid parameters                      

    lon0=           central meridian for TM projection in degrees      
                    (default uses the 60 standard UTM longitude zones) 
    xoff=500000     false Easting (default: UTM)                       
    ysoff=10000000  false Northing, southern hemisphere (default: UTM) 
    ynoff=0         false Northing, northern hemisphere (default: UTM) 

 Notes:                                                                
    Universal Transverse Mercator (UTM) coordinates are defined between
    latitudes 80S (-80) and 84N (84). Longitude values must be between 
    -180 degrees (west) and 179.999... degrees (east).                 

    Latitudes are read from sy and gy (N positive), and longitudes     
    are read from sx and gx (E positive).                              
    The UTM zone is determined from the receiver coordinates gy and gx.

    Use suazimuth to calculate shot-receiver azimuths and offsets.     

 Reference ellipsoids:                                                 
    An ellipsoid may be specified by its semimajor axis a and its      
    flattening f, or one of the following ellipsoids may be selected   
    by its index idx (semimajor axes in meters):                       
     0  Sphere with radius of 6371000 m                                
     1  Airy 1830                                                      
     2  Australian National 1965                                       
     3  Bessel 1841 (Ethiopia, Indonesia, Japan, Korea)                
     4  Bessel 1841 (Namibia)                                          
     5  Clarke 1866                                                    
     6  Clarke 1880                                                    
     7  Everest (Brunei, E. Malaysia)                                  
     8  Everest (India 1830)                                           
     9  Everest (India 1956)                                           
    10  Everest (Pakistan)                                             
    11  Everest (W. Malaysia, Singapore 1948)                          
    12  Everest (W. Malaysia 1969)                                     
    13  Geodetic Reference System 1980 (GRS 1980)                      
    14  Helmert 1906                                                   
    15  Hough 1960                                                     
    16  Indonesian 1974                                                
    17  International 1924 / Hayford 1909                              
    18  Krassovsky 1940                                                
    19  Modified Airy                                                  
    20  Modified Fischer 1960                                          
    21  South American 1969                                            
    22  World Geodetic System 1972 (WGS 1972)                          
    23  World Geodetic System 1984 (WGS 1984) / NAD 1983               


 UTM grid:
 The Universal Transverse Mercator (UTM) system is a world wide
 coordinate system defined between 80S and 84N. It divides the
 Earth into 60 six-degree zones. Zone number 1 has its central
 meridian at 177W (-177 degrees), and numbers increase eastward.

 Within each zone, an Easting of 500,000 m is assigned to its 
 central meridian to avoid negative coordinates. On the northern
 hemisphere, Northings start at 0 m at the equator and increase 
 northward. On the southern hemisphere a false Northing of 
 10,000,000 m is applied, i.e. Northings start at 10,000,000 m at 
 the equator and decrease southward.

 Coordinate encoding (sx,sy,gx,gy):
    counit=1  units of length (coordinates are not converted)
    counit=2  seconds of arc
    counit=3  decimal degrees 
    counit=4  degrees, minutes and seconds encoded as integer DDDMMSS 
              with scalco=1 or DDDMMSS.ss with scalco=-100 (see segy.h)
 Units of length are also assumed, if counit <= 0 or counit >= 5.


 Author: 
    Nils Maercklin, RISSC, University of Naples, Italy, March 2007

 References:
 NIMA (2000). Department of Defense World Geodetic System 1984 - 
    its definition and relationships with local geodetic systems.
    Technical Report TR8350.2. National Imagery and Mapping Agency, 
    Geodesy and Geophysics Department, St. Louis, MO. 3rd edition.
 J. P. Snyder (1987). Map Projections - A Working Manual. 
    U.S. Geological Survey Professional Paper 1395, 383 pages.
    U.S. Government Printing Office.


 Trace header fields accessed: sx, sy, gx, gy, scalco, counit
 Trace header fields modified: sx, sy, gx, gy, scalco, counit

\end{verbatim}
\pagebreak
\begin{verbatim}
 SUXEDIT - examine segy diskfiles and edit headers			

 suxedit diskfile  (open for possible header modification if writable)	
 suxedit <diskfile  (open read only)					

 The following commands are recognized:				
 number	read in that trace and print nonzero header words	
 <CR>		go to trace one step away (step is initially -1)	
 +		read in next trace (step is set to +1)			
 -		read in previous trace (step is set to -1)		
 dN		advance N traces (step is set to N)			
 %		print some percentiles of the trace data		
 r		print some ranks (rank[j] = jth smallest datum) 	
 p [n1 [n2]]  	tab plot sample n1 to n2 on current trace		
 g [tr1 tr2] ["opts"] 	wiggle plot (graph) the trace		
				[traces tr1 to tr2]			
 f		wiggle plot the Fourier transform of the trace		
 ! key=val  	change a value in a field (e.g. ! tracr=101)		
 ?		print help file						
 q		quit							

 NOTE: sample numbers are 1-based (first sample is 1).			


 Credits:
	SEP: Einar Kjartansson, Shuki Ronen, Stew Levin
	CWP: Jack K. Cohen

 Trace header fields accessed: ns
 Trace header fields modified: ntr (only for internal plotting)

\end{verbatim}
\pagebreak
\begin{verbatim}
 SUINTERPFOWLER - interpolate output image from constant velocity panels
	   built by SUTIFOWLER or CVS					

 These parameters should be specified the same as in SUTIFOWLER:	
 vmin=1500.		minimum velocity				
 vmax=2500.		maximum velocity				
 nv=21			number of velocity panels			
 etamin=0.10		minimum eta value				
 etamax=0.25		maximum eta value				
 neta=11		number of eta values				
 ncdps=1130		number of cdp points				

 If these parameters are specified so that nvstack>5, then the input 	
 data are assumed to come from CVS and the SUTIFOWLER parameters are ignored.
 nvstack=0		number of constant velocity stack panels output by CVS
 vminstack=1450	minimum velocity specified for CVS		
 vscale=1.0		scale factor for velocity functions		

 These parameters specify the desired output (time,velocity,eta) model	
 at each cdp location. The sequential cdp numbers should be specified in
 increasing order from 0 to 'ncdps-1' at from 1 to 'ncdps' control point
 locations. (Time values are in seconds.)				
 cdp=0			cdp number for (t,v,eta) triplets (specify more than
 				once if needed)				
 t=0.			array of times for (t,v,eta) triplets (specify more
				than once if needed)			
 v=1500.		array of velocities for (t,v,eta) triplets (specify
				more than once if needed)		
 eta=0.		array of etas for (t,v,eta) triplets (specify more
				than once if needed)			

 Note: This is a simple research code based on linear interpolation.	
 There are no protections against aliasing built into the code beyond	
 suggesting that this program have a knowledgable user. A final version
 should do a better job taking care of endpoint conditions.



 Author: (Visitor from Mobil) John E. Anderson, Spring 1994 

\end{verbatim}
\pagebreak
\begin{verbatim}
 SUINTERP - interpolate traces using automatic event picking		

           suinterp < stdin > stdout					

 ninterp=1    number of traces to output between each pair of input traces
 nxmax=500    maximum number of input traces				
 freq1=4.     starting corner frequency of unaliased range		
 freq2=20.    ending corner frequency of unaliased range		
 deriv=0      =1 means take vertical derivative on pick section        
              (useful if interpolating velocities instead of seismic)  
 linear=0     =0 means use 8 point sinc temporal interpolation         
              =1 means use linear temporal interpolation               
              (useful if interpolating velocities instead of seismic)  
 lent=5       number of time samples to smooth for dip estimate	
 lenx=1       number of traces to smooth for dip estimate		
 lagc=400     number of ms agc for dip estimate			
 xopt=0       0 compute spatial derivative via FFT			
                 (assumes input traces regularly spaced and relatively	
                  noise-free)						
              1 compute spatial derivative via differences		
                 (will work on irregulary spaced data)			
 iopt=0     0 = interpolate
            1 = output low-pass model: useful for QC if interpolator failing
            2 = output dip picks in units of samples/trace		

 verbose=0	verbose = 1 echoes information				

 tmpdir= 	 if non-empty, use the value as a directory path	
		 prefix for storing temporary files; else if the	
	         the CWP_TMPDIR environment variable is set use		
	         its value for the path; else use tmpfile()		

 Notes:								
 This program outputs 'ninterp' interpolated traces between each pair of
 input traces.  The values for lagc, freq1, and freq2 are only used for
 event tracking. The output data will be full bandwidth with no agc.  The
 default parameters typically will do a satisfactory job of interpolation
 for dips up to about 12 ms/trace.  Using a larger value for freq2 causes
 the algorithm to do a better job on the shallow dips, but to fail on the
 steep dips.  Only one dip is assumed at each time sample between each pair
 of input traces.							

 The key assumption used here is that the low frequency data are unaliased
 and can be used for event tracking. Those dip picks are used to interpolate
 the original full-bandwidth data, giving some measure of interpolation
 at higher frequencies which otherwise would be aliased.  Using iopt equal
 to 1 allows you to visually check whether the low-pass picking model is
 aliased.								

 Trace headers for interpolated traces are not updated correctly.	
 The output header for an interpolated traces equals that for the preceding
 trace in the original input data.  The original input traces are passed
 through this module without modification.				

 The place this code is most likely to fail is on the first breaks.	

 Example run:    suplane | suinterp | suxwigb &			



 Credit: John Anderson (visiting scholar from Mobil) July 1994

 Trace header fields accessed: ns, dt


\end{verbatim}
\pagebreak
\begin{verbatim}
 SUOCEXT - smaller Offset EXTrapolation via Offset Continuation        
           method for common-offset gathers                            

 suocext <stdin >stdout cdpmin= cdpmax= dxcdp= noffmix= offextr= [...]	

 Required Parameters:							
 cdpmin=	minimum cdp (integer number) for which to apply DMO	
 cdpmax=	maximum cdp (integer number) for which to apply DMO	
 dxcdp=	distance between adjacent cdp bins (m)			
 noffmix=	number of offsets to mix (see notes)			
 offextr=	offset to extrapolate					

 Optional Parameters:							
 tdmo=0.0	times corresponding to rms velocities in vdmo (s)	
 vdmo=1500.0	rms velocities corresponding to times in tdmo (m/s)	
 sdmo=1.0	DMO stretch factor; try 0.6 for typical v(z)		
 fmax=0.5/dt	maximum frequency in input traces (Hz)			
 verbose=0	=1 for diagnostic print					
 tmpdir=	if non-empty, use the value as a directory path	prefix	
		for storing temporary files; else if the CWP_TMPDIR	
		environment variable is set use	its value for the path;	
		else use tmpfile()					

 Notes:								
 Input traces should be sorted into common-offset gathers.  One common- 
 offset gather ends and another begins when the offset field of the trace
 headers changes. One common-offset gather usually is enough.		

 The cdp field of the input trace headers must be the cdp bin NUMBER, NOT
 the cdp location expressed in units of meters or feet.		

 The number of offsets to mix (noffmix) should typically equal the ratio of
 the shotpoint spacing to the cdp spacing.  This choice ensures that every
 cdp will be represented in each offset mix.  Traces in each mix will	
 contribute through DMO to other traces in adjacent cdps within that mix.

 The tdmo and vdmo arrays specify a velocity function of time that is	
 used to implement a first-order correction for depth-variable velocity.
 The times in tdmo must be monotonically increasing.			

 For each offset, the minimum time at which a non-zero sample exists is 
 used to determine a mute time.  Output samples for times earlier than this", 
 mute time will be zeroed.  Computation time may be significantly reduced
 if the input traces are zeroed (muted) for early times at large offsets.

 A term for better amplitude reconstruction was added to Hale's formulation.

 Credits: Carlos E. Theodoro (modification of Hale's SUDMOFK program)

 Technical Reference:
	C. Theodoro & K. Larner, 1998
      Extrapolation of seismic data to small offsets (CWP-276). 

	Dip-Moveout Processing - SEG Course Notes
	Dave Hale, 1988

	Bleistein, Cohen & Jaramillo, 1997
      True amplitude transformation to zero offset of data from 
      curved reflectors (CWP-262). 

 Trace header fields accessed:  ns, dt, delrt, offset, cdp.
 Trace header fields modified:  offset.

\end{verbatim}
\pagebreak
\begin{verbatim}
 SUGAZMIGQ - SU version of Jeno GAZDAG's phase-shift migration 	
	     for zero-offset data, with attenuation Q.			

 sugazmig <infile >outfile vfile= [optional parameters]		

 Optional Parameters:							
 dt=from header(dt) or	.004	time sampling interval			
 dx=from header(d2) or 1.0	midpoint sampling interval		
 ft=0.0			first time sample			
 ntau=nt(from data)	number of migrated time samples			
 dtau=dt(from header)	migrated time sampling interval			
 ftau=ft		first migrated time sample			
 tmig=0.0		times corresponding to interval velocities in vmig
 vmig=1500.0	interval velocities corresponding to times in tmig	
 vfile=		name of file containing velocities		
 Q=1e6			quality factor					
 ceil=1e6		gain ceiling beyond which migration ceases	

 verbose=0	verbose = 1 echoes information				

 tmpdir= 	 if non-empty, use the value as a directory path	
		 prefix for storing temporary files; else if the	
	         the CWP_TMPDIR environment variable is set use		
	         its value for the path; else use tmpfile()		

 Note: ray bending effects not accounted for in this version.		

 The tmig and vmig arrays specify an interval velocity function of time.
 Linear interpolation and constant extrapolation is used to determine	
 interval velocities at times not specified.  Values specified in tmig	
 must increase monotonically.						

 Alternatively, interval velocities may be stored in a binary file	
 containing one velocity for every time sample in the data that is to be
 migrated.  If vfile is specified, then the tmig and vmig arrays are ignored.

 Caveat: Adding Q is a first attempt to address GPR issues.		

 
 Credits: 
  Constant Q attenuation correction by Chuck Oden 5 May 2004
  CWP John Stockwell 12 Oct 1992
 	Based on a constant v version by Dave Hale.

 Trace header fields accessed: ns, dt, delrt, d2
 Trace header fields modified: ns, dt, delrt
 
\end{verbatim}
\pagebreak
\begin{verbatim}
 SUINVXZCO - Seismic INVersion of Common Offset data for a smooth 	
             velocity function V(X,Z) plus a slowness perturbation vp(x,z)

     suinvvxzco <infile >outfile [optional parameters] 		

 Required Parameters:							
 vfile                  file containing velocity array v[nx][nz]	
 nx=                    number of x samples (2nd dimension) in velocity
 nz=                    number of z samples (1st dimension) in velocity
 nxm=			number of midpoints of input traces		

 Optional Parameters:							
 dt= or from header (dt) 	time sampling interval of input data	
 offs= or from header (offset) 	source-receiver offset	 	
 dxm= or from header (d2) 	sampling interval of midpoints 		
 fxm=0		first midpoint in input trace				
 nxd=5		skipped number of midpoints (see note)			
 dx=50.0	x sampling interval of velocity				
 fx=0.0	first x sample of velocity				
 dz=50.0	z sampling interval of velocity				
 nxb=nx/2	band centered at midpoints (see note)			
 nxc=0		hozizontal range in which velocity is changed		
 nzc=0		vertical range in which velocity is changed		
 fxo=0.0	x-coordinate of first output trace 			
 dxo=15.0	horizontal spacing of output trace 			
 nxo=101	number of output traces 				",	
 fzo=0.0	z-coordinate of first point in output trace 		
 dzo=15.0	vertical spacing of output trace 			
 nzo=101	number of points in output trace			",	
 fmax=0.25/dt	Maximum frequency set for operator antialiasing		
 ang=180	Maximum dip angle allowed in the image			
 ls=0		=1 for line source; =0 for point source			
 pert=0	=1 calculate time correction from v_p[nx][nz]		
 vpfile	file containing slowness perturbation array v_p[nx][nz]	
 verbose=1              =1 to print some useful information		

 Notes:								
 Traveltime and amplitude are calculated by finite difference which	
 is done only in one of every NXD midpoints; in the skipped midpoint, 	
 interpolation is used to calculate traveltime and amplitude.		", 
 For each midpoint, traveltime and amplitude are calculated in the 	
 horizontal range of (xm-nxb*dx, xm+nxb*dx). Velocity is changed by 	
 constant extropolation in two upper trianglar corners whose width is 	
 nxc*dx and height is nzc*dz.						

 Eikonal equation will fail to solve if there is a polar turned ray.	
 In this case, the program shows the related geometric information. 	
 There are three ways to remove the turned ray: smoothing velocity, 	
 reducing nxb, and increaing nxc and nzc (if the turned ray occurs  	
 in the shallow areas). To prevent traveltime distortion from a over	
 smoothed velocity, traveltime is corrected based on the slowness 	
 perturbation.								

 Offsets are signed - may be positive or negative. 			




 Author:  Zhenyue Liu, 08/28/93,  Colorado School of Mines 

 Reference:
 Bleistein, N., Cohen, J. K., and Hagin, F., 1987,
  Two-and-one-half dimensional Born inversion with an arbitrary reference
         Geophysics Vol. 52, no.1, 26-36.


\end{verbatim}
\pagebreak
\begin{verbatim}
 SUINVZCO3D - Seismic INVersion of Common Offset data with V(Z) velocity
             function in 3D						

     suinvzco3d <infile >outfile [optional parameters] 		

 Required Parameters:							
 vfile                  file containing velocity array v[nz]		
 nz=                    number of z samples (1st dimension) in velocity
 nxm=			number of midpoints of input traces		
 ny=			number of lines 				

 Optional Parameters:							
 dt= or from header (dt) 	time sampling interval of input data	
 offs= or from header (offset) 	source-receiver offset	 	
 dxm= or from header (d2) 	sampling interval of midpoints 		
 fxm=0                  first midpoint in input trace			
 nxd=5			skipped number of midpoints (see note)		
 dx=50.0                x sampling interval of velocity		
 fx=0.0                 first x sample of velocity			
 dz=50.0                z sampling interval of velocity		
 nxb=nx/2		band centered at midpoints (see note)		
 fxo=0.0                x-coordinate of first output trace 		
 dxo=15.0		horizontal spacing of output trace 		
 nxo=101                number of output traces 			",	
 fyo=0.0		y-coordinate of first output trace		
 dyo=15.0		y-coordinate spacing of output trace		
 nyo=101		number of output traces in y-direction		
 fzo=0.0                z-coordinate of first point in output trace 	
 dzo=15.0               vertical spacing of output trace 		
 nzo=101                number of points in output trace		",	
 fmax=0.25/dt		Maximum frequency set for operator antialiasing 
 ang=180		Maximum dip angle allowed in the image		
 verbose=1              =1 to print some useful information		

 Notes:									

 This algorithm is based on formula (50) in Geophysics Vol. 51, 	
 1552-1558, by Cohen, J., Hagin, F., and Bleistein, N.			

 Traveltime and amplitude are calculated by ray tracing.		
 Interpolation is used to calculate traveltime and amplitude.		", 
 For each midpoint, traveltime and amplitude are calculated in the 	
 horizontal range of (xm-nxb*dx, xm+nxb*dx). Velocity is changed by 	
 linear interpolation in two upper trianglar corners whose width is 	
 nxc*dx and height is nzc*dz.						",	

 Eikonal equation will fail to solve if there is a polar turned ray.	
 In this case, the program shows the related geometric information. 	
 
 Offsets are signed - may be positive or negative. 			", 

\end{verbatim}
\pagebreak
\begin{verbatim}
SUKDMIG2D - Kirchhoff Depth Migration of 2D poststack/prestack data	

    sukdmig2d  infile=  outfile=  ttfile=   [parameters] 		

 Required parameters:							
 infile=stdin		file for input seismic traces			
 outfile=stdout	file for common offset migration output  	
 ttfile=		file for input traveltime tables		

 ...  The following 9 parameters describe traveltime tables:		
 fzt= 			first depth sample in traveltime table		
 nzt= 			number of depth samples in traveltime table	
 dzt=			depth interval in traveltime table		
 fxt=			first lateral sample in traveltime table	
 nxt=			number of lateral samples in traveltime table	
 dxt=			lateral interval in traveltime table		
 fs= 			x-coordinate of first source			
 ns= 			number of sources				
 ds= 			x-coordinate increment of sources		

 Optional Parameters:							
 dt= or from header (dt) 	time sampling interval of input data	
 ft= or from header (ft) 	first time sample of input data		
 dxm= or from header (d2) 	sampling interval of midpoints 		
 fzo=fzt		    z-coordinate of first point in output trace	
 dzo=0.2*dzt		vertical spacing of output trace 		
 nzo=5*(nzt-1)+1 	number of points in output trace		",	
 fxo=fxt		    x-coordinate of first output trace 		
 dxo=0.5*dxt		horizontal spacing of output trace 		
 nxo=2*(nxt-1)+1  	number of output traces 			",	
 off0=0		   	first offest in output 			
 doff=99999		offset increment in output 			
 noff=1	 	number of offsets in output 			",	
 absoff=0      flag for using absolute offsets of input traces		
               =0 means use offset=gx-sx                		
               =1 means use abs(gx-sx)                  		
 limoff=0      flag for only using input traces that fall within the range
               of defined output offset bins (off0,doff,noff) 		
               =0 means use all input traces                 		
               =1 means limit traces used by offset           		
 fmax=0.25/dt		frequency-highcut for input traces		
 offmax=99999		maximum absolute offset allowed in migration 	
 aperx=nxt*dxt/2  	migration lateral aperature 			
 angmax=60		migration angle aperature from vertical 	
 v0=1500(m/s)		reference velocity value at surface		",	
 dvz=0.0  		reference velocity vertical gradient		

 ls=1			flag for line source				
 jpfile=stderr		job print file name 				

 mtr=100  		print verbal information at every mtr traces	
 ntr=100000		maximum number of input traces to be migrated	
 npv=0			flag of computing quantities for velocity analysis
 rscale=1000.0 	scaling for roundoff error suppression		

   ...if npv>0 specify the following three files:			
 tvfile=tvfile		input file of traveltime variation tables	
			tv[ns][nxt][nzt]				
 csfile=csfile		input file of cosine tables cs[ns][nxt][nzt]	
 outfile1=outfile1	file containning additional migration output   	
			with extra amplitude				

 Notes:								
 1. Traveltime tables were generated by program rayt2d (or other ones)	
    on relatively coarse grids, with dimension ns*nxt*nzt. In the	
    migration process, traveltimes are interpolated into shot/gephone 	
    positions and output grids.					
 2. Input seismic traces must be SU format and can be any type of 	
    gathers (common shot, common offset, common CDP, and so on).	", 
 3. Migrated traces are output in CDP gathers if velocity analysis	
    is required, with dimension nxo*noff*nzo.  			", 
 4. If the offset value of an input trace is not in the offset array 	
    of output, the nearest one in the array is chosen. 		
 5. Memory requirement for this program is about			
    	[ns*nxt*nzt+noff*nxo*nzo+4*nr*nzt+5*nxt*nzt+npa*(2*ns*nxt*nzt   
	+noff*nxo*nzo+4*nxt*nzt)]*4 bytes				
    where nr = 1+min(nxt*dxt,0.5*offmax+aperx)/dxo. 			
 6. Amplitudes are computed using the reference velocity profile, v(z),
    specified by the parameters v0= and dvz=.				
 7. Input traces must specify source and receiver positions via the header
    fields tr.sx and tr.gx. Offset is computed automatically.		
 8. if limoff=0, input traces from outside the range defined by off0, doff, 
    noff, will get migrated into the extremal offset bins/planes.  E.g. if 
    absoff=0 and limoff=0, all traces with gx<sx will get migrated into the 
    off0 bin.


 Author:  Zhenyue Liu, 03/01/95,  Colorado School of Mines 
 Modifcations:
    Gary Billings, Talisman Energy, Sept 2005:  added absoff, limoff.

 Trace header fields accessed: ns, dt, delrt, d2
 Trace header fields modified: sx, gx

 
\end{verbatim}
\pagebreak
\begin{verbatim}
SUKDMIG3D - Kirchhoff Depth Migration of 3D poststack/prestack data	

 	sukdmig3d datain= dataout= [parameters] 			

 Required parameters:							
 ttfile	 file for input tttables			       	

 Optional Parameters:							

 datain=stdin	 file for input seismic traces				
 dataout=stdout file for common offset migration output		
 crfile=NULL    file for cos theta and ray paths                       

   The following 17 parameters describe tttables: (from ttfile header)	
 fxgd= or from header (f1)		first x-sample in tttable	
 nxt= or from header (ns)		number of x-samples in tttable	
 dxgd= or from header (d1)		x-interval in tttable 		
 fygd= or from header (f2)		first y-sample in tttable 	
 nyt= or from header (ntr)		number of y-samples in tttable	
 dygd= or from header (d2)		y-interval in tttable		
 ixsf= or from header (sdel) 	        x in dxgd of first source	
 nxs= or from header (nhs) 		number of sources in x		
 ixsr= or from header (swdep) 	        ratio of source & gd spacing    
 iysf= or from header (gdel)	        y in dygd of first source 	
 nys= or from header (nvs)		number of sources in y          
 iysr= or from header (gwdep)	        ratio of source & gd spacing    
 fzs= or from header (sdepth/1000)	first depth sample in tttable	
 nzs= or from header (duse) 		number of depth samples in tttable
 dzs= or from header (ep/1000)		depth interval in tttable	
 nxgd= or from header (selev)		x size of the traveltime region 
 nygd= or from header (gelev)		y size of the traveltime region 
 multit= or from header (scalel)       number of multivalued traveltime

   The following two parameters are from data header			
 dt= or from header (dt)	time sampling interval of input data	
 ft= or from header (ft)	first time sample of input data		
 dxm= or from header (d2)      mid point spacing of input data         

 Default: output is 5 times finer in depth and 2 times finer in lateral
 fzo=fzs	z-coordinate of first point in output trace		
 dzo=0.2*dzs	vertical spacing of output trace (5 times finer)	
 nzo=5*(nzs-1)+1 number of points in output trace (5 times finer)	",	
 fxo=fxgd	x-coordinate of first output trace	 		
 dxo=0.5*dxgd	horizontal spacing of output trace (2 times finer)	
 nxo=2*(nxgd-1)+1 number of output traces (2 times finer)		
 fyo=fygd	y-coordinate of first output trace			
 dyo=0.5*dygd	horizontal spacing of output trace (2 times finer)	
 nyo=2*(nygd-1)+1 number of output traces (2 times finer)		

 Default: poststack migration						",	
 fxoffset=0		first offest in output in x			
 fyoffset=0		first offest in output in y			
 dxoffset=99999	offset increment in output in x	 		
 dyoffset=99999	offset increment in output in y			
 nxoffset=1		number of offsets in output in x		
 nyoffset=1		number of offsets in output in y		",	
 xoffsetmax=99999	x-maximum absolute offset allowed in migration 	
 yoffsetmax=99999	y-maximum absolute offset allowed in migration  
 xaper=nxt*dxgd/2.5	migration lateral aperature in x		
 yaper=nyt*dygd/2.5	migration lateral aperature in y		
 angmax=60             max angle to handle                             
 fmax=0.25/dt          max frequency in the data                       
 jpfile=stderr		job print file name 				
 pptr=100		print verbal information at every pptr traces	
 ntrmax=100000		maximum number of input traces to be migrated	
 ls=0                  point =0 line source =1                         

 Notes:								
 1. Traveltime tables were generated by program SUTETRARAY (or other	
    ones) on very sparse tetrahedral model, with dimension nys*nxs*nzs 
    *nyt*nxt.                                                          
 2. Input seismic traces must be SU format and can be any type of 	
    gathers (common shot, common offset, common CDP, and so on).	", 
 3. Migrated traces are output in CDP gathers if velocity analysis	
    is required, with dimension nyoffset*nxoffset*nyo*nxo*nzo.		", 
 4. If the offset value of an input trace is not in the offset array 	
    of output, the nearest one in the array is chosen. 		
 5. Memory requirement for this program is about			
    [nys*nxs*nzs*nyt*nxt+nyoffset*nxoffset*nxo*nyo*nzo+	       	
    nys*nxo*nzo*nyt*nxt]                                               
 6. Input traces must specify source and receiver positions via header	
    fields tr.sx and tr.gx, as well as tr.sy and tr.gy. Offset is 	
    computed automatically.						

 Disclaimer:								
 This is a research code that will take considerable work to get into	
 the form of a a production level 3D migration code. The code is	
 offered as is, along with tetramod and sutetraray, to provide a starting
 point for researchers who wish to write their own 3D migration codes.



 Author:  Zhaobo Meng, 01/10/97,  Colorado School of Mines 

 Trace header fields accessed: ns, dt, delrt, d2
 Trace header fields modified: sx, gx

\end{verbatim}
\pagebreak
\begin{verbatim}
 SUKTMIG2D - prestack time migration of a common-offset	
	section with the double-square root (DSR) operator	


   suktmig2d < infile vfile= [parameters]  > outfile		

 Required Parameters:						
 vfile=	rms velocity file (units/s) v(t,x) as a function
		of time						
 dx=		distance (units) between consecutive traces	

 Optional parameters:						
 fcdpdata=tr.cdp	first cdp in data			
 firstcdp=fcdpdata	first cdp number in velocity file	
 lastcdp=from header	last cdp number in velocity file	
 dcdp=from header	number of cdps between consecutive traces
 angmax=40	maximum aperture angle for migration (degrees)	
 hoffset=.5*tr.offset		half offset (m)			
 nfc=16	number of Fourier-coefficients to approximate	
		low-pass					
		filters. The larger nfc the narrower the filter	
 fwidth=5 	high-end frequency increment for the low-pass	
 		filters						
 		in Hz. The lower this number the more the number
		of lowpass filters to be calculated for each 	
		input trace.					

 Caveat: this code may need some work				
 Notes:							
 Data must be preprocessed with sufrac to correct for the	
 wave-shaping factor using phasefac=.25 for 2D migration.	

 Input traces must be sorted into offset and cdp number.	
 The velocity file consists of rms velocities for all CMPs as a
 function of vertical time and horizontal position v(t,x)	
 in C-style binary floating point numbers.  It's easiest to 	
 supply v(t,x) that has the same dimensions as the input data to
 be migrated. Note that time t is the fast dimension in these  
 the input velocity file.					

 The units may be feet or meters, as long as these are		
 consistent.							
 Antialias filter is performed using (Gray,1992, Geoph. Prosp), 
 using nc low- pass filtered copies of the data. The cutoff	
 frequencies are calculated  as fractions of the Nyquist	
 frequency.							

 The maximum allowed angle is 80 degrees(a 10 degree taper is 
 applied to the end of the aperture)				

define LOOKFAC 2       /* Look ahead factor for npfaro  
define PFA_MAX 720720  /* Largest allowed nfft	  


 Prototype of functions used internally
void lpfilt(int nfc, int nfft, float dt, float fhi, float *filter);

segy intrace; 	/* input traces
segy outtrace;	/* migrated output traces

int
main(int argc, char **argv)
{
	int i,k,imp,iip,it,ix,ifc;	/* counters
	int ntr,nt;			/* x,t

	int verbose;	/* is verbose?				*/
	int nc;		/* number of low-pass filtered versions	*/
			/*  of the data for antialiasing	*/
	int nfft,nf;	/* number of frequencies		*/
	int nfc;	/* number of Fourier coefficients for low-pass filter
	int fwidth;	/* high-end frequency increment for the low-pass
			/* filters 				*/
	int firstcdp=0;	/* first cdp in velocity file		*/
	int lastcdp=0;	/* last cdp in velocity file		*/
	int oldcdp=0;	/* temporary storage			*/
	int fcdpdata=0;	/* first cdp in the data		*/
	int olddeltacdp=0;
	int deltacdp;
	int ncdp=0;	/* number of cdps in the velocity file	*/
	int dcdp=0;	/* number of cdps between consecutive traces

	float dx=0.0;	/* cdp sample interval
	float hoffset=0.0;  /* half receiver-source
	float p=0.0;	/* horizontal slowness of the migration operator
	float pmin=0.0;	/* maximum horizontal slowness for which there's
			/* no aliasing of the operator
	float dt;	/* t sample interval
	float h;	/* offset
	float x;	/* aperture distance
	float xmax=0.0;	/* maximum aperture distance

	float obliq;	/* obliquity factor
	float geoms;	/* geometrical spreading factor
	float angmax;   /* maximum aperture angle

	float mp,ip;	/* mid-point and image-point coordinates
	float t;	/* time
	float t0;	/* vertical traveltime
	float tmax;	/* maximum time

	float fnyq;	/* Nyquist frequency
	float ang;	/* aperture angle
	float angtaper=0.0;	/* aperture-angle taper
	float v;		/* velocity
  
	float *fc=NULL;		/* cut-frequencies for low-pass filters
	float *filter=NULL;	/* array of low-pass filter values

	float **vel=NULL;	/* array of velocity values from vfile
	float **data=NULL;	/* input data array*/
	float **lowpass=NULL;   /* low-pass filtered version of the trace
	float **mig=NULL;	/* output migrated data array

	register float *rtin=NULL,*rtout=NULL;/* real traces
	register complex *ct=NULL;   /* complex trace

	/* file names
	char *vfile="";		/* name of velocity file
	FILE *vfp=NULL;
	FILE *tracefp=NULL;	/* temp file to hold traces*/
	FILE *hfp=NULL;		/* temp file to hold trace headers

	float datalo[8], datahi[8];
	int itb, ite;
	float firstt, amplo, amphi;

	cwp_Bool check_cdp=cwp_false;	/* check cdp in velocity file	*/

	/* Hook up getpar to handle the parameters
	initargs(argc,argv);
	requestdoc(0);
	
	/* Get info from first trace
	if (!gettr(&intrace))  err("can't get first trace");
	nt=intrace.ns;
	dt=(float)intrace.dt/1000000;
	tmax=(nt-1)*dt;

	MUSTGETPARFLOAT("dx",&dx);
	MUSTGETPARSTRING("vfile",&vfile);
	if (!getparfloat("angmax",&angmax)) angmax=40;
	if (!getparint("firstcdp",&firstcdp)) firstcdp=intrace.cdp;
	if (!getparint("fcdpdata",&fcdpdata)) fcdpdata=intrace.cdp;
	if (!getparfloat("hoffset",&hoffset)) hoffset=.5*intrace.offset;
	if (!getparint("nfc",&nfc)) nfc=16;
	if (!getparint("fwidth",&fwidth)) fwidth=5;
	if (!getparint("verbose",&verbose)) verbose=0;

	h=hoffset;

	/* Store traces in tmpfile while getting a count of number of traces
	tracefp = etmpfile();
	hfp = etmpfile();
	ntr = 0;
	do {
		++ntr;

		/* get new deltacdp value
		deltacdp=intrace.cdp-oldcdp;

		/* read headers and data
		efwrite(&intrace,HDRBYTES, 1, hfp);
		efwrite(intrace.data, FSIZE, nt, tracefp);

		/* error trappings.
		/* ...did cdp value interval change?
		if ((ntr>3) && (olddeltacdp!=deltacdp)) {

			if (verbose) {
			warn("cdp interval changed in data");	
			warn("ntr=%d olddeltacdp=%d deltacdp=%d"
				,ntr,olddeltacdp,deltacdp);
		 	check_cdp=cwp_true;
			}
		}
		
		/* save cdp and deltacdp values
		oldcdp=intrace.cdp;
		olddeltacdp=deltacdp;

	} while (gettr(&intrace));

	/* get last cdp  and dcdp
	if (!getparint("lastcdp",&lastcdp)) lastcdp=intrace.cdp; 
	if (!getparint("dcdp",&dcdp))	dcdp=deltacdp - 1;


	checkpars();

	/* error trappings
	if ( (firstcdp==lastcdp) 
		|| (dcdp==0) 
		|| (check_cdp==cwp_true) ) warn("Check cdp values in data!");

	/* rewind trace file pointer and header file pointer
	erewind(tracefp);
	erewind(hfp);

	/* total number of cdp's in data
	ncdp=lastcdp-firstcdp+1;

	/* Set up FFT parameters
	nfft = npfaro(nt, LOOKFAC*nt);
	if(nfft>= SU_NFLTS || nfft >= PFA_MAX)
	  err("Padded nt=%d -- too big",nfft);
	nf = nfft/2 + 1;

	/* Determine number of filters for antialiasing
	fnyq= 1.0/(2*dt);
	nc=ceil(fnyq/fwidth);
	if (verbose)
		warn(" The number of filters for antialiasing is nc= %d",nc);

	/* Allocate space
	data = alloc2float(nt,ntr);
	lowpass=alloc2float(nt,nc+1);
	mig=   alloc2float(nt,ntr);
	vel=   alloc2float(nt,ncdp);
	fc = alloc1float(nc+1);
	rtin= ealloc1float(nfft);
	rtout= ealloc1float(nfft);
	ct= ealloc1complex(nf);
	filter= alloc1float(nf);

	/* Read data from temporal array
	for (ix=0; ix<ntr; ++ix){
		efread(data[ix],FSIZE,nt,tracefp);
	}

	/* read velocities
	vfp=efopen(vfile,"r");
	efread(vel[0],FSIZE,nt*ncdp,vfp);
	efclose(vfp);

	/* Zero all arrays
	memset((void *) mig[0], 0,nt*ntr*FSIZE);
	memset((void *) rtin, 0, nfft*FSIZE);
	memset((void *) filter, 0, nf*FSIZE);
	memset((void *) lowpass[0], 0,nt*(nc+1)*FSIZE);

	/* Calculate cut frequencies for low-pass filters
	for(i=1; i<nc+1; ++i){
		fc[i]= fnyq*i/nc;
	}

	/* Start the migration process
	/* Loop over input mid-points first
	if (verbose) warn("Starting migration process...\n");
	for(imp=0; imp<ntr; ++imp){
		float perc;

		mp=imp*dx; 
		perc=imp*100.0/(ntr-1);
		if(fmod(imp*100,ntr-1)==0 && verbose)
			warn("migrated %g\n ",perc);

		/* Calculate low-pass filtered versions 
		/* of the data to be used for antialiasing
		for(it=0; it<nt; ++it){
			rtin[it]=data[imp][it];
		}
		for(ifc=1; ifc<nc+1; ++ifc){
			memset((void *) rtout, 0, nfft*FSIZE);
			memset((void *) ct, 0, nf*FSIZE);
			lpfilt(nfc,nfft,dt,fc[ifc],filter);
			pfarc(1,nfft,rtin,ct);

			for(it=0; it<nf; ++it){
				ct[it] = crmul(ct[it],filter[it]);
			}
			pfacr(-1,nfft,ct,rtout);
			for(it=0; it<nt; ++it){ 
				lowpass[ifc][it]= rtout[it]; 
			}
		}

		/* Loop over vertical traveltimes
		for(it=0; it<nt; ++it){
			int lx,ux;

			t0=it*dt;
			v=vel[imp*dcdp+fcdpdata-1][it];
			xmax=tan((angmax+10.0)*PI/180.0)*v*t0;
			lx=MAX(0,imp - ceil(xmax/dx)); 
			ux=MIN(ntr,imp + ceil(xmax/dx));
	
		/* loop over output image-points to the left of the midpoint
		for(iip=imp; iip>lx; --iip){
			float ts,tr;
			int fplo=0, fphi=0;
			float ref,wlo,whi;

			ip=iip*dx; 
			x=ip-mp; 
			ts=sqrt( pow(t0/2,2) + pow((x+h)/v,2) );
			tr=sqrt( pow(t0/2,2) + pow((h-x)/v,2) );
			t= ts + tr;
			if(t>=tmax) break;
			geoms=sqrt(1/(t*v));
	  		obliq=sqrt(.5*(1 + (t0*t0/(4*ts*tr)) 
					- (1/(ts*tr))*sqrt(ts*ts - t0*t0/4)*sqrt(tr*tr - t0*t0/4)));
	  		ang=180.0*fabs(acos(t0/t))/PI;  
	  		if(ang<=angmax) angtaper=1.0;
	  		if(ang>angmax) angtaper=cos((ang-angmax)*PI/20);
	  		/* Evaluate migration operator slowness p to determine
			/* the low-pass filtered trace for antialiasing
			pmin=1/(2*dx*fnyq);
			p=fabs((x+h)/(pow(v,2)*ts) + (x-h)/(pow(v,2)*tr));
				if(p>0){fplo=floor(nc*pmin/p);}
				if(p==0){fplo=nc;}
				ref=fmod(nc*pmin,p);
				wlo=1-ref;
				fphi=++fplo;
				whi=ref;
				itb=MAX(ceil(t/dt)-3,0);
				ite=MIN(itb+8,nt);
				firstt=(itb-1)*dt;
				/* Move energy from CMP to CIP
				if(fplo>=nc){
					for(k=itb; k<ite; ++k){
						datalo[k-itb]=lowpass[nc][k];
					}
					ints8r(8,dt,firstt,datalo,0.0,0.0,1,&t,&amplo);
					mig[iip][it] +=geoms*obliq*angtaper*amplo;
				} else if(fplo<nc){
					for(k=itb; k<ite; ++k){
						datalo[k-itb]=lowpass[fplo][k];
						datahi[k-itb]=lowpass[fphi][k];
					}
					ints8r(8,dt,firstt,datalo,0.0,0.0,1,&t,&amplo);
					ints8r(8,dt,firstt,datahi,0.0,0.0,1,&t,&amphi);
					mig[iip][it] += geoms*obliq*angtaper*(wlo*amplo + whi*amphi);
				}
			}

			/* loop over output image-points to the right of the midpoint
			for(iip=imp+1; iip<ux; ++iip){
				float ts,tr;
				int fplo=0, fphi;
				float ref,wlo,whi;

				ip=iip*dx; 
				x=ip-mp; 
				t0=it*dt;	  
				ts=sqrt( pow(t0/2,2) + pow((x+h)/v,2) );
				tr=sqrt( pow(t0/2,2) + pow((h-x)/v,2) );
				t= ts + tr;
				if(t>=tmax) break;
				geoms=sqrt(1/(t*v));
				obliq=sqrt(.5*(1 + (t0*t0/(4*ts*tr)) 
					- (1/(ts*tr))*sqrt(ts*ts 
						- t0*t0/4)*sqrt(tr*tr 
								- t0*t0/4)));
				ang=180.0*fabs(acos(t0/t))/PI;   
				if(ang<=angmax) angtaper=1.0;
				if(ang>angmax) angtaper=cos((ang-angmax)*PI/20.0);

				/* Evaluate migration operator slowness p to determine the 
				/* low-pass filtered trace for antialiasing
				pmin=1/(2*dx*fnyq);
				p=fabs((x+h)/(pow(v,2)*ts) + (x-h)/(pow(v,2)*tr));
				if(p>0){
					fplo=floor(nc*pmin/p);
				}
				if(p==0){
					fplo=nc;
				}

				ref=fmod(nc*pmin,p);
				wlo=1-ref;
				fphi=fplo+1;
				whi=ref;
				itb=MAX(ceil(t/dt)-3,0);
				ite=MIN(itb+8,nt);
				firstt=(itb-1)*dt;

				/* Move energy from CMP to CIP
				if(fplo>=nc){
					for(k=itb; k<ite; ++k){
						datalo[k-itb]=lowpass[nc][k];
					}
					ints8r(8,dt,firstt,datalo,0.0,0.0,1,&t,&amplo);
					mig[iip][it] +=geoms*obliq*angtaper*amplo;
				} else if(fplo<nc){
					for(k=itb; k<ite; ++k){
						datalo[k-itb]=lowpass[fplo][k];
						datahi[k-itb]=lowpass[fphi][k];
					}
					ints8r(8,dt,firstt,datalo,0.0,0.0,1,&t,&amplo);
					ints8r(8,dt,firstt,datahi,0.0,0.0,1,&t,&amphi);
					mig[iip][it] += geoms*obliq*angtaper*(wlo*amplo + whi*amphi);
				}
			}

		}
	} 

	/* Output migrated data
	erewind(hfp);
	for (ix=0; ix<ntr; ++ix) {
		efread(&outtrace, HDRBYTES, 1, hfp);
		for (it=0; it<nt; ++it) {
			outtrace.data[it] = mig[ix][it];
		}
		puttr(&outtrace);
	}

	efclose(hfp);

	return(CWP_Exit());
}

void
lpfilt(int nfc, int nfft, float dt, float fhi, float *filter)
lpfilt -- low-pass filter using Lanczos Smoothing 
	(R.W. Hamming:"Digital Filtering",1977)
Input: 
nfc	number of Fourier coefficients to approximate ideal filter
nfft	number of points in the fft
dt	time sampling interval
fhi	cut-frequency

Output:
filter  array[nf] of filter values
Notes: Filter is to be applied in the frequency domain   
Author: CWP: Carlos Pacheco   2006   
{
	int i,j;  /* counters
	int nf;   /* Number of frequencies (including Nyquist)
	float onfft;  /* reciprocal of nfft
	float fn; /* Nyquist frequency
	float df; /* frequency interval
	float dw; /* frequency interval in radians
	float whi;/* cut-frequency in radians
	float w;  /* radian frequency

	nf= nfft/2 + 1;
	onfft=1.0/nfft;
	fn=1.0/(2*dt);
	df=onfft/dt;
	whi=fhi*PI/fn;
	dw=df*PI/fn;

	for(i=0; i<nf; ++i){
		filter[i]= whi/PI;
		w=i*dw;

		for(j=1; j<nfc; ++j){
			float c= sin(whi*j)*sin(PI*j/nfc)*2*nfc/(PI*PI*j*j);
			filter[i] +=c*cos(j*w);
		}
	}
}
  

  
  
  

  
  
  
\end{verbatim}
\pagebreak
\begin{verbatim}
 SUMIGFD - 45-90 degree Finite difference depth migration for		
           zero-offset data.						

   sumigfd <infile >outfile vfile= [optional parameters]		

 Required Parameters:							
 nz=		number of depth sapmles					
 dz=		depth sampling interval					
 vfile=	name of file containing velocities			
 		(see Notes below concerning format of this file)	

 Optional Parameters:							
 dt=from header(dt) or .004    time sampling interval			
 dx=from header(d2) or 1.0	midpoint sampling interval		
 dip=45,65,79,80,87,89,90  	Maximum angle of dip reflector		

 tmpdir=	if non-empty, use the value as a directory path		
		prefix for storing temporary files; else if the		
		the CWP_TMPDIR environment variable is set use		
		its value for the path; else use tmpfile()		
 
 Notes:								", 
 The computation cost by dip angle is 45=65=79<80<87<89<90		
 
 The input velocity file \'vfile\' consists of C-style binary floats.	", 
 The structure of this file is vfile[iz][ix]. Note that this means that
 the x-direction is the fastest direction instead of z-direction! Such a
 structure is more convenient for the downward continuation type	
 migration algorithm than using z as fastest dimension as in other SU	
 programs. (In C  v[iz][ix] denotes a v(x,z) array, whereas v[ix][iz]  
 denotes a v(z,x) array, the opposite of what Matlab and Fortran	
 programmers may expect.)						", 
 
 Because most of the tools in the SU package (such as  unif2, unisam2,	
 and makevel) produce output with the structure vfile[ix][iz], you will
 need to transpose the velocity files created by these programs. You may
 use the SU program \'transp\' in SU to transpose such files into the	
 required vfile[iz][ix] structure.					


 
 Credits: CWP Baoniu Han, April 20th, 1998

 Trace header fields accessed: ns, dt, delrt, d2
 Trace header fields modified: ns, dt, delrt

\end{verbatim}
\pagebreak
\begin{verbatim}
 SUMIGFFD - Fourier finite difference depth migration for		
	    zero-offset data. This method is a hybrid migration which	
	    combines the advantages of phase shift and finite difference", 
	    migrations.							

 sumigffd <infile >outfile vfile= [optional parameters]		

 Required Parameters:						  	
 nz=		   number of depth sapmles			 	", 
 dz=		   depth sampling interval			 	
 vfile=		name of file containing velocities	      	

 Optional Parameters:						  	
 dt=from header(dt) or .004    time sampling interval		  	
 dx=from header(d2) or 1.0     midpoint sampling interval	  	
 ft=0.0			first time sample			
 fz=0.0			first depth sample		      	

 tmpdir=	if non-empty, use the value as a directory path		
		prefix for storing temporary files; else if the		
		the CWP_TMPDIR environment variable is set use		
		its value for the path; else use tmpfile()		
 
 The input velocity file \'vfile\' consists of C-style binary floats.  ",  
 The structure of this file is vfile[iz][ix]. Note that this means that
 the x-direction is the fastest direction instead of z-direction! Such a
 structure is more convenient for the downward continuation type	
 migration algorithm than using z as fastest dimension as in other SU  ", 
 programs. (In C  v[iz][ix] denotes a v(x,z) array, whereas v[ix][iz]  
 denotes a v(z,x) array, the opposite of what Matlab and Fortran	
 programmers may expect.)						", 

 Because most of the tools in the SU package (such as  unif2, unisam2, ", 
 and makevel) produce output with the structure vfile[ix][iz], you will
 need to transpose the velocity files created by these programs. You may
 use the SU program \'transp\' in SU to transpose such files into the  
 required vfile[iz][ix] structure.					




 Credits: CWP Baoniu Han, July 21th, 1997


 Trace header fields accessed: ns, dt, delrt, d2
 Trace header fields modified: ns, dt, delrt

\end{verbatim}
\pagebreak
\begin{verbatim}
 SUMIGGBZOAN - MIGration via Gaussian beams ANisotropic media (P-wave)	

 sumiggbzoan <infile >outfile vfile= nt= nx= nz= [optional parameters]	

 Required Parameters:							
 a3333file=		name of file containing a3333(x,z)		
 nx=                    number of inline samples (traces)		
 nz=                    number of depth samples			

 Optional Parameters:							
 dt=tr.dt               time sampling interval				
 dx=tr.d2               inline sampling interval (trace spacing)	
 dz=1.0                 depth sampling interval			
 fmin=0.025/dt          minimum frequency				
 fmax=10*fmin           maximum frequency				
 amin=-amax             minimum emergence angle; must be > -90 degrees	
 amax=60                maximum emergence angle; must be < 90 degrees	
 bwh=0.5*vavg/fmin      beam half-width; vavg denotes average velocity	
 verbose=0		 silent, =1 chatty 				

 Files for general anisotropic parameters confined to a vertical plane:
 a1111file=		name of file containing a1111(x,z)		
 a1133file=          	name of file containing a1133(x,z)		
 a1313file=          	name of file containing a1313(x,z)		
 a1113file=          	name of file containing a1113(x,z)		
 a3313file=          	name of file containing a3313(x,z)		

 For transversely isotropic media Thomsen's parameters could be used:	
 deltafile=		name of file containing delta(x,z)		
 epsilonfile=		name of file containing epsilon(x,z)		
 a1313file=          	name of file containing a1313(x,z)		

 if anisotropy parameters are not given the program considers		", 
 the medium to be isotropic.						


 Credits:
	CWP: Tariq Alkhalifah,  based on MIGGBZO by Dave Hale
      CWP: repackaged as an SU program by John Stockwell, April 2006
      
   Technical Reference:

      Alkhailfah, T., 1993, Gaussian beam migration for
      anisotropic media: submitted to Geophysics.

	Cerveny, V., 1972, Seismic rays and ray intensities 
	in inhomogeneous anisotropic media: 
	Geophys. J. R. Astr. Soc., 29, 1--13.

	Hale, D., 1992, Migration by the Kirchhoff, 
	slant stack, and Gaussian beam methods:
      CWP,1992 Report 121, Colorado School of Mines.

	Hale, D., 1992, Computational Aspects of Gaussian
      Beam migration:
     	CWP,1992 Report 121, Colorado School of Mines.




\end{verbatim}
\pagebreak
\begin{verbatim}
 SUMIGGBZO - MIGration via Gaussian Beams of Zero-Offset SU data	

 sumiggbzo <infile >outfile vfile=  nz= [optional parameters]		

 Required Parameters:							
 vfile=                 name of file containing v(z,x)			
 nz=                    number of depth samples			

 Optional Parameters:							
 dt=from header		time sampling interval			
 dx=from header(d2) or 1.0	spatial sampling interval 		
 dz=1.0                 depth sampling interval			
 fmin=0.025/dt          minimum frequency				
 fmax=10*fmin           maximum frequency				
 amin=-amax             minimum emergence angle; must be > -90 degrees	
 amax=60                maximum emergence angle; must be < 90 degrees	
 bwh=0.5*vavg/fmin      beam half-width; vavg denotes average velocity	
 verbose=0		 =0 silent; =1 chatty				

 Note: spatial units of v(z,x) must be the same as those of dx.	
 v(z,x) is represented numerically in C-style binary floats v[x][z],	
 where the depth direction is the fast direction in the data. Such	
 models can be created with unif2 or makevel.				

(In C  v[iz][ix] denotes a v(x,z) array, whereas v[ix][iz]  		
 denotes a v(z,x) array, the opposite of what Matlab and Fortran	
 programmers may expect.)						", 

 Caveat:								
 In the event of a "Segmentation Violation" try reducing the value of
 the "bwh" parameter. Run program with verbose=1 do see the default	
 value.								

 Credits:

 CWP: Dave Hale (algorithm), Jack K. Cohen, and John Stockwell
 (reformatting for SU)


\end{verbatim}
\pagebreak
\begin{verbatim}
 SUMIGPREFD --- The 2-D prestack common-shot 45-90 degree		
			finite-difference depth migration. 		

    sumigprefd <indata >outfile [parameters] 				", 

 Required Parameters:							",  
 nxo=		number of total horizontal output samples		
 nxshot=	number of shot gathers to be migrated			
 nz=		number of depth sapmles					
 dx=		horizontal sampling interval				",	
 dz=		depth sampling interval				 	
 vfile=	velocity profile, it must be binary format (see Notes)	
  
 Optional Parameters:							
 dip=79	the maximum dip to migrate, possible values are:	
		45,65,79,80,87,89,90 degrees				
		The computation cost is 45=65=79<80<87<89<90		
 fmax=25	peak frequency of Ricker wavelet used as source wavelet	
 f1=5,f2=10,f3=40,f4=50	 frequencies to build a Hamming window	

 lpad=9999,rpad=9999	number of zero traces padded on both		
			sides of depth section to determine the		
			migration aperature, the default values 	
			are using the full aperature.			
 verbose=0		silent, =1 additional runtime information	

 Notes:								
 The input velocity file \'vfile\' consists of C-style binary floats.  
 The structure of this file is vfile[iz][ix]. Note that this means that
 the x-direction is the fastest direction instead of z-direction! Such a
 structure is more convenient for the downward continuation type	
 migration algorithm than using z as fastest dimension as in other SU  
 programs.								

 Because most of the tools in the SU package (such as  unif2, unisam2, ", 
 and makevel) produce output with the structure vfile[ix][iz], you will
 need to transpose the velocity files created by these programs. You may
 use the SU program \'transp\' in SU to transpose such files into the  
 required vfile[iz][ix] structure.					
 (In C  v[iz][ix] denotes a v(x,z) array, whereas v[ix][iz]  		
 denotes a v(z,x) array, the opposite of what Matlab and Fortran	
 programmers may expect.)						", 

 Also, sx must be monotonically increasing throughout the dataset, and 
 and gx must be monotonically increasing within a shot. You may resort	
 your data with \'susort\', accordingly.				

 The scalco header field is honored so this field must be set correctly.
 See selfdocs of \'susort\', \'suchw\'. Also:   sukeyword scalco	



 Credits: CWP, Baoniu Han, bhan@dix.mines.edu, April 19th, 1998
	  Modified: Chris Stolk, 11 Dec 2005, - changed data input
		    to remove erroneous time delay. 
	  Modified: CWP, John Stockwell 26 Sept 2006 - replaced Han's
	  "goto-loop" in two places with "do { }while loops".
	  Fixed it so that sx, gx, and scalco are honored.



 Trace header fields accessed: ns, dt, delrt, d2, sx, gx, 
 Trace header fields modified: ns, dt, delrt


\end{verbatim}
\pagebreak
\begin{verbatim}
SUMIGPREFFD - The 2-D prestack common-shot Fourier finite-difference	
		depth  migration.					

  sumigpreffd <indata >outfile [parameters]				", 

 Required Parameters:							",  
 nxo=	   number of total horizontal output samples			
 nxshot=	number of shot gathers to be migrated			
 nz=		number of depth sapmles					
 dx=		horizontal sampling interval				
 dz=		depth sampling interval					
 vfile=	 velocity profile, it must be binary format.		
  
 Optional Parameters:							
 fmax=25	the peak frequency of Ricker wavelet used as source wavelet
 f1=5,f2=10,f3=40,f4=50	frequencies to build a Hamming window	
 lpad=9999,rpad=9999		number of zero traces padded on both	
				sides of depth section to determine the 
				migration aperature, the default	
				values are using the full aperature.	
 verbose=0		silent, =1 additional runtime information	
  
 Notes:								
 The input velocity file consists of C-style binary floats.		",  
 The structure of this file is vfile[iz][ix]. Note that this means that
 the x-direction is the fastest direction instead of z-direction! Such a
 structure is more convenient for the downward continuation type	
 migration algorithm than using z as fastest dimension as in other SU  ", 
 programs.								

 Because most of the tools in the SU package (such as  unif2, unisam2, ", 
 and makevel) produce output with the structure vfile[ix][iz], you will
 need to transpose the velocity files created by these programs. You may
 use the SU program \'transp\' in SU to transpose such files into the  
 required vfile[iz][ix] structure.					
 (In C  v[iz][ix] denotes a v(x,z) array, whereas v[ix][iz]  		
 denotes a v(z,x) array, the opposite of what Matlab and Fortran	
 programmers may expect.)						", 

 Also, sx must be monotonically increasing throughout the dataset, and 
 and gx must be monotonically increasing within a shot. You may resort 
 your data with \'susort\', accordingly.				

 The scalco header field is honored so this field must be set correctly.
 See selfdocs of \'susort\', \'suchw\'. Also:   sukeyword scalco	



 Credits: CWP, Baoniu Han, bhan@dix.mines.edu, April 19th, 1998

	  Modified: Chris Stolk, 11 Dec 2005, - changed data input
		    to remove erroneous time delay.
	  Modified: CWP, John Stockwell 26 Sept 2006 - replaced Han's
	  "goto-loop" with  "do { }while loops".
	  Fixed it so that sx, gx, and scalco are honored.



 Trace header fields accessed: ns, dt, delrt, d2
 Trace header fields modified: ns, dt, delrt


\end{verbatim}
\pagebreak
\begin{verbatim}
 char *sdoc[] = {
 
 " SUMIGPREPSPI --- The 2-D PREstack commom-shot Phase-Shift-Plus 	
 "			interpolation depth MIGration.			
 
 "   sumigprepspi <indata >outfile [parameters] 			", 
 
 " Required Parameters:						   	
 
 " nxo=     number of total horizontal output samples			
 " nxshot=  number of shot gathers to be migrated		   	
 " nz=      number of depth samples				 	
 " dx=      horizontal sampling interval			  	",   
 " dz=      depth sampling interval				 	
 " vfile=   velocity profile, it must be binary format.                 
 
 " Optional Parameters:						   	
 " fmax=25    the peak frequency of Ricker wavelet used as source wavelet
 " f1=5,f2=10,f3=40,f4=50     frequencies to build a Hamming window     
 " lpad=9999,rpad=9999        number of zero traces padded on both	
 "                            sides of depth section to determine the	
 "                            migration aperture, the default values    
 "                            are using the full aperture.              
 " nflag=0    normalization of cross-correlation:                       
 "            0: none, 1: by source wave field                          
 " verbose=0  silent, =1 additional runtime information	                
   
 " Notes:								
 " The input velocity file \'vfile\' consists of C-style binary floats.	",  
 " The structure of this file is vfile[iz][ix]. Note that this means that
 " the x-direction is the fastest direction instead of z-direction! Such a
 " structure is more convenient for the downward continuation type	
 " migration algorithm than using z as fastest dimension as in other SU  ", 
 " programs.						   		
 
 " Because most of the tools in the SU package (such as  unif2, unisam2, ", 
 " and makevel) produce output with the structure vfile[ix][iz], you will
 " need to transpose the velocity files created by these programs. You may
 " use the SU program \'transp\' in SU to transpose such files into the  
 " required vfile[iz][ix] structure.					
 
 " (In C  v[iz][ix] denotes a v(x,z) array, whereas v[ix][iz]  		
 " denotes a v(z,x) array, the opposite of what Matlab and Fortran	
 " programmers may expect.)						", 
 
 " Also, sx must be monotonically increasing throughout the dataset, and 
 " and gx must be monotonically increasing within a shot. You may resort 
 " your data with \'susort\', accordingly.				
 
 " The scalco header field is honored so this field must be set correctly.
 " See selfdocs of \'susort\', \'suchw\'. Also:   sukeyword scalco	
 


  * Credits: CWP, Baoniu Han, bhan@dix.mines.edu, April 19th, 1998
  *	  Modified: Chris Stolk, 11 Dec 2005, - changed data input
  *		    to remove erroneous time delay.
  *	  Modified: CWP, John Stockwell 26 Sept 2006 - replaced Han's
  *	  "goto-loop" in two places with "do { }while loops".
  *	  Fixed it so that sx, gx, and scalco are honored.
  *
  *
  * Trace header fields accessed: ns, dt, delrt, d2
  * Trace header fields modified: ns, dt, delrt
 

\end{verbatim}
\pagebreak
\begin{verbatim}
 SUMIGPRESP - The 2-D prestack common-shot split-step Fourier		", 
		migration 						

   sumigpresp <indata >outfile [parameters]				", 

 Required Parameters:							
 nxo=	   number of total horizontal output samples			
 nxshot=	number of shot gathers to be migrated			
 nz=	    number of depth sapmles					
 dx=	    horizontal sampling interval				
 dz=	    depth sampling interval					
 vfile=	 velocity profile, it must be binary format.		
  
 Optional Parameters:						   	
 fmax=25	The peak frequency of Ricker wavelet used as source wavelet
 f1=5,f2=10,f3=40,f4=50	 frequencies to build a Hamming window	
 lpad=9999,rpad=9999	    number of zero traces padded on both    	
				sides of depth section to determine the 
				migration aperature, the default values 
				are using the full aperature.		
 verbose=0             silent, =1 additional runtime information       
  
 Notes:								
 The input velocity file consists of C-style binary floats.	    	
 The structure of this file is vfile[iz][ix]. Note that this means that
 the x-direction is the fastest direction instead of z-direction! Such a
 structure is more convenient for the downward continuation type	
 migration algorithm than using z as fastest dimension as in other SU  
 programs.								

 Because most of the tools in the SU package (such as  unif2, unisam2, ", 
 and makevel) produce output with the structure vfile[ix][iz], you will
 need to transpose the velocity files created by these programs. You may
 use the SU program \'transp\' in SU to transpose such files into the  
 required vfile[iz][ix] structure.					
 (In C  v[iz][ix] denotes a v(x,z) array, whereas v[ix][iz]  		
 denotes a v(z,x) array, the opposite of what Matlab and Fortran	
 programmers may expect.)						", 

 Also, sx must be monotonically increasing throughout the dataset, and 
 and gx must be monotonically increasing within a shot. You may resort 
 your data with \'susort\', accordingly.                               

 The scalco header field is honored so this field must be set correctly.
 See selfdocs of \'susort\', \'suchw\'. Also:   sukeyword scalco       



 Credits: CWP, Baoniu Han, bhan@dix.mines.edu, April 19th, 1998
          Modified: Chris Stolk, 11 Dec 2005, - changed data input
                    to remove erroneous time delay.
          Modified: CWP, John Stockwell 26 Sept 2006 - replaced Han's
          "goto-loop" in two places with "do { }while loops".
          Fixed it so that sx, gx, and scalco are honored.


 Trace header fields accessed: ns, dt, delrt, d2, sx, gx, scalco
 Trace header fields modified: ns, dt, delrt


\end{verbatim}
\pagebreak
\begin{verbatim}
 SUMIGPSPI - Gazdag's phase-shift plus interpolation depth migration   
            for zero-offset data, which can handle the lateral         
            velocity variation.                                        

 sumigpspi <infile >outfile vfile= [optional parameters]               
 
 Required Parameters:							
 nz=		number of depth sapmles					
 dz=		depth sampling interval					
 vfile=	name of file containing velocities			
		(Please see Notes below concerning the format of vfile)	

 Optional Parameters:                                                  
 dt=from header(dt) or .004    time sampling interval                  
 dx=from header(d2) or 1.0     midpoint sampling interval              

 tmpdir=        if non-empty, use the value as a directory path        
                prefix for storing temporary files; else if the        
                the CWP_TMPDIR environment variable is set use         
                its value for the path; else use tmpfile()             

 Notes:								
 The input velocity file 'vfile' consists of C-style binary floats.	
 The structure of this file is vfile[iz][ix]. Note that this means that
 the x-direction is the fastest direction instead of z-direction! Such a
 structure is more convenient for the downward continuation type	
 migration algorithm than using z as fastest dimension as in other SU	
 programs. (In C  v[iz][ix] denotes a v(x,z) array, whereas v[ix][iz]	
 denotes a v(z,x) array, the opposite of what Matlab and Fortran	
 programmers may expect.)						

 Because most of the tools in the SU package (such as  unif2, unisam2,	
 and makevel) produce output with the structure vfile[ix][iz], you will
 need to transpose the velocity files created by these programs. You may
 use the SU program 'transp' in SU to transpose such files into the	
 required vfile[iz][ix] structure.					




 Credits: CWP, Baoniu Han, April 20th, 1998

 Trace header fields accessed: ns, dt, delrt, d2
 Trace header fields modified: ns, dt, delrt

\end{verbatim}
\pagebreak
\begin{verbatim}
 SUMIGPS - MIGration by Phase Shift with turning rays			

 sumigps <stdin >stdout [optional parms]				

 Required Parameters:							
 	None								

 Optional Parameters:							
 dt=from header(dt) or .004	time sampling interval			
 dx=from header(d2) or 1.0	distance between sucessive cdp's	
 ffil=0,0,0.5/dt,0.5/dt  trapezoidal window of frequencies to migrate	
 tmig=0.0		times corresponding to interval velocities in vmig
 vmig=1500.0		interval velocities corresponding to times in tmig
 vfile=		binary (non-ascii) file containing velocities v(t)
 nxpad=0		number of cdps to pad with zeros before FFT	
 ltaper=0		length of linear taper for left and right edges", 
 verbose=0		=1 for diagnostic print				


 tmpdir= 	 if non-empty, use the value as a directory path	
		 prefix for storing temporary files; else if the	
	         the CWP_TMPDIR environment variable is set use		
	         its value for the path; else use tmpfile()		

 Notes:								
 Input traces must be sorted by either increasing or decreasing cdp.	

 The tmig and vmig arrays specify an interval velocity function of time.
 Linear interpolation and constant extrapolation is used to determine	
 interval velocities at times not specified.  Values specified in tmig	
 must increase monotonically.						

 Alternatively, interval velocities may be stored in a binary file	
 containing one velocity for every time sample.  If vfile is specified,
 then the tmig and vmig arrays are ignored.				

 The time of first sample is assumed to be zero, regardless of the value
 of the trace header field delrt.					

 Credits:
	CWP: Dave Hale (originally called supsmig.c)

  Trace header fields accessed:  ns, dt, d2

\end{verbatim}
\pagebreak
\begin{verbatim}
 SUMIGPSTI - MIGration by Phase Shift for TI media with turning rays	

 sumigpsti <stdin >stdout [optional parms]				

 Required Parameters:							
 	None								

 Optional Parameters:							
 dt=from header(dt) or .004	time sampling interval			
 dx=from header(d2) or 1.0	distance between sucessive cdp's	
 ffil=0,0,0.5/dt,0.5/dt  trapezoidal window of frequencies to migrate	
 tmig=0.0	times corresponding to interval velocities in vmig	
 vnmig=1500.0	interval NMO velocities corresponding to times in tmig	
 vmig=1500.0	interval velocities corresponding to times in tmig	
 etamig=0.0	interval eta values corresponding to times in tmig	
 vnfile=	binary (non-ascii) file containing NMO velocities vn(t)	
 vfile=	binary (non-ascii) file containing velocities v(t)	
 etafile=	binary (non-ascii) file containing eta values eta(t)	
 nxpad=0	number of cdps to pad with zeros before FFT		
 ltaper=0	length of linear taper for left and right edges		", 
 verbose=0	=1 for diagnostic print					

 Notes:								
 Input traces must be sorted by either increasing or decreasing cdp.	

 The tmig, vnmig, vmig and etamig arrays specify an interval values	
 function of time. Linear interpolation and constant extrapolation is	
 used to determine interval velocities at times not specified.  Values	
 specified in tmig must increase monotonically.			
 Alternatively, interval velocities may be stored in a binary file	
 containing one velocity for every time sample.  If vnfile is specified,
 then the tmig and vnmig arrays are ignored.				

 The time of first sample is assumed to be zero, regardless of the value
 of the trace header field delrt.					

 Trace header fields accessed:  ns and dt				


\end{verbatim}
\pagebreak
\begin{verbatim}
 SUMIGSPLIT - Split-step depth migration for zero-offset data.         

 sumigsplit <infile >outfile vfile= [optional parameters]              

 Required Parameters:                                                  
 nz=                   number of depth sapmles                         
 dz=                   depth sampling interval                         
 vfile=                name of file containing velocities              

 Optional Parameters:                                                  
 dt=from header(dt) or .004    time sampling interval                  
 dx=from header(d2) or 1.0     midpoint sampling interval              
 ft=0.0                        first time sample                       
 fz=                           first depth sample                      

 tmpdir=        if non-empty, use the value as a directory path        
                prefix for storing temporary files; else if the        
                the CWP_TMPDIR environment variable is set use         
                its value for the path; else use tmpfile()             


 Notes:                                                                
 The input velocity file \'vfile\' consists of C-style binary floats.  
 The structure of this file is vfile[iz][ix]. Note that this means that
 the x-direction is the fastest direction instead of z-direction! Such a
 structure is more convenient for the downward continuation type       
 migration algorithm than using z as fastest dimension as in other SU  
 programs.                                                     	

 Because most of the tools in the SU package (such as  unif2, unisam2, 
 and makevel) produce output with the structure vfile[ix][iz], you will
 need to transpose the velocity files created by these programs. You may
 use the SU program \'transp\' in SU to transpose such files into the  
 required vfile[iz][ix] structure.                                     
 (In C  v[iz][ix] denotes a v(x,z) array, whereas v[ix][iz]  		
 denotes a v(z,x) array, the opposite of what Matlab and Fortran	
 programmers may expect.)						", 

  

 Credits: CWP Baoniu Han, July 21th, 1997

 Reference:
  Stoffa, P. L. and Fokkema, J. T. and Freire, R. M. and Kessinger, W. P.,
  1990, Split-step Fourier migration, Geophysics, 55, 410-421.


 Trace header fields accessed: ns, dt, delrt, d2
 Trace header fields modified: ns, dt, delrt


\end{verbatim}
\pagebreak
\begin{verbatim}
 SUMIGTK - MIGration via T-K domain method for common-midpoint stacked data

 sumigtk <stdin >stdout dxcdp= [optional parms]			

 Required Parameters:							
 dxcdp                   distance between successive cdps		

 Optional Parameters:							
 fmax=Nyquist            maximum frequency				
 tmig=0.0                times corresponding to interval velocities in vmig
 vmig=1500.0             interval velocities corresponding to times in tmig
 vfile=                  binary (non-ascii) file containing velocities v(t)
 nxpad=0                 number of cdps to pad with zeros before FFT	
 ltaper=0                length of linear taper for left and right edges", 
 verbose=0               =1 for diagnostic print			

 tmpdir= 	 if non-empty, use the value as a directory path	
		 prefix for storing temporary files; else if the	
	         the CWP_TMPDIR environment variable is set use		
	         its value for the path; else use tmpfile()		

 Notes:								
 Input traces must be sorted by either increasing or decreasing cdp.	

 The tmig and vmig arrays specify an interval velocity function of time.
 Linear interpolation and constant extrapolation is used to determine	
 interval velocities at times not specified.  Values specified in tmig	
 must increase monotonically.						

 Alternatively, interval velocities may be stored in a binary file	
 containing one velocity for every time sample.  If vfile is specified,
 then the tmig and vmig arrays are ignored.				

 The time of first sample is assumed to be zero, regardless of the value
 of the trace header field delrt.					

 The migration is a reverse time migration in the (t,k) domain. In the	
 first step, the data g(t,x) are Fourier transformed x->k into the	",	
 the time-wavenumber domain g(t,k).					

 Then looping over wavenumbers, the data are then reverse-time		
 finite-difference migrated, wavenumber by wavenumber.  The resulting	
 migrated data m(tau,k), now in the tau (migrated time) and k domain,	
 are inverse fourier transformed back into m(tau,xout) and written out.",	


 Credits:
	CWP: Dave Hale

 Trace header fields accessed:  ns and dt

\end{verbatim}
\pagebreak
\begin{verbatim}
 SUMIGTOPO2D - Kirchhoff Depth Migration of 2D postack/prestack data	
	     from the (variable topography) recording surface		

    sumigtopo2d  infile=  outfile=  [parameters] 			

 Required parameters:							
 infile=stdin		file for input seismic traces			
 outfile=stdout	file for common offset migration output  	
 ttfile		file for input traveltime tables		
   The following 9 parameters describe traveltime tables:		
 fzt 			first depth sample in traveltime table		
 nzt 			number of depth samples in traveltime table	
 dzt			depth interval in traveltime table		
 fxt			first lateral sample in traveltime table	
 nxt			number of lateral samples in traveltime table	
 dxt			lateral interval in traveltime table		
 fs 			x-coordinate of first source			
 ns 			number of sources				
 ds 			x-coordinate increment of sources		

 fxi                   x-coordinate of the first input trace           
 dxi                   horizontal spacing of input data                
 nxi                   number of input trace locations in surface      

 Optional Parameters:							
 dt= or from header (dt) 	time sampling interval of input data	
 ft= or from header (ft) 	first time sample of input data		
 dxm= or from header (d2) 	sampling interval of midpoints 		
 surf="0,0;99999,0"  Recording surface "x1,z1;x2,z2;x3,z3;...
 fzo=fzt               z-coordinate of first point in output trace 	
 dzo=0.2*dzt		vertical spacing of output trace 		
 nzo=5*(nzt-1)+1 	number of points in output trace		",	
 fxo=fxt               x-coordinate of first output trace 		
 dxo=0.5*dxt		horizontal spacing of output trace 		
 nxo=2*(nxt-1)+1  	number of output traces 			",	
 off0=0               	first offest in output 				
 doff=99999		offset increment in output 			
 noff=1       		number of offsets in output 			",	
 fmax=0.25/dt		frequency-highcut for input traces		
 offmax=99999		maximum absolute offset allowed in migration 	
 aperx=nxt*dxt/2  	migration lateral aperature 			
 angmax=60		migration angle aperature from vertical 	
 v0=1500(m/s)		reference velocity value at surface		",	
 dvz=0.0  		reference velocity vertical gradient		
 ls=1	                flag for line source				
 jpfile=stderr		job print file name 				
 mtr=100  		print verbal information at every mtr traces	
 ntr=100000		maximum number of input traces to be migrated	

 Notes:								
 1. Traveltime tables were generated by program rayt2dtopo (or any 	
    other one that considers topography )on relatively coarse grids,	
    with dimension ns*nxt*nzt. In the migration process, traveltimes	
    are interpolated into shot/gephone positions and output grids.	
 2. Input seismic traces must be SU format and can be any type of 	
    gathers (common shot, common offset, common CDP, and so on).	", 
 3. Migrated traces are output in CDP gathers if velocity analysis	
    is required, with dimension nxo*noff*nzo.  			", 
 4. If the offset value of an input trace is not in the offset array 	
    of output, the nearest one in the array is chosen. 		
 5. Amplitudes are computed using the reference velocity profile, v(z),
    specified by the parameters v0= and dvz=.				
 6. Input traces must specify source and receiver positions via the header
    fields tr.sx and tr.gx. Offset is computed automatically.		


 Author:  Zhenyue Liu, 03/01/95,  Colorado School of Mines

	    Trino Salinas, 07/01/96, Colorado School of Mines,
          extended the code to migrate data from a nonflat 
          recording surface.

 References :

 Bleistein, N., Cohen, J., and Hagin, F., 1987, Two and one-half
   dimensional Born inversion with arbitrary reference: Geophysics,
   52, 26-36.

 Liu,Z., 1993, A Kirchhoff approach to seismic modeling and 
   pre-stack depth migration: CWP Annual Report, CWP, Colorado
   School of Mines.

 Wiggins, J. W., 1984, Kirchhoff integral extrapolation and
   migration of nonplanar data: Geophysics, 49, 1239-
   1248.

\end{verbatim}
\pagebreak
\begin{verbatim}
 SUSTOLT - Stolt migration for stacked data or common-offset gathers	

 sustolt <stdin >stdout cdpmin= cdpmax= dxcdp= noffmix= [...]		

 Required Parameters:							
 cdpmin=		  minimum cdp (integer number) in dataset	
 cdpmax=		  maximum cdp (integer number) in dataset	
 dxcdp=		  distance between adjacent cdp bins (m)	

 Optional Parameters:							
 noffmix=1		number of offsets to mix (for unstacked data only)
 tmig=0.0		times corresponding to rms velocities in vmig (s)
 vmig=1500.0		rms velocities corresponding to times in tmig (m/s)
 smig=1.0		stretch factor (0.6 typical if vrms increasing)
 vscale=1.0		scale factor to apply to velocities		
 fmax=Nyquist		maximum frequency in input data (Hz)		
 lstaper=0		length of side tapers (# of traces)		
 lbtaper=0		length of bottom taper (# of samples)		
 verbose=0		=1 for diagnostic print				
 tmpdir=		if non-empty, use the value as a directory path	
			prefix for storing temporary files; else if the	
			the CWP_TMPDIR environment variable is set use	
			its value for the path; else use tmpfile()	

 Notes:								
 If unstacked traces are input, they should be NMO-corrected and sorted
 into common-offset  gathers.  One common-offset gather ends and another
 begins when the offset field of the trace headers changes. If both	
 NMO and DMO are applied, then this is equivalent to prestack time 	
 migration (though the velocity profile is assumed v(t), only).	

 The cdp field of the input trace headers must be the cdp bin NUMBER, NOT
 the cdp location expressed in units of meters or feet.		

 The number of offsets to mix (noffmix) should be specified for	
 unstacked data only.	noffmix should typically equal the ratio of the	
 shotpoint spacing to the cdp spacing.	 This choice ensures that every	
 cdp will be represented in each offset mix.  Traces in each mix will	
 contribute through migration to other traces in adjacent cdps within	
 that mix.								

 The tmig and vmig arrays specify a velocity function of time that is	
 used to implement Stolt's stretch for depth-variable velocity.  The	
 stretch factor smig is often referred to as the "W" factor.		
 The times in tmig must be monotonically increasing.			

 Credits:
	CWP: Dave Hale c. 1990

 Trace header fields accessed:  ns, dt, delrt, offset, cdp

\end{verbatim}
\pagebreak
\begin{verbatim}
 SUTIFOWLER   VTI constant velocity prestack time migration		
	      velocity analysis via Fowler's method			

 sutifowler ncdps=250 vmin=1500 vmax=6000 dx=12.5			

 Required Parameter:							
 ncdps=		number of input cdp's				
 Optional Parameters:							
 choose=1	1 do full prestack time migration			
		2 do only DMO						
		3 do only post-stack migrations				
		4 do only stacking velocity analysis			
 getcvstacks=0	flag to set to 1 if inputting precomputed cvstacks	
		(vmin, nvstack, and ncdps must match SUCVS4FOWLER job)	
 vminstack=vmin	minimum velocity panel in m/s in input cvstacks	
 etamin=0.		minimum eta (see paper by Tariq Alkhalifah)	
 etamax=0.5	maximum eta (see paper by Tariq Alkhalifah)		
 neta=1	number of eta values to image				
 d=0.		Thomsen's delta						
 vpvs=0.5	assumed vp/vs ratio (not critical -- default almost always ok)
 dx=25.	cdp x increment						
 vmin=1500.	minimum velocity panel in m/s to output			
 vmax=8000.	maximum velocity panel in m/s to output			
 nv=75	 number of velocity panels to output				
 nvstack=180	number of stacking velocity panels to compute		
		     ( Let offmax be the maximum offset, fmax be	
		     the maximum freq to preserve, and tmute be		
		     the starting mute time in sec on offmax, then	
		     the recommended value for nvstack would be		
		     nvstack = 4 +(offmax*offmax*fmax)/(0.6*vmin*vmin*tmute)
		     ---you may want to make do with less---)		
 nxpad=0	  number of traces to padd for spatial fft		
		     Ideally nxpad = (0.5*tmax*vmax+0.5*offmax)/dx	
 lmute=24	 length of mute taper in ms				
 nonhyp=1	  1 if do mute at 2*offset/vmin to avoid non-hyperbolic 
				moveout, 0 otherwise			
 lbtaper=0	length of bottom taper in ms				
 lstaper=0	length of side taper in traces				
 dtout=1.5*dt	output sample rate in s,   note: typically		
				fmax=salias*0.5/dtout			
 mxfold=120	maximum number of offsets/input cmp			
 salias=0.8	fraction of output frequencies to force within sloth	
		     antialias limit.  This controls muting by offset of
		     the input data prior to computing the cv stacks	
		     for values of choose=1 or choose=2.		
 file=sutifowler	root name for temporary files			
 p=not		Enter a path name where to put temporary files.		
	  	specified  Can enter multiple times to use multiple disk
		systems.						
		     The default uses information from the .VND file	
		     in the current directory if it exists, or puts 	
		     unique temporary files in the current directory.	
 ngroup=20	Number of cmps per velocity analysis group.		
 printfile=stderr    The output file for printout from this program.	

 Required trace header words on input are ns, dt, cdp, offset.		
 On output, trace headers are rebuilt from scratch with		
 ns - number of samples						
 dt - sample rate in usec						
 cdp - the output cmp number (0 based)					
 offset - the output velocity						
 tracf	- the output velocity index (0 based)				
 fldr - index for velocity analysis group (0 based, groups of ngroup cdps)
 ep - central cmp for velocity analysis group				
 igc - index for choice of eta (0 based)				
 igi - eta*100								
 sx=gx	- x coordinate as icmp*dx					
 tracl=tracr -sequential trace count (1 based)				

 Note: Due to aliasing considerations, the small offset-to-depth	
 ratio assumption inherent in the TI DMO derivation, and the		
 poor stacking of some large-offset events associated with TI non-hyperbolic
 moveout, a fairly stiff initial mute is recommended for the		
 long offsets.  As a result, this method may not work well		
 where you have multiple reflections to remove via stacking.		

 Note: The temporary files can be split over multiple disks by building
 a .VND file in your working directory.  The .VND file is ascii text	
 with the first line giving the number of directories followed by	
 successive lines with one line per directory name.			

 Note: The output data order has primary key equal to cdp, secondary	
 key equal to eta, and tertiary key equal to velocity.			

 Credits:
	CWP: John Anderson (visitor to CSM from Mobil) Spring 1993


\end{verbatim}
\pagebreak
\begin{verbatim}
 SUALFORD - trace by trace Alford Rotation of shear wave data volumes  

 sualford inS11=file1 inS22=file2 inS12=file3 inS21=file4		
 outS11=file5 outS22=file6 outS12=file7 outS21=file8 [optional         
 parameters]                                                           

 Required Parameters:                                                  
 inS11=	input data volume for the 11 component			
 inS12=	input data volume for the 12 component			
 inS21=	input data volume for the 21 component			
 inS22=	input data volume for the 22 component			
 outS11=	output data volume for the 11 component			
 outS12=	output data volume for the 11 component			
 outS21=	output data volume for the 11 component			
 outS22=	output data volume for the 11 component			

 Optional parameters:                                                  
 angle_inc=               sets the increment to the angle by which	
                         the data sets are rotated. The minimum is     
                         set to be 1 degree and default is 5.          
 Az_key=                  to set the header field storing the azimuths	
                         for the fast shear wave on the output volumes 
 Q_key=                   to set the header field storing the quality	
                         factors of performed optimum rotations        
 lag_key=                 to set the header field storing the lag in	
                         miliseconds the fast and slow shear components
 xcorr_key=               to set the header field containing the maxi-	
                         mum normalized cross-correlation between the	
                         and slow shear waves.                         
 taper=		  2*taper+1 is the length of the sample overlap 
			  between the unrotated data with the rotated   
			  data on the traces. The boundary between them 
 			  is defined by time windowning.                
				taper = -1, for no-overlap		
				taper = 0, for overlap of one sample	
				taper =>1, for use of cosine scale to   
					   to interpolate between the 	
					   unrotated and rotated parts	
					   of the traces		

 taperwin=               another taper used to taper the data within   
			  the window of analysis to diminish the effect 
                         of data near the window edges.In this way one 
                         can focus on a given reflector. Also given in 
                         number of samples                             

 maxlag=		  maximum limit in ms for the lag between fast  
 			  and slow shear waves. If this threshold is 	
			  attained or surpassed, the quality factor for	
			  the rotation is zeroed as well as all the     
			  parameters found for that certain rotation 	


 ntraces=		  number of traces to be used per computation   
			  ntraces=3 will use three adjacent traces to   
		          compute the angle of rotation                 "

 Notes:                                                                

 The Alford Rotation is a method to rotate the four components         
 of a shear wave survey into its natural coordinate system, where      
 the fast and slow shear correspond to the inline to inline shear (S11)
 and xline to xline (S22) volumes, respectively.                       

 This Alford Rotation code tries to maximize the energy in the         
 diagonal volumes, i.e., S11 and S22, while minimizing the energy      
 in the off-diagonals, i.e., in volumes S12 and S21, in a trace by     
 trace manner. It then returns the new rotated volumes, saving the     
 the quality factor for the rotation and azimuth angle of the fast     
 shear wave direction for each trace headers of the new rotated S11    
 volume.                                                               

 The fields in the header containing the Azimuth and Quality factor    
 and the sample lag between fast and slow shear are otrav, grnolf and  
 grnofr, respectively, by default. The values are multiplied by ten in 
 the case of the angles and by a thousand for quality factors. To      
 change this defaults use the optional parameters Az_key, Q_key and    
 lag_key                                                             	

 
 modified header fields:                                               
 the ones specified by Az_key, Q_key, lag_key and xcorr_key. By default
 these are otrav, grnlof, tstat and grnors, respectively.            	

 Credits:
	CWP: Rodrigo Felicio Fuck
      Code translated and adapted from original version in Fortran
      by Ted Schuck (1993)


 Schuck, E. L. , 1993, Multicomponent, three dimensional seismic 
 characterization of a fractured coalbed methane reservoir, 
 Cedar Hill Field, San Juan County, New Mexico, Ph.D. Thesis,
 Colorado School of Mines


\end{verbatim}
\pagebreak
\begin{verbatim}
 SUEIPOFI - EIgenimage (SVD) based POlarization FIlter for             
            three-component data                                       

 sueipofi <stdin >stdout [optional parameters]                         

 Required parameters:                                                  
    none                                                               

 Optional parameters:                                                  
    dt=(from header)  time sampling intervall in seconds               
    wl=0.1            SVD time window length in seconds                
    pwr=1.0           exponent of filter weights                       
    interp=cubic      interpolation between initially calculated       
                      weights, choose "cubic" or "linear
    verbose=0         1 = echo additional information                  

    file=polar        base name for additional output file(s) of       
                      filter weights (see flags below)                 
    rl1=0             1 = rectilinearity along first principal axis    
    rl2=0             1 = rectilinearity along second principal axis   
    pln=0             1 = planarity                                    


 Notes:                                                                
    Three adjacent traces are considered as one three-component        
    dataset.                                                           

    The filter is the sum of the first two eigenimages of the singular 
    value decomposition (SVD) of the signal matrix (time window).      
    Weighting functions depending on linearity and planarity of the    
    signal are applied, additionally. To avoid edge effects, these are 
    interpolated linearily or via cubic splines between initially      
    calculated values of non-overlapping time windows.                 
    The algorithm is based on the assumption that the particle motion  
    trajectory is essentially 2D (elliptical polarization).            

 Caveat:                                                               
    Cubic spline interpolation may result in filter weights exceeding  
    the set of values of initial weights. Weights outside the valid    
    interval [0.0, 1.0] are clipped.                                   


 
 Author: Nils Maercklin, 
         GeoForschungsZentrum (GFZ) Potsdam, Germany, 2001.
         E-mail: nils@gfz-potsdam.de


 References:
    Franco, R. de, and Musacchio, G., 2000: Polarization Filter with
       Singular Value Decomposition, submitted to Geophysics and
       published electronically in Geophysics online (www.geo-online.org).
    Jurkevics, A., 1988: Polarization analysis of three-comomponent
       array data, Bulletin of the Seismological Society of America, 
       vol. 78, no. 5.
    Press, W. H., Teukolsky, S. A., Vetterling, W. T., and Flannery, B. P.
       1996: Numerical Recipes in C - The Art of Scientific Computing,
       Cambridge University Press, Cambridge.

 Trace header fields accessed: ns, dt
 Trace header fields modified: none

\end{verbatim}
\pagebreak
\begin{verbatim}
 SUHROT - Horizontal ROTation of three-component data			

 suhrot <stdin >stdout [optional parameters]				

 Required parameters:							
 none									

 Optional parameters:							
 angle=rad	unit of angles, choose "rad", "deg", or "gon
 inv=0		1 = inverse rotation (counter-clockwise)		
 verbose=0	1 = echo angle for each 3-C station			

 a=...		array of user-supplied rotation angles			
 x=0.0,...	array of corresponding header value(s)			
 key=tracf	header word defining 3-C station ("x")		

 ... or input angles from files:					
 n=0		 number of x and a values in input files		
 xfile=...   file containing the x values as specified by the		
 				"key" parameter			
 afile=...   file containing the a values				

 Notes:								
 Three adjacent traces are considered as one three-component		
 dataset.								
 By default, the data will be rotated from the Z-North-East (Z,N,E)	
 coordinate system into Z-Radial-Transverse (Z,R,T).			

	If one of the parameters "a=" or "afile=" is set, the data	
	are rotated by these user-supplied angles. Specified x values	
	must be monotonically increasing or decreasing, and afile and	
	xfile are files of binary (C-style) floats.			


 
 Author: Nils Maercklin,
		 Geophysics, Kiel University, Germany, 1999.


 Trace header fields accessed: ns, sx, sy, gx, gy, key=keyword
 Trace header fields modified: trid
 

\end{verbatim}
\pagebreak
\begin{verbatim}
 SULTT - trace by trace, sample by sample, rotation of shear wave data 
	  volumes using the Linear Transform Technique of Li & Crampin  
	  (1993)							

 sultt inS11=file1 inS22=file2 inS12=file3 inS21=file4 [optional       
 parameters]                                                           

 optional parameters:							

 mode		determines what the linear transform will compute 	
			mode=1, computes asymmetry indexes		
 			mode=2, computes Polarization and main       	
				reflectivity series.			
			mode=3, same as above, but using eigenvalues    

 		mode=3 is more robust estimation for Polarization angle 
		than mode=2. In both cases the reflectivity series is   
		computed in the same way. mode=1 outputs two other data 
		volumes, each containing the asymmetry parameters theta,
		and gamma. The other two modes only output an extra da- 
		ta volume, the instant polarization alpha		

 outSij	defines the names of the output seismic data files,     
		i and j equal either 1 or 2				

 alpha, gamma	name for optional output volumes: instanteneous polarity
 theta		, alpha; theta, for angle misalignment between source   
		and receiver; gamma, the medium asymmetric response     
		coefficient						

 wl		running window acting on traces (in samples)		

 ntraces	number of traces to be average for each location	

 CAVEAT								

 Naming convention for off-diagonal volumes:				
 S12 - Inline source, Xline receiver					
 S21 - Xline source, Inline receiver					

 the running will always have an odd number of samples, despite the    
 input length.								


 Credits:
	CWP/RCP: Rodrigo Felicio Fuck
      Code based on algorithms presented in Li & Crampin (1993) and 
	Li & MacBeth (1997)


	Li, X.Y., and Crampin, S., 1993, Linear-transform techniques for 
		processing shear-wave anisotropy in four-component
		seismic data, Geophysics, 58, 240-256.
	Li, X.Y., and MacBeth, C., 1997, Data-Matrix asymmetry and polar-
		ization changes from multicomponent surface seismics 

\end{verbatim}
\pagebreak
\begin{verbatim}
 SUPOFILT - POlarization FILTer for three-component data               

 supofilt <stdin >stdout [optional parameters]                         

 Required parameters:                                                  
    dfile=polar.dir   file containing the 3 components of the          
                      direction of polarization                        
    wfile=polar.rl    file name of weighting polarization parameter    

 Optional parameters:                                                  
    dt=(from header)  time sampling intervall in seconds               
    smooth=1          1 = smooth filter operators, 0 do not            
    sl=0.05           smoothing window length in seconds               
    wpow=1.0          raise weighting function to power wpow           
    dpow=1.0          raise directivity functions to power dpow        
    verbose=0         1 = echo additional information                  


 Notes:                                                                
    Three adjacent traces are considered as one three-component        
    dataset.                                                           

    This program SUPOFILT is an extension to the polarization analysis 
    program supolar. The files wfile and dfile are SU files as written 
    by SUPOLAR.                                                        


 
 Author: Nils Maercklin, 
         GeoForschungsZentrum (GFZ) Potsdam, Germany, 1999-2000.
         E-mail: nils@gfz-potsdam.de
 

 References:
    Benhama, A., Cliet, C. and Dubesset, M., 1986: Study and
       Application of spatial directional filtering in three 
       component recordings.
       Geophysical Prospecting, vol. 36.
    Kanasewich, E. R., 1981: Time Sequence Analysis in Geophysics, 
       The University of Alberta Press.
    Kanasewich, E. R., 1990: Seismic Noise Attenuation, 
       Handbook of Geophysical Exploration, Pergamon Press, Oxford.
 

 Trace header fields accessed: ns, dt

\end{verbatim}
\pagebreak
\begin{verbatim}
 SUPOLAR - POLarization analysis of three-component data               

 supolar <stdin [optional parameters]                                  

 Required parameters:                                                  
    none                                                               

 Optional parameters:                                                  
    dt=(from header)  time sampling intervall in seconds               
    wl=0.1            correlation window length in seconds             
    win=boxcar        correlation window shape, choose "boxcar",     
                      "hanning", "bartlett", or "welsh
    file=polar        base of output file name(s)                      
    rl=1              1 = rectilinearity evaluating 2 eigenvalues,     
                      2, 3 = rectilinearity evaluating 3 eigenvalues   
    rlq=1.0           contrast parameter for rectilinearity            
    dir=1             1 = 3 components of direction of polarization    
                          (the only three-component output file)       
    tau=0             1 = global polarization parameter                
    ellip=0           1 = principal, subprincipal, and transverse      
                          ellipticities e21, e31, and e32              
    pln=0             1 = planarity measure                            
    f1=0              1 = flatness or oblateness coefficient           
    l1=0              1 = linearity coefficient                        
    amp=0             1 = amplitude parameters: instantaneous,         
                          quadratic, and eigenresultant ir, qr, and er 
    theta=0           1, 2, 3 = incidence angle of principal axis      
    phi=0             1, 2, 3 = horizontal azimuth of principal axis   
    angle=rad         unit of angles theta and phi, choose "rad",    
                      "deg", or "gon
    all=0             1, 2, 3 = set all output flags to that value     
    verbose=0         1 = echo additional information                  


 Notes:                                                                
    Three adjacent traces are considered as one three-component        
    dataset.                                                           
    Correct calculation of angles theta and phi requires the first of  
    these traces to be the vertical component, followed by the two     
    horizontal components (e.g. Z, N, E, or Z, inline, crossline).     
    Significant signal energy on Z is necessary to resolve the 180 deg 
    ambiguity of phi (options phi=2,3 only).                           

    Each calculated polarization attribute is written into its own     
    SU file. These files get the same base name (set with "file=")   
    and the parameter flag as an extension (e.g. polar.rl).            

    In case of a tapered correlation window, the window length wl may  
    have to be increased compared to the boxcar case, because of their 
    smaller effective widths (Bartlett, Hanning: 1/2, Welsh: 1/3).     

 Range of values:                                                      
    parameter     option  interval                                     
    rl            1, 2    0.0 ... 1.0   (1.0: linear polarization)     
    rl            3      -1.0 ... 1.0                                  
    tau, l1       1       0.0 ... 1.0   (1.0: linear polarization)     
    pln, f1       1       0.0 ... 1.0   (1.0: planar polarization)     
    e21, e31, e32 1       0.0 ... 1.0   (0.0: linear polarization)     
    theta         1      -pi/2... pi/2  rad                            
    theta         2, 3    0.0 ... pi/2  rad                            
    phi           1      -pi/2... pi/2  rad                            
    phi           2      -pi  ... pi    rad   (see notes above)        
    phi           3       0.0 ... 2 pi  rad   (see notes above)        



 
 Author: Nils Maercklin, 
         GeoForschungsZentrum (GFZ) Potsdam, Germany, 1998-2001.
         E-mail: nils@gfz-potsdam.de
 

 References:
    Jurkevics, A., 1988: Polarization analysis of three-component
       array data. Bulletin of the Seismological Society of America, 
       vol. 78, no. 5.
    Kanasewich, E. R., 1981: Time Sequence Analysis in Geophysics.
       The University of Alberta Press.
    Kanasewich, E. R., 1990: Seismic Noise Attenuation.
       Handbook of Geophysical Exploration, Pergamon Press, Oxford.
    Meyer, J. H. 1988: First Comparative Results of Integral and
       Instantaneous Polarization Attributes for Multicomponent Seismic
       Data. Institut Francais du Petrole.
    Press, W. H., Teukolsky, S. A., Vetterling, W. T., and Flannery, B. P.
       1996: Numerical Recipes in C - The Art of Scientific Computing.
       Cambridge University Press, Cambridge.
    Samson, J. C., 1973: Description of the Polarisation States of Vector
       Processes: Application to ULF Electromagnetic Fields.
       Geophysical Journal vol. 34, p. 403-419.
    Sheriff, R. E., 1991: Encyclopedic Dictionary of Exploration
       Geophysics. 3rd ed., Society of Exploration Geophysicists, Tulsa.

 Trace header fields accessed: ns, dt
 Trace header fields modified: none

\end{verbatim}
\pagebreak
\begin{verbatim}
 SUADDNOISE - add noise to traces					

 suaddnoise <stdin >stdout  sn=20  noise=gauss  seed=from_clock	

 Required parameters:							
 	if any of f=f1,f2,... and amp=a1,a2,... are specified by the user
	and if dt is not set in header, then dt is mandatory		

 Optional parameters:							
 	sn=20			signal to noise ratio			
 	noise=gauss		noise probability distribution		
 				=flat for uniform; default Gaussian	
 	seed=from_clock		random number seed (integer)		
	f=f1,f2,...		array of filter frequencies (as in sufilter)
	amps=a1,a2,...		array of filter amplitudes		
 	dt= (from header)	time sampling interval (sec)		
	verbose=0		=1 for echoing useful information	

 	tmpdir=	 if non-empty, use the value as a directory path	
		 prefix for storing temporary files; else if the	
	         the CWP_TMPDIR environment variable is set use		
	         its value for the path; else use tmpfile()		

 Notes:								
 Output = Signal +  scale * Noise					

 scale = (1/sn) * (absmax_signal/sqrt(2))/sqrt(energy_per_sample)	

 If the signal is already band-limited, f=f1,f2,... and amps=a1,a2,...	
 can be used, as in sufilter, to bandlimit the noise traces to match	
 the signal band prior to computing the scale defined above.		

 Examples of noise bandlimiting:					
 low freqency:    suaddnoise < data f=40,50 amps=1,0 | ...		
 high freqency:   suaddnoise < data f=40,50 amps=0,1 | ...		
 near monochromatic: suaddnoise < data f=30,40,50 amps=0,1,0 | ...	
 with a notch:    suaddnoise < data f=30,40,50 amps=1,0,1 | ...	
 bandlimited:     suaddnoise < data f=20,30,40,50 amps=0,1,1,0 | ...	


 Credits:
	CWP: Jack Cohen, Brian Sumner, Ken Larner
		John Stockwell (fixed filtered noise option)

 Notes:
	At S/N = 2, the strongest reflector is well delineated, so to
	see something 1/nth as strong as this dominant reflector
	requires S/N = 2*n.

 Trace header field accessed: ns


\end{verbatim}
\pagebreak
\begin{verbatim}
 SUHARLAN - signal-noise separation by the invertible linear		
	    transformation method of Harlan, 1984			

   suharlan <infile >outfile  [optional parameters]			

 Required Parameters:						 	
 <none>								

 Optional Parameters:							
 FLAGS:								
 niter=1	number of requested iterations				
 anenv=1	=1 for positive analytic envelopes			
		=0 for no analytic envelopes (not recommended)		
 scl=0		=1 to scale output traces (not recommended)		
 plot=3	=0 for no plots. =1 for 1-D plots only			
		=2 for 2-D plots only. =3 for all plots			
 norm=1	=0 not to normalize reliability values			
 verbose=1	=0 not to print processing information			
 rgt=2		=1 for uniform random generator				
		=2 for gaussian random generator			
 sts=1		=0 for no smoothing (not recommended)			

 tmpdir= 	 if non-empty, use the value as a directory path	
		 prefix for storing temporary files; else if the	
	         the CWP_TMPDIR environment variable is set use		
	         its value for the path; else use tmpfile()		

 General Parameters:							
 dx=20		offset sampling interval (m)				
 fx=0	  	offset on first trace (m)				
 dt=0.004	time sampling interval (s)				

 Tau-P Transform Parameters:						
 gopt=1	=1 for parabolic transform. =2 for Foster/Mosher	
		=3 for linear. =4 for absolute value of linear		
 pmin1=-400	minimum moveout at farthest offset for fwd transf(ms)	
 pmax1=400	maximum moveout at farthest offset for fwd transf(ms)	
 pmin2=pmin1	minimum moveout at farthest offset for inv transf(ms)	
 pmax2=pmax1	maximum moveout at farthest offset for inv transf(ms)	
 np=100	number of p-values for taup transform			
 prewhite=0.01	prewhitening value (suggested between 0.1 and 0,01)	
 offref=2000	reference offset for p-values (m)			
 depthref=500	reference depth for Foster/Mosher taup (if gopt=4)	
 pmula=pmax1	maximum p-value preserved in the data (ms)		
 pmulb=pmax1	minimum p-value muted on the data (ms)			
 ninterp=0	number of traces to interpolate in input data		

 Extraction Parameters:						
 nintlh=50	number of intervals (bins) in histograms		
 sditer=5	number of steepest descent iterations to compute ps	
 c=0.04	maximum noise allowed in a sample of signal(%)		
 rel1=0.5	reliability value for first pass of the extraction	
 rel2=0.75	reliability value for second pass of the extraction	

 Smoothing Parameters:							", 
 r1=10		number of points for damped lsq vertical smoothing	
 r2=2		number of points for damped lsq horizontal smoothing	


 Output Files:								
 signal=out_signal 	name of output file for extracted signal	
 noise=out_noise 	name of output file for extracted noise		


 Notes:								
 The signal-noise separation algorithm was developed by Dr. Bill Harlan
 in 1984. It can be used to separate events that can be focused by a	
 linear transformation (signal) from events that can't (noise). The	
 linear transform is whatever is well siuted for the application at	
 hand. Here, only the discrete Radon transform is used, so the program	
 is capable of separating events focused by that transform (linear,	
 parabolic or time-invariantly hyperbolic). Should other transform be	
 required, the changes to the program will be relatively		
 straightforward.							

 The reliability parameter is the most critical one to determine what	
 to extract as signal and what to reject as noise. It should be tested	
 for every dataset. The way to test it is to start with a small value,	
 say 0.1 or 0.01. If too much noise is present in the extracted noise,	
 it is too low. If too much signal was extracted, that is, part of the	
 signal was lost, it is too big. All other parameters have good default
 values and should perhaps not be changed in a first encounter with the
 program. The transform parameters are also critical. They should be	
 chosen such that no aliasing is present and such that the range of	
 interesting slopes is spanned by the transform but not much more. The 
 program suradon.c has more documentation on the transform paramters.	



 Credits:
 	Gabriel Alvarez CWP (1995) 
	Some subroutines are direct translations to C from Fortran versions
 	written by Dr. Bill Harlan (1984)

 References:

 	Harlan, S., Claerbout, J., and Roca, F. (1984), Signal/noise
	separation and velocity estimation, Geophysics, v. 49, no. 11,
	p 1869-1880. 

 	Harlan, S. (1988), Separation of signal and noise applied to
	vertical seismic profiles, Geophysics, v. 53, no. 7,
	p 932-946. 

	Alvarez, G. (1995), Comparison of moveout-based approaches to
	ground roll and multiple suppression, MSc., Department of 
	Geophysics, Colorado School of Mines, (Chapter 3 deals
	exclusively with this method).


\end{verbatim}
\pagebreak
\begin{verbatim}
 SUJITTER - Add random time shifts to seismic traces			

     sujitter <stdin >stdout  [optional parameters]	 		

 Required parameters:							
	none								
 Optional Parameters:							
 	seed=from_clock    	random number seed (integer)            
	min=1 			minimum random time shift (samples)	
	max=1 			maximum random time shift (samples)	
	pon=1 			shift can be positive or negative	
				=0 shift is positive only		
	fldr=0 			each trace has new shift		
				=1 new shift when fldr header field changes
 Notes:								
 Useful for simulating random statics. See also:  suaddstatics		


 Credits:
	U of Houston: Chris Liner 
	UH:  Chris added fldr, min, pon options 12/10/08


\end{verbatim}
\pagebreak
\begin{verbatim}
 SUFLIP - flip a data set in various ways			

 suflip <data1 >data2 flip=1 verbose=0				

 Required parameters:						
 	none							

 Optional parameters:						
 	flip=1 	rotational sense of flip			
 			+1  = flip 90 deg clockwise		
 			-1  = flip 90 deg counter-clockwise	
 			 0  = transpose data			
 			 2  = flip right-to-left		
 			 3  = flip top-to-bottom		
 	tmpdir=	 if non-empty, use the value as a directory path
		 prefix for storing temporary files; else if	
	         the CWP_TMPDIR environment variable is set use	
	         its value for the path; else use tmpfile()	

 	verbose=0	verbose = 1 echoes flip info		

 NOTE:  tr.dt header field is lost if flip=-1,+1.  It can be	
        reset using sushw.					

 EXAMPLE PROCESSING SEQUENCES:					
   1.	suflip flip=-1 <data1 | sushw key=dt a=4000 >data2	

   2.	suflip flip=2 <data1 | suflip flip=2 >data1_again	

   3.	suflip tmpdir=/scratch <data1 | ...			

 Caveat:  may fail on large files.				

 Credits:
	CWP: Chris Liner, Jack K. Cohen, John Stockwell

 Caveat:
	right-left flip (flip = 2) and top-bottom flip (flip = 3)
	don't require the matrix approach.  We sacrificed efficiency
	for uniform coding.

 Trace header fields accessed: ns, dt
 Trace header fields modified: ns, dt, tracl

\end{verbatim}
\pagebreak
\begin{verbatim}
 SUFWMIX -  FX domain multidimensional Weighted Mix			

	sufwmix < stdin > stdout [optional parameters]			

 Required parameters:							
 key=key1,key2,..	Header words defining mixing dimension		
 dx=d1,d2,..		Distance units for each header word		
 Optional parameters:							
 keyg=ep		Header word indicating the start of gather	
 vf=0			=1 Do a frequency dependent mix			
 vmin=5000		Velocity of the reflection slope		
			than should not be attenuated			
 Notes:								
 Trace with the header word mark set to one will be			
 the output trace 							
  (a work in progress)							


 Credits:  

  Potash Corporation: Balazs Nemeth, Saskatoon Saskatchewan CA,
   given to CWP in 2008



\end{verbatim}
\pagebreak
\begin{verbatim}
 SUMATH - do math operation on su data 		

 suop <stdin >stdout op=mult					

 Required parameters:						
	none							

 Optional parameter:						
	op=mult		operation flag				
			--------- operations -----------------	
			add   : o = i + a    (o=out; i=in)	
			sub   : o = i - a			
			mult  : o = i * a  			
			div   : o = i / a  			
			pow   : o = i ^ a			
			spow  : o = sgn(i) * abs(i) ^ a  	
			--------- operation parameter --------	
	a=1							
	copy=1		n>1 copy each trace n times		

 Note:								
 There is overlap between this program and "sugain" and	
 "suop

 Credits:

	U Arkansas: Chris Liner Jun 2013

 Notes:

\end{verbatim}
\pagebreak
\begin{verbatim}
 SUMIX - compute weighted moving average (trace MIX) on a panel	
	  of seismic data						

 sumix <stdin >sdout 							
 mix=.6,1,1,1,.6	array of weights for weighted average		


 Note: 								
 The number of values defined by mix=val1,val2,... determines the number
 of traces to be averaged, the values determine the weights.		

 Examples: 								
 sumix <stdin mix=.6,1,1,1,.6 >sdout 	(default) mix over 5 traces weights
 sumix <stdin mix=1,1,1 >sdout 	simple 3 trace moving average	


 Author:
	CWP: John Stockwell, Oct 1995

 Trace header fields accessed: ns

\end{verbatim}
\pagebreak
\begin{verbatim}
 SUOP2 - do a binary operation on two data sets			

 suop2 data1 data2 op=diff [trid=111] >stdout				

 Required parameters:							
 	none								

 Optional parameter:							
 	op=diff		difference of two panels of seismic data	
 			=sum  sum of two panels of seismic data		
 			=prod product of two panels of seismic data	
 			=quo quotient of two panels of seismic data	
 			=ptdiff differences of a panel and single trace	
 			=ptsum sum of a panel and single trace		
 			=ptprod product of a panel and single trace	
 			=ptquo quotient of a panel and single trace	
 			=zipper do "zipper" merge of two panels	
			=zippol convert polar to rectangular and then zip

  trid=FUNPACKNYQ	output trace identification code. (This option  
 			is active only for op=zipper)			
			For SU version 39-43 FUNPACNYQ=111		
 			(See: sukeyword trid     for current value)	


 Note1: Output = data1 "op" data2 with the header of data1		

 Note2: For convenience and backward compatibility, this		
 	program may be called without an op code as:			

 For:  panel "op" panel  operations: 				
 	susum  file1 file2 == suop2 file1 file2 op=sum			
 	sudiff file1 file2 == suop2 file1 file2 op=diff			
 	suprod file1 file2 == suop2 file1 file2 op=prod			
 	suquo  file1 file2 == suop2 file1 file2 op=quo			

 For:  panel "op" trace  operations: 				
 	suptsum  file1 file2 == suop2 file1 file2 op=ptsum		
 	suptdiff file1 file2 == suop2 file1 file2 op=ptdiff		
 	suptprod file1 file2 == suop2 file1 file2 op=ptprod		
 	suptquo  file1 file2 == suop2 file1 file2 op=ptquo		

 Note3: If an explicit op code is used it must FOLLOW the		
	filenames.							

 Note4: With op=quo and op=ptquo, divide by 0 is trapped and 0 is returned.

 Note5: Weighted operations can be specified by setting weighting	
	coefficients for the two datasets:				
	w1=1.0								
	w2=1.0								

 Note6: With op=zipper, it is possible to set the output trace id code 
 		(See: sukeyword trid)					
  This option processes the traces from two files interleaving its samples.
  Both files must have the same trace length and must not longer than	
  SU_NFLTS/2  (as in SU 39-42  SU_NFLTS=32768).			

  Being "tr1" a trace of data1 and "tr2" the corresponding trace of
  data2, The merged trace will be :					

  tr[2*i]= tr1[i]							
  tr[2*i+1] = tr2[i]							

  The default value of output tr.trid is that used by sufft and suifft,
  which is the trace id reserved for the complex traces obtained through
  the application of sufft. See also, suamp.				

 Note 7: op=zippol is like op=zipper, but the input samples are polar	
	(amplitude and phase) and are converted to cartesian (real, imag)
	before interleaving them.					

 For operations on non-SU binary files  use:farith 			

 Credits:
	SEP: Shuki Ronen
	CWP: Jack K. Cohen
	CWP: John Stockwell, 1995, added panel op trace options.
	: Fernando M. Roxo da Motta <petro@roxo.org> - added zipper op

 Notes:
	If efficiency becomes important consider inverting main loop
	and repeating operation code within the branches of the switch.

\end{verbatim}
\pagebreak
\begin{verbatim}
 SUOP - do unary arithmetic operation on segys 		

 suop <stdin >stdout op=abs					

 Required parameters:						
	none							

 Optional parameter:						
	op=abs		operation flag				
			abs   : absolute value			
			avg   : remove average value		
			ssqrt : signed square root		
			sqr   : square				
			ssqr  : signed square			
			sgn   : signum function			
			exp   : exponentiate			
			sexp  : signed exponentiate		
			slog  : signed natural log		
			slog2 : signed log base 2		
			slog10: signed common log		
			cos   : cosine				
			sin   : sine				
			tan   : tangent				
			cosh  : hyperbolic cosine		
			sinh  : hyperbolic sine			
			tanh  : hyperbolic tangent		
			cnorm : norm complex samples by modulus ", 
			norm  : divide trace by Max. Value	
			db    : 20 * slog10 (data)		
			neg   : negate value			
			posonly : pass only positive values	
			negonly : pass only negative values	
                       sum   : running sum trace integration   
                       diff  : running diff trace differentiation
                       refl  : (v[i+1] - v[i])/(v[i+1] + v[i]) 
			mod2pi : modulo 2 pi			
			inv   : inverse				
			rmsamp : rms amplitude			
                       s2v   : sonic to velocity (ft/s) conversion     
                       s2vm  : sonic to velocity (m/s) conversion     
                       d2m   : density (g/cc) to metric (kg/m^3) conversion 
                       drv2  : 2nd order vertical derivative 
                       drv4  : 4th order vertical derivative 
                       integ : top-down integration            
                       spike : local extrema to spikes         
                       saf   : spike and fill to next spike    
                       freq  : local dominant freqeuncy        
                       lnza  : preserve least non-zero amps    
                       --------- window operations ----------- 
                       mean  : arithmetic mean                 
                       despike  : despiking based on median filter
                       std   : standard deviation              
                       var   : variance                        
       nw=21           number of time samples in window        
                       --------------------------------------- 
			nop   : no operation			

 Note:	Binary ops are provided by suop2.			
 Operations inv, slog, slog2, and slog10 are "punctuated",	", 
 meaning that if, the input contains 0 values,			
 0 values are returned.					",	

 For file operations on non-SU format binary files use:  farith

 Credits:

 CWP: Shuki Ronen, Jack K Cohen (c. 1987)
  Toralf Foerster: norm and db operations, 10/95.
  Additions by Reg Beardsley, Chris Liner, and others.

 Notes:
	If efficiency becomes important consider inverting main loop
      and repeating operation code within the branches of the switch.

	Note on db option.  The following are equivalent:
	... | sufft | suamp | suop op=norm | suop op=slog10 |\
		sugain scale=20| suxgraph style=normal

	... | sufft | suamp | suop op=db | suxgraph style=normal

\end{verbatim}
\pagebreak
\begin{verbatim}
 SUPERMUTE - permute or transpose a 3d datacube	 		

 supermute <stdin >sdout	 					

 Required parameters:							
 none									

 Optional parameters:							
 n1=ns from header		number of samples in the fast direction	
 n2=ntr from header		number of samples in the med direction	",	
 n3=1				number of samples in the slow direction	

 o1=1				new fast direction			
 o2=2				new med direction			
 o3=3				new slow direction			

 d1=1				output interval in new fast direction	
 d2=1				output interval in new med direction	
 d3=1				output interval in new slow direction	

 Notes:								
 header fields d1 and d2 default to d1=1.0 and d2=1.0			

 Credits:

	VT: Matthias Imhof

 Trace header fields accessed: ns, ntr
 Trace header fields modified: d1=1, f1=1, d2=1, f2=1, ns, ntr

\end{verbatim}
\pagebreak
\begin{verbatim}
 SUSIMAT - Correlation similarity matrix for two traces.		
 		 Output is zero lag of cross-correlation of traces,	
 		 or linear regression correlation coefficient.		
 		 Horizontal axis is time of trace 1, vertical is trace 2.

 susimmat <data12 sufile=data2 >dataout 				

 Required parameters:							
 sufile=		file containing SU traces to use as filter	

 Optional parameters:							
 panel=0		use only the first trace of sufile as filter	
 		      	=1 compute sim matrix trace by trace an entire  
                        gather						
 mt=21			operator window length (odd integer)		
 eps=1e-3		stability parameter				
 taper=0		no taper to data fragments			
 			=1 apply exponential taper (1/e at ends)	
 method=1		use xcorrelation as similarity meausure		
 			=2 same but normalized by (rms+eps)		
 			=3 use linear regression CC			
 			=4 use mt-dimensional vector angle		

 EXAMPLE PROCESSING SEQUENCES:						
   1. Look for all possible alignments of OBC P and Z data  		
   	susimmat < P_Z.su sufile=OBC_P.su  mt=71 > P_Zxcor.su		

 Note:  xcor window is collapsed as needed to compute edge values	
 It is quietly assumed that the time sampling interval on the  single  
 trace									
 and the output traces is the same as that on the traces in the input  
 file.  								
 The sufile may actually have more than one trace, but only the first  
 trace is used when panel=0. When panel=1 the number of traces in the 
 sufile MUST be  the same as the number of traces in the input.	

 Credits:
	U Arkansas: Chris Liner (originally 11/2009 at U Houston)
      CWP: John Stockwell, some i/o modifications  Jul 2015

 References: 
 Liner, Christopher L., and Robert G. Clapp. "Nonlinear pairwise 
	 alignment of seismic traces." The Leading Edge 23.11 (2004): 1146-1150.
 Liner, C. L, and R. G. Clapp (2004), Nonlinear pairwise alignment of 
	 seismic traces GEOPHYSICS, VOL. 69, NO. 6 
	(NOVEMBER-DECEMBER 2004); P. 1552–1559, 7 FIGS.  10.1190/1.1836828 
 
 Caveat:

 Trace header fields accessed: ns, dt
 Trace header fields modified: ns, dt, tracl

\end{verbatim}
\pagebreak
\begin{verbatim}
 SUVCAT -  append one data set to another, with or without an  ", 
           overlapping	region.  Data in the overlap may be     
           determined by one of several methods.               

 suvcat data1 data2 >stdout					

 Required parameters:						
        none							

 Optional parameters for overlapping sections:			

  taplen=0    Length of overlap in integer number of           
                  samples.(Default is 0.)                      

  taptype=0    Type of taper or combination method in the	
                  overlap region.  0 - average                 
                                   1 - maximum magnitude       
                                   2 - cosine scaled           
                                   3 - summation               

 Computational Notes:						
 
 This program vertically concatenates traces from data2 onto   
 the end of the corresponding traces in data1, with a region   
 of overlap, defined by taplen.  Data in the overlapping       ", 
 region is combined by the method specified by taptype. The    
 currently available methods are:                              

     taptype=0    output is assigned the unweighted average of 
                  each point in the overlap                    
     taptype=1    output is assigned the value of the maximum  
                  absolute value of each point in the overlap  
     taptype=2    output is assigned the weighted average of   
                  each point in the overlap, where the output  
                  is the sum of cos(x) times the values on the 
                  first section, and 1-cos(x) times the values 
                  on the second section, where x is factor that
                  goes from 0 to pi/2 across the overlap. This 
                  favors the upper section in the upper part of
                  the overlap, and favors the lower section in 
                  the lower part of the overlap.               
     taptype=3    output is assigned the sum of the amplitudes 
                  at each sample in the overlap                


 Credits:
	CWP: Jack K. Cohen, Michel Dietrich (Original SUVCAT)
	     Steven D. Sheaffer (modifed to include overlap) 
 IfG Kiel: Thies Beilecke (added taptype=3)

 Trace header fields accessed:  ns
 Trace header fields modified:  ns

\end{verbatim}
\pagebreak
\begin{verbatim}
 SUVLENGTH - Adjust variable length traces to common length   	

 suvlength <vdata >stdout					

 Required parameters:						
 	none							

 Optional parameters:						
 	ns=	output number of samples (default: 1st trace ns)
\end{verbatim}
\pagebreak
\begin{verbatim}
 SUFBPICKW - First break auto picker				

   sufbpickw < infile >outfile					

 Required parameters:						
  none								
 Optional parameters:						
 keyg=ep						 	
 window=.03	Length of forward and backward windows (s)	
 test=1	Output the characteristic function	 	
		This can be used for testing window size	
 Template							
 o=		offset...				  	
 t=		time pairs for defining first break search	
			window centre				
 tdv=.05	Half length of the search window		

 If the template is specified the maximum value of the		
 characteristic function is searched in the window		
  defined by the template only.Default is the whole trace.	

  The time of the pick is stored in header word unscale 	

   
 segy data 
segy *trp;				/* SEGY trace array
segy trtp;				/* SEGY trace
int 
main( int argc, char *argv[] )
{
	
	segy **rec_o;	   /* trace header+data matrix  
	
	cwp_String keyg;
	cwp_String typeg;		
	Value valg;
		   	
	int first=0;		/* true when we passed the first gather
	int ng=0;
	float dt;
	int nt;
	int ntr;

	unsigned int np;	/* Number of points in pick template
	float *t=NULL;		/* array defining pick template times
	float *o=NULL;		/* array defining pick template offsets
	
	float window;
	int iwindow;
	int *itimes;
	int *itimes2;
	float *offset;
	float tdv;
	int itdv;
	float *find;
	int nowindow=0;
	int test=0;
	
	FILE *ttp;
		
	initargs(argc, argv);
   	requestdoc(1);
	
	if (!countparval("t")) {
		np=2;
		nowindow=1;
	} else {
		np=countparval("t");
	}
	
	if(!nowindow) { 
		t  = ealloc1float(np);
		o  = ealloc1float(np);
	
		if( np == countparval("o")) {
			getparfloat("t",t);
			getparfloat("o",o);
		} else {
			err(" t and o has different number of elements\n");
		}
	}



	ttp = efopen("test.su","w");
		
	if (!getparstring("keyg", &keyg)) keyg ="ep";
	
	if (!getparfloat("window",&window)) window =0.02;
	if (!getparfloat("tdv",&tdv)) tdv = -1.0;
	if (!getparint("test",&test)) test = 1;
	
        checkpars();
	/* get information from the first header
	rec_o = get_gather(&keyg,&typeg,&valg,&nt,&ntr,&dt,&first);
	
	iwindow=NINT(window/dt);
	if(tdv==-1.0) {
		itdv=nt;
	} else {
		itdv=NINT(2.0*tdv/dt);
	}
	
	if(ntr==0) err("Can't get first record\n");
	do {
		ng++;
		
		itimes = ealloc1int(ntr);
		itimes2 = ealloc1int(ntr);
		offset = ealloc1float(ntr);
		
		/* Phase 1
		/* Loop through traces
		{ int itr,ifbt;
		  int it;
		  float fbt;
		  float ampb;		
		  float ampf;
		  float *wf,*wb;
		  float *cf;		/* Characteristic function
		  
		  
		  cf=ealloc1float(nt);
		  find=ealloc1float(nt);
		  for(it=0;it<nt;it++)
		  	find[it]=(float)it;
		  
		  for(itr=0;itr<ntr;itr++) {
		  
		  	memset( (void *) cf, (int) '\0', nt*FSIZE);
			
			if(nowindow) {
				ifbt=0;
			} else {
		  	/* Linear inperpolation of estimtated fb time
				offset[itr] =(*rec_o[itr]).offset; 
		  		intlin(np,o,t,t[0],t[np-1],1,&offset[itr],&fbt);
		  		ifbt = NINT(fbt/dt);
			}
			
		  
			wb = &(*rec_o[itr]).data[0];
			wf = &(*rec_o[itr]).data[iwindow];
			ampb = sasum(iwindow,wb,1)+FLT_EPSILON;
			ampf = sasum(iwindow,wf,1)+FLT_EPSILON;
		  	for(it=iwindow;it<nt-iwindow-1;it++) {
				cf[it] = ampf/ampb;
				/* setup next window
				ampb -= fabs((*rec_o[itr]).data[it-iwindow]);
				ampf -= fabs((*rec_o[itr]).data[it]);
				ampb += fabs((*rec_o[itr]).data[it]);
				ampf += fabs((*rec_o[itr]).data[it+iwindow]);
			}
			/* Smooth the characteristic function
			smooth_segmented_array(&find[iwindow],&cf[iwindow],nt-iwindow-1,iwindow,1,5);
		  	
			/* find the maximum*/
			it=MIN(MAX(ifbt-itdv/2,0),nt-1);
			

			wb = &cf[it];
			
			itimes[itr] = isamax(itdv,wb,1)+it;
			
			/* Final check
			if(itimes[itr] == ifbt-itdv/2 || itimes[itr] == ifbt+itdv/2)
				itimes[itr]=0;
				 
			(*rec_o[itr]).unscale=dt*itimes[itr];
			
			if(test)
				memcpy( (void *) &(*rec_o[itr]).data[0], (const void *) cf, nt*FSIZE);

		  }
		  free1float(cf);
		  free1float(find);
		}

		free1int(itimes);
		free1int(itimes2);
		free1float(offset);
		
		rec_o = put_gather(rec_o,&nt,&ntr);
		rec_o = get_gather(&keyg,&typeg,&valg,&nt,&ntr,&dt,&first);
	} while(ntr);
	return EXIT_SUCCESS;

}
\end{verbatim}
\pagebreak
\begin{verbatim}
 SUFNZERO - get Time of First Non-ZERO sample by trace              

  sufnzero <stdin >stdout [optional parameters] 			

 Required parameters:							
	none								

 Optional parameters: 							
	mode=first   	Output time of first non-zero sample		
	             	=last for last non-zero sample			
	             	=both for both first & last non-zero samples    

	min=0   	Threshold value for considering as zero         
			Any abs(sample)<min is considered to be zero	

	key=key1,...	Keyword(s) to print				


 Credits:
      Geocon : Garry Perratt
	based on surms by the same, itself based on sugain & sumax by:
	CWP : John Stockwell

\end{verbatim}
\pagebreak
\begin{verbatim}
 SUPICKAMP - pick amplitudes within user defined and resampled window	

   supickamp <stdin >stdout d2=  [optional parameters]			

 Required parameters:							
 d2=		   sampling interval for slow dimension			
			(required if key-parameter not specified)	
 Optional parameters:							
 key= 			Key header word specifying trace offset 	
 			(alternatively, specify d2,x2beg)		

 x_above=		array of lateral position values   		
 			(upper window corner)				
 t_above=		array of time values   				
 			(upper window corner)				

  ... or input via files:						
 t_xabove=		file containing time and lateral position values
 			(upper window corner)				
 t_xbelow=		file containing time and lateral position values
 			(lower window corner)				
 wl=		   	window width if t_xbelow is not specified	
			(No windowing if not specified)		 	

 dt_resamp=dt	  resampling interval within pick window	  	
			(dt has to come from trace headers)		
 tmin=0		minimum time in input trace			
 x2beg=0		first lateral position				
 format=ascii 		write ascii data to stdout			
			 =binary for binary floats to stdout		
 verbose=1 		writes complete  pick information into outpar   
			=2 writes complete pick information into outpar	
			   in tab-delimited column format		
 outpar=/dev/tty	output parameter file; contains output		
					from verbose			
 arg1=max		output (first dimension) to stdout		
 arg2=i2		output (second dimension) to stdout		
			(see notes for other options)			
 Notes: 								

 Window can be defined using						
 (1)   vectors x_above, t_above, [wl]					
 (2)   file  t_xabove, [wl]	or					
 (3)   files t_xabove, t_xbelow					

 files t_xabove, t_xbelow can be generated using xwigb's picking	
 algorithm. The lateral positions have to be monotonically increasing  
 or decreasing for both vector and file input.				
 verbose=1 or 2 writes min, max, abs[max], energy and associated times	
 tmin,tmax,tabs to outpar, together with global values. verbose=0	
 only outputs global values.						
 Acceptable arg-parameters for lateral positions are			
 (1) x2   (2) i2 = trace number					

 If key=keyword is set, then the values of x2 are taken from the header
 field represented by the keyword (for example key=offset)		
 Type	sukeyword -o   to see the complete list of SU keywords.		



 Credits:

	CWP: Andreas Rueger July 06, 1996
	MTU: David Forel,   Jan. 26, 2005  Add verbose=2 option

\end{verbatim}
\pagebreak
\begin{verbatim}
 SUCVS4FOWLER --compute constant velocity stacks for input to Fowler codes

 Required Parameter:							
 ncdps=		number of input cdp gathers			
 Optional Parameters:							
 vminstack=1500.	minimum velocity panel in m/s to output		
 nvstack=180		number of stacking velocity panels to compute	
			( Let offmax be the maximum offset, fmax be	
			the maximum freq to preserve, and tmute be	
			the starting mute time in sec on offmax, then	
			the recommended value for nvstack would be	
			nvstack = 4 +(offmax*offmax*fmax)/(0.6*vmin*vmin*tmute)
			---you may want to make do with less---)		
 lmute=24		length of mute taper in ms			
 nonhyp=1		1 if do mute at 2*offset/vhyp to avoid		
			non-hyperbolic moveout, 0 otherwise		
 vhyp=2500.		velocity to use for non-hyperbolic moveout mute	
 lbtaper=0		length of bottom taper in ms			
 lstaper=0		length of side taper in traces			
 dtout=1.5*dt		output sample rate in s,			
			note: typically fmax=salias*0.5/dtout		
 mxfold=120		maximum number of offsets/input cmp		
 salias=0.8		fraction of output frequencies to force within sloth
			antialias limit.  This controls muting by offset of
			the input data prior to computing the cv stacks	
			for values of choose=1 or choose=2.		
 Required trace header words on input are ns, dt, cdp, offset.		



 Author:  (Visitor to CSM from Mobil): John E. Anderson, Spring 1994
 
	References:

	Fowler, P., 1988, Ph.D. Thesis, Stanford University.
	Anderson, J.E., Alkhalifah, T., and Tsvankin, I., 1994, Fowler
		DMO and time migration for transversely isotropic media,
		1994 CWP project review



\end{verbatim}
\pagebreak
\begin{verbatim}
 SUDIVSTACK -  Diversity Stacking using either average power or peak   
               power within windows                                    


 Required parameters:                                                  
    none                                                               

 Optional parameters:                                                  
    key=tracf        key header word to stack on                       
    winlen=0.064     window length in seconds.                         
                     typical choices: 0.064, 0.128, 0.256,             
                                      0.512, 1.024, 2.048, 4.096       
                     if not specified the entire trace is used         

    peak=1           peak power option default is average power        

 Notes:                                                                
    Diversity stacking is a noise reduction technique used in the      
    summation of duplicate data. Each trace is scaled by the inverse   
    of its average power prior to stacking.  The composite trace is    
    then renormalized by dividing by the sum of the scalers used.      

    This program stacks adjacent traces having the same key header     
    word, which can be specified by the key parameter. The default     
    is "tracf" (trace number within field record).                   
    For more information on key header words, type "sukeyword -o".   

 Examples:                                                             
    For duplicate field shot records:                                  
        susort < field.data tracf | sudivstack > stack.data            
    For CDP ordered data:                                              
        sudivstack < cdp.data key=cdp > stack.data                     


 
 Author: Mary Palen-Murphy,
         Masters' candidate, Colorado School of Mines,
         Geophysics Department, 1994

 Implementation of "key=" option: Nils Maercklin,
         GeoForschungsZentrum (GFZ) Potsdam, Germany, 2002.

 References:

    Embree, P.,1968, Diversity seismic record stacking method and system:
        U.S. patent 3,398,396.
    Gimlin, D. R., and Smith, J. W., 1980, A comparison of seismic trace 
        summing techniques: Geophysics, vol. 45, pages 1017-1041.

 Trace header fields accessed: ns, dt, key=keyword
 Trace header fields modified: tracl


\end{verbatim}
\pagebreak
\begin{verbatim}
 SUPWS - Phase stack or phase-weighted stack (PWS) of adjacent traces	
	 having the same key header word				

 supws <stdin >stdout [optional parameters]				

 Required parameters:							
	none								

 Optional parameters:						 	
	key=cdp	   key header word to stack on				
	pwr=1.0	   raise phase stack to power pwr			
	dt=(from header)  time sampling intervall in seconds		
	sl=0.0		window length in seconds used for smoothing	
			of the phase stack (weights)			
	ps=0		0 = output is PWS, 1 = output is phase stack	
	verbose=0	 1 = echo additional information		

 Note:								 	
	Phase weighted stacking is a tool for efficient incoherent noise
	reduction. An amplitude-unbiased coherency measure is designed	
	based on the instantaneous phase, which is used to weight the	
	samples of an ordinary, linear stack. The result is called the	
	phase-weighted stack (PWS) and is cleaned from incoherent noise.
	PWS thus permits detection of weak but coherent arrivals.	

	The phase-stack (coherency measure) has values between 0 and 1.	

	If the stacking is over cdp and the PWS option is set, then the	
	offset header field is set to zero. Otherwise, output traces get
	their headers from the first trace of each data ensemble to stack,
	including the offset field. Use "sushw" afterwards, if this is
	not acceptable.							



 Author: Nils Maercklin,
	 GeoForschungsZentrum (GFZ) Potsdam, Germany, 2001.
	 E-mail: nils@gfz-potsdam.de

 References:
	B. L. N. Kennett, 2000: Stacking three-component seismograms.
	 Geophysical Journal International, vol. 141, p. 263-269.
	M. Schimmel and H. Paulssen, 1997: Noise reduction and detection
	 of weak , coherent signals through phase-weighted stacks.
	 Geophysical Journal International, vol. 130, p. 497-505.
	M. T. Taner, A. F. Koehler, and R. E. Sheriff, 1979: Complex
	 seismic trace analysis. Geophysics, vol. 44, p. 1041-1063.

 Trace header fields accessed: ns
 Trace header fields modified: nhs, offset


\end{verbatim}
\pagebreak
\begin{verbatim}
 SURECIP - sum opposing offsets in prepared data (see below)	

 surecip <stdin >stdout	 		               	

 Sum traces with equal positive and negative offsets (i.e. assume
 reciprocity holds). 						

 Usage:							
	suabshw <data >absdata					
	susort cdp offset <absdata | surecip >sumdata		

 Note that this processing stream can be simply evoked by:	

	recip data sumdata					


 Credits:
	SEP: Shuki Ronen
	CWP: Jack Cohen

 Caveat:
	The assumption is that this operation is not a mainstay processing
	item.  Hence the recommended implemention via the 'recip' shell
	script.  If it becomes a mainstay, then a much faster code can
	quickly drummed up by incorporating portions of suabshw and
	susort.

 Trace header fields accessed: ns
 Trace header fields modified: nhs, tracl, sx, gx

\end{verbatim}
\pagebreak
\begin{verbatim}
 SUSTACK - stack adjacent traces having the same key header word

     sustack <stdin >stdout [Optional parameters]		

 Required parameters:						
 	none							

 Optional parameters: 						
 	key=cdp		header key word to stack on		
 	normpow=1.0	each sample is divided by the		
			normpow'th number of non-zero values	
			stacked (normpow=0 selects no division)	
	repeat=0	=1 repeats the stack trace nrepeat times
	nrepeat=10	repeats stack trace nrepeat times in	
	          	output file				
 	verbose=0	verbose = 1 echos information		

 Notes:							
 ------							
 The offset field is set to zero on the output traces, unless	
 the user is stacking with key=offset. In that case, the value 
 of the offset field is left unchanged. 		        

 Sushw can be used afterwards if this is not acceptable.	

 For VSP users:						
 The stack trace appears ten times in the output file when	
 setting repeat=1 and nrepeat=10. Corridor stacking can be	
 achieved by properly muting the upgoing data with SUMUTE	
 before stacking.						


 Credits:
	SEP: Einar Kjartansson
	CWP: Jack K. Cohen, Dave Hale
	CENPET: Werner M. Heigl - added repeat trace functionality

 Note:
	The "valxxx" subroutines are in su/lib/valpkge.c.  In particular,
      "valcmp" shares the annoying attribute of "strcmp" that
		if (valcmp(type, val, valnew) {
			...
		}
	will be performed when val and valnew are different.

 Trace header fields accessed: ns
 Trace header fields modified: nhs, tracl, offset

\end{verbatim}
\pagebreak
\begin{verbatim}
 SUADDSTATICS - ADD random STATICS on seismic data			

 suaddstatics required parameters [optional parameters] > stdout	

 Required parameters:							
 shift=		the static shift will be generated 	 	
			randomly in the interval [+shift,-shif] (ms)	
 sources=		number of source locations			
 receivers=		number of receiver locations			
 cmps=			number of common mid point locations		
 maxfold=		maximum fold of input data			
 datafile=		name and COMPLETE path of the input file	

 Optional parameters:							
 dt=tr.dt			time sampling interval (ms)		
 seed=getpid()		 seed for random number generator		
 verbose=0			=1 print useful information		

 Notes:								
 Input data should be sorted into cdp gathers.				

 SUADDSTATICS applies static time shifts in a surface consistent way on
 seismic data sets. SUADDSTATICS writes the static time shifts in the  
 header field TSTAT. To perform the actual shifts the user should use 	
 the program SUSTATIC after SUADDSTATICS. SUADDSTATICS outputs the	
 corrupted data set to stdout.						

 Header field used by SUADDSTATICS: cdp, sx, gx, tstat, dt.		



 Credits: CWP Wences Gouveia, 11/07/94,  Colorado School of Mines

\end{verbatim}
\pagebreak
\begin{verbatim}
 SURANDSTAT - Add RANDom time shifts STATIC errors to seismic traces	

     surandstat <stdin >stdout  [optional parameters]	 		

 Required parameters:							
	none								
 Optional Parameters:							
 	seed=from_clock    	random number seed (integer)            
	max=tr.dt 		maximum random time shift (ms)		
	scale=1.0		scale factor for shifts			


 Credits:
	U Houston: Chris Liner c. 2009


\end{verbatim}
\pagebreak
\begin{verbatim}
 SURESSTAT - Surface consistent source and receiver statics calculation

   suresstat fn=  [optional parameters]				

 Required parameters: 							
 fn=		seismic file				
 ssol=		output file source statics				
 rsol=		output file receiver statics				

 Optional parameters:							
 ntpick=50 	maximum static shift (samples)         			
 niter=5 	number of iterations					
 imax=100000 	largest shot (fldr),reciver(tracf) or cmp(cdp) number	
 sub=0 	subtract super trace 1 from super trace 2 (=1)		
 		sub=0 strongly biases static to a value of 0		
 mode=0 	use global maximum in cross-correllation window		
		=1 choose the peak perc=percent smaller than the global max.
 perc=10. 	percent of global max (used only for mode=1)		
 verbose=0 	print diagnostic output (verbose=1)                     

 Notes:								
 Estimates surface-consistent source and receiver statics, meaning that
 there is one static correction value estimated for each shot and receiver
 position.								

 The method employed here is based on the method of Ronen and Claerbout:
 Geophysics 50, 2759-2767 (1985).					

 The input data are NMO-corrected and sorted into shot gathers ( fldr)  
 Rreceiver id position should be stored in headerword tracf	        
 The output files are binary files containing the source and receiver	
 statics, as a function of shot number (trace header fldr) and      	
 receiver station number (trace header tracf). 			

 The code builds a supertrace1 and supertrace2, which are subsequently	
 cross-correllated. The program then picks the time lag associated with
 the largest peak in the cross-correllation according to two possible	
 criteria set by the parameter "mode". If mode=0, the maximum of the	
 cross-correllation window is chosen. If mode=1, the program will pick 
 a peak which is up to perc=percent smaller than the global maximum, but
 closer to zero lag than the global maximum.	(Choosing mode=0 is	
 recommended.)								

 The geometry can be irregular: the program simply computes a static 	
 correction for each shot record (fldr=1 to fldr=nshot), with any missing 
 shots being assigned a static of 0.  A static correction for each    	
 receiver station (tracf=1 to tracf=nr) is calculated, with missing    
 receivers again assigned a static of 0.                               ", 
 To window out the most cohherent region use suwind tmin= tmax= and 	
 save the result into a file. This will reduce the amount of time  	
 the code will spent on scaning the file,since the file is much smaller
 The ntpick parameter sets the maximum allowable shift desired (in	
   samples NOT time).							

 To apply the static corrections, use sustatic with hdrs=3		

 Reference:

  Ronen, J. and Claerbout, J., 1985, Surface-consistent residual statics
      estimation  by stack-power maximization: Geophysics, vol. 50,
      2759-2767.

 Credits:
	CWP: Timo Tjan, 4 October 1994

      rewritten by Thomas Pratt, USGS, Feb. 2000. 
      Modified by A. Bitri, BRGM-France, Apr. 2015
 Trace header fields accessed: ns, dt, tracf, fldr, cdp

\end{verbatim}
\pagebreak
\begin{verbatim}
 SUSTATICB - Elevation static corrections, apply corrections from	
	      headers or from a source and receiver statics file	
	      (beta submitted by J. W. Neese)				

     sustaticB <stdin >stdout  [optional parameters]	 		

 Required parameters:							
	none								
 Optional Parameters:							
	v0=v1 or user-defined	or from header, weathering velocity	
	v1=user-defined		or from header, subweathering velocity	
	hdrs=0			=1 to read statics from headers		
 				=2 to read statics from files		
				=3 to read from output files of suresstat
	sign=1			apply static correction (add tstat values)
				=-1 apply negative of tstat values	
 Options when hdrs=2 and hdrs=3:					
	sou_file=		input file for source statics (ms) 	
	rec_file=		input file for receiver statics (ms) 	
	ns=240 		(2)number of sources; (3) max fldr	
	nr=335 			number of receivers 			
	no=96 			number of offsets			

 Notes:								
 For hdrs=1, statics calculation is not performed, statics correction  
 is applied to the data by reading statics (in ms) from the header.	

 For hdrs=0, field statics are calculated, and				
 	input field sut is assumed measured in ms.			
 	output field sstat = 10^scalel*(sdel - selev + sdepth)/swevel	
 	output field gstat = sstat - sut/1000.				
 	output field tstat = sstat + gstat + 10^scalel*(selev - gelev)/wevel

 For hdrs=2, statics are surface consistently obtained from the 	
 statics files. The geometry should be regular.			
 The source- and receiver-statics files should be unformated C binary 	
 floats and contain the statics (in ms) as a function of surface location.

 For hdrs=3, statics are read from the output files of suresstat, with 
 the same options as hdrs=2 (but use no=max traces per shot and assume 
 that ns=max fldr number and nr=max receiver number).			
 For each shot number (trace header fldr) and each receiver number     
 (trace header tracf) the program will look up the appropriate static  
 correction.  The geometry need not be regular as each trace is treated
 independently.							

 Caveat:  The static shifts are computed with the assumption that the  
 desired datum is sea level (elevation=0). You may need to shift the	
 selev and gelev header values via  suchw.				
 Example: subtracting min(selev,gelev)=25094431			

 suchw < CR290.su key1=selev,gelev key2=selev,gelev key3=selev,gelev \\ 
            a=-25094431,-25094431 b=1,1 c=0,0 > CR290datum.su		

 Credits:
	CWP: Jamie Burns

	CWP: Modified by Mohammed Alfaraj, 11/10/1992, for reading
	     statics from headers and including sign (+-) option

      CWP: Modified by Timo Tjan, 29 June 1995, to include input of
           source and receiver statics from files. 

	modified by Thomas Pratt, USGS, Feb, 2000 to read statics from
 	     the output files of suresstat

 Logic changed by JWN to fix options hdrs=2,3 ???

 Trace header fields accessed:  ns, dt, delrt, gelev, selev,
	sdepth, gdel, sdel, swevel, sut, scalel, fldr, tracf
 Trace header fields modified:  sstat, gstat, tstat


\end{verbatim}
\pagebreak
\begin{verbatim}
 SUSTATICRRS - Elevation STATIC corrections, apply corrections from	
	      headers or from a source and receiver statics file,	
	      includes application of Residual Refraction Statics	

     sustaticrrs <stdin >stdout  [optional parameters]	 		

 Required parameters:							
	none								
 Optional Parameters:							
	v0=v1 or user-defined	or from header, weathering velocity	
	v1=user-defined		or from header, subweathering velocity	
	hdrs=0			=1 to read statics from headers		
 				=2 to read statics from files		
	sign=1			=-1 to subtract statics from traces(up shift)
 Options when hdrs=2:							
	sou_file=		input file for source statics (ms) 	
	rec_file=		input file for receiver statics (ms) 	
	ns=240 			number of sources 			
	nr=335 			number of receivers 			
	no=96 			number of offsets			

 Options when hdrs=3:                                                  
       blvl_file=              base of the near-surface model file (sampled
                                  at CMP locations)                    
       refr_file=              horizontal reference datum file (sampled at
                                  CMP locations)                       
       nsamp=                  number of midpoints on line             
       fx=                     first x location in velocity model      
       dx=                     midpoint interval                       
       V_r=                    replacement velocity                    
       mx=                     number of velocity model samples in     
                                  lateral direction                    
       mz=                     number of velocity model samples in     
                                  vertical direction                   
       dzv=                    velocity model depth interval           
       vfile=                  near-surface velocity model             

 Options when hdrs=4:                                                  
       nsamp=                  number of midpoints on line             
       fx=                     first x location in velocity model      ", 
       dx=                     midpoint interval                       ", 

 Options when hdrs=5:                                                  
       none                                                            

 Notes:								
 For hdrs=1, statics calculation is not performed, statics correction  
 is applied to the data by reading statics (in ms) from the header.	

 For hdrs=0, field statics are calculated, and				
 	input field sut is assumed measured in ms.			
 	output field sstat = 10^scalel*(sdel - selev + sdepth)/swevel	
 	output field gstat = sstat - sut/1000.				
 	output field tstat = sstat + gstat + 10^scalel*(selev - gelev)/wevel

 For hdrs=2, statics are surface consistently obtained from the 	
 statics files. The geometry should be regular.			
 The source- and receiver-statics files should be unformated C binary 	
 floats and contain the statics (in ms) as a function of surface location.

 For hdrs=3, residual refraction statics and average refraction statics
 are computed.  For hdrs=4, residual refraction statics are applied,   
 and for hdrs=5, average refraction statics are applied (Cox, 1999).   
 These three options are coupled in many data processing sequences:    
 before stack residual and average refraction statics are computed but 
 only residual refractions statics are applied, and after stack average
 refraction statics are applied.  Refraction statics are often split   
 like this to avoid biasing stacking velocities.  The files blvl_file  
 and refr_file are the base of the velocity model defined in vfile and 
 the final reference datum, as described by Cox (1999), respectively.  
 Residual refraction statics are stored in the header field gstat, and 
 the average statics are stored in the header field tstat.  V_r is the 
 replacement velocity as described by Cox (1999).  The velocity file,  
 vfile, is designed to work with a horizontal upper surface defined in 
 refr_file.  If the survey has irregular topography, the horizontal    
 upper surface should be above the highest topographic point on the    
 line, and the velocity between this horizontal surface and topography 
 should be some very large value, such as 999999999, so that the       
 traveltimes through that region are inconsequential.                  

 Credits:
	CWP: Jamie Burns

	CWP: Modified by Mohammed Alfaraj, 11/10/1992, for reading
	     statics from headers and including sign (+-) option

      CWP: Modified by Timo Tjan, 29 June 1995, to include input of
           source and receiver statics from files. 

      CWP: Modified by Chris Robinson, 11/2000, to include the splitting
           of refraction statics into residuals and averages

 Trace header fields accessed:  ns, dt, delrt, gelev, selev,
	sdepth, gdel, sdel, swevel, sut, scalel
 Trace header fields modified:  sstat, gstat, tstat

 References:

 Cox, M., 1999, Static corrections for seismic reflection surveys:
    Soc. Expl. Geophys.


\end{verbatim}
\pagebreak
\begin{verbatim}
 SUSTATIC - Elevation static corrections, apply corrections from	
	      headers or from a source and receiver statics file	

     sustatic <stdin >stdout  [optional parameters]	 		

 Required parameters:							
	none								
 Optional Parameters:							
	v0=v1 or user-defined	or from header, weathering velocity	
	v1=user-defined		or from header, subweathering velocity	
	hdrs=0			=1 to read statics from headers		
 				=2 to read statics from files		
				=3 to read from output files of suresstat
	sign=1			apply static correction (add tstat values)
				=-1 apply negative of tstat values	
 Options when hdrs=2 and hdrs=3:					
	sou_file=		input file for source statics (ms) 	
	rec_file=		input file for receiver statics (ms) 	
	ns=240 			number of souces 			
	nr=335 			number of receivers 			
	no=96 			number of offsets			

 Notes:								
 For hdrs=1, statics calculation is not performed, statics correction  
 is applied to the data by reading statics (in ms) from the header.	

 For hdrs=0, field statics are calculated, and				
 	input field sut is assumed measured in ms.			
 	output field sstat = 10^scalel*(sdel - selev + sdepth)/swevel	
 	output field gstat = sstat - sut/1000.				
 	output field tstat = sstat + gstat + 10^scalel*(selev - gelev)/wevel

 For hdrs=2, statics are surface consistently obtained from the 	
 statics files. The geometry should be regular.			
 The source- and receiver-statics files should be unformated C binary 	
 floats and contain the statics (in ms) as a function of surface location.

 For hdrs=3, statics are read from the output files of suresstat, with 
 the same options as hdrs=2 (but use no=max traces per shot and assume 
 that ns=max shot number and nr=max receiver number).			
 For each shot number (trace header fldr) and each receiver number     
 (trace header tracf) the program will look up the appropriate static  
 correction.  The geometry need not be regular as each trace is treated
 independently.							

 Caveat:  The static shifts are computed with the assumption that the  
 desired datum is sea level (elevation=0). You may need to shift the	
 selev and gelev header values via  suchw.				
 Example: subtracting min(selev,gelev)=25094431			

 suchw < CR290.su key1=selev,gelev key2=selev,gelev key3=selev,gelev \\ 
            a=-25094431,-25094431 b=1,1 c=0,0 > CR290datum.su		

 Credits:
	CWP: Jamie Burns

	CWP: Modified by Mohammed Alfaraj, 11/10/1992, for reading
	     statics from headers and including sign (+-) option

      CWP: Modified by Timo Tjan, 29 June 1995, to include input of
           source and receiver statics from files. 

	modified by Thomas Pratt, USGS, Feb, 2000 to read statics from
 	     the output files of suresstat

 Trace header fields accessed:  ns, dt, delrt, gelev, selev,
	sdepth, gdel, sdel, swevel, sut, scalel, fldr, tracf
 Trace header fields modified:  sstat, gstat, tstat


\end{verbatim}
\pagebreak
\begin{verbatim}
 SUILOG -- time axis inverse log-stretch of seismic traces	

 suilog nt= ntmin=  <stdin >stdout 				

 Required parameters:						
 	nt= 	nt output from sulog prog			
 	ntmin= 	ntmin output from sulog prog			
 	dt= 	dt output from sulog prog			
 Optional parameters:						
 	none							

 NOTE:  Parameters needed by suilog to reconstruct the 	
	original data may be input via a parameter file.	

 EXAMPLE PROCESSING SEQUENCE:					
 		sulog outpar=logpar <data1 >data2		
 		suilog par=logpar <data2 >data3			

 	where logpar is the parameter file			


 
 Credits:
	CWP: Shuki Ronen, Chris Liner

 Caveats:
 	amplitudes are not well preserved

 Trace header fields accessed: ns, dt
 Trace header fields modified: ns, dt

\end{verbatim}
\pagebreak
\begin{verbatim}
 SULOG -- time axis log-stretch of seismic traces		

 sulog [optional parameters] <stdin >stdout 			

 Required parameters:						
	none				 			

 Optional parameters:						
	ntmin= .1*nt		minimum time sample of interest	
	outpar=/dev/tty		output parameter file, contains:
				number of samples (nt=)		
				minimum time sample (ntmin=)	
				output number of samples (ntau=)
	m=3			length of stretched data	
				is set according to		
					ntau = nextpow(m*nt)	
	ntau= pow of 2		override for length of stretched
				data (useful for padding zeros	
				to avoid aliasing)		

 NOTES:							
	ntmin is required to avoid taking log of zero and to 	
	keep number of outsamples (ntau) from becoming enormous.
        Data above ntmin is zeroed out.			

	The output parameters will be needed by suilog to 	
	reconstruct the original data. 				

 EXAMPLE PROCESSING SEQUENCE:					
		sulog outpar=logpar <data1 >data2		
		suilog par=logpar <data2 >data3			


 Credits:
	CWP: Shuki, Chris

 Caveats:
 	Amplitudes are not well preserved.

 Trace header fields accessed: ns, dt
 Trace header fields modified: ns, dt

\end{verbatim}
\pagebreak
\begin{verbatim}
 SUNMO_a - NMO for an arbitrary velocity function of time and CDP with	     
		experimental Anisotropy options				     
  sunmo <stdin >stdout [optional parameters]				     

 Optional Parameters:							     
 tnmo=0,...		NMO times corresponding to velocities in vnmo	     
 vnmo=1500,..		NMO velocities corresponding to times in tnmo	     
 anis1=0		two anisotropy coefficients making up quartic term   
 anis2=0		in traveltime curve, corresponding to times in tnmo  
 cdp=			CDPs for which vnmo & tnmo are specified (see Notes) 
 smute=1.5		samples with NMO stretch exceeding smute are zeroed  
 lmute=25		length (in samples) of linear ramp for stretch mute  
 sscale=1		=1 to divide output samples by NMO stretch factor    
 invert=0		=1 to perform (approximate) inverse NMO		     
 upward=0		=1 to scan upward to find first sample to kill	

 Notes:								     
 For constant-velocity NMO, specify only one vnmo=constant and omit tnmo.   

 NMO interpolation error is less than 1% for frequencies less than 60% of   
 the Nyquist frequency.						     

 Exact inverse NMO is impossible, particularly for early times at large     
 offsets and for frequencies near Nyquist with large interpolation errors.  

 The "offset" header field must be set.				     
 Use suazimuth to set offset header field when sx,sy,gx,gy are all	     
 nonzero. 							   	     

 For NMO with a velocity function of time only, specify the arrays	     
	   vnmo=v1,v2,... tnmo=t1,t2,...				     
 where v1 is the velocity at time t1, v2 is the velocity at time t2, ...    
 The times specified in the tnmo array must be monotonically increasing.    
 Linear interpolation and constant extrapolation of the specified velocities
 is used to compute the velocities at times not specified.		     
 The same holds for the anisotropy coefficients as a function of time only. 

 For NMO with a velocity function of time and CDP, specify the array	     
	   cdp=cdp1,cdp2,...						     
 and, for each CDP specified, specify the vnmo and tnmo arrays as described 
 above. The first (vnmo,tnmo) pair corresponds to the first cdp, and so on. 
 Linear interpolation and constant extrapolation of 1/velocity^2 is used    
 to compute velocities at CDPs not specified.				     

 Anisotropy option:							     
 Caveat, this is an experimental option,				     

 The anisotropy coefficients anis1, anis2 permit non-hyperbolicity due	     
 to layering, mode conversion, or anisotropy. Default is isotropic NMO.     

 The same holds for the anisotropy coefficients as a function of time and   
 CDP.									     

 Moveout is defined by							     

   1		 anis1							     
  --- x^2 + ------------- x^4.						     
  v^2	     1 + anis2 x^2						     

 Note: In general, the user should set the cdp parameter.  The default is   
	to use tr.cdp from the first trace and assume only one cdp.	  
 Caveat:								     
 Nmo cannot handle negative moveout as in triplication caused by	     
 anisotropy. But negative moveout happens necessarily for negative anis1 at 
 sufficiently large offsets. Then the error-negative moveout- is printed.   
 Check anis1. An error (anis2 too small) is also printed if the	     
 denominator of the quartic term becomes negative. Check anis2. These errors
 are prompted even if they occur in traces which would not survive the	     
 NMO-stretch threshold. Chop off enough far-offset traces (e.g. with suwind)
 if anis1, anis2 are fine for near-offset traces.			     


 Credits:
	SEP: Shuki, Chuck Sword
	CWP: Shuki, Jack, Dave Hale, Bjoern Rommel
      Modified: 08/08/98 - Carlos E. Theodoro - option for lateral offset
      Modified: 07/11/02 - Sang-yong Suh -
	  added "upward" option to handle decreasing velocity function.
      CWP: Sept 2010: John Stockwell
	  replaced Carlos Theodoro's fix
	  and added the instruction in the selfdoc to use suazimuth to set 
	    offset so that it accounts for lateral offset
      note that by the segy standard "scalel" does not scale the offset
      field
 Technical Reference:
	The Common Depth Point Stack
	William A. Schneider
	Proc. IEEE, v. 72, n. 10, p. 1238-1254
	1984

 Trace header fields accessed: ns, dt, delrt, offset, cdp, scalel

\end{verbatim}
\pagebreak
\begin{verbatim}
 SUNMO - NMO for an arbitrary velocity function of time and CDP	     

  sunmo <stdin >stdout [optional parameters]				     

 Optional Parameters:							     
 tnmo=0,...		NMO times corresponding to velocities in vnmo	     
 vnmo=1500,...		NMO velocities corresponding to times in tnmo	     
 cdp=			CDPs for which vnmo & tnmo are specified (see Notes) 
 smute=1.5		samples with NMO stretch exceeding smute are zeroed  
 lmute=25		length (in samples) of linear ramp for stretch mute  
 sscale=1		=1 to divide output samples by NMO stretch factor    
 invert=0		=1 to perform (approximate) inverse NMO		     
 upward=0		=1 to scan upward to find first sample to kill	     
 voutfile=		if set, interplolated velocity function v[cdp][t] is 
			output to named file.			     	     
 Notes:								     
 For constant-velocity NMO, specify only one vnmo=constant and omit tnmo.   

 NMO interpolation error is less than 1% for frequencies less than 60% of   
 the Nyquist frequency.						     

 Exact inverse NMO is impossible, particularly for early times at large     
 offsets and for frequencies near Nyquist with large interpolation errors.  

 The "offset" header field must be set.				     
 Use suazimuth to set offset header field when sx,sy,gx,gy are all	     
 nonzero. 							   	     

 For NMO with a velocity function of time only, specify the arrays	     
	   vnmo=v1,v2,... tnmo=t1,t2,...				     
 where v1 is the velocity at time t1, v2 is the velocity at time t2, ...    
 The times specified in the tnmo array must be monotonically increasing.    
 Linear interpolation and constant extrapolation of the specified velocities
 is used to compute the velocities at times not specified.		     
 The same holds for the anisotropy coefficients as a function of time only. 

 For NMO with a velocity function of time and CDP, specify the array	     
	   cdp=cdp1,cdp2,...						     
 and, for each CDP specified, specify the vnmo and tnmo arrays as described 
 above. The first (vnmo,tnmo) pair corresponds to the first cdp, and so on. 
 Linear interpolation and constant extrapolation of 1/velocity^2 is used    
 to compute velocities at CDPs not specified.				     

 The format of the output interpolated velocity file is unformatted C floats
 with vout[cdp][t], with time as the fast dimension and may be used as an   
 input velocity file for further processing.				     

 Note that this version of sunmo does not attempt to deal with	anisotropy.  
 The version of sunmo with experimental anisotropy support is "sunmo_a


 Credits:
	SEP: Shuki Ronen, Chuck Sword
	CWP: Shuki Ronen, Jack, Dave Hale, Bjoern Rommel
      Modified: 08/08/98 - Carlos E. Theodoro - option for lateral offset
      Modified: 07/11/02 - Sang-yong Suh -
	  added "upward" option to handle decreasing velocity function.
      CWP: Sept 2010: John Stockwell
	  1. replaced Carlos Theodoro's fix 
	  2. added  the instruction in the selfdoc to use suazimuth to set 
	      offset so that it accounts for lateral offset. 
        3. removed  Bjoren Rommel's anisotropy stuff. sunmo_a is the 
           version with the anisotropy parameters left in.
        4. note that scalel does not scale the offset field in
           the segy standard.
 Technical Reference:
	The Common Depth Point Stack
	William A. Schneider
	Proc. IEEE, v. 72, n. 10, p. 1238-1254
	1984

 Trace header fields accessed: ns, dt, delrt, offset, cdp, scalel

\end{verbatim}
\pagebreak
\begin{verbatim}
 SUREDUCE - convert traces to display in reduced time		", 

 sureduce <stdin >stdout rv=					

 Required parameters:						
	dt=tr.dt	if not set in header, dt is mandatory	

 Optional parameters:						
	rv=8.0		reducing velocity in km/sec		",	

 Note: Useful for plotting refraction seismic data. 		
 To remove reduction, do:					
 suflip < reduceddata.su flip=3 | sureduce rv=RV > flip.su	
 suflip < flip.su flip=3 > unreduceddata.su			

 Trace header fields accessed: dt, ns, offset			
 Trace header fields modified: none				


 Author: UC Davis: Mike Begnaud  March 1995


 Trace header fields accessed: ns, dt, offset

\end{verbatim}
\pagebreak
\begin{verbatim}
 SURESAMP - Resample in time                                       

 suresamp <stdin >stdout  [optional parameters]                    

 Required parameters:                                              
     none                                                          

 Optional Parameters:                                              
    nt=tr.ns    number of time samples on output                   
    dt=         time sampling interval on output                   
                default is:                                        
                tr.dt/10^6     seismic data                        
                tr.d1          non-seismic data                    
    tmin=       time of first sample in output                     
                default is:                                        
                tr.delrt/10^3  seismic data                        
                tr.f1          non-seismic data                    
    rf=         resampling factor;                                 
                if defined, set nt=nt_in*rf and dt=dt_in/rf        
    verbose=0   =1 for advisory messages                           


 Example 1: (assume original data had dt=.004 nt=256)              
    sufilter <data f=40,50 amps=1.,0. |                            
    suresamp nt=128 dt=.008 | ...                                  
 Using the resampling factor rf, this example translates to:       
    sufilter <data f=40,50 amps=1.,0. | suresamp rf=0.5 | ...      

 Note the typical anti-alias filtering before sub-sampling!        

 Example 2: (assume original data had dt=.004 nt=256)              
    suresamp <data nt=512 dt=.002 | ...                            
 or use:                                                           
    suresamp <data rf=2 | ...                                      

 Example 3: (assume original data had d1=.1524 nt=8192)            
    sufilter <data f=0,1,3,3.28 amps=1,1,1,0 |                     
    suresamp <data nt=4096 dt=.3048 | ...                          

 Example 4: (assume original data had d1=.5 nt=4096)               
    suresamp <data nt=8192 dt=.25 | ...                            


 Credits:
    CWP: Dave (resamp algorithm), Jack (SU adaptation)
    CENPET: Werner M. Heigl - modified for well log support
    RISSC: Nils Maercklin 2006 - minor fixes, added rf option

 Algorithm:
    Resampling is done via 8-coefficient sinc-interpolation.
    See "$CWPROOT/src/cwp/lib/intsinc8.c" for technical details.

 Trace header fields accessed:  ns, dt, delrt, d1, f1, trid
 Trace header fields modified:  ns, dt, delrt (only when set tmin)
                                d1, f1 (only when set tmin)

\end{verbatim}
\pagebreak
\begin{verbatim}
 SUSHIFT - shifted/windowed traces in time				

 sushift <stdin >stdout [tmin= ] [tmax= ]				

 tmin=			min time to pass				
 tmax=			max time to pass				
 dt=                    sample rate in microseconds 			
 fill=0.0               value to place in padded samples 		

 (defaults for tmin and tmax are calculated from the first trace.	
 verbose=		1 echos parameters to stdout			

 Background :								
 tmin and tmax must be given in seconds				

 In the high resolution single channel seismic profiling the sample 	
 interval is short, the shot rate and the number of samples are high.	
 To reduce the file size the delrt time is changed during a profiling	
 trip. To process and display a seismic section a constant delrt is	
 needed. This program does this job.					

 The SEG-Y header variable delrt (delay in ms) is a short integer.	
 That's why in the example shown below delrt is rounded to 123 !	

   ... | sushift tmin=0.1234 tmax=0.2234 | ...				

 The dt= and fill= options are intended for manipulating velocity	
 volumes in trace format.  In particular models which were hung	
 from the water bottom when created & which then need to have the	
 water layer added.							



 Author:

 Toralf Foerster
 Institut fuer Ostseeforschung Warnemuende
 Sektion Marine Geologie
 Seestrasse 15
 D-18119 Rostock, Germany

 Trace header fields accessed: ns, delrt
 Trace header fields modified: ns, delrt

\end{verbatim}
\pagebreak
\begin{verbatim}
 SUTAUPNMO - NMO for an arbitrary velocity function of tau and CDP	

  sutaupnmo <stdin >stdout [optional parameters]			

 Optional Parameters:							
 tnmo=0,...		NMO times corresponding to velocities in vnmo	
 vnmo=1500,...		NMO velocities corresponding to times in tnmo	
 cdp=			CDPs for which vnmo & tnmo are specified (see Notes) 
 smute=1.5		samples with NMO stretch exceeding smute are zeroed  
 lmute=25		length (in samples) of linear ramp for stretch mute  
 sscale=1		=1 to divide output samples by NMO stretch factor    

 Notes:								

 For constant-velocity NMO, specify only one vnmo=constant and omit tnmo.

 For NMO with a velocity function of tau only, specify the arrays	
	   vnmo=v1,v2,... tnmo=t1,t2,...				
 where v1 is the velocity at tau t1, v2 is the velocity at tau t2, ...    
 The taus specified in the tnmo array must be monotonically increasing.    
 Linear interpolation and constant extrapolation of the specified velocities
 is used to compute the velocities at taus not specified.		

 For NMO with a velocity function of tau and CDP, specify the array	
	   cdp=cdp1,cdp2,...						
 and, for each CDP specified, specify the vnmo and tnmo arrays as described 
 above. The first (vnmo,tnmo) pair corresponds to the first cdp, and so on. 
 Linear interpolation and constant extrapolation of velocity^2 is used	 
 to compute velocities at CDPs not specified.				

 Moveout is defined by							

  tau^2 + tau^2.p^2.vel^2						

 Note: In general, the user should set the cdp parameter.  The default is   
	to use tr.cdp from the first trace and assume only one cdp.	 
 Caveat:								
 Taunmo should handle triplication					

 NMO interpolation error is less than 1% for frequencies less than 60% of   
 the Nyquist frequency.						

 Exact inverse NMO is not implemented, nor has anisotropy		
 Example implementation:						
   sutaup dx=25 option=2 pmin=0 pmax=0.0007025 < cmpgather.su |	
   supef minlag=0.2 maxlag=0.8 |					
   sutaupnmo tnmo=0.5,2,4 vnmo=1500,2000,3200 smute=1.5 |		
   sumute key=tracr mode=1 ntaper=20 xmute=1,30,40,50,85,15  		
				 tmute=7.8,7.8,4.5,3.5,2.0,0.35 |	
   sustack key=cdp | ... [...]						


 Credits:
	 Durham, Richard Hobbs modified from SUNMO credited below
	SEP: Shuki Ronen, Chuck Sword
	CWP: Shuki Ronen, Jack K. Cohen , Dave Hale

 Technical Reference:
	van der Baan papers in geophysics (2002 & 2004)

 Trace header fields accessed: ns, dt, delrt, offset, cdp, sy

\end{verbatim}
\pagebreak
\begin{verbatim}
 SUTSQ -- time axis time-squared stretch of seismic traces	

 sutsq [optional parameters] <stdin >stdout 			

 Required parameters:						
	none				 			

 Optional parameters:						
       tmin= .1*nt*dt  minimum time sample of interest		
                       (only needed for forward transform)	
       dt= .004       output sample rate			
                       (only needed for inverse transform)	
       flag= 1        1=forward transform: time to time squared
                     -1=inverse transform: time squared to time

 Note: The output of the forward transform always starts with	
 time squared equal to zero.  'tmin' is used to avoid aliasing	
 the early times.
	


 Caveats:
 	Amplitudes are not well preserved.

 Trace header fields accessed: ns, dt
 Trace header fields modified: ns, dt

\end{verbatim}
\pagebreak
\begin{verbatim}
 SUTTOZ - resample from time to depth					

 suttoz <stdin >stdout [optional parms]				

 Optional Parameters:							
 nz=1+(nt-1)*dt*vmax/(2.0*dz)   number of depth samples in output	
 dz=vmin*dt/2		depth sampling interval (defaults avoids aliasing)
 fz=v(ft)*ft/2		first depth sample				
 t=0.0,...		times corresponding to interval velocities in v
 v=1500.0,...		interval velocities corresponding to times in v
 vfile=		  binary (non-ascii) file containing velocities v(t)
 verbose=0		>0 to print depth sampling information		

 Notes:								
 Default value of nz set to avoid aliasing				
 The t and v arrays specify an interval velocity function of time.	

 Note that t and v are given  as arrays of floats separated by commas,  
 for example:								
 t=0.0,0.01,.2,... v=1500.0,1720.0,1833.5,... with the number of t values
 equaling the number of v values. The velocities are linearly interpolated
 to make a continuous, piecewise linear v(t) profile.			

 Linear interpolation and constant extrapolation is used to determine	
 interval velocities at times not specified.  Values specified in t	
 must increase monotonically.						

 Alternatively, interval velocities may be stored in a binary file	
 containing one velocity for every time sample.  If vfile is specified,
 then the t and v arrays are ignored.					

 see selfdoc of suztot  for depth to time conversion			

 Trace header fields accessed:  ns, dt, and delrt			
 Trace header fields modified:  trid, ns, d1, and f1			


 Credits:
	CWP: Dave Hale c. 1992


\end{verbatim}
\pagebreak
\begin{verbatim}
 SUZTOT - resample from depth to time					

 suztot <stdin >stdout [optional parms]				

 Optional Parameters:							
 nt=1+(nz-1)*2.0*dz/(vmax*dt)    number of time samples in output	
 dt=2*dz/vmin		time sampling interval (defaults avoids aliasing)
 ft=2*fz/v(fz)		first time sample				
 z=0.0,...		depths corresponding to interval velocities in v
 v=1500.0,...		interval velocities corresponding to depths in v
 vfile=		binary (non-ascii) file containing velocities v(z)
 verbose=0		>0 to print depth sampling information		

 Notes:								
 Default value of nt set to avoid aliasing				
 The z and v arrays specify an interval velocity function of depth.	

 Note that z and v are given  as arrays of floats separated by commas,  
 for example:								
 z=0.0,100,200,... v=1500.0,1720.0,1833.5,... with the number of z values
 equaling the number of v values. The velocities are linearly interpolated
 to produce a piecewise linear v(z) profile. This fact must be taken into
 account when attempting to use this program as the inverse of suttoz.	

 Linear interpolation and constant extrapolation is used to determine	
 interval velocities at times not specified.  Values specified in z	
 must increase monotonically.						

 Alternatively, interval velocities may be stored in a binary file	
 containing one velocity for every time sample.  If vfile is specified,
 then the z and v arrays are ignored.					

 see the selfdoc of   suttoz  for time to depth conversion		
 Trace header fields accessed:  ns, dt, and delrt			
 Trace header fields modified:  trid, ns, d1, and f1			


 Credits:
	CWP: John Stockwell, 2005, 
            based on suttoz.c written by Dave Hale c. 1992


\end{verbatim}
\pagebreak
\begin{verbatim}
 SUGET  - Connect SU program to file descriptor for input stream.	

    suget fd=$1 | next_su_module					

 This program is for interfacing " outside processing systems 
 with SU. Typically, an outside system would execute the su command file
 and a file descriptor would be passed by an outside system to		
 the su command file so that output data from the outside system	
 could be piped into the su programs executing inside the command file.

 Example:    suget fd=$1 | next_su_module				

      fd=-1        file_descriptor_for_input_stream			
      verbose=0    minimal listing					
                   =1  asks for message with each trace processed.	


 Author: John Anderson (visiting scholar from Mobil) July 1994

\end{verbatim}
\pagebreak
\begin{verbatim}
 SUPUT - Connect SU program to file descriptor for output stream.	

       su_module | suput fp=$1						

 This program is for interfacing " outside processing systems 
 with SU. Typically, the outside system would execute the SU command file.
 The outside system provides the file descriptor it would like to read	
 from to the command file to be an argument for suput.			

 Example: su_module | suput fp=$1					

       fd=-1       file_descriptor_for_output_stream_from_su		
       verbose=0   minimal listing					
                   =1  asks for message with each trace processed.	


 Author: John Anderson (visiting scholar from Mobil) July 1994


\end{verbatim}
\pagebreak
\begin{verbatim}

 SUADDEVENT - add a linear or hyperbolic moveout event to seismic data 

 suaddevent <stdin >stdout [optional parameters]		       

 Required parameters:						  
       none								

 Optional parameters:						  
     type=nmo    =lmo for linear event 				
     t0=1.0      zero-offset intercept time IN SECONDS			
     vel=3000.   moveout velocity in m/s				
     amp=1.      amplitude						
     dt=	 must provide if 0 in headers (seconds)		

 Typical usage: 
     sunull nt=500 dt=0.004 ntr=100 | sushw key=offset a=-1000 b=20 \\ 
     | suaddevent v=1000 t0=0.05 type=lmo | suaddevent v=1800 t0=0.8 \
     | sufilter f=8,12,75,90 | suxwigb clip=1 &	     		



 Credits:
      Gary Billings, Talisman Energy, May 1996, Apr 2000, June 2001

 Note:  code is inefficient in that to add a single "spike", with sinc
	interpolation, an entire trace is generated and added to 
	the input trace.  In fact, only a few points needed be created
	and added, but the current coding avoids the bookkeeping re
	which are the relevant points!

\end{verbatim}
\pagebreak
\begin{verbatim}
 SUDGWAVEFORM - make Gaussian derivative waveform in SU format		

  sudgwaveform >stdout  [optional parameters]				


 Optional parameters:							
 n=2    	order of derivative (n>=1)				
 fpeak=35	peak frequency						
 nfpeak=n*n	max. frequency = nfpeak * fpeak				
 nt=128	length of waveform					
 shift=0	additional time shift in s (used for plotting)		
 sign=1	use =-1 to change sign					
 verbose=0	=0 don't display diagnostic messages			
               =1 display diagnostic messages				
 Notes:								
 This code computes a waveform that is the n-th order derivative of a	
 Gaussian. The variance of the Gaussian is specified through its peak	
 frequency, i.e. the frequency at which the amplitude spectrum of the	
 Gaussian has a maximum. nfpeak is used to compute maximum frequency,	
 which in turn is used to compute the sampling interval. Increasing	
 nfpeak gives smoother plots. In order to have a (pseudo-) causal	
 pulse, the program computes a time shift equal to sqrt(n)/fpeak. An	
 additional shift can be applied with the parameter shift. A positive	
 value shifts the waveform to the right.				

 Examples:								
 2-loop Ricker: dgwaveform n=1	>ricker2.su				
 3-loop Ricker: dgwaveform n=2 >ricker3.su				
 Sonic transducer pulse: dgwaveform n=10 fpeak=300 >sonic.su		

 To display use suxgraph. For example:					
 dgwaveform n=10 fpeak=300 | suxgraph style=normal &			

 For other seismic waveforms, please use "suwaveform".		


 Credits:

	Werner M. Heigl, February 2007

 This copyright covers parts that are not part of the original
 CWP/SU: Seismic Un*x codes called by this program:

 Copyright (c) 2007 by the Society of Exploration Geophysicists.
 For more information, go to http://software.seg.org/2007/0004 .
 You must read and accept usage terms at:
 http://software.seg.org/disclaimer.txt before use.

 Revision history:
 Original SEG version by Werner M. Heigl, Apache E&P Technology,
 February 2007.

 Jan 2010 - subroutines deriv_n_gauss and hermite_n_polynomial moved
 to libcwp.a
/
\end{verbatim}
\pagebreak
\begin{verbatim}
 SUEA2DF - SU version of (an)elastic anisotropic 2D finite difference 	
		forward modeling, 4th order in space			

 suea2df > outfile c11file= c55file  [optional parameters]		

 Required Parameters:							
 c11file=c11_file	c11 voigt elasticity parameter filename		
 c55file=c55_file	c55 voigt elasticity parameter filename		

 Optional Parameters:							
 rhofile=rho_file	density filename				
			(if rhofile is not set, rho=1000 is assumed)	
 Anisotropy parameters:						
 aniso=0	 	 =1 - include anisotropy parameters		
 mode=0		=0 output particle velocity, =1 output stresses 
			(snapshots only)				

 ... the next 3 parameters become active only when aniso=1....		
 c13file=c13_file	c13 voigt elasticity parameter filename		
 c33file=c33_file	c33 voigt elasticity parameter filename		
 c15file=c15_file	c15 voigt elasticity parameter filename		
 c35file=c35_file	c35 voigt elasticity parameter filename		

 Attenuation parameters:						 
 qsw=0		 switch to include attenuation =1 - include		
 ... the next parameter becomes active only when qsw=1....	     	
 qfile=Q_file	  Q parameter filename	    				

 dt=0.001		time sampling interval (s)			
 ft=0.0 		first time (s)				 	
 lt=1.0 		last time (s)					

 nx=200		number of values in slow (x-direction)		
 dx=10.0	 	spatial sampling interval (m) x-coor		
 fx=-1000		first x coor (m)				

 nz=100		number of values in fast (z)-dimension		
 dz=dx			spatial sampling interval (m) z-coor		
 fz=0			firstz coor (m)  				

 Source parameters:							
 sx=0			source x position (m)				
 sz=500		source location (m)  				
 stype='p'		source type					
			  p: P-wave					
			  v: velocity					
			 pw: P plane-wave				
 sang=0		for stype='pw': plane wave angle		
 wtype='dg'		wave type					
 			 dg: Gaussian derivative 			
 			 ga: Gaussian		 			
 			 ri: Ricker					
 			 sp: spike, sp2: double spike   		
 ts=0.05		source duration (s)				
 favg=50		source average frequency			

 Attenuation parameters:						
 qsw=0		 	switch to include attenuation =1 - include	

 Boundary condition parameters:					
 bc=10,10,10,10 	Top,left,bottom,right boundary condition	
 			=0 none						
 			=1 symmetry 					
 			=2 free surface (top only)			
 			>2 absorbing (value indicates width of absorbing
			layer	 					
 bc_a=0.95;		bc initial taper value for absorbing boundary  
 bc_r=0.;		bc exponential factor for absorbing boundary  	
 			variables are scaled by bc_a*pow(i,-bc_r)	

 Optional output parameters:						
 sofile=		name of source file				
 snfile=		name of file containing for snapshots		
 snaptime=		times of snapshots i.e. snaptime=0.1,0.2,0.3	

 vsx=			x coordinate of vertical line of seismograms	
 hsz=			z coordinate of horizontal line of seismograms	
 vrslfile="vsp.su"	output file for vertical line of seismograms[nz][nt]
 hsfile="hs.su" 	output file for horizontal line of seismograms[nx][nt]
 tsw=0		 	switch to use shear stress only in non-fluid	
			media - may help reduce dispersion tsw=1. If	
			tsw=0 then standard calculation	  		
 verbose=0		=1 to print progress on screen			

 Notes:								
 1) The outfile contains information generated by the input parameters,
    such as memory allocation, stability, etc. If your input file does	
    not work, check this file first.					

 The model is specified as binary files of stiffness parameters and    
 densities. These may be created any way the user desires. The program 
 unif2 or makevel may be used to generate densities, and the program	
 unif2aniso may be used to generate the stiffnesses. You will need to	
 transpose these files (stiffnesses and densities), as the input	
 format for suea2df assumes that the fast dimesion is the horizontal or 
 the x-dimension. You may do this via					

  transp n1=nz < c11_file > transp_c11_file				

 If aniso=1 then the program will expect the additional stiffnesss files.

 If qsw=1 unif2anis can be used to generate the Q values on a grid	
 These value also need to be transposed, as with the stiffnesses.	

Output files (always generated)					
	hsfile								
	vrslfile							
	hsfile.chd - header for hsfile					
	vrslfile.chd - header for vrslfile				
	hsfile.mod - model file						

 Output files (if requested)						
	sofile - ascii source file					
	snfile  - su format snapshots file				

 Caveat:								
 A common error in using this program is to compute stiffnesses with	
 a specified density, but forget to specify this density as the rhofile.

 
 Credits: UU GEOPHY Chris Juhlin 15 May 1999
 Copyright (c) Uppsala University, 1998.
 All rights reserved.			
 Parts of program use Seismic Unix Package - CSM
 Changes - C. Juhlin

 1. Fixed upgrading of stresses. There was an error in the coding for
 the Tzz term, c15 was being used instead of c35. This only caused
 problems for dipping anisotropic layers

 2. Added some header information for hutput of snapshots.

 3. 2001-01-30: Added option to set absorbing bc constants bc_a and bc_r 

 4. 2001-02-23: Corrected bug in outputting model boundaries to standard
 output in 
 routine get_econst

 5. 2001-04-26: Added option for updating velocities to only use 
 shear stress if material is non-fluid, this appears to reduce dispersion at 
 near grazing angles for fluid-solid boundary. Set tsw=1 to invoke

 6. 2001-05-14: Changed loop in free-surface boundary condition for velocties
 Thanks goes to Mike Holzrichter for pointing out this problem and the wrong
 scaling factor in the updating.

 7. 2001-05-16: Changed set_layers function to avoid negative indexing.
 Thanks goes to Mike Holzrichter for pointing out this problem

 8. 2001-05-17: Modified make_seis to take into account VSP geometry
 correctly and not store unnecessary data.

 9. 2001-08-21: Fixed set_layers so mode fills properly in depth. Earlier
 versions were accessing incorrect array locations at last defined depth.

 10. 2003-04-21: Fixed boundary conditions.

 11. 2003-05-02: Extended the model area by half the grid spacing on the RHS.
 This makes the model area symmetric allowing a plane wave source to be
 introduced into the model (stype=pw). The w, txx and tzz grids contain now
 one more column than the u and txz grids.

 12. 2003-05-02: Added routines to allow plane waves to be introduced at a 
 specified angle (sang) into the model with functions add_pw_source_V and
 add_pw_source_S.

 13. 15 Oct 2005 -- tossed all the model building stuff. Read models
	from binary files made by  unif2aniso (CWP:John Stockwell)

 14. 25 Feb 2008 -- Fixed attenutation option (qsw=1) so that Q values are
	from binary files made by makevel or similar program

 15. 1 April 2010 -- Changed free surface velocity BC back to original.
	Someone had changed the scaling factor from 2.0 to 0.5 in fs4v_bc_top

 Algorithm based on Juhlin (1995, Geophys. Prosp.)
	and Levander (1988, Geophysics)
 Attenuation included as in Graves (1996, BSSA)



\end{verbatim}
\pagebreak
\begin{verbatim}
 SUFCTANISMOD - Flux-Corrected Transport correction applied to the 2D
	  elastic wave equation for finite difference modeling in 	
	  anisotropic media						

 sufctanismod > outfile [optional parameters]				
		outfile is the final wavefield snapshot x-component	
		x-component of wavefield snapshot is in snapshotx.data	
		y-component of wavefield snapshot is in snapshoty.data	
		z-component of wavefield snapshot is in snapshotz.data	

 Optional Output Files:						
 reflxfile=	reflection seismogram file name for x-component		
		no output produced if no name specified	 		
 reflyfile=	reflection seismogram file name for y-component		
		no output produced if no name specified	 		
 reflzfile=	reflection seismogram file name for z-component		
		no output produced if no name specified	 		
 vspxfile=	VSP seismogram file name for x-component		
		no output produced if no name specified	 		
 vspyfile=	VSP seismogram file name for y-component		
		no output produced if no name specified	 		
 vspzfile=	VSP seismogram file name for z-component		
		no output produced if no name specified	 		

 suhead=1      To get SU-header output seismograms (else suhead=0)	

 New parameter:							
     
 Optional Parameters:							
 mt=1          number of time steps per output snapshot  		
 dofct=1 	1 do the FCT correction					
		0 do not do the FCT correction 				
 FCT Related parameters:						
 eta0=0.03	diffusion coefficient					
		typical values ranging from 0.008 to 0.06		
		about 0.03 for the second-order method 			
		about 0.012 for the fourth-order method 		
 eta=0.04	anti-diffusion coefficient 				
		typical values ranging from 0.008 to 0.06		
		about 0.04 for the second-order method  		
		about 0.015 for the fourth-order method 		
 fctxbeg=0 	x coordinate to begin applying the FCT correction	
 fctzbeg=0 	z coordinate to begin applying the FCT correction	
 fctxend=nx 	x coordinate to stop applying the FCT correction	
 fctzend=nz 	z coordinate to stop applying the FCT correction	

 deta0dx=0.0	gradient of eta0 in x-direction  d(eta0)/dx		
 deta0dz=0.0	gradient of eta0 in z-direction  d(eta0)/dz		
 detadx=0.0	gradient of eta in x-direction 	 d(eta)/dx		
 detadz=0.0	gradient of eta in z-direction 	 d(eta)/dz		

 General Parameters:							
 order=2	2 second-order finite-difference 			
		4 fourth-order finite-difference 			

 nt=200        number of time steps 			 		
 dt=0.004	time step  						

 nx=100 	number of grid points in x-direction 			
 nz=100 	number of grid points in z-direction 			

 dx=0.02	spatial step in x-direction 				
 dz=0.02	spatial step in z-direction 				

 sx=nx/2	source x-coordinate (in gridpoints)			
 sz=nz/2	source z-coordinate (in gridpoints)			

 fpeak=20	peak frequency of the wavelet 				

 receiverdepth=sz  depth of horizontal receivers (in gridpoints)      
 vspnx=sx			x grid loc of vsp				

 verbose=0     silent operation							
				=1 for diagnostic messages, =2 for more		

 wavelet=1	1 AKB wavelet						
 		2 Ricker wavelet 					
		3 impulse 						
		4 unity 						

 isurf=2	1 absorbing surface condition 				
		2 free surface condition 				
		3 zero surface condition 				

 source=1	1 point source 						
 		2 sources are located on a given refelector 	        ", 
			(two horizontal and one dipping reflectors) 	
 		3 sources are located on a given dipping refelector     ", 

 sfile= 	the name of input source file, if no name specified then
		use default source location. (source=1 or 2) 		

 Density and Elastic Parameters:					
 dfile= 	the name of input density file,                         
               if no name specified then                             
		assume a linear density profile with ...		
 rho00=2.0	density at (0, 0) 					
 drhodx=0.0	density gradient in x-direction  d(rho)/dx		
 drhodz=0.0	density gradient in z-direction  d(rho)/dz		

 afile= 	name of input elastic param.  (c11) aa file, if no name 
		specified then, assume a linear profile with ...	
 aa00=2.0	elastic parameter at (0, 0) 				
 daadx=0.0	parameter gradient in x-direction  d(aa)/dx		
 daadz=0.0	parameter gradient in z-direction  d(aa)/dz		

 cfile= 	name of input elastic param. (c33)  cc file, if no name 
		specified then, assume a linear profile with ...	
 cc00=2.0	elastic parameter at (0, 0) 				
 dccdx=0.0	parameter gradient in x-direction  d(cc)/dx		
 dccdz=0.0	parameter gradient in z-direction  d(cc)/dz		

 ffile= 	name of input elastic param.  (c13) ff file, if no name 
		specified then, assume a linear profile with ...	
 ff00=2.0	elastic parameter at (0, 0) 				
 dffdx=0.0	parameter gradient in x-direction  d(ff)/dx		
 dffdz=0.0	parameter gradient in z-direction  d(ff)/dz		

 lfile= 	name of input elastic param.  (c44) ll file, if no name 
		specified then, assume a linear profile with ...	
 ll00=2.0	elastic parameter at (0, 0) 				
 dlldx=0.0	parameter gradient in x-direction  d(ll)/dx		
 dlldz=0.0	parameter gradient in z-direction  d(ll)/dz		

 nfile= 	name of input elastic param. (c66)  nn file, if no name 
		specified then, assume a linear profile with ...	
 nn00=2.0	elastic parameter at (0, 0) 				
 dnndx=0.0	parameter gradient in x-direction  d(nn)/dx		
 dnndz=0.0	parameter gradient in z-direction  d(nn)/dz		

 Optimizations:							
 The moving boundary option permits the user to restrict the computations
 of the wavefield to be confined to a specific range of spatial coordinates.
 The boundary of this restricted area moves with the wavefield		
 movebc=0	0 do not use moving boundary optimization		
		1 use moving boundaries					



 Author: Tong Fei,	Center for Wave Phenomena, 
		Colorado School of Mines, Dec 1993
 Some additional features by: Stig-Kyrre Foss, CWP
		Colorado School of Mines, Oct 2001
 New features (Oct 2001): 
 - setting receiver depth
 - outputfiles with SU-headers
 - additional commentary
 Modifications (Mar 2010) Chris Liner, U Houston
 - added snapshot mt param to parallel sufdmod2d functionality
 - added verbose and some basic info echos
 - error check that source loc is in grid
 - dropped mbx1 etc from selfdoc (they were internally computed)
 - moved default receiver depth to source depth
 - added vspnx to selfdoc and moved default vspnx to source x
 - changed sy in selfdoc to sz (typo)
 - fixed bug in vsp file(s) allocation: was [nt,nx] now is [nt,nz]


 
Notes:
	This program performs seismic modeling for elastic anisotropic 
	media with vertical axis of symmetry.  
	The finite-difference method with the FCT correction is used.

	Stability condition:	vmax*dt /(sqrt(2)*min(dx,dz)) < 1
	
	Two major stages are used in the algorithm:
	(1) conventional finite-difference wave extrapolation
	(2) followed by an FCT correction 

References:
	The detailed algorithm description is given in the article
	"Elimination of dispersion in finite-difference modeling 
	and migration"	in CWP-137, project review, page 155-174.

	Original reference to the FCT method:
	Boris, J., and Book, D., 1973, Flux-corrected transport. I.
	SHASTA, a fluid transport algorithm that works: 
	Journal of Computational Physics, vol. 11, p. 38-69.

/
\end{verbatim}
\pagebreak
\begin{verbatim}
 SUFDMOD1 - Finite difference modelling (1-D 1rst order) for the	
 acoustic wave equation						"

 sufdmod1 <vfile >sfile nz= tmax= sz= [optional parameters]		

 Required parameters :							
 <vfile or vfile=	binary file containing velocities[nz]		
 >sfile or sfile=	SU file containing seimogram[nt]		
 nz=		 number of z samples				   	
 tmax=		maximum propagation time				
 sz=		 z coordinate of source					

 Optional parameters :							
 dz=1	   z sampling interval						
 fz=0.0	 first depth sample					
 rz=1	   coordinate of receiver					
 sz=1	   coordinate of source						
 dfile=	 binary input file containing density[nz]		
 wfile=	 output file for wave field (snapshots in a SU trace panel)
 abs=0,1	absorbing conditions on top and bottom			
 styp=0	 source type (0: gauss, 1: ricker 1, 2: ricker 2)	
 freq=15.0	approximate source center frequency (Hz)		
 nt=1+tmax/dt   number od time samples (dt determined for numerical	
 stability)								
 zt=1	   trace undersampling factor for trace and snapshots	 	
 zd=1	   depth undersampling factor for snapshots		   	
 press=1	to record the pressure field; 0 records the particle	
		velocity						
 verbose=0	=1 for diagnostic messages				

 Notes :								
  This program uses a first order explicit velocity/pressure  finite	
  difference equation.							
  The source function is applied on the pressure component.		
  If no density file is given, constant density is assumed	 	
  Wavefield  can be easily viewed with suximage, user must provide f2=0
  to the ximage program in order to  get correct time labelling	
  Seismic trace is shifted in order to get a zero phase source		
  Source begins and stop when it's amplitude is 10^-4 its maximum	
  Time and depth undersampling only modify the output trace and snapshots.
  These parameters are useful for keeping snapshot file small and	
  the number of samples under SU_NFLTS.				

NULL  };

float source (float t, int styp, float dt, float dz, float t0, float alpha);

int main (int argc, char **argv)
{
	float *rv;	/* array of rock velocity from cfile
	float *rd;	/* array of rock density from dfile on p knots
	float *rd1_5;	/* array of rock density from dfile on v knots
	float *p;	/* pressure
	float *v;	/* particle velocity
	float tmax, dt, t0;	/* maximum time , time step,  		*/
				/* time delay for near causal source	*/
	float vmax;		/* maximum rock velocity		*/
	int verbose;		/* is verbose?				*/
	int nz, nt;		/* number of z samples, time samples	*/
	float fz, dz;		/* first sample depth spatial depth	*/
	float sz;		/* source coordinate			*/
	int abs[2];		/* array of absorbing conditions	*/
	int isz;		/* source location index		*/
	float rz;		/* receiver depth
	int irz;		/* zcoordinate (in samples) of the source
	int iz, it, itsis;	/* counter
	int ies;		/* end of source index
	int press;		/* to choose  between pressure or particle
				/* velocity
	float t;		/* time
	int td=1, zd=1;		/* time and depth decimation
	segy snapsh, sismo;	/* recording of the seismic field,
				/*  seismogram
	char *dfile="";		/* density file name
	char *wfile="";		/* seismogram file name
	char *velfile="";	  /* velocity file name
	char *sfile="";		/* velocity file name
	float freq=0.0;		/* source center freq
	float alpha=0.0;	/* source exp		*/
	float epst0=0.0;	/* source first amp ratio
	int styp;		/* source type

	FILE *seisfp=stdout;	/* pointer to seismic trace output file
	FILE *wavefp=NULL;	/* pointer to wave field output file
	FILE *velocityfp=stdin;	/* pointer to input velocity file
	FILE *densityfp=NULL;	/* pointer to input density file


	/* hook up getpar to handle the parameters
	initargs (argc, argv);
	requestdoc(0);

	/* verbose
	if (!getparint ("verbose",&verbose)) verbose=0;

	/* get required parameters
	if (!getparint ("nz",&nz)) err("must specify nz ! ");
	if (verbose) warn("nz= %d", nz);

	if (!getparfloat ("tmax",&tmax)) err("must specify tmax ! ");
	if (verbose) warn("tmax= %f", tmax);

	if (!getparfloat("sz", &sz)) err ("must specify sz ! ");
	if (verbose) warn("sz= %f", sz);
	
	

	/* get optional parameters
	if (!getparint ("nt", &nt)) nt=0; 
	if (verbose) warn("nt= %d", nt);
	if (!getparint ("styp", &styp)) styp=0;
	if (verbose) warn("styp= %d ", styp);
	if (!getparfloat ("dz", &dz)) dz=1;
	if (!getparfloat ("fz", &fz)) fz=0.0;

	/* source coordinates to samples
	isz=NINT((sz-fz)/dz);
	if (verbose) warn( "source on knot number %d ", isz);


	if (!getparfloat("rz", &rz)) rz=0.0;
	irz = NINT ((rz-fz)/dz);
	if (verbose) warn("receiver depth : %f on knot # %d ", rz, irz);

	if (!getparfloat("freq", &freq)) freq=15.0;
	if (verbose) warn("frequency : %f  Hz",freq);

	getparstring ("velfile", &velfile);
	if (verbose) {
		if (*velfile != '\0' ) warn("Velocity file : %s ",velfile);
		else warn("Velocity file supplied via stdin");
	}

	getparstring ("sfile", &sfile);
	if (verbose) {
		if (*sfile != '\0' ) warn("Output trace file : %s ",sfile);
		else warn("Output trace via stdout");
	}

	getparstring ("dfile", &dfile);
	if (verbose) {
		if (*dfile != '\0' ) warn("Density file : %s ",dfile);
		else warn("No density file supplied ");
	}

	getparstring ("wfile", &wfile);
	if (verbose) {
		if (*wfile != '\0' ) warn("Wave file : %s ",wfile);
		else warn("No wave file requested ");
	}

	if ( NINT((float) nz/((float) zd)) + 1  >SU_NFLTS) {
		warn ("Too many depth points : impossible to output wave field. Increase zd ?");
		*wfile='\0';
	}
	

	/* get absorbing conditions
	if (!getparint("abs",abs)) {  abs[0]=0; abs[1]=1;  }
	if (verbose) {
		if (abs[0]==1) warn("absorbing condition on top ");
		if (abs[1]==1) warn("absorbing condition on bottom ");
	}
	/* get decimation coefficients
	if (!getparint("td",&td)) td=1 ;
	if (verbose) warn("time decimation ccoefficent: %d ",td);
	if (!getparint("zd",&zd)) zd=1 ;
	if (verbose) warn("depth decimation ccoefficent: %d ",zd);

	/* choose pressure or particle velocity
	if (!getparint("press", &press)) press=1 ;
	if ((press != 0) && (press != 1)) err ("press must equal 0 or 1");
	if (verbose) {
		if (press==1) warn( "program will output pressure values");
		else if (press==0) warn( "program will output particle velocity values");
	}
		

	/* allocate space
	p=alloc1float(nz);
	v=alloc1float(nz);
	rv=alloc1float(nz);
	rd=alloc1float(nz);
	rd1_5=alloc1float(nz);

	/* read velocity file
	if (*velfile != '\0' ) {
		if ((velocityfp=fopen(velfile,"r"))=='\0') err("cannot open velfile=%s ",velfile);
	}
	if (efread (rv, sizeof(float), nz, velocityfp)!=nz) 
	   err("cannot read %d velocity values ", nz);

	/* read density file  and linearly inderpolate on corrrect location
	if (*dfile != '\0') {
		if ((densityfp=fopen(dfile,"r"))=='\0') err("cannot open dfile=%s ",dfile);
		if (fread(rd,sizeof(float), nz, densityfp)!=nz) err("error reading dfile %s",dfile);
		fclose(densityfp);
	}
	else for (iz=0; iz<nz; iz++) rd[iz]=2500;
	for (iz=0; iz<nz-1; iz++) rd1_5[iz]=(rd[iz]+rd[iz+1])/2;
	rd1_5[nz-1]=rd[nz-1];

	/* time step computation
	vmax=0;
	for (iz=0; iz<nz; iz++) if (rv[iz]>vmax) vmax=rv[iz];if (verbose) warn( "vmax= %f ", vmax);
	dt=dz/1.414/vmax/2; if (verbose) warn( "time step dt= %f ", dt);

	/* maximum number of iterations
	if (nt==0) nt=1+tmax/dt;
	if (verbose) warn( "number of time steps nt= %d ", nt);
	if (NINT( (float) nt/((float)td))+1>SU_NFLTS) err("too many time steps. Increase td ?");

	/* source parameter computation
	   alpha=2*9.8696*freq*freq;

	/* time shift to get a t0 centered source

	if ((styp==0) || (styp == 2)) epst0=fabs(source (0, styp, dt, dz, 0, alpha) / 1e4);
	else if (styp==1) epst0=fabs(source (1/sqrt(2*alpha), styp, dt, dz, 0, alpha)) / 1e4;
	if (verbose) warn( "epst0 = %f ", epst0);

	t=tmax+dt;
	do t=t-dt; while (fabs(source(t, styp, dt, dz, 0, alpha))<epst0);
	t0=t;
	ies=2*t/dt;

	if (verbose) warn("time shift t0 = %f s", t0);

 array initialization
	for (iz=0; iz<nz; iz++) {  v[iz]=0; p[iz]=0;  }

	if (*wfile != '\0') {
		wavefp=fopen (wfile,"w");
		snapsh.d1=dz*zd; snapsh.f1=fz ; snapsh.ns=nz/zd+1; snapsh.d2=dt*td; snapsh.f2=0; 
		/* snapsh.f2=0 is useless since 0 is the "no value" code for SU headers
	}
	/* propagation computation
	itsis=0;
	for (it=0; it<=nt; it++) {
		t=it*dt;
		if (abs[0]==1) p[0]=(p[0]*(1-rv[0]*dt/dz)+2*rd[0]*rv[0]*rv[0]*dt/dz*v[0])/(1+rv[0]*dt/dz);
		else p[0]=0;
		for (iz=1; iz<nz; iz++) p[iz]=p[iz]+rd[iz]*rv[iz]*rv[iz]*dt/dz*(v[iz]-v[iz-1]);
		if (abs[1]!=1) p[nz-1]=0;
		if (it<ies) {
		p[isz]=p[isz]+source(t, styp, dt, dz, t0, alpha);
		}

		for (iz=0; iz<nz-1; iz++) v[iz]=v[iz]+dt/rd1_5[iz]/dz*(p[iz+1]-p[iz]);
		
		if (abs[1] != 1) v[nz-1]=0;
		else
		v[nz-1]=((rd1_5[nz-1]*dz-dt*rd[nz-1]*rv[nz-1])*v[nz-1]-2*dt*p[nz-1])/(rd1_5[nz-1]*dz+dt*rd[nz-1]*rv[nz-1]);

	  if (it % td == 0) {
		   if (press==1) 
			sismo.data[itsis]=p[irz];
		   else
			sismo.data[itsis]=v[irz];
		   itsis++;
		}

		if ((*wfile!='\0') && (it % td == 0)) {
		if (press==1) 
			for (iz=0; iz<nz/zd; ++iz) snapsh.data[iz]=p[iz*zd];
		else
			for (iz=0; iz<nz/zd; ++iz) snapsh.data[iz]=v[iz*zd];

	   	fputtr(wavefp, &snapsh);
		}

	}

	if (*wfile!='\0') fclose (wavefp);

	sismo.dt=td*dt*1e6;
	sismo.ns=itsis;
	sismo.delrt=-t0*1000;
	sismo.trid=TREAL;
	sismo.tracl=1;

	if (*sfile != '\0') seisfp=efopen(sfile,"w");
	fputtr (seisfp, &sismo);
	

return(CWP_Exit());
}

float source (float t, int styp, float dt, float dz, float t0, float alpha)
{
	float x=0.0, sou=0.0;
	x=-alpha*(t-t0)*(t-t0);
	if (x>-40) {
	 if (styp==0) sou=exp(x);
	 	else if (styp==1) sou=-2*alpha*(t-t0)*exp(x);
	 	else if (styp==2) sou=2*alpha*(1+2*x)*exp(x);
		}
	else sou=0;
	return sou/dz*dt*1e8;
}
\end{verbatim}
\pagebreak
\begin{verbatim}
 SUFDMOD2_PML - Finite-Difference MODeling (2nd order) for acoustic wave
    equation with PML absorbing boundary conditions.			
 Caveat: experimental PML absorbing boundary condition version,	
may be buggy!								

 sufdmod2_pml <vfile >wfile nx= nz= tmax= xs= zs= [optional parameters]

 Required Parameters:							
 <vfile		file containing velocity[nx][nz]		
 >wfile		file containing waves[nx][nz] for time steps	
 nx=			number of x samples (2nd dimension)		
 nz=			number of z samples (1st dimension)		
 xs=			x coordinates of source				
 zs=			z coordinates of source				
 tmax=			maximum time					

 Optional Parameters:							
 nt=1+tmax/dt		number of time samples (dt determined for stability)
 mt=1			number of time steps (dt) per output time step	

 dx=1.0		x sampling interval				
 fx=0.0		first x sample					
 dz=1.0		z sampling interval				
 fz=0.0		first z sample					

 fmax = vmin/(10.0*h)	maximum frequency in source wavelet		
 fpeak=0.5*fmax	peak frequency in ricker wavelet		

 dfile=		input file containing density[nx][nz]		
 vsx=			x coordinate of vertical line of seismograms	
 hsz=			z coordinate of horizontal line of seismograms	
 vsfile=		output file for vertical line of seismograms[nz][nt]
 hsfile=		output file for horizontal line of seismograms[nx][nt]
 ssfile=		output file for source point seismograms[nt]	
 verbose=0		=1 for diagnostic messages, =2 for more		

 abs=1,1,1,1		Absorbing boundary conditions on top,left,bottom,right
 			sides of the model. 				
 		=0,1,1,1 for free surface condition on the top		

 ...PML parameters....                                                 
 pml_max=1000.0        PML absorption parameter                        
 pml_thick=0           half-thickness of pml layer (0 = do not use PML)

 Notes:								
 This program uses the traditional explicit second order differencing	
 method. 								

 Two different absorbing boundary condition schemes are available. The 
 first is a traditional absorbing boundary condition scheme created by 
 Hale, 1990. The second is based on the perfectly matched layer (PML)	
 method of Berenger, 1995.						



 Authors:  CWP:Dave Hale
           CWP:modified for SU by John Stockwell, 1993.
           CWP:added frequency specification of wavelet: Craig Artley, 1993
           TAMU:added PML absorbing boundary condition: 
               Michael Holzrichter, 1998
           CWP/WesternGeco:corrected PML code to handle density variations:
               Greg Wimpey, 2006

 References: (Hale's absobing boundary conditions)
 Clayton, R. W., and Engquist, B., 1977, Absorbing boundary conditions
 for acoustic and elastic wave equations, Bull. Seism. Soc. Am., 6,
	1529-1540. 

 Clayton, R. W., and Engquist, B., 1980, Absorbing boundary conditions
 for wave equation migration, Geophysics, 45, 895-904.

 Hale, D.,  1990, Adaptive absorbing boundaries for finite-difference
 modeling of the wave equation migration, unpublished report from the
 Center for Wave Phenomena, Colorado School of Mines.

 Richtmyer, R. D., and Morton, K. W., 1967, Difference methods for
 initial-value problems, John Wiley & Sons, Inc, New York.

 Thomee, V., 1962, A stable difference scheme for the mixed boundary problem
 for a hyperbolic, first-order system in two dimensions, J. Soc. Indust.
 Appl. Math., 10, 229-245.

 Toldi, J. L., and Hale, D., 1982, Data-dependent absorbing side boundaries,
 Stanford Exploration Project Report SEP-30, 111-121.

 References: (PML boundary conditions)
 Jean-Pierre Berenger, ``A Perfectly Matched Layer for the Absorption of
 Electromagnetic Waves,''  Journal of Computational Physics, vol. 114,
 pp. 185-200.

 Hastings, Schneider, and Broschat, ``Application of the perfectly
 matched layer (PML) absorbing boundary condition to elastic wave
 propogation,''  Journal of the Accoustical Society of America,
 November, 1996.

 Allen Taflove, ``Electromagnetic Modeling:  Finite Difference Time
 Domain Methods'', Baltimore, Maryland: Johns Hopkins University Press,
 1995, chap. 7, pp. 181-195.


 Trace header fields set: ns, delrt, tracl, tracr, offset, d1, d2,
                          sdepth, trid

\end{verbatim}
\pagebreak
\begin{verbatim}
 SUFDMOD2 - Finite-Difference MODeling (2nd order) for acoustic wave equation

 sufdmod2 <vfile >wfile nx= nz= tmax= xs= zs= [optional parameters]	

 Required Parameters:							
 <vfile		file containing velocity[nx][nz]		
 >wfile		file containing waves[nx][nz] for time steps	
 nx=			number of x samples (2nd dimension)		
 nz=			number of z samples (1st dimension)		
 xs=			x coordinates of source, or, alternatively, the name
			of a file that contains the x- and z-coordinates,
			with the number of pairs as the first record and
			the actual pairs of (x,z) locations following.  
 zs=			z coordinates of source				
 tmax=			maximum time					

 Optional Parameters:							
 sstrength=1.0		strength of source				
 pw=0			use point or extended source geometry parameters
 			=1  use horizontal plane wave source 		
 pwt=20		amp taper on ends of line src (in grid points)  
 mono=0		use ricker wavelet as source function 		
 			=1  use single frequency src (freq=2*fpeak)	
 nt=1+tmax/dt		number of time samples (dt determined for stability)
 mt=1			number of time steps (dt) per output time step	

 dx=1.0		x sampling interval				
 fx=0.0		first x sample					
 dz=1.0		z sampling interval				
 fz=0.0		first z sample					

 fmax = vmin/(10.0*h)	maximum frequency in source 			
 fpeak=0.5*fmax	peak frequency in ricker wavelet		

 dfile=		input file containing density[nx][nz]		
 vsx=			x coordinate of vertical line of seismograms	
 hsz=			z coordinate of horizontal line of seismograms	
 vsfile=		output file for vertical line of seismograms[nz][nt]
 hsfile=		output file for horizontal line of seismograms[nx][nt]
 ssfile=		output file for source point seismograms[nt]	
 verbose=0		=1 for diagnostic messages, =2 for more		
 abs=1,1,1,1		absorbing boundary conditions on top,left,bottom,right
			sides of the model. 				
 			=0,1,1,1 for free surface condition on the top	

 Notes:								

 This program uses the traditional explicit second order differencing	
 method. 								



 Authors:  CWP:Dave Hale
           CWP:modified for SU by John Stockwell, 1993.
           U Houston: added plane wave and monochromatic wave 
                        source options.  Chris Liner, 2010


 Trace header fields set: sx, gx, ns, delrt, tracl, tracr, offset, d1, d2,
                          sdepth, trid

 Modifications: Tony Kocurko (TK:)
                Memorial University in Newfoundland and Labrador
                - Allow user to supply the name of a file containing
                  shot point locations, rather than supplying them
                  as values to the xs= and zs= command line arguments.
                - Correct the calculation of izs[is].

 Technical reference:
 Kelly, K. R., R. W. Ward, S. Treitel, and R. M. Alford (1976),
 Synthetic Seismograms: A finite-difference approach, 
 Geophysics, Vol. 41. No. I (February, 1976), p. 2-27.


\end{verbatim}
\pagebreak
\begin{verbatim}
 SUGOUPILLAUDPO - calculate Primaries-Only impulse response of a lossless
	      GOUPILLAUD medium for plane waves at normal incidence	

 sugoupillaudpo < stdin > stdout [optional parameters]		      

 Required parameters:							     
	none								

 Optional parameters:						       
	l=1	   source layer number; 1 <= l <= tr.ns		  
		      Source is located at the top of layer l.		     
	k=1	   receiver layer number; 1 <= k			 
		      Receiver is located at the top of layer k.	    
	tmax	  number of output time-samples;			
		      default: long enough to capture all primaries	 
	pV=1	  flag for vector field seismogram		      
		      (displacement, velocity, acceleration);	       
		      =-1 for pressure seismogram.			  
	verbose=0     silent operation, =1 list warnings		    

 Input: Reflection coefficient series:				      

			       impedance[i]-impedance[i+1]		   
		       r[i] = -----------------------------		  
			       impedance[i]+impedance[i+1]		   

	r[0]= surface refl. coef. (as seen from above)		      
	r[n]= refl. coef. of the deepest interface			  

 Input file is to be in SU format, i.e., binary floats with a SU header.    

 Remarks:								   
 1. For vector fields, a buried source produces a spike of amplitude 1      
 propagating downwards and a spike of amplitude -1 propagating upwards.     
 A buried pressure source produces spikes of amplitude 1 both in the up-    
 and downward directions.						   
    A surface source induces only a downgoing spike of amplitude 1 at the   
 top of the first layer (both for vector and pressure fields).	      
 2. The sampling interval dt in the header of the input reflectivity file   
 is interpreted as a two-way traveltime thicknes of the layers. The sampling
 interval of the output seismogram is the same as that of the input file.   

 
 Credits:
	CWP: Albena Mateeva, April 2001.



\end{verbatim}
\pagebreak
\begin{verbatim}
 SUGOUPILLAUD - calculate 1D impulse response of	 		
     non-absorbing Goupillaud medium					

 sugoupillaud < stdin > stdout [optional parameters]			

 Required parameters:							
	none								

 Optional parameters:							
l=1source layer number; 1 <= l <= tr.ns				
			Source is located at the top of layer l.	
	k=1		receiver layer number; 1 <= k			
Receiver is located at the top of layer k.				
tmax  number of output time-samples; default:				
tmax=NINT((2*tr.ns-(l-1)-(k-1))/2)  if k < tr.ns			
			tmax=k				if k >=tr.ns	
pV=1  flag for vector field seismogram					
	(displacement, velocity, acceleration);				
=-1 for pressure seismogram.						
verbose=0  silent operation, =1 list warnings				

 Input: Reflection coefficient series:					

	 impedance[i]-impedance[i+1]					
 r[i] = ----------------------------- 					
	 impedance[i]+impedance[i+1]					

	r[0]= surface refl. coef. (as seen from above)			
r[n]= refl. coef. of the deepest interface				

 Input file is to be in SU format, i.e., binary floats with a SU header.

 Remarks:								
 1. For vector fields, a buried source produces a spike of amplitude 1	
 propagating downwards and a spike of amplitude -1 propagating upwards.
 A buried pressure source produces spikes of amplitude 1 both in the up
 and downward directions.						

 A surface source induces only a downgoing spike of amplitude 1 at the	
 top of the first layer (both for vector and pressure fields).		
 2. The sampling interval dt in the header of the input reflectivity file
 is interpreted as a two-way traveltime thicknes of the layers. The sampling
 interval of the output seismogram is the same as that of the input file.

 
 Credits:
	CWP: Albena Mateeva, May 2000, a summer project at Western Geophysical


 ANOTATION used in the code comments [arises from the use of z-transforms]:
		Z-sampled: sampling interval equal to the TWO-way 
			traveltime of the layers; 
		z-sampled: sampling interval equal to the ONE-way
			traveltime of the layers;

 REFERENCES:

	1. Ganley, D. C., 1981, A method for calculating synthetic seismograms 
	which include the effects of absorption and dispersion. 
	Geophysics, Vol.46, No. 8, p. 1100-1107.
 
	The burial of the source is based on the Appendix of that article.

	2. Robinson, E. A., Multichannel Time Series Analysis with Digital 
	Computer Programs: 1983 Goose Pond Press, 2nd edition.

	The recursive polynomials Q, P used in this code are described
	in Chapter 3 of the book: Wave Propagation in Layered Media.

	My polynomial multiplication and division functions "prod" and
	"pratio" are based on Robinson's Fortran subroutines in Chapter 1.

	4. Clearbout, J. F., Fundamentals of Geophysical Data Processing with
	Applications to Petroleum Prospecting: 1985 Blackwell Scientific 
	Publications.

	Chapter 8, Section 3: Introduces recursive polynomials F, G in a 
	more intuitive way than Robinson.
	
	The connection between the Robinson's P_k, Q_k and Clearbout's 
	F_k, G_k is:
				P_k(Z) = F_k(Z)
				Q_k(Z) = - Z^(k) G_k(1/Z)


\end{verbatim}
\pagebreak
\begin{verbatim}
 SUIMP2D - generate shot records for a line scatterer	
           embedded in three dimensions using the Born	
	    integral equation				",							

 suimp2d [optional parameters] >stdout			

 Optional parameters					
	nshot=1		number of shots			
	nrec=1		number of receivers		
	c=5000		speed				
	dt=.004		sampling rate			
	nt=256		number of samples		
	x0=1000		point scatterer location	
	z0=1000		point scatterer location	
	sxmin=0		first shot location		
	szmin=0		first shot location		
	gxmin=0		first receiver location		
	gzmin=0		first receiver location		
	dsx=100		x-step in shot location		
	dsz=0	 	z-step in shot location		
	dgx=100		x-step in receiver location	
	dgz=0		z-step in receiver location	

 Example:						
	suimp2d nrec=32 | sufilter | supswigp | ...	


 Credits:
	CWP: Norm Bleistein, Jack K. Cohen


 Theory: Use the 3D Born integral equation (e.g., Geophysics,
 v51, n8, p1554(7)). Use 2-D delta function for alpha and do
 remaining y-integral by stationary phase.

 Note: Setting a 2D offset in a single offset field beats the
       hell out of us.  We did _something_.

 Trace header fields set: ns, dt, tracl, tracr, fldr, sx, selev,
                          gx, gelev, offset

\end{verbatim}
\pagebreak
\begin{verbatim}
SUIMP3D - generate inplane shot records for a point 	
          scatterer embedded in three dimensions using	
          the Born integral equation			",							

suimp3d [optional parameters] >stdout 			

Optional parameters					
	nshot=1		number of shots			
	nrec=1		number of receivers		
	c=5000		speed				
	dt=.004		sampling rate			
	nt=256		number of samples		
	x0=1000		point scatterer location	
	y0=0		point scatterer location	
	z0=1000		point scatterer location	
   dir=0		do not include direct arrival	
	            =1 include direct arrival	
	sxmin=0		first shot location		
	symin=0		first shot location		
	szmin=0		first shot location		
	gxmin=0		first receiver location		
	gymin=0		first receiver location		
	gzmin=0		first receiver location		
	dsx=100		x-step in shot location		
	dsy=0	 	y-step in shot location		
	dsz=0	 	z-step in shot location		
	dgx=100		x-step in receiver location	
	dgy=0		y-step in receiver location	
	dgz=0		z-step in receiver location	

 Example:                                              
       suimp3d nrec=32 | sufilter | supswigp | ...     


 Credits:
	CWP: Norm Bleistein, Jack K. Cohen
  UHouston: Chris Liner 2010 (added direct arrival option)


 
 Theory: Use the 3D Born integral equation (e.g., Geophysics,
 v51, n8, p1554(7)). Use 3-D delta function for alpha.

 Note: Setting a 3D offset in a single offset field beats the
       hell out of us.  We did _something_.

 Trace header fields set: ns, dt, tracl, tracr, fldr, tracf,
                          sx, sy, selev, gx, gy, gelev, offset

\end{verbatim}
\pagebreak
\begin{verbatim}
 SUIMPEDANCE - Convert reflection coefficients to impedances.  

 suimpedance <stdin >stdout [optional parameters]		

 Optional Parameters:					  	
 v0=1500.	Velocity at first sample (m/sec)		
 rho0=1.0e6	Density at first sample  (g/m^3)		

 Notes:							
 Implements recursion [1-R(k)]Z(k) = [1+R(k)]Z(k-1).		
 The input traces are assumed to be reflectivities, and thus are
 expected to have amplitude values between -1.0 and 1.0.	


 Credits:
	SEP: Stew Levin

 Trace header fields accessed: ns
 

\end{verbatim}
\pagebreak
\begin{verbatim}
 SUKDSYN2D - Kirchhoff Depth SYNthesis of 2D seismic data from a	
             migrated seismic section					

   sukdsyn2d  infile  outfile [parameters] 				

 Required parameters: 							
 infile=stdin		input migrated section				
 outfile=stdout	file for output seismic traces  		
 ttfile		file for input traveltime tables		

 The following 9 parameters describe traveltime tables:		
 fzt= 			first depth sample 				
 nzt= 			number of depth samples 			
 dzt=			depth interval 					
 fxt=			first lateral sample 				
 nxt=			number of lateral samples 			
 dxt=			lateral interval 				
 fs=			x-coordinate of first source			
 ns=			number of sources				
 ds=			x-coordinate increment of sources		

 The following 6 parameters describe the migration section:		
 fz=                   first z-coordinate in migrated section 		
 dz=     		vertical spacing of migrated section 		
 nz=           	number of depth points in migrated section	
 fx=                   first x-coordinate of migrated section 		
 dx=     		horizontal spacing of migrated section 		
 nx=           	number of lateral points in migrated section  	

 Optional Parameters:							
 nt=501        	number of time samples				
 dt=0.004      	time sampling interval (sec)			
 ft=0.0        	first time (sec)				
 nxo=1                 number of source-receiver offsets		
 dxo=25                offset sampling interval  			
 fxo=0.0               first offset  					
 nxs=101               number of shotpoints  				
 dxs=25                shotpoint sampling interval  			
 fxs=0.0               first shotpoint 				
 fmax=1/(4*dt)         maximum frequency in migration section (Hz)	
 aperx=nxt*dxt/2  	modeling lateral aperature 			
 angmax=60		modeling angle aperature from vertical		
 v0=1500(m/s)		reference velocity value at surface		
 dvz=0.0  		reference velocity vertical gradient		
 ls=1	                flag for line source				
 jpfile=stderr		job print file name 				
 mtr=100  		print verbal information at every mtr traces	

 Notes:								
 This program takes a migrated seismic section and a set of travel time
 tables generated using rayt2d for a specific background velocity model
 and generates synthetic seismic data in the form of common shot gathers.
 (Common offset gathers may be generated by using nxo=1.) (Demigration.)

 This program is a tool which may be used for the migration residual	
 statics estimation technique of Tjan, Audebert, and Larner 1994.	

1. The traveltime tables are generated by program rayt2d (or other ones)
   on relatively coarse grids, with dimension ns*nxt*nzt. In the	
   modeling process, traveltimes are interpolated into shot/geophone 	
   positions and migration section grids. 				
2. The input migration section must be an array of binary floats (no SU
   headers).								", 
3. The synthesized traces are output in common-shot gathers in SU format.
4. The memory requirement for this program is about			
    	(ns*nxt*nzt+nx*nz+4*nr*nzt+3*nxt*nzt)*4 bytes 			
    where nr = 1+min(nxt*dxt,0.5*offmax+aperx)/dxo. 			


 Author:  CWP: Zhenyue Liu, 07/24/95,  Colorado School of Mines 

 References: 

 Tjan, T., F. Audebert, and K. Larner, 1994,
    Prestack migration for residual statics estimation in complex media
    (Appeared in 1994 Project Review, CWP-153.)

 Tjan, T., 1995, Residual statics estimation for data from structurally
    complex areas using prestack depth migration: M.Sc. thesis, Colorado
    School of Mines. (In progress.)

 Larner, K., and Tjan, T., 1995, Simultaneous statics and velocity
    estimation for data from structurally complex areas.
    (Appeared in 1995 Project Review, CWP-185.)


 Trace header fields set: ns, dt, delrt, tracl, tracr, fldr, tracf
                          offset, sx, gx, trid, counit


\end{verbatim}
\pagebreak
\begin{verbatim}
 SUNHMOSPIKE - generates SPIKE test data set with a choice of several   
   Non-Hyperbolic MOveouts						

   sunhmospike [optional parameters] > out_data_file  			

 Optional parameters:							
	nt=300	  number of time samples				
	ntr=20	  number of traces					
	dt=0.001	time sample rate in seconds			
	offref=2000	reference offset				

	gopt=		1 = parabolic transform model			
			2 = Foster/Mosher pseudo hyperbolic option model
			3 = linear tau-p model				
	depthref=400	reference depth used when gopt=2		
	offinc=100	offset increment				
	nspk=4	  number of events					

	p1 = 0		event moveout for event #1 in ms on reference offset
	t1 = 100	intercept time ms event #1			
	a1 = 1.0	amplitude for event #1				

	p2 = 200	event moveout for event #2 in ms on reference offset
	t2 = 100	intercept time ms for spike #2			
	a2 = 1.0	amplitude for event #2				

	p3 = 0;		event moveout for event #3 in ms on reference offset
	t3 = 200	intercept time for spike #3			
	a3 = 1.0	amplitude for event #3				
	p4 = 120	 event moveout for event #4 in ms on reference offset

	t4 = 200	intercept time s for spike #4			
	a4 = 1.0	amplitude for event #4				

	cdp = 1	 output cdp number					

 Notes:								
 Creates a common cdp su data file with up to four spike events	
 for impulse response studies for suradon, and sutifowler		


 Credits:
	CWP: Shuki Ronen, Chris Liner, 
      Modified: CWP   by John Anderson, April, 1994, to have
       appropriate trace header words and default values 
       for SUTIFOWLER tests


\end{verbatim}
\pagebreak
\begin{verbatim}
 SUNULL - create null (all zeroes) traces	 		

 sunull nt=   [optional parameters] >outdata			

 Required parameter						
 	nt=		number of samples per trace		

 Optional parameters						
 	ntr=5		number of null traces to create		
 	dt=0.004	time sampling interval			

 Rationale: It is sometimes useful to insert null traces	
	 between "panels" in a shell loop.			

 See also: sukill, sumute, suzero				


 Credits:
	CWP: Jack K. Cohen

 Trace header fields set: ns, dt, tracl

\end{verbatim}
\pagebreak
\begin{verbatim}
SUPLANE - create common offset data file with up to 3 planes	

suplane [optional parameters] >stdout	 			

Optional Parameters:						
 npl=3			number of planes			
 nt=64 		number of time samples			
 ntr=32		number of traces			
 taper=0		no end-of-plane taper			
			= 1 taper planes to zero at the end	
 offset=400 		offset					
 dt=0.004	 	time sample interval in seconds		
...plane 1 ...							
	dip1=0		dip of plane #1 (ms/trace)		
 	len1= 3*ntr/4	HORIZONTAL extent of plane (traces)	
	ct1= nt/2	time sample for center pivot	 	
	cx1= ntr/2	trace for center pivot			
...plane 2 ...							
	dip2=4		dip of plane #2 (ms/trace)		
	len2= 3*ntr/4	HORIZONTAL extent of plane (traces)	
	ct2= nt/2	time sample for center pivot 		
	cx2= ntr/2	trace for center pivot			
...plane 3 ...							
	dip3=8		dip of plane #3 (ms/trace)		
	len3= 3*ntr/4	HORIZONTAL extent of plane (traces)	
	ct3= nt/2	time sample for center pivot		
	cx3= ntr/2	trace for center pivot			

 liner=0	use parameters					
			= 1 parameters set for 64x64 data set   
			with separated dipping planes.		

 Credits:
	CWP: Chris Liner

 Trace header fields set: ns, dt, offset, tracl

\end{verbatim}
\pagebreak
\begin{verbatim}
 SURANDSPIKE - make a small data set of RANDom SPIKEs 		

   surandspike [optional parameters] > out_data_file  		

 Creates a common offset su data file with random spikes	

 Optional parameters:						
	n1=500 			number of time samples		
	n2=200			number of traces		
 	dt=0.002 		time sample rate in seconds	
	nspk=20			number of spikes per trace	
	amax=0.2		abs(max) spike value		
	mode=1			different spikes on each trace	
				=2 same spikes on each trace	
 	seed=from_clock    	random number seed (integer)    


 Credits:
	ARAMCO: Chris Liner

 Trace header fields set: ns, dt, offset

\end{verbatim}
\pagebreak
\begin{verbatim}
 SUREMAC2D - Acoustic 2D Fourier method modeling with high accuracy     
             Rapid Expansion Method (REM) time integration              

 suremac2d [parameters]                                                 

 Required parameters:                                                   

 opflag=     0: variable density wave equation                          
             1: constant density wave equation                          
             2: non-reflecting wave equation                            

 nx=         number of grid points in horizontal direction              
 nz=         number of grid points in vertical direction                
 nt=         number of time samples                                     
 dx=         spatial increment in horizontal direction                  
 dz=         spatial increment in vertical direction                    
 dt=         time sample interval in seconds                            
 isx=        grid point # of horizontal source positions                
 isz=        grid point # of vertical source positions                  

 Optional parameters:                                                   
 fx=0.0      first horizontal coordinate                                
 fz=0.0      first vertical coordinate                                  
 irx=        horizontal grid point # of vertical receiver lines         
 irz=        vertical grid point # of horizontal receiver lines         
 w=0.1       width of spatial source distribution (see notes)           
 sflag=2     source time function                                       
             0: user supplied source function                           
             1: impulse (spike at t=0)                                  
             2: Ricker wavelet                                          
 fmax=       maximum frequency of Ricker (default) wavelet              
 amps=1.0    amplitudes of sources                                      
 prec=0      1: precompute Bessel coefficients b_k (see notes)          
             2: use precomputed Bessel coefficients b_k                 
 fsflag=0    1: perform run with free surface b.c.                      
 vmaxu=      user-defined maximum velocity                              
 dtsnap=0.0  time interval in seconds of wave field snapshots           
 iabso=1     apply absorbing boundary conditions (0: none)              
 abso=0.1    damping parameter for absorbing boundaries                 
 nbwx=20     horizontal width of absorbing boundary                     
 nbwz=20     vertical width of absorbing boundary                       
 verbose=0   1: show parameters used                                    
             2: print maximum amplitude at every expansion term         

 velfile=vel          velocity filename                                 
 densfile=dens        density filename                                  
 sname=wavelet.su     user supplied source time function filename       
 sepxname=sectx.su    x-direction pressure sections filename            
 sepzname=sectz.su    z-direction pressure sections filename            
 snpname=snap.su      pressure snapshot filename                        
 jpfile=stderr        diagnostic output                                 

 Notes:                                                                 
  0. The combination of the Fourier method with REM time integration    
     allows the computation of synthetic seismograms which are free     
     of numerical grid dispersion. REM has no restriction on the        
     time step size dt. The Fourier method requires at least two        
     grid points per shortest wavelength.                               
  1. nx and nz must be valid numbers for pfafft transform lengths.      
     nx and nz must be odd numbers (unless opflag=1). For valid         
     numbers see e.g. numbers in structure 'nctab' in source file       
     $CWPROOT/src/cwp/lib/pfafft.c.                                     
  2. Velocities (and densities) are stored as plain C style files       
     of floats where the fast dimension is along the z-direction.       
  3. Units must be consistent, e.g. m, s and m/s.                       
  4. A 20 grid points wide border at the sides and the bottom of        
     the modeling grid is used for sponge boundary conditions           
     (default: iabso=1).                                                
     Source and receiver lines should be placed some (e.g. 10) grid     
     points away from the absorbing boundaries in order to reduce       
     reflections due to obliquely incident wavefronts.                  
  5. Dominant frequency is about fmax/2 (sflag=2), absolute maximum     
     is delayed by 3/fmax from beginning of wavelet.                    
  6. If opflag!=1 the source should be not a spike in space; the        
     parameter w determines at which distance (in grid points) from     
     the source's center the Gaussian weight decays to 10 percent       
     of its maximum. w=2 may be a reasonable choice; however, the       
     waveform will be distorted.                                        
  7. Horizontal and vertical receiver line sections are written to      
     separate files. Each file can hold more than one line.             
  8. Parameter vmaxu may need to be chosen larger than the highest      
     propagation velocity if the modeling run becomes unstable.         
     This happens if the largest eigenvalue of the modeling             
     operator L is larger than estimated from the largest velocity      
     due to variations of the density.                                  
     In particular if using the variable density acoustic wave          
     equation the eigenvalues depend also on the density and it is      
     impossible to estimated the largest eigenvalue analytically.       
  9. Bessel coefficients can be precomputed (prec=1) and stored on      
     disk to save CPU time when several shots need to be run.           
     In this case computation of Bessel coefficients can be skipped     
     and read from disk file for reuse (prec=2).                        
     For reuse of Bessel coefficients the user may need to define       
     the overall maximum velocity (vmaxu).                              
 10. If snapshots are not required, a spike source (sflag=1) may be     
     applied and the resulting impulse response seismograms can be      
     convolved later with a desired wavelet.                            
 11. The free surface (fsflag=1) does not coincide with the first       
     vertical grid index (0). It appears to be half a grid spacing      
     above that position.                                               



  Acoustic 2D Fourier method modeling with REM time integration

  Reference: 
  Kosloff, D., Fihlo A.Q., Tessmer, E. and Behle, A., 1989,
    Numerical solution of the acoustic and elastic wave equations by a
    new rapid expansion method, Geophysical Prospecting, 37, 383-394
  
 Credits:
      University of Hamburg: Ekkehart Tessmer, October 2012

\end{verbatim}
\pagebreak
\begin{verbatim}
 SUREMEL2DAN - Elastic anisotropic 2D Fourier method modeling with high 
               accuracy Rapid Expansion Method (REM) time integration   

 suremel2dan [parameters]                                               

 Required parameters:                                                   

 nx=         number of grid points in horizontal direction              
 nz=         number of grid points in vertical direction                
 nt=         number of time samples                                     
 dx=         spatial increment in horizontal direction                  
 dz=         spatial increment in vertical direction                    
 dt=         time sample interval in seconds                            
 isx=        grid point # of horizontal source positions                
 isz=        grid point # of vertical source positions                  
 styp=       source types (pressure, shear, single forces)              
 samp=       amplitudes of sources                                      
 amode=      0: isotropic,  1: anisotropic                              
 vmax=       global maximum velocity (only if amode=1)                  
 vmin=       global minimum velocity (only if amode=1)                  

 Optional parameters:                                                   
 fx=0.0      first horizontal coordinate                                
 fz=0.0      first vertical coordinate                                  
 irx=        horizontal grid point # of vertical receiver lines         
 irz=        vertical grid point # of horizontal receiver lines         
 rxtyp=      types of horizontal receiver lines                         
 rztyp=      types of vertical receivers lines                          
 sntyp=      types of snapshots                                         
             0: P,  1: S,  2: UX,  3: UZ                                
 w=0.1       width of spatial source distribution (see notes)           
 sflag=2     source time function                                       
             0: user supplied source function                           
             1: impulse (spike at t=0)                                  
             2: Ricker wavelet                                          
 fmax=       maximum frequency of Ricker (default) wavelet              
 amps=1.0    amplitudes of sources                                      
 prec=0      1: precompute Bessel coefficients b_k (see notes)          
             2: use precomputed Bessel coefficients b_k                 
 vmaxu=      user-defined maximum velocity                              
 dtsnap=0.0  time interval in seconds of wave field snapshots           
 iabso=1     apply absorbing boundary conditions (0: none)              
 abso=0.1    damping parameter for absorbing boundaries                 
 nbwx=20     horizontal width of absorbing boundary                     
 nbwz=20     vertical width of absorbing boundary                       
 verbose=0   1: show parameters used                                    
             2: print maximum amplitude at every expansion term         

 c11file=c11       c11 filename                                         
 c13file=c13       c13 filename                                         
 c15file=c15       c15 filename                                         
 c33file=c33       c33 filename                                         
 c35file=c35       c35 filename                                         
 c55file=c55       c55 filename                                         
 vpfile=vp         P-velocity filename                                  
 vsfile=vs         S-velocity filename                                  
 densfile=dens     density filename                                     

 sname=wavelet.su  user supplied source time function filename          

 Basenames of seismogram and snapshot files:                            
 xsect=xsect_     x-direction section files basename                    
 zsect=zsect_     z-direction section files basename                    
 snap=snap_       snapshot files basename                               

 jpfile=stderr        diagnostic output                                 

 Notes:                                                                 
  0. The combination of the Fourier method with REM time integration    
     allows the computation of synthetic seismograms which are free     
     of numerical grid dispersion. REM has no restriction on the        
     time step size dt. The Fourier method requires at least two        
     grid points per shortest wavelength.                               
  1. nx and nz must be valid numbers for pfafft transform lengths.      
     nx and nz must be odd numbers. For valid numbers see e.g.          
     numbers in structure 'nctab' in source file                        
     $CWPROOT/src/cwp/lib/pfafft.c.                                     
  2. Velocities and densities are stored as plain C style files         
     of floats where the fast dimension is along the z-direction.       
  3. Units must be consistent, e.g. m, s and m/s.                       
  4. A 20 grid points wide border at the sides and the bottom of        
     the modeling grid is used for sponge boundary conditions           
     (default: iabso=1).                                                
     Source and receiver lines should be placed some (e.g. 10) grid     
     points away from the absorbing boundaries in order to reduce       
     reflections due to obliquely incident wavefronts.                  
  5. Dominant frequency is about fmax/2 (sflag=2), absolute maximum     
     is delayed by 3/fmax from beginning of wavelet.                    
  6. If source is not single force (i.e. pressure or shear source)      
     it should be not a spike in space; the parameter w determines      
     at which distance (in grid points) from the source's center        
     the Gaussian weight decays to 10 percent of its maximum.           
     w=2 may be a reasonable choice; however, the waveform will be      
     distorted.                                                         
  7. Horizontal and vertical receiver line sections are written to      
     separate files. Each file can hold more than one line.             
  8. Parameter vmaxu may need to be chosen larger than the highest      
     propagation velocity if the modeling run becomes unstable.         
     This happens if the largest eigenvalue of the modeling             
     operator L is larger than estimated from the largest velocity      
     due to variations of the density.                                  
  9. Bessel coefficients can be precomputed (prec=1) and stored on      
     disk to save CPU time when several shots need to be run.           
     In this case computation of Bessel coefficients can be skipped     
     and read from disk file for reuse (prec=2).                        
     For reuse of Bessel coefficients the user may need to define       
     the overall maximum velocity (vmaxu).                              
 10. If snapshots are not required, a spike source (sflag=1) may be     
     applied and the resulting impulse response seismograms can be      
     convolved later with a desired wavelet.                            
 11. Output is written to SU style files.                               ", 
     Basenames of seismogram and snapshot output files will be          
     extended by the type of the data (p, s, ux, or uz).                
     Additionally seismogram files will be consecutively numbered.      

 Caveat:                                                                
     Time sections and snapshots are kept entirely in memory during     
     run time. Therefore, lots of time section and snapshots may        
     eat up a large amount of memory.                                   


  Elastic anisotropic 2D Fourier method modeling with REM time integration

  Reference: 
  Kosloff, D., Fihlo A.Q., Tessmer, E. and Behle, A., 1989,
    Numerical solution of the acoustic and elastic wave equations by a
    new rapid expansion method, Geophysical Prospecting, 37, 383-394
  
 Credits:
      University of Hamburg: Ekkehart Tessmer, July 2013

\end{verbatim}
\pagebreak
\begin{verbatim}
 SUSPIKE - make a small spike data set 			

 suspike [optional parameters] > out_data_file  		

 Creates a common offset su data file with up to four spikes	
 for impulse response studies					

 Optional parameters:						
	nt=64 		number of time samples			
	ntr=32		number of traces			
 	dt=0.004 	time sample rate in seconds		
 	offset=400 	offset					
	nspk=4		number of spikes			
	ix1= ntr/4	trace number (from left) for spike #1	
	it1= nt/4 	time sample to spike #1			
	ix2 = ntr/4	trace for spike #2			
	it2 = 3*nt/4 	time for spike #2			
	ix3 = 3*ntr/4;	trace for spike #3			
	it3 = nt/4;	time for spike #3			
	ix4 = 3*ntr/4;	trace for spike #4			
	it4 = 3*nt/4;	time for spike #4			


 Credits:
	CWP: Shuki Ronen, Chris Liner

 Trace header fields set: ns, dt, offset

\end{verbatim}
\pagebreak
\begin{verbatim}
 SUSYNCZ - SYNthetic seismograms for piecewise constant V(Z) function	
	   True amplitude (primaries only) modeling for 2.5D		

  susyncz > outfile [parameters]					

 Required parameters:							
 none									

 Optional Parameters:							
 ninf=4        number of interfaces (not including upper surface)	
 dip=5*i       dips of interfaces in degrees (i=1,2,3,4)		
 zint=100*i    z-intercepts of interfaces at x=0 (i=1,2,3,4)		
 v=1500+ 500*i velocities below surface & interfaces (i=0,1,2,3,4)	
 rho=1,1,1,1,1 densities below surface & interfaces (i=0,1,2,3,4)	
 nline=1	number of (identical) lines				
 ntr=32        number of traces					
 dx=10         trace interval						
 tdelay=0      delay in recording time after source initiation		
 dt=0.004      time interval						
 nt=128        number of time samples					

 Notes:								
 The original purpose of this code was to create some nontrivial	
 data for Brian Sumner's CZ suite.					

 The program produces zero-offset data over dipping reflectors.	

 In the original fortran code, some arrays had the index		
 interval 1:ninf, as a natural way to index over the subsurface	
 reflectors.  This indexing was preserved in this C translation.	
 Consequently, some arrays in the code do not use the 0 "slot".	

 Example:								
	susyncz | sufilter | sugain tpow=1 | display_program		

 Trace header fields set: tracl, ns, dt, delrt, ntr, sx, gx		


 Credits:
 	CWP: Brian Sumner, 1983, 1985, Fortran design and code 
      CWP: Stockwell & Cohen, 1995, translation to C 



\end{verbatim}
\pagebreak
\begin{verbatim}
 SUSYNLVCW - SYNthetic seismograms for Linear Velocity function	
 		for mode Converted Waves				

 susynlvcw >outfile [optional parameters]				

 Optional Parameters:							
 nt=101		number of time samples				
 dt=0.04		time sampling interval (sec)			
 ft=0.0		first time (sec)				
 nxo=1			number of source-receiver offsets		
 dxo=0.05		offset sampling interval (km)			
 fxo=0.0		first offset (km, see notes below)		
 xo=fxo,fxo+dxo,...	array of offsets (use only for non-uniform offsets)
 nxm=101		number of midpoints (see notes below)		
 dxm=0.05		midpoint sampling interval (km)		
 fxm=0.0		first midpoint (km)				
 nxs=101		number of shotpoints (see notes below)		
 dxs=0.05		shotpoint sampling interval (km)		
 fxs=0.0		first shotpoint (km)				
 x0=0.0		distance x at which v00 is specified		
 z0=0.0		depth z at which v00 is specified		
 v00=2.0		velocity at x0,z0 (km/sec)			
 gamma=1.0		velocity ratio, upgoing/downgoing		
 dvdx=0.0		derivative of velocity with distance x (dv/dx)	
 dvdz=0.0		derivative of velocity with depth z (dv/dz)	
 fpeak=0.2/dt		peak frequency of symmetric Ricker wavelet (Hz)	
 ref="1:1,2;4,2"	reflector(s):  "amplitude:x1,z1;x2,z2;x3,z3;...
 smooth=0		=1 for smooth (piecewise cubic spline) reflectors
 er=0			=1 for exploding reflector amplitudes		
 ls=0			=1 for line source; default is point source	
 ob=1			=1 to include obliquity factors		
 sp=1			=1 to account for amplitude spreading		
 			=0 for constant amplitudes throught out		
 tmin=10.0*dt		minimum time of interest (sec)			
 ndpfz=5		number of diffractors per Fresnel zone		
 verbose=0		=1 to print some useful information		

 Notes:								

 Offsets are signed - may be positive or negative.  Receiver locations	
 are computed by adding the signed offset to the source location.	

 Specify either midpoint sampling or shotpoint sampling, but not both.	
 If neither is specified, the default is the midpoint sampling above.	

 More than one ref (reflector) may be specified.  When obliquity factors
 are included, then only the left side of each reflector (as the x,z	
 reflector coordinates are traversed) is reflecting.  For example, if x
 coordinates increase, then the top side of a reflector is reflecting.	
 Note that reflectors are encoded as quoted strings, with an optional	
 reflector amplitude: preceding the x,z coordinates of each reflector.	
 Default amplitude is 1.0 if amplitude: part of the string is omitted.	

 Note that gamma<1 implies P-SV mode conversion, gamma>1 implies SV-P,	
 and gamma=1 implies no mode conversion.				



 based on Dave Hale's code susynlv, but modified
 by Mohammed Alfaraj to handle mode conversion
 Date of modification: 01/07/92

 Trace header fields set: trid, counit, ns, dt, delrt,
				tracl. tracr, fldr, tracf,
				cdp, cdpt, d2, f2, offset, sx, gx

\end{verbatim}
\pagebreak
\begin{verbatim}
 SUSYNLVFTI - SYNthetic seismograms for Linear Velocity function in a  
              Factorized Transversely Isotropic medium			

 susynlvfti >outfile [optional parameters]				

 Optional Parameters:							
 nt=101		number of time samples				
 dt=0.04		time sampling interval (sec)			
 ft=0.0		first time (sec)				
 kilounits=1            input length units are km or kilo-feet         
                        =0 for m or ft                                 
                        Note: Output (sx,gx,offset) are always m or ft 
 nxo=1			number of source-receiver offsets		
 dxo=0.05		offset sampling interval (kilounits)		
 fxo=0.0		first offset (kilounits, see notes below)	
 xo=fxo,fxo+dxo,...    array of offsets (use only for non-uniform offsets)
 nxm=101		number of midpoints (see notes below)		
 dxm=0.05		midpoint sampling interval (kilounits)		
 fxm=0.0		first midpoint (kilounits)			
 nxs=101		number of shotpoints (see notes below)		
 dxs=0.05		shotpoint sampling interval (kilounits)		
 fxs=0.0		first shotpoint (kilounits)			
 x0=0.0		distance x at which v00 is specified		
 z0=0.0		depth z at which v00 is specified		
 v00=2.0		velocity at x0,z0 (kilounits/sec)		
 dvdx=0.0		derivative of velocity with distance x (dv/dx)	
 dvdz=0.0		derivative of velocity with depth z (dv/dz)	
 fpeak=0.2/dt		peak frequency of symmetric Ricker wavelet (Hz)	
 ref=1:1,2;4,2		reflector(s):  "amplitude:x1,z1;x2,z2;x3,z3;...
 smooth=0		=1 for smooth (piecewise cubic spline) reflectors
 er=0			=1 for exploding reflector amplitudes		
 ls=0			=1 for line source; default is point source	
 ob=0			=1 to include obliquity factors			
 tmin=10.0*dt		minimum time of interest (sec)			
 ndpfz=5		number of diffractors per Fresnel zone		
 verbose=1		=1 to print some useful information		

 For transversely isotropic media:					
 angxs=0.0		angle of symmetry axis with the vertical (degrees)
 define the media using either						
 a=1.0		corresponding to the ratio of elastic coef.(c1111/c3333)
 f=0.4		corresponding to the ratio of elastic coef. (c1133/c3333)
 l=0.3		corresponding to the ratio of elastic coef. (c1313/c3333)
 Alternately use Tompson\'s parameters:				
 delta=0	Thomsen's 1986 defined parameter			
 epsilon=0	Thomsen's 1986 defined parameter			
 ntries=40	number of iterations in Snell's law and offset searches 
 epsx=.001	lateral offset tolerance				
 epst=.0001	reflection time tolerance				
 nitmax=12	max number of iterations in travel time integrations	

 Notes:								

 Offsets are signed - may be positive or negative.  Receiver locations	
 are computed by adding the signed offset to the source location.	

 Specify either midpoint sampling or shotpoint sampling, but not both.	
 If neither is specified, the default is the midpoint sampling above.	

 More than one ref (reflector) may be specified.  When obliquity factors
 are included, then only the left side of each reflector (as the x,z	
 reflector coordinates are traversed) is reflecting.  For example, if x
 coordinates increase, then the top side of a reflector is reflecting.	
 Note that reflectors are encoded as quoted strings, with an optional	
 reflector amplitude: preceding the x,z coordinates of each reflector.	
 Default amplitude is 1.0 if amplitude: part of the string is omitted.	

 Concerning the choice of delta and epsilon. The difference between delta", 
 and epsilon should not exceed one. A possible break down of the program
 is the result. This is caused primarly by the break down in the two point", 
 ray-tracing. Also keep the values of delta and epsilon between 2 and -2.
\end{verbatim}
\pagebreak
\begin{verbatim}
 SUSYNLV - SYNthetic seismograms for Linear Velocity function		

 susynlv >outfile [optional parameters]				

 Optional Parameters:							
 nt=101                 number of time samples				
 dt=0.04                time sampling interval (sec)			
 ft=0.0                 first time (sec)				
 kilounits=1            input length units are km or kilo-feet		
			 =0 for m or ft					
                        Note: Output (sx,gx,offset) are always m or ft 
 nxo=1                  number of source-receiver offsets		
 dxo=0.05               offset sampling interval (kilounits)		
 fxo=0.0                first offset (kilounits, see notes below)	
 xo=fxo,fxo+dxo,...     array of offsets (use only for non-uniform offsets)
 nxm=101                number of midpoints (see notes below)		
 dxm=0.05               midpoint sampling interval (kilounits)		
 fxm=0.0                first midpoint (kilounits)			
 nxs=101                number of shotpoints (see notes below)		
 dxs=0.05               shotpoint sampling interval (kilounits)	
 fxs=0.0                first shotpoint (kilounits)			
 x0=0.0                 distance x at which v00 is specified		
 z0=0.0                 depth z at which v00 is specified		
 v00=2.0                velocity at x0,z0 (kilounits/sec)		
 dvdx=0.0               derivative of velocity with distance x (dv/dx)	
 dvdz=0.0               derivative of velocity with depth z (dv/dz)	
 fpeak=0.2/dt           peak frequency of symmetric Ricker wavelet (Hz)
 ref="1:1,2;4,2"        reflector(s):  "amplitude:x1,z1;x2,z2;x3,z3;...
 smooth=0               =1 for smooth (piecewise cubic spline) reflectors
 er=0                   =1 for exploding reflector amplitudes		
 ls=0                   =1 for line source; default is point source	
 ob=1                   =1 to include obliquity factors		
 tmin=10.0*dt           minimum time of interest (sec)			
 ndpfz=5                number of diffractors per Fresnel zone		
 verbose=0              =1 to print some useful information		

Notes:								
Offsets are signed - may be positive or negative.  Receiver locations	
are computed by adding the signed offset to the source location.	

Specify either midpoint sampling or shotpoint sampling, but not both.	
If neither is specified, the default is the midpoint sampling above.	

More than one ref (reflector) may be specified. Do this by putting	
additional ref= entries on the commandline. When obliquity factors	
are included, then only the left side of each reflector (as the x,z	
reflector coordinates are traversed) is reflecting.  For example, if x	
coordinates increase, then the top side of a reflector is reflecting.	
Note that reflectors are encoded as quoted strings, with an optional	
reflector amplitude: preceding the x,z coordinates of each reflector.	
Default amplitude is 1.0 if amplitude: part of the string is omitted.	


 Credits: CWP Dave Hale, 09/17/91,  Colorado School of Mines
	    UTulsa Chris Liner 5/22/03 added kilounits flag

 Trace header fields set: trid, counit, ns, dt, delrt,
				tracl. tracr, fldr, tracf,
				cdp, cdpt, d2, f2, offset, sx, gx

\end{verbatim}
\pagebreak
\begin{verbatim}
 SUSYNVXZCS - SYNthetic seismograms of common shot in V(X,Z) media via	
 		Kirchhoff-style modeling				

 susynvxzcs<vfile >outfile  nx= nz= [optional parameters]		

 Required Parameters:							
 <vfile        file containing velocities v[nx][nz]			
 >outfile      file containing seismograms of common ofset		
 nx=           number of x samples (2nd dimension) in velocity 
 nz=           number of z samples (1st dimension) in velocity 

 Optional Parameters:							
 nt=501        	number of time samples				
 dt=0.004      	time sampling interval (sec)			
 ft=0.0        	first time (sec)				
 fpeak=0.2/dt		peak frequency of symmetric Ricker wavelet (Hz)	
 nxg=			number of receivers of input traces		
 dxg=15		receiver sampling interval (m)			
 fxg=0.0		first receiver (m)				
 nxd=5         	skipped number of receivers			
 nxs=1			number of offsets				
 dxs=50		shot sampling interval (m)			
 fxs=0.0		first shot (m)				
 dx=50         	x sampling interval (m)				
 fx=0.         	first x sample (m)				
 dz=50         	z sampling interval (m)				
 nxb=nx/2    	band width centered at midpoint (see note)	
 nxc=0         hozizontal range in which velocity is changed	
 nzc=0         vertical range in which velocity is changed	
 pert=0        =1 calculate time correction from v_p[nx][nz]	
 vpfile        file containing slowness perturbation array v_p[nx][nz]	
 ref="1:1,2;4,2"	reflector(s):  "amplitude:x1,z1;x2,z2;x3,z3;...
 smooth=0		=1 for smooth (piecewise cubic spline) reflectors
 ls=0			=1 for line source; =0 for point source		
 tmin=10.0*dt		minimum time of interest (sec)			
 ndpfz=5		number of diffractors per Fresnel zone		
 cable=1		roll reciever spread with shot			
 			=0 static reciever spread			
 verbose=0		=1 to print some useful information		

 Notes:								
 This algorithm is based on formula (58) in Geo. Pros. 34, 686-703,	
 by N. Bleistein.							

 Traveltime and amplitude are calculated by finite difference which	
 is done only in one of every NXD receivers; in skipped receivers, 	
 interpolation is used to calculate traveltime and amplitude.		", 
 For each receiver, traveltime and amplitude are calculated in the 	
 horizontal range of (xg-nxb*dx, xg+nxb*dx). Velocity is changed by 	
 constant extropolation in two upper trianglar corners whose width is 	
 nxc*dx and height is nzc*dz.						

 Eikonal equation will fail to solve if there is a polar turned ray.	
 In this case, the program shows the related geometric information. 	
 There are three ways to remove the turned rays: smoothing velocity, 	
 reducing nxb, and increaing nxc and nzc (if the turned ray occurs  	
 in shallow areas). To prevent traveltime distortion from an over-	
 smoothed velocity, traveltime is corrected based on the slowness 	
 perturbation.								

 More than one ref (reflector) may be specified.			
 Note that reflectors are encoded as quoted strings, with an optional	
 reflector amplitude: preceding the x,z coordinates of each reflector.	
 Default amplitude is 1.0 if amplitude: part of the string is omitted.	



	Author: Zhenyue Liu, 07/20/92, Center for Wave Phenomena
		Many subroutines borrowed from Dave Hale's program: SUSYNLV

		Trino Salinas, 07/30/96, fixed a bug in the geometry
		setting to allow the spread move with the shots.

		Chris Liner 12/10/08  added cable option, set fldr header word

 Trace header fields set: trid, counit, ns, dt, delrt,
				tracl. tracr, fldr, tracf,
				sx, gx

\end{verbatim}
\pagebreak
\begin{verbatim}
 SUSYNVXZ - SYNthetic seismograms of common offset V(X,Z) media via	
 		Kirchhoff-style modeling				

 susynvxz >outfile [optional parameters]				

 Required Parameters:							
 <vfile		file containing velocities v[nx][nz]		
 nx=			number of x samples (2nd dimension)		
 nz=			number of z samples (1st dimension)		
 Optional Parameters:							
 nxb=nx		band centered at midpoint			
 nxd=1			skipped number of midponits			
 dx=100		x sampling interval (m)				
 fx=0.0		first x sample					
 dz=100		z sampling interval (m)				
 nt=101		number of time samples				
 dt=0.04		time sampling interval (sec)			
 ft=0.0		first time (sec)				
 nxo=1		 	number of offsets				
 dxo=50		offset sampling interval (m)			
 fxo=0.0		first offset (m)				
 nxm=101		number of midpoints				
 dxm=50		midpoint sampling interval (m)			
 fxm=0.0		first midpoint (m)				
 fpeak=0.2/dt		peak frequency of symmetric Ricker wavelet (Hz)	
 ref="1:1,2;4,2"	reflector(s):  "amplitude:x1,z1;x2,z2;x3,z3;...
 smooth=0		=1 for smooth (piecewise cubic spline) reflectors
 ls=0			=1 for line source; default is point source	
 tmin=10.0*dt		minimum time of interest (sec)			
 ndpfz=5		number of diffractors per Fresnel zone		
 verbose=0		=1 to print some useful information		

 Notes:								
 This algorithm is based on formula (58) in Geo. Pros. 34, 686-703,	
 by N. Bleistein.							

 Offsets are signed - may be positive or negative.			", 
 Traveltime and amplitude are calculated by finite differences which	
 is done only in part of midpoints; in the skiped midpoint, interpolation
 is used to calculate traveltime and amplitude.			", 

 More than one ref (reflector) may be specified.			
 Note that reflectors are encoded as quoted strings, with an optional	
 reflector amplitude: preceding the x,z coordinates of each reflector.	
 Default amplitude is 1.0 if amplitude: part of the string is omitted.	



   CWP:  Zhenyue Liu, 07/20/92
	Many subroutines borrowed from Dave Hale's program: SUSYNLV

 Trace header fields set: trid, counit, ns, dt, delrt,
				tracl. tracr,
				cdp, cdpt, d2, f2, offset, sx, gx

\end{verbatim}
\pagebreak
\begin{verbatim}
 SUVIBRO - Generates a Vibroseis sweep (linear, linear-segment,
			dB per Octave, dB per Hertz, T-power)	

 suvibro [optional parameters] > out_data_file			

 Optional Parameters:						
 dt=0.004		time sampling interval			
 sweep=1	  	linear sweep			  	
 		  	=2 linear-segment			
 		  	=3 decibel per octave	 		
 		  	=4 decibel per hertz	  		
 		  	=5 t-power				
 swconst=0.0		sweep constant (see note)		
 f1=10.0		sweep frequency at start		
 f2=60.0		sweep frequency at end			
 tv=10.0		sweep length				
 phz=0.0		initial phase (radians=1 default)	
 radians=1		=0 degrees				
 fseg=10.0,60.0	frequency segments (see notes)		
 tseg=0.0,10.0		time segments (see notes)		
 t1=1.0		length of taper at start (see notes)	
 t2=1.0		length of taper at end (see notes)	
 taper=1		linear					
		  	=2 sine					
			=3 cosine				
			=4 gaussian (+/-3.8)			
			=5 gaussian (+/-2.0)			

 Notes:							
 The default tapers are linear envelopes. To eliminate the	
 taper, choose t1=t2=0.0.					

 "swconst" is active only with nonlinear sweeps, i.e. when	
 sweep=3,4,5.							", 
 "tseg" and "fseg" arrays are used when only sweep=2	

 Sweep is a modulated cosine function.				


 Author: CWP: Michel Dietrich
   Rewrite: Tagir Galikeev, CWP,  7 October 1994

 Trace header fields set: ns, dt, tracl, sfs, sfe, slen, styp

\end{verbatim}
\pagebreak
\begin{verbatim}
 SUWAVEFORM - generate a seismic wavelet				

 suwaveform <stdin >stdout [optional parameters]			

 Required parameters:						  	
	one of the optional parameters listed below			

 Optional parameters:						  	
	type=akb	wavelet type					
		   akb:	AKB wavelet defined by max frequency fpeak	
		   berlage: Berlage wavelet				
		   gauss:   Gaussian wavelet defined by frequency fpeak	
		   gaussd:  Gaussian first derivative wavelet		
		   ricker1: Ricker wavelet defined by frequency fpeak	
		   ricker2: Ricker wavelet defined by half and period	
		   spike:   spike wavelet, shifted by time tspike	
		   unit:	unit wavelet, i.e. amplitude = 1 = const.

	dt=0.004	time sampling interval in seconds		
	ns=		if set, number of samples in  output trace	

	fpeak=20.0	peak frequency of a Berlage, Ricker, or Gaussian,
		   and maximum frequency of an AKB wavelet in Hz	

	half=1/fpeak   Ricker wavelet "ricker2": half-length		
	period=c*half  Ricker wavelet "ricker2": period (c=sqrt(6)/pi)
	distort=0.0	Ricker wavelet "ricker2": distortion factor	
	decay=4*fpeak  Berlage wavelet: exponential decay factor in 1/sec
	tn=2	   Berlage wavelet: time exponent			
	ipa=-90	Berlage wavelet: initial phase angle in degrees		
	tspike=0.0	Spike wavelet: time at spike in seconds		
	verbose=0	1: echo output wavelet length			


 Notes:								
	If ns is not defined, the program determines the trace length	
	depending on the dominant signal period.			   

	The Ricker wavelet "ricker1" and the Gaussian wavelet "gauss
	are zero-phase. For these two wavelets, the trace header word	
	delrt is set such that the peak amplitude is at t=0 seconds.	
	If this is not acceptable, use "sushw key=delrt a=0".		

	The Ricker wavelets can be defined either by the peak frequency	
	fpeak ("ricker1") or by its half-length, the period, and a	
	distortion factor ("ricker2"). "ricker" is an acceptable	
	alias for "ricker1".						

	The Berlage wavelet is defined by the peak frequency fpeak, a time 
	time exponent tn describing the wavelet shape at its beginning,	
	and an exponential decay factor describing the amplitude decay	
	towards later times. The parameters tn and decay are non-negative, 
	real numbers; tn is typically a small integer number and decay a   
	multiple of the dominant signal period 1/fpeak. Additionally, an   
	initial phase angle can be given; use -90 or 90 degrees for	
	zero-amplitude at the beginning.				   

	For an AKB wavelet, fpeak is the maximum frequency; the peak	
	frequency is about 1/3 of the fpeak value.			 

	The output wavelet can be normalized or scaled with "sugain".	
	Use "suvibro" to generate a Vibroseis sweep.			

 Example:								
 A normalized, zero-phase Ricker wavelet with a peak frequency		
 of 15 Hz is generated and convolved with a spike dataset:		

	suwaveform type=ricker1 fpeak=15 | sugain pbal=1 > wavelet.su	
	suplane npl=1 | suconv sufile=wavelet.su | suxwigb		

 Gaussian and derivatives of Gaussians:				
 Use "sudgwaveform" to generate these				

 Caveat:								
	This program does not check for aliasing.			



 Author: 
	Nils Maercklin, RISSC, University of Napoli, Italy, 2006

 References:
	Aldridge, D. F. (1990). The Berlage wavelet. 
	Geophysics, vol. 55(11), p. 1508-1511.
	Alford, R., Kelly, K., and Boore, D. (1947). Accuracy
	of finite-difference modeling of the acoustic wave
	equation. Geophysics, vol. 39, p. 834-842. (AKB wavelet)
	Sheriff, R. E. (2002). Encyclopedic dictionary of 
	applied geophysics. Society of Exploration Geophysicists,
	Tulsa. (Ricker wavelet, page 301)

 Notes:
	For more information on the wavelets type "sudoc waveforms" 
	or have a look at "$CWPROOT/src/cwp/lib/waveforms.c".

 Credits: 
	CWP, the authors of the subroutines in "waveforms.c".

 Trace header fields set: ns, dt, trid, delrt

\end{verbatim}
\pagebreak
\begin{verbatim}
 SUGAUSSTAPER - Multiply traces with gaussian taper		

 sugausstaper < stdin > stdout [optional parameters]		

 Required Parameters:					   	
   <none>							

 Optional parameters:					   	
 key=offset    keyword of header field to weight traces by 	
 x0=300        key value defining the center of gaussian window", 
 xw=50         width of gaussian window in units of key value 	

 Notes:							
 Traces are multiplied with a symmetrical gaussian taper 	
  	w(t)=exp(-((key-x0)/xw)**2)				
 unlike "sutaper" the value of x0 defines center of the taper
 rather than the edges of the data.				

 Credits:

	Thomas Bohlen, formerly of TU Bergakademie, Freiberg GDR
      most recently of U Karlsruhe
          04.01.2002

 Trace header fields accessed: ns

\end{verbatim}
\pagebreak
\begin{verbatim}
 SURAMP - Linearly taper the start and/or end of traces to zero.	

 suramp <stdin >stdout [optional parameters]				

 Required parameters:							
 	if dt is not set in header, then dt is mandatory		

 Optional parameters							
	tmin=tr.delrt/1000	end of starting ramp (sec)		
	tmax=(nt-1)*dt		beginning of ending ramp (sec)		
 	dt = (from header)	sampling interval (sec)			

 The taper is a linear ramp from 0 to tmin and/or tmax to the		
 end of the trace.  Default is a no-op!				


 Credits:

	CWP: Jack K. Cohen, Ken Larner 

 Trace header fields accessed: ns, dt, delrt

\end{verbatim}
\pagebreak
\begin{verbatim}
 SUTAPER - Taper the edge traces of a data panel to zero.	


 sutaper <stdin >stdout [optional parameters]		  

 Optional Parameters:					  
 ntr=tr.ntr	number of traces. If tr.ntr is not set, then	
 		ntr is mandatory				
 tr1=0	 number of traces to be tapered at beginning	
 tr2=tr1	number of traces to be tapered at end		
 min=0.		minimum amplitude factor of taper		
 tbeg=0		length of taper (ms) at trace start		
 tend=0		length of taper (ms) at trace end		
 taper=1	taper type					
		 =1 linear (default)			   
		 =2 sine					
		 =3 cosine					
		 =4 gaussian (+/-3.8)			  
		 =5 gaussian (+/-2.0)			  

 Notes:							
   To eliminate the taper, choose tbeg=0. and tend=0. and tr1=0


 Credits:

	CWP: Chris Liner, Jack K. Cohen

 Trace header fields accessed: ns, ntr
 
 Rewrite: Tagir Galikeev, October 2002

\end{verbatim}
\pagebreak
\begin{verbatim}
 SUTXTAPER - TAPER in (X,T) the edges of a data panel to zero.	

 sutxtaper <stdin >stdout [optional parameters]		

 Optional Parameters:                                          
 low=0.    	minimum amplitude factor of taper		
 tbeg=0    	length of taper (ms) at trace start		
 tend=0     	length of taper (ms) at trace end		
 taper=1       taper type                                      
                 =1 linear (default)                           
                 =2 sine                                       
                 =3 cosine                                     
                 =4 gaussian (+/-3.8)                          
                 =5 gaussian (+/-2.0)                          
 key=tr	set key to compute x-domain taper weights	
               default is using internal tracecount (tr)       
 tr1=0         number of traces to be tapered at beg (key=tr)	
 tr2=tr1       number of traces to be tapered at end (key=tr)	

 min=0.	minimum value of key where taper starts (amp=1.)
 max=0.	maximum value of key where taper starts (amp=1.)
 dx=1. 	length of taper (in key units)			
		if key=tr (unset) length is tr1 and (ntr-tr2)	

 Notes:                                                        
   Taper type is used for trace (x-domain) tapering as well 	
   as for time domain tapering.				
   The taper is applied to all traces <tr1 (or key<min) and    
   >tr2 (or key >max) and all time samples <tbeg and >tend. 	
   Taper weights are amp*1 for traces n tr1<n<tr2 and computed	
   for all other traces corresponding to the taper typ.	
   If key is given the taper length is defined by dx, otherwise
   the length of taper is tr1 and (ntr-tr2) respectively.	
   To eliminate the taper, choose tbeg=0. and tend=0. and tr1=0
   If key is set, min,max values take precedence over tr1,tr2.	


 Credits: (based on sutaper)

	CWP: Chris Liner, Jack K. Cohen

 Trace header fields accessed: ns
 
 Rewrite: Tagir Galikeev, October 2002
 Rewrite: Gerald Klein, IFM-GEOMAR, April 2004

\end{verbatim}
\pagebreak
\begin{verbatim}
 SUAMP - output amp, phase, real or imag trace from			
 	(frequency, x) domain data					

 suamp <stdin >stdout mode=amp						

 Required parameters:							
 none									
 Optional parameter:							
 mode=amp	output flag		 				
 		=amp	output amplitude traces				
 		=logamp	output log(amplitude) traces			
 			=phase	output phase traces			
 			=ouphase output unwrapped phase traces (oppenheim)
 			=suphase output unwrapped phase traces (simple)	
 			=real	output real parts			
 	     	=imag	output imag parts	 			
 jack=0	=1  divide value at zero frequency by 2   		
		(operative only for mode=amp)				

 .... phase unwrapping options	..... 					
 unwrap=1	 |dphase| > pi/unwrap constitutes a phase wrapping	
			(operative only for mode=suphase)		
 trend=1	remove linear trend from the unwrapped phase		
 zeromean=0	assume phase(0)=0.0, else assume phase is zero mean	
 smooth=0	apply damped least squares smoothing to unwrapped phase 
 r=10.0	    ... damping coefficient, only active when smooth=1	

 Notes:								
 	The trace returned is half length from 0 to Nyquist. 		

 Example:								
 	sufft <data | suamp >amp_traces					
 Example: 								
	sufft < data > complex_traces					
 	 suamp < complex_traces mode=real > real_traces			
 	 suamp < complex_traces mode=imag > imag_traces			

 Note: the inverse of the above operation is: 				
	suop2 real_traces imag_traces op=zipper > complex_traces	

 Note: Explanation of jack=1 						
 The amplitude spectrum is the modulus of the complex output of	
 the fft. f(0) is thus the average over the range of integration	
 of the transform. For causal functions, or equivalently, half		
 transforms, f(0) is 1/2 of the average over the full range.		
 Most oscillatory functions encountered in wave applications are	
 zero mean, so this is usually not an issue.				

 Note: Phase unwrapping: 						

 The mode=ouphase uses the phase unwrapping method of Oppenheim and	
 Schaffer, 1975. 							
 The mode=suphase generates unwrapped phase assuming that jumps	
 in phase larger than pi/unwrap constitute a phase wrapping.		

 Credits:
	CWP: Shuki Ronen, Jack K. Cohen c.1986

 Notes:
	If efficiency becomes important consider inverting main loop
      and repeating extraction code within the branches of the switch.

 Trace header fields accessed: ns, trid
 Trace header fields modified: ns, trid

\end{verbatim}
\pagebreak
\begin{verbatim}
 SUANALYTIC - use the Hilbert transform to generate an ANALYTIC	
		(complex) trace						

 suanalytic <stdin >sdout 						

 Optional Parameter:							
 phaserot=		phase rotation in degrees of complex trace	

 Notes:								

 The output are complex valued traces. The analytic trace is defined as",  
   ctr[ i ] = indata[i] + i hilb[indata[t]]				
 where the imaginary part is the hilbert tranform of the original trace

 The Hilbert transform is computed in the direct (time) domain		

 If phaserot is set, then a phase rotated complex trace is produced	
   ctr[ i ] = cos[phaserot]*indata[i] + i sin[phaserot]* hilb[indata[t]]

 Use "suamp" to extract real, imaginary, amplitude (modulus), etc 	
 Exmple:								
 suanalytic < sudata | suamp mode=amp | suxgraph 			



 Use "suattributes" for instantaneous phase, frequency, etc.		


 Credits:
    CWP: John Stockwell, based on suhilb by Jack K. Cohen.

 Trace header fields accessed: ns, trid

 Technical references:
 Oppenheim, A. V. and Schafer, R. W. (1999).
     Discrete-Time Signal Processing. Prentice Hall Signal Processing Series.
     Prentice Hall, New Jersey, 2.
 Taner, M. T., F. Koehler, and R. E. Sheriff, 1979, Complex seismic 
    trace analysis: Geophysics, 44, 1041-1063. 


\end{verbatim}
\pagebreak
\begin{verbatim}
 SUCCEPSTRUM - Compute the complex CEPSTRUM of a seismic trace 	"

  sucepstrum < stdin > stdout					   	

 Required parameters:						  	
	none								
 Optional parameters:						  	
 sign1=1		sign of real to complex transform		
 sign2=-1		sign of complex to complex (inverse) transform	

 ...phase unwrapping .....						
 mode=ouphase		Oppenheim's algorithm for phase unwrapping	
			=suphase  simple unwrap phase			
 unwrap=1	 |dphase| > pi/unwrap constitutes a phase wrapping	
			(operative only for mode=suphase)		

 trend=1		deramp the phase, =0 do not deramp the phase	
 zeromean=0		assume phase starts at 0,  =1 phase is zero mean

 Notes:								
 The cepstrum is defined as the fourier transform of the the decibel   
 spectrum, as though it were a time domain signal.			

 CC(t) = FT[ln[T(omega)] ] = FT[ ln|T(omega)| + i phi(omega) ]		
	T(omega) = |T(omega)| exp(i phi(omega))				
       phi(omega) = unwrapped phase of T(omega)			

 Phase unwrapping:							
 The mode=ouphase uses the phase unwrapping method of Oppenheim and	
 Schaffer, 1975, which operates integrating the derivative of the phase

 The mode=suphase generates unwrapped phase assuming that jumps	
 in phase larger than dphase=pi/unwrap constitute a phase wrapping. In this case
 the jump in phase is replaced with the average of the jumps in phase  
 on either side of the location where the suspected phase wrapping occurs.

 In either mode, the user has the option of de-ramping the phase, by   
 removing its linear trend via trend=1 and of deciding whether the 	
 phase starts at phase=0 or is of  zero mean via zeromean=1.		


 Author: John Stockwell, Dec 2010
 			based on sucepstrum.c by:

 Credits:
 Balazs Nemeth of Potash Corporation of Saskatchewan Inc. 
			given to CWP in 2008


\end{verbatim}
\pagebreak
\begin{verbatim}
 SUCCWT - Complex continuous wavelet transform of seismic traces	

 succwt < tdata.su > tfdata.su	[optional parameters]			

 Required Parameters:							
 None									

 Optional Parameters:							
 noct=5	Number of octaves (int)					
 nv=10		Number of voices per octave (int)			
 fmax=Nyq	Highest frequency in transform				
 p=-0.5	Power of scale value normalizing CWT			
		=0 for amp-preserved spec. decomp.			
 c=1/(2*fmax)	Time-domain inverse gaussian damping parameter		
		(bigger c means more wavelet oscillations,		
		default gives minimal oscillations)			
 k=1		Use complex Morlet as wavelet transform kernel		
		=2 use Fourier kernel ... Exp[i 2 pi f t]		
 fs=1		Use dyadic freq sampling (CWT standard, honors		
		noct, nv)						
		=2 use linear freq sampling (Fourier standard)		
 df=1		Frequency sample interval in Hz (used only for fs=2)	
			NOTE: not yet implimented (hardwired to df=1) 	
 dt=(from tr.dt)	Sample interval override (in secs, if time data)
 verbose=0	 Run silent, except echo c value. (=1 for more info)	

 Examples:								
 This generates amplitude spec of the CWT impulse response (IR).	
  suspike ntr=1 ix1=1 nt=125 | succwt | suamp | suximage & 		
 Real part of Fourier IR with linear freq sampling:			
 suspike ntr=1 ix1=1 nt=125 | succwt k=2 fs=2 | suamp mode=real | suximage &
 Real part of Fourier IR with dyadic freq sampling: 			
 suspike ntr=1 ix1=1 nt=125 | succwt k=2 | suamp mode=real | suximage &

 Inverse CWT: (within a constant scale factor)				
	... | succwt p=-1 | suamp mode=real | sustack key=cdp > inv.su	

 Notes:								
 1. Total number of scales: nscale = noct*nv				
 2. Each input trace spawns nscale complex output traces		
 3. Lowest frequency in the transform is fmax/( 2^(noct-1/nv) )	
 4. Header field (cdp) used as cwt spectrum counter			
 5. Header field (cdpt) used as scale counter within cwt spectrum	
 6. Header field (gut) holds number of cwt scales `na'			
 7. Header field (unscale) holds CWT scale `a'				

 Header fields set: tracl, cdp, cdpt, unscale, gut			



 Copyright (c) University of Tulsa, 2003-4.
 All rights reserved.			
 Author:	UTulsa: Chris Liner, SEP: Bob Clapp

 todo:
	fix fs=2 case to allow df not equal to 1
 History:
 6/18/04
	major overhaul by Clapp, including fourier implementation.
	Speedup ~ 41 times	(4100 %)
 2/20/04
	made p=-0.5 default
 2/16/04
	added p option to experiment with CWT normalization
 2/12/04
	replace fb (bandwidth parameter) with c (t-domain gaussian damping const.)
 2/10/04 --- in sync with EAS paper in prep
	changed morlet scaling (c = 1) to preserve time-domain peak amplitude
	changed morlet exp sign to std CWT definition (conjugate) and 
	mathematica result that only gives positive freq gaussian with neg exp
 1/26/04
	added linear frequency sampling option
 1/23/04
	figured out fb and made it a getpar
	key: Look at real ccwt output and determine fb by number of 
		oscillations desired:	Default gives -+-+-+-
 1/20/04
	beefed up verbose output 
	dimension wavelet to length 2*nt and change correlation call
	... this is done to avoid conv edge effects
 1/19/04
	added fourier wavlet option for comparison with Fourier Transform action
 1/17/04
	complex morlet amp scaling now set to preserve first scale amp with IR 
 1/16/04
	added dt getpar to handle depth input properly
	preserves first tracl so tracl is ok after spice
 11/11/03
	initial version

 Trace header fields set: tracl, cdp, cdpt, unscale, gut

\end{verbatim}
\pagebreak
\begin{verbatim}
 SUCEPSTRUM - transform to the CEPSTRal domain				

  sucepstrum <stdin >sdout sign1=1 					

 Required parameters:							
 none									

 Optional parameters:							
 sign1=1			sign in exponent of fft			
 sign2=-1			sign in exponent of ifft		
 dt=from header		sampling interval			
 verbose=1			=0 to stop advisory messages		

 .... phase unwrapping options .....				   	
 mode=ouphase	Oppenheim's phase unwrapping				
		=suphase simple jump detecting phase unwrapping		
 unwrap=1       |dphase| > pi/unwrap constitutes a phase wrapping	
 	 	=0 no phase unwrapping	(in mode=suphase  only)		
 trend=1	remove linear trend from the unwrapped phase	   	
 zeromean=0     assume phase(0)=0.0, else assume phase is zero mean	
 smooth=0      apply damped least squares smoothing to unwrapped phase 
 r=10     ... damping coefficient, only active when smooth=1           

 Notes:								
 The complex log fft of a function F(t) is given by:			
 clogfft(F(t)) = log(FFT(F(t)) = log|F(omega)| + iphi(omega)		
 where phi(omega) is the unwrapped phase. Note that 0< unwrap<= 1.0 	
 allows phase unwrapping to be tuned, if desired. 			

 The ceptrum is the inverse Fourier transform of the log fft of F(t) 	
 F(t_c) =cepstrum(F(t)) = INVFFT[log(FFT(F(t))]			
                        =INVFFT[log|F(omega)| + iphi(omega)]		
 Here t_c is the cepstral time domain. 				

 To facilitate further processing, the sampling interval		
 in quefrency and first quefrency (0) are set in the			
 output header.							

 Caveats: 								
 No check is made that the data ARE real time traces!			

 Use suminphase to make minimum phase representations of signals 	

 Credits:
      CWP: John Stockwell, June 2013 based on
	sufft by:
	CWP: Shuki Ronen, Chris Liner, Jack K. Cohen
	CENPET: Werner M. Heigl - added well log support
	U Montana: Bob Lankston - added m_unwrap_phase feature

 Note: leave dt set for later inversion

 Trace header fields accessed: ns, dt, d1, f1
 Trace header fields modified: ns, d1, f1, trid

\end{verbatim}
\pagebreak
\begin{verbatim}
 SUCLOGFFT - fft real time traces to complex log frequency domain traces

 suclogftt <stdin >sdout sign=1 					

 Required parameters:							
 none									

 Optional parameters:							
 sign=1			sign in exponent of fft			
 dt=from header		sampling interval			
 verbose=1		=0 to stop advisory messages			

 .... phase unwrapping options .....				   	
 mode=suphase	simple jump detecting phase unwrapping			
 		=ouphase  Oppenheim's phase unwrapping			
 unwrap=1       |dphase| > pi/unwrap constitutes a phase wrapping	
 	 	=0 no phase unwrapping	(in mode=suphase  only)		
 trend=1	remove linear trend from the unwrapped phase	   	
 zeromean=0     assume phase(0)=0.0, else assume phase is zero mean	

 Notes:								
 clogfft(F(t)) = log(FFT(F(t)) = log|F(omega)| + iphi(omega)		
 where phi(omega) is the unwrapped phase. Note that 0< unwrap<= 1.0 	
 allows phase unwrapping to be tuned, if desired. 			

 To facilitate further processing, the sampling interval		
 in frequency and first frequency (0) are set in the			
 output header.							

 suclogfft unwrap=0 | suiclogfft is not quite a no-op since the trace	
 length will usually be longer due to fft padding.			

 Caveats: 								
 No check is made that the data ARE real time traces!			

 Output is type complex. To view amplitude, phase or real, imaginary	
 parts, use    suamp 							
 PI/unwrap = minimum dphase is assumed to constitute a wrap in phase	
 for suphase unwrapping only 					

 Examples: 								
 suclogfft < stdin | suamp mode=real | .... 				
 suclogfft < stdin | suamp mode=imag | .... 				

 The real and imaginary parts of the complex log spectrum are the	
 respective amplitude and phase (unwrapped) phase spectra of the input	
 signal. 								

 Example:  Homomorphic wavelet estimation 				
 suclogfft < shotgather | suamp mode=real | sustack key=dt > real.su	
 suclogfft < shotgather | suamp mode=imag | sustack key=dt > imag.su	
 suop2 real.su imag.su op=zipper | suiclogfft | suminphase > wavelet.su




 Credits:
      CWP: John Stockwell, Dec 2010 based on
	sufft by:
	CWP: Shuki Ronen, Chris Liner, Jack K. Cohen
	CENPET: Werner M. Heigl - added well log support
	U Montana: Bob Lankston - added m_unwrap_phase feature

 Note: leave dt set for later inversion

 Trace header fields accessed: ns, dt, d1, f1
 Trace header fields modified: ns, d1, f1, trid

\end{verbatim}
\pagebreak
\begin{verbatim}
 SUCWT - generates Continous Wavelet Transform amplitude, regularity	
         analysis in the wavelet basis					

     sucwt < stdin [Optional parameters ] > stdout			

 Required Parameters:							
 none									

 Optional Parameters:							
 base=10	Base value for wavelet transform scales			
 first=-1	First exponent value for wavelet transform scales	
 expinc=0.01	Exponent increment for wavelet transform scales		
 last=1.5	Last exponent value for wavelet transform scales	

 Wavelet Parameters:							
 wtype=0		2nd derivative of Gaussian (Mexican hat)	
			=1 4th derivative of Gaussian (witch's hat)	
			=2 6th derivative of Gaussian (wizard's hat)	
 nwavelet=1024		number of samples in the wavelet		
 xmin=-20		minimum x value wavelet is computed		
 xcenter=0		center x value  wavelet is computed 		
 xmax=20		maximum x value wavelet is computed		
 sigma=1		sharpness parameter ( sigma > 1 sharper)	

 verbose=0		silent, =1 chatty				
 holder=0		=1 compute Holder regularity estimate		
 divisor=1.0		a floating point number >= 1.0 (see notes)	

 Notes: 								
 This is the CWT version of the time frequency analysis notion that is 
 applied in sugabor.							
 The parameter base is the base of the power that is applied to scale	
 the wavelet. Some mathematical literature assume base 2. Base 10 works
 well here.								

 Default option yields an output similar to that of sugabor. With the  
 parameter holder=1 an estimate of the instantaneous Holder regularity 
 (the Holder exponent) is output for each input data value. The result 
 is a Holder exponent trace for each corresponding input data trace.	

 The strict definition of the Holder exponent is the maximum slope of  
 the rise of the spectrum in the log(amplitude) versus log(scale) domain:

 divisor=1.0 means the exponent is computed simply by fitting a line   
 through all of the values in the transform. A value of divisor>1.0    
 indicates that the Holder exponent is determined as the max of slopes 
 found in (total scales)/divisor length segments.			

 Some experimentation with the parameters nwavelet, first, last, and   
 expinc may be necessary before a desirable output is obtained. The	
 most effective way to proceed is to perform a number of tests with    
 holder=0 to determine the range of first, last, and expinc that best  
 represents the data in the wavelet domain. Then experimentation with  
 holder=1 and values of divisor>=1.0 may proceed.			



 Credits: 
	CWP: John Stockwell, Nov 2004
 inspired in part by "bhpcwt" in the BHP_SU package, code written by
	BHP: Michael Glinsky,	c. 2002, based loosely on a Matlab CWT function

 References: 
         
 Li C.H., (2004), Information passage from acoustic impedence to
 seismogram: Perspectives from wavelet-based multiscale analysis, 
 Journal of Geophysical Research, vol. 109, B07301, p.1-10.
         
 Mallat, S. and  W. L. Hwang, (1992),  Singularity detection and
 processing with wavelets,  IEEE Transactions on information, v 38,
 March 1992, p.617 - 643.
         


\end{verbatim}
\pagebreak
\begin{verbatim}
 SUFFT - fft real time traces to complex frequency traces		

 suftt <stdin >sdout sign=1 						

 Required parameters:							
 none									

 Optional parameters:							
 sign=1			sign in exponent of fft			
 dt=from header		sampling interval			
 verbose=1		=0 to stop advisory messages			

 Notes: To facilitate further processing, the sampling interval	
 in frequency and first frequency (0) are set in the			
 output header.							

 sufft | suifft is not quite a no-op since the trace			
 length will usually be longer due to fft padding.			

 Caveats: 								
 No check is made that the data IS real time traces!			

 Output is type complex. To view amplitude, phase or real, imaginary	
 parts, use    suamp 							

 Examples: 								
 sufft < stdin | suamp mode=amp | .... 				
 sufft < stdin | suamp mode=phase | .... 				
 sufft < stdin | suamp mode=uphase | .... 				
 sufft < stdin | suamp mode=real | .... 				
 sufft < stdin | suamp mode=imag | .... 				


 Credits:

	CWP: Shuki Ronen, Chris Liner, Jack K. Cohen
	CENPET: Werner M. Heigl - added well log support

 Note: leave dt set for later inversion

 Trace header fields accessed: ns, dt, d1, f1
 Trace header fields modified: ns, d1, f1, trid

\end{verbatim}
\pagebreak
\begin{verbatim}
 SUGABOR -  Outputs a time-frequency representation of seismic data via
	        the Gabor transform-like multifilter analysis technique 
		presented by Dziewonski, Bloch and  Landisman, 1969.	

    sugabor <stdin >stdout [optional parameters]			

 Required parameters:					 		
	if dt is not set in header, then dt is mandatory		

 Optional parameters:							
	dt=(from header)	time sampling interval (sec)		
	fmin=0			minimum frequency of filter array (hz)	
	fmax=NYQUIST 		maximum frequency of filter array (hz)	
	beta=3.0		ln[filter peak amp/filter endpoint amp]	
	band=.05*NYQUIST	filter bandwidth (hz) 			
	alpha=beta/band^2	filter width parameter			
	verbose=0		=1 supply additional info		
	holder=0		=1 output Holder regularity estimate	
				=2 output linear regularity estimate	

 Notes: This program produces a muiltifilter (as opposed to moving window)
 representation of the instantaneous amplitude of seismic data in the	
 time-frequency domain. (With Gaussian filters, moving window and multi-
 filter analysis can be shown to be equivalent.)			

 An input trace is passed through a collection of Gaussian filters	
 to produce a collection of traces, each representing a discrete frequency
 range in the input data. For each of these narrow bandwidth traces, a 
 quadrature trace is computed via the Hilbert transform. Treating the narrow
 bandwidth trace and its quadrature trace as the real and imaginary parts
 of a "complex" trace permits the "instantaneous" amplitude of each
 narrow bandwidth trace to be compute. The output is thus a representation
 of instantaneous amplitude as a function of time and frequency.	

 Some experimentation with the "band" parameter may necessary to produce
 the desired time-frequency resolution. A good rule of thumb is to run 
 sugabor with the default value for band and view the image. If band is
 too big, then the t-f plot will consist of stripes parallel to the frequency
 axis. Conversely, if band is too small, then the stripes will be parallel
 to the time axis. 							

 Caveat:								
 The Gabor transform is not a wavelet transform, but rather are sharp	
 frame basis. However, it is nearly a Morlet continuous wavelet transform
 so the concept of Holder regularity may have some meaning. If you are	
 computing Holder regularity of, say, a migrated seismic section, then
 set band to 1/3 of the frequency band of your data.			

 Examples:								
    suvibro | sugabor | suximage					
    suvibro | sugabor | suxmovie n1= n2= n3= 				
     (because suxmovie scales it's amplitudes off of the first panel,  
      may have to experiment with the wclip and bclip parameters	
    suvibro | sugabor | supsimage | ... ( your local PostScript utility)


 Credits:

	CWP: John Stockwell, Oct 1994
      CWP: John Stockwell Oct 2004, added holder=1 option
 Algorithm:

 This programs takes an input seismic trace and passes it
 through a collection of truncated Gaussian filters in the frequency
 domain.

 The bandwidth of each filter is given by the parameter "band". The
 decay of these filters is given by "alpha", and the number of filters
 is given by nfilt = (fmax - fmin)/band. The result, upon inverse
 Fourier transforming, is that nfilt traces are created, with each
 trace representing a different frequency band in the original data.

 For each of the resulting bandlimited traces, a quadrature (i.e. pi/2
 phase shifted) trace is computed via the Hilbert transform. The 
 bandlimited trace constitutes a "complex trace", with the bandlimited
 trace being the "real part" and the quadrature trace being the 
 "imaginary part".  The instantaneous amplitude of each bandlimited
 trace is then computed by computing the modulus of each complex trace.
 (See Taner, Koehler, and Sheriff, 1979, for a discussion of complex
 trace analysis.

 The final output for a given input trace is a map of instantaneous
 amplitude as a function of time and frequency.

 This is not a wavelet transform, but rather a redundant frame
 representation.

 References: 	Dziewonski, Bloch, and Landisman, 1969, A technique
		for the analysis of transient seismic signals,
		Bull. Seism. Soc. Am., 1969, vol. 59, no.1, pp.427-444.

		Taner, M., T., Koehler, F., and Sheriff, R., E., 1979,
		Complex seismic trace analysis, Geophysics, vol. 44,
		pp.1041-1063.

 		Chui, C., K.,1992, Introduction to Wavelets, Academic
		Press, New York.

 Trace header fields accessed: ns, dt, trid, ntr
 Trace header fields modified: tracl, tracr, d1, f2, d2, trid, ntr

\end{verbatim}
\pagebreak
\begin{verbatim}
 SUHILB - Hilbert transform					

 suhilb <stdin >sdout 						

 Note: the transform is computed in the direct (time) domain   


 Credits:
	CWP: Jack Cohen   
      CWP: John Stockwell, modified to use Dave Hale's hilbert() subroutine

 Trace header fields accessed: ns, trid

\end{verbatim}
\pagebreak
\begin{verbatim}
 SUICEPSTRUM - fft of complex log frequency traces to real time traces

  suicepstrum <stdin >sdout sign2=-1				

 Required parameters:						
 	none							

 Optional parameter:						
 	sign1=1		sign in exponent of first fft		
 	sign2=-1	sign in exponent of inverse fft		
	sym=0		=1 center  output 			
	dt=tr.dt	time sampling interval (s) from header	
			if not set assumed to be .004s		
 Output traces are normalized by 1/N where N is the fft size.	

 Note:								
 The forward  cepstral transform is the			
   F(t_c) = InvFFT[ln[FFT(F(t))]] 				
 The inverse  cepstral transform is the			
   F(t) = InvFFT[exp[FFT(F(t_c))]] 				

 Here t_c is the cepstral time (quefrency) domain 		

 Credits:
 
   CWP: John Stockwell, Dec 2010 based on
     suifft.c by:
	CWP: Shuki Ronen, Chris Liner, Jack K. Cohen,  c. 1989

 Trace header fields accessed: ns, trid
 Trace header fields modified: ns, trid

\end{verbatim}
\pagebreak
\begin{verbatim}
 SUICLOGFFT - fft of complex log frequency traces to real time traces

  suiclogftt <stdin >sdout sign=-1				

 Required parameters:						
 	none							

 Optional parameter:						
 	sign=-1		sign in exponent of inverse fft		
	sym=0		=1 center  output 			
 Output traces are normalized by 1/N where N is the fft size.	

 Note:								
 Nominally this is the inverse to the complex log fft, but	
 suclogfft | suiclogfft is not quite a no-op since the trace	
 	length will usually be longer due to fft padding.	


 Example:  Homomorphic wavelet estimation                              
 suclogfft < shotgather | suamp mode=real | sustack key=dt > real.su   
 suclogfft < shotgather | suamp mode=imag | sustack key=dt > imag.su   
 suop2 real.su imag.su op=zipper | suiclogfft | suminphase > wavelet.su




 Credits:
 
   CWP: John Stockwell, Dec 2010 based on
     suifft.c by:
	CWP: Shuki Ronen, Chris Liner, Jack K. Cohen,  c. 1989

 Trace header fields accessed: ns, trid
 Trace header fields modified: ns, trid

\end{verbatim}
\pagebreak
\begin{verbatim}
 SUIFFT - fft complex frequency traces to real time traces	

 suiftt <stdin >sdout sign=-1					

 Required parameters:						
 	none							

 Optional parameter:						
 	sign=-1		sign in exponent of inverse fft		

 Output traces are normalized by 1/N where N is the fft size.	

 Note: sufft | suifft is not quite a no-op since the trace	
 	length will usually be longer due to fft padding.	


 Credits:

	CWP: Shuki, Chris, Jack

 Trace header fields accessed: ns, trid
 Trace header fields modified: ns, trid

\end{verbatim}
\pagebreak
\begin{verbatim}
 SUMINPHASE - convert input to minimum phase				

 suminphase <stdin >stdout [optional parameters]	 		

 Required parameters:					 		
	if dt is not set in header, then dt is mandatory		

 Optional parameters:							
	sign1=1		sign of first transform	(1 or -1)		
	sign2=-1	sign of second transform (-1 or 1)		
    	pnoise=1.e-9	   white noise in spectral routine		
	verbose=0		=1 for advisory messages		


 Example:  Homomorphic wavelet estimation                              

 suclogfft < shotgather | suamp mode=real | sustack key=dt > real.su   
 suclogfft < shotgather | suamp mode=imag | sustack key=dt > imag.su   
 suop2 real.su imag.su op=zipper | suiclogfft | suminphase > wavelet.su


 Credits:
      SEAM Project: Bruce VerWest c. 2013
 
 Trace header fields accessed: ns, dt, d1

\end{verbatim}
\pagebreak
\begin{verbatim}
 SUPHASEVEL - Multi-mode PHASE VELocity dispersion map computed
              from shot record(s)				

 suphasevel <infile >outfile [optional parameters]		

 Optional parameters:						
 fv=330	minimum phase velocity (m/s)			
 nv=100	number of phase velocities			
 dv=25		phase velocity step (m/s)			
 fmax=50	maximum frequency to process (Hz)		
		=0 process to nyquist				
 norm=0	do not normalize by amplitude spectrum		
		=1 normalize by amplitude spectrum		
 verbose=0	verbose = 1 echoes information			

 Notes:  Offsets read from headers.			 	
  1. output is complex frequency data				
  2. offset header word must be set (signed offset is ok)	
  3. norm=1 tends to blow up aliasing and other artifacts	
  4. For correct suifft later, use fmax=0			
  5. For later processing outtrace.dt=domega			
  6. works for 2D or 3D shots in any offset order		


 Using this program:						

 First: use 							
 	suspecfx < shotrecord.su | suximage			
 to see what the maximum bandwidth is in your data. This will	
 give you an idea about the possible value for fmax.		

 Second: Plot your data or some subset of your data via:	
     suxwigb < shotrecord.su key=offset			

 You can then estimate the range of phase velocities by looking
 at the maximum and minimum slopes of arrivals in your data.	
 This will allow you do set first velocity fv and the increment
 in velocity dv, that make sense for your data.		

 You can pick values of offset and time by placing the cursor  
 on the desired location on the plot and pressing the \'s\' key
 The picks will appear in your terminal window. 		

 When displaying, don't forget to use suamp to compute the	
 modulus of the complex values that this program puts out.	

   suphasevel < shotrecord.su [parameters] | suamp | suximage	



 Credits:

	UHouston: Chris Liner June2008 (cloned from suspecfk)

  This code implements the following integral transform
             _
            /
  u(w,v) = / k(w,x,v) u(w,x) dx
         _/
  where
	u(w,v) is the phase velocity dispersion image
	k(w,x,v) is the transform kernel.... exp(-i w x / v)
	u(w,x) = FT[u(t,x)] is the input shot record(s) 
         _/
 Reference: Park, Miller, and Xia (1998, SEG Abstracts)

 Trace header fields accessed: dt, offset, ns
 Trace header fields modified: nx,dt,trid,d1,f1,d2,f2,tracl

\end{verbatim}
\pagebreak
\begin{verbatim}
 SURADON - compute forward or reverse Radon transform or remove multiples
           by using the parabolic Radon transform to estimate multiples
           and subtract.						

     suradon <stdin >stdout [Optional Parameters]			

 Optional Parameters:							
 choose=0    0  Forward Radon transform				
             1  Compute data minus multiples				
             2  Compute estimate of multiples				
             3  Compute forward and reverse transform			
             4  Compute inverse Radon transform			
 igopt=1     1  parabolic transform: g(x) = offset**2			
             2  Foster/Mosher psuedo hyperbolic transform		
                   g(x) = sqrt(depth**2 + offset**2)			
             3  Linear tau-p: g(x) = offset				
             4  abs linear tau-p: g(x) = abs(offset)			
 offref=2000.    reference maximum offset to which maximum and minimum	
                 moveout times are associated				
 interoff=0.     intercept offset to which tau-p times are associated	
 pmin=-200       minimum moveout in ms on reference offset		
 pmax=400        maximum moveout in ms on reference offset		
 dp=16           moveout increment in ms on reference offset		
 pmula=80        moveout in ms on reference offset where multiples begin
                     at maximum time					
 pmulb=200       moveout in ms on reference offset where multiples begin
                     at zero time					
 depthref=500.   Reference depth for Foster/Mosher hyperbolic transform
 nwin=1          number of windows to use through the mute zone	
 f1=60.          High-end frequency before taper off			
 f2=80.          High-end frequency					
 prewhite=0.1    Prewhitening factor in percent.			
 cdpkey=cdp      name of header word for defining ensemble		
 offkey=offset   name of header word with spatial information		
 nxmax=240       maximum number of input traces per ensemble		
 ltaper=7	  taper (integer) for mute tapering function		

 Optimizing Parameters:						
 The following parameters are occasionally used to avoid spatial aliasing
 problems on the linear tau-p transform.  Not recommended for other	
 transforms...								
 ninterp=0      number of traces to interpolate between each input trace
                   prior to computing transform			
 freq1=4.0      low-end frequency in Hz for picking (good default: 3 Hz)
                (Known bug: freq1 cannot be zero) 
 freq2=20.0     high-end frequency in Hz for picking (good default: 20 Hz)
 lagc=400       length of AGC operator for picking (good default: 400 ms)
 lent=5         length of time smoother in samples for picker		
                     (good default: 5 samples)				
 lenx=7         length of space smoother in samples for picker		
                     (good default: 1 sample)				
 xopt=1         1 = use differences for spatial derivative		
                        (works with irregular spacing)			
                0 = use FFT derivative for spatial derivatives		
                      (more accurate but requires regular spacing and	
                      at least 16 input tracs--will switch to differences
                      automatically if have less than 16 input traces)	


 Credits:
	CWP: John Anderson (visitor to CSM from Mobil) Spring 1993

 Multiple removal notes:
	Usually the input data are NMO corrected CMP gathers.  The
	first pass is to compute a parabolic Radon transform and
 	identify the multiples in the transform domain.  Then, the
 	module is run on all the data using "choose=1" to estimate
 	and subtract the multiples.  See the May, 1993 CWP Project
	Review for more extensive documentation.

 NWIN notes:
	The parabolic transform runs with higher resolution if the
 	mute zone is honored.  When "nwin" is specified larger than
   	one (say 6), then multiple windows are used through the mute
 	zone.  It is assumed in this case that the input data are
 	sorted by the offkey header item from small offset to large
 	offset.  This causes the code to run 6 times longer.  The
      mute time is taken from the "muts" header word.
      You may have to manually set this header field yourself, if
      it is not already set.

 References:
 Anderson, J. E., 1993, Parabolic and linear 2-D, tau-p transforms
       using the generalized radon tranform, in May 11-14, 1993
       Project Review, Consortium Project on Seismic Inverse methods
       for Complex Structures, CWP-137, Center for Wave Phenomena
       internal report.
 Other References cited in above paper:
 Beylkin, G,.1987, The discrete Radon transform: IEEE Transactions
       of Acoustics, Speech, and Signal Processing, 35, 162-712.
 Chapman, C.H.,1981, Generalized Radon transforms and slant stacks:
       Geophysical Journal of the Royal Astronomical Society, 66,
       445-453.
 Foster, D. J. and Mosher, C. C., 1990, Multiple supression
       using curvilinear Radon transforms: SEG Expanded Abstracts 1990,
       1647-1650.
 Foster, D. J. and Mosher, C. C., 1992, Suppression of multiples
       using the Radon transform: Geophysics, 57, No. 3, 386-395.
 Gulunay, N., 1990, F-X domain least-squares Tau-P and Tau-Q: SEG
       Expanded Abstracts 1990, 1607-1610.
 Hampson, D., 1986, Inverse velocity stacking for multiple elimination:
       J. Can. Soc. Expl. Geophs., 22, 44-55.
 Hampson, D., 1987, The discrete Radon transform: a new tool for image
       enhancement and noise suppression: SEG Expanded Abstracts 1978,
       141-143.
 Johnston, D.E., 1990, Which multiple suppression method should I use?
       SEG Expanded Abstracts 1990, 1750-1752.

 Trace header words accessed: ns, dt, cdpkey, offkey, muts

\end{verbatim}
\pagebreak
\begin{verbatim}
 SUSLOWFT - Fourier Transforms by a (SLOW) DFT algorithm (Not an FFT)

 suslowft <stdin >sdout sign=1 				

 Required parameters:						
 	none							

 Optional parameters:						
 	sign=1			sign in exponent of fft		
 	dt=from header		sampling interval		

 Trace header fields accessed: ns, dt				
 Trace header fields modified: ns, dt, trid			

 Notes: To facilitate further processing, the sampling interval
       in frequency and first frequency (0) are set in the	
	output header.						
 Warning: This program is *not* fft based. Use only for demo 	
 	   purposes, *not* for large data processing.		

 	No check is made that the data are real time traces!	
 suslowft | suslowift is not quite a no-op since the trace     
 length will usually be longer due to fft padding.             

 Caveats:                                                              
 No check is made that the data IS real time traces!                   

 Output is type complex. To view amplitude, phase or real, imaginary   
 parts, use    suamp                                                   

 Examples:                                                             
 suslowft < stdin | suamp mode=amp | ....                                 
 suslowft < stdin | suamp mode=phase | ....                               
 suslowft < stdin | suamp mode=real | ....                                
 suslowft < stdin | suamp mode=imag | ....                                



 Credits:

	CWP: Shuki, Chris, Jack

 Note: leave dt set for later inversion


\end{verbatim}
\pagebreak
\begin{verbatim}
 SUSLOWIFT - Fourier Transforms by (SLOW) DFT algorithm (Not an FFT)
             complex frequency to real time domain traces 	

 suslowift <stdin >sdout sign=-1 				

 Required parameters:						
 	none							

 Optional parameters:						
 	sign=-1			sign in exponent of fft		
 	dt=from header		sampling interval		

 Trace header fields accessed: ns, dt				
 Trace header fields modified: ns, dt, trid			

 Notes: To facilitate further processing, the sampling interval
       in frequency and first frequency (0) are set in the	
	output header.						

 Warning: This program is *not* fft based. Use only for demo 	
 	   purposes, *not* for large data processing.		

 	No check is made that the data are real time traces!	


 Credits:

	CWP: John Stockwell c. 1993
       based on suifft:   Shuki Ronen, Chris Liner, Jack K. Cohen

 Note: leave dt set for later inversion


\end{verbatim}
\pagebreak
\begin{verbatim}
 SUSPECFK - F-K Fourier SPECtrum of data set			

 suspecfk <infile >outfile [optional parameters]		

 Optional parameters:						

 dt=from header		time sampling interval		
 dx=from header(d2) or 1.0	spatial sampling interval	

 verbose=0	verbose = 1 echoes information			

 tmpdir= 	 if non-empty, use the value as a directory path
		 prefix for storing temporary files; else if the
	         the CWP_TMPDIR environment variable is set use	
	         its value for the path; else use tmpfile()	

 Note: To facilitate further processing, the sampling intervals
       in frequency and wavenumber as well as the first	
	frequency (0) and the first wavenumber are set in the	
	output header (as respectively d1, d2, f1, f2).		

 Note: The relation: w = 2 pi F is well known, but there	
	doesn't	seem to be a commonly used letter corresponding	
	to F for the spatial conjugate transform variable.  We	
	use K for this.  More specifically we assume a phase:	
		i(w t - k x) = 2 pi i(F t - K x).		
	and F, K define our notion of frequency, wavenumber.	


 Credits:

	CWP: Dave (algorithm), Jack (reformatting for SU)

 Trace header fields accessed: ns, dt, d2
 Trace header fields modified: tracl, ns, dt, trid, d1, f1, d2, f2

\end{verbatim}
\pagebreak
\begin{verbatim}
 SUSPECFX - Fourier SPECtrum (T -> F) of traces 		

 suspecfx <infile >outfile 					

 Note: To facilitate further processing, the sampling interval	
       in frequency and first frequency (0) are set in the	
	output header.						


 Credits:

	CWP: Dave (algorithm), Jack (reformatting for SU)

 Trace header fields accessed: ns, dt
 Trace header fields modified: ns, dt, trid, d1, f1

\end{verbatim}
\pagebreak
\begin{verbatim}
 SUSPECK1K2 - 2D (K1,K2) Fourier SPECtrum of (x1,x2) data set		

 suspeck1k2 <infile >outfile [optional parameters]			

 Optional parameters:							

 d1=from header(d1) or 1.0	spatial sampling interval in first (fast)
				   dimension				
 d2=from header(d2) or 1.0	spatial sampling interval in second	
				 (slow)  dimension			

 verbose=0		verbose = 1 echoes information			

 tmpdir= 	 	if non-empty, use the value as a directory path
		 	prefix for storing temporary files; else if the
	         	the CWP_TMPDIR environment variable is set use	
	         	its value for the path; else use tmpfile()	

 Notes:								
 Because the data are assumed to be purely spatial (i.e. non-seismic), 
 the data are assumed to have trace id (30), corresponding to (z,x) data

 To facilitate further processing, the sampling intervals in wavenumber
 as well as the first frequency (0) and the first wavenumber are set in
 the output header (as respectively d1, d2, f1, f2).			

 The relation: w = 2 pi F is well known for frequency, but there	
 doesn't seem to be a commonly used letter corresponding to F for the	
 spatial conjugate transform variables.  We use K1 and K2 for this.	
 More specifically we assume a phase:					
		-i(k1 x1 + k2 x2) = -2 pi i(K1 x1 + K2 x2).		
 and K1, K2 define our respective wavenumbers.				


 Credits:
     CWP: John Stockwell, 26 April 1995, based on original code by
          Dave Hale and Jack Cohen	

 Trace header fields accessed: ns, d1, d2, trid
 Trace header fields modified: tracl, ns, dt, trid, d1, f1, d2, f2

\end{verbatim}
\pagebreak
\begin{verbatim}
 SUTAUP - forward and inverse T-X and F-K global slant stacks		

    sutaup <infile >outfile  [optional parameters]                 	

 Optional Parameters:                                                  
 option=1			=1 for forward F-K domian computation	
				=2 for forward T-X domain computation	
				=3 for inverse F-K domain computation	
				=4 for inverse T-X domain computation	
 dt=tr.dt (from header) 	time sampling interval (secs)           
 nx=ntr   (counted from data)	number of horizontal samples (traces)	
 dx=1				horizontal sampling interval (m)	
 npoints=71			number of points for rho filter		
 pmin=0.0			minimum slope for Tau-P transform (s/m)	
 pmax=.006			maximum slope for Tau-P transform (s/m)	
 np=nx				number of slopes for Tau-P transform	
 ntau=nt			number of time samples in Tau-P domain  
 fmin=3			minimum frequency of interest 	        
 xmin=0			offset on first trace	 	        

 verbose=0	verbose = 1 echoes information				

 tmpdir= 	 if non-empty, use the value as a directory path	
		 prefix for storing temporary files; else if the	
	         the CWP_TMPDIR environment variable is set use		
	         its value for the path; else use tmpfile()		

 Notes:                                                                
 The cascade of a forward and inverse  tau-p transform preserves the	
 relative amplitudes in a data panel, but not the absolute amplitudes  
 meaning that a scale factor must be applied to data output by such a  
 a cascade before the output may be compared to the original data.	
 This is a characteristic of the algorithm employed in this program.	
 (Suradon does not have this problem.)					



 Credits: CWP: Gabriel Alvarez, 1995.

 Reference:       
    Levin, F., editor, 1991, Slant-Stack Processing, Geophysics Reprint 
         Series #14, SEG Press, Tulsa.

 Trace header fields accessed: ns, dt
 Trace header fields modified: dt,d2,f2

\end{verbatim}
\pagebreak
\begin{verbatim}
 SUWFFT - Weighted amplitude FFT with spectrum flattening 0->Nyquist	

 suwfft <stdin | suifft >sdout 					

 Required parameters:							
 none									

 Optional parameters:							
 w0=0.75		weight for AmpSpectrum[f-df]			
 w1=1.00		weight for AmpSpectrum[f].. center value	
 w2=0.75		weight for AmpSpectrum[f+df]			

 Notes: 								
 1. output format is same as sufft					
 2. suwfft | suifft is not quite a no-op since the trace		
    length will usually be longer due to fft padding.			
 3. using w0=0 w1=1 w2=0  gives truly flat spectrum, for other	        
    weight choices the spectrum retains some of its original topograpy 

 Examples: 								
 1. boost data bandwidth to 10-90 Hz					
     suwfft < data.su | suifft | sufilter f=5,8,90,100 | suximage 	
 1. view amplitude spectrum after flattening				
     suwfft < data.su | suamp | suximage 				

 Caveat: The process of cascading the forward and inverse Fourier	
  transforms may result in the output trace length being greater than 	
  the input trace length owing to zero padding. The user may wish to	
  apply suwind to return the number of samples per trace to the original
  value:  Here NS is the number of samples per trace on the original data
  			... | suwind itmax=NS | ... 			

 Credits:

	UHouston: Chris Liner 

 Note: Concept from UTulsa PhD thesis of Bassel Al-Moughraby

 Trace header fields accessed: ns, dt
 Trace header fields modified: ns, d1, f1, trid

\end{verbatim}
\pagebreak
\begin{verbatim}
 SUZEROPHASE - convert input to zero phase equivalent			

 suzerophase <stdin >stdout [optional parameters]	 		

 Required parameters:					 		
	if dt is not set in header, then dt is mandatory		

 Optional parameters:							
    	t0=1.0			time of peak value t0 			
	verbose=0		=1 for advisory messages		



 Credits:
	c. 2011 Bruce VerWest as part of the SEAM project
 
 Trace header fields accessed: ns, dt, d1

\end{verbatim}
\pagebreak
\begin{verbatim}
 SURELANAN - REsiduaL-moveout semblance ANalysis for ANisotropic media	

 surelan refl= npicks=    [optional parameters]			

 Required parameters:							
 reflector file: reflec =						
 number of points in the reflector file =				

 Optional Parameters:							
 nr1=51		number of r1-parameter samples   		
 dr1=0.01              r1-parameter sampling interval			
 fr1=-0.25             first value of r1-parameter			
 nr2=51		number of r2-parameter samples   		
 dr2=0.01              r2-parameter sampling interval			
 fr2=-0.25             first value of r2-parameter			
 dzratio=5             ratio of output to input depth sampling intervals
 nsmooth=dzratio*2+1   length of semblance num and den smoothing window
 verbose=0             =1 for diagnostic print on stderr		
 method=linear		for linear interpolation of the interface       
 			=mono for monotonic cubic interpolation of interface
 			=akima for Akima's cubic interpolation of interface 
 			=spline for cubic spline interpolation of interface 

 Note: 								
 1. This program is part of Debashish Sarkar's anisotropic model building
 technique. 								
 2. Input migrated traces should be sorted by cdp - surelan outputs a 	
    group of semblance traces every time cdp changes.  Therefore, the  
    output will be useful only if cdp gathers are input.  		
 3. The residual-moveout semblance for cdp gathers is based		
	on z(h)*z(h) = z(0)*z(0) + r1*h^2 + r2*h^4/[h^2+z(0)^2] where z 
	depth and h is the half-offset.   				

\end{verbatim}
\pagebreak
\begin{verbatim}
 SURELAN - compute residual-moveout semblance for cdp gathers based	
	on z(h)*z(h) = z(0)*z(0) + r*h*h where z depth and h offset.	

 surelan <stdin >stdout   [optional parameters]			

 Optional Parameters:							
 nr=51			number of r-parameter samples   		
 dr=0.01               r-parameter sampling interval			
 fr=-0.25               first value of b-parameter			
 smute=1.5             samples with RMO stretch exceeding smute are zeroed
 dzratio=5             ratio of output to input depth sampling intervals
 nsmooth=dzratio*2+1   length of semblance num and den smoothing window
 verbose=0             =1 for diagnostic print on stderr		

 Note: 								
 1. This program is part of Zhenyue Liu's velocity analysis technique.	
 2. Input migrated traces should be sorted by cdp - surelan outputs a 	
    group of semblance traces every time cdp changes.  Therefore, the  
    output will be useful only if cdp gathers are input.  		
 3. The parameter r may take negative values. The range of r can be 	
     controlled by maximum of (z(h)*z(h)-z(0)*z(0))/(h*h)   		
\end{verbatim}
\pagebreak
\begin{verbatim}
  SUTIVEL -  SU Transversely Isotropic velocity table builder		
	computes vnmo or vphase as a function of Thomsen's parameters and
	theta and optionally interpolate to constant increments in slowness

 Optional Parameters:							
 a=2500.		alpha (vertical p velocity)			
 b=1250.		beta (vertical sv velocity)			
 e=.20			epsilon (horiz p-wave anisotropy)		
 d=.10			delta (strange parameter)			
 maxangle=90.0		max angle in degrees				
 nangle=9001		number of angles to compute			
 verbose=0		set to 1 to see full listing			
 np=8001		number of slowness values to output		
 option=1		1=output vnmo(p) (result used for TI DMO)	
			2=output vnmo(theta) in degrees			
			3=output vnmo(theta) in radians			
			4=output vphase(p)				
			5=output vphase(theta) in degrees		
			6=output vphase(theta) in radians		
			7=output first derivative vphase(p)		
			8=output first derivative vphase(theta) in degrees
			9=output first derivative vphase(theta) in radians
			10=output second derivative vphase(p)		
			11=output second derivative vphase(theta) in degrees
			12=output second derivative vphase(theta) in radians
			13=( 1/vnmo(0)^2 -1/vnmo(theta)^2 )/p^2 test vs theta
			   (result should be zero for all theta for d=e)
			14=return vnmo(p) for weak anisotropy		
 normalize=0		=1 means scale vnmo by cosine and scale vphase by
 			    1/sqrt(1+2*e*sin(theta)*sin(theta)		
	 		   (only useful for vphase when d=e for constant
				result)					
			=0 means output vnmo or vphase unnormalized	

 Output on standard output is ascii text with:				
 line   1: number of values						
 line   2: abscissa increment (p or theta increment, always starts at zero)
 line 3-n: one value per line						



 Author: (visitor to CSM form Mobil) John E. Anderson, Spring 1994

\end{verbatim}
\pagebreak
\begin{verbatim}
 SUVEL2DF - compute stacking VELocity semblance for a single time in   
			    over Vnmo and eta in 2-D			

    suvel2df <stdin >stdout [optional parameters]			

 Required Parameters:							
 tn			zero-offset time of reflection			
 offsetm		Maximum offset considerd			

 Optional Parameters:							
 nv=50			number of velocities				
 dv=50.0		velocity sampling interval			
 fv=1500.0		first velocity					
 nvh=50		number of horizotal velocities			
 dvh=50.0		horizontal velocity sampling interval		
 fvh=1500.0		first horizontal velocity			
 xod=1.5		maximum offset-to-depth ratio to resolve	
 dtratio=5		ratio of output to input time sampling intervals
 nsmooth=dtratio*2+1	length of semblance num and den smoothing window
 verbose=0		=1 for diagnostic print on stderr		
 vavg=fv+0.5*(nv-1)*dv   average velocity used in the search		

 Notes:								
 Semblance is defined by the following quotient:			

		 n-1		 					
		[ sum q(t,j) ]^2					
		 j=0		 					
	s(t) = ------------------					
		 n-1		 					
		n sum [q(t,j)]^2					
		 j=0		 					

 where n is the number of non-zero samples after muting.		
 Smoothing (nsmooth) is applied separately to the numerator and denominator
 before computing this semblance quotient.				

 Input traces should be sorted by cdp - suvel2df outputs a group of	
 semblance traces every time cdp changes.  Therefore, the output will	
 be useful only if cdp gathers are input.				


 Credits:
	CWP: Tariq Alkhalifah,  February 1997
 Trace header fields accessed:  ns, dt, delrt, offset, cdp.
 Trace header fields modified:  ns, dt, offset.

\end{verbatim}
\pagebreak
\begin{verbatim}
 SUVELAN_NCCS - compute stacking VELocity panel for cdp gathers	     
		using Normalized CrossCorrelation Sum 	                     

 suvelan_uccs <stdin >stdout [optional parameters]			     

 Optional Parameters:							     
 nx=tr.cdpt              number of traces in cdp			     
 nv=50                   number of velocities				     
 dv=50.0                 velocity sampling interval			     
 fv=1500.0               first velocity				     
 smute=1.5               samples with NMO stretch exceeding smute are zeroed
 dtratio=5               ratio of output to input time sampling intervals   
 nsmooth=dtratio*2+1     length of smoothing window                         
 verbose=0               =1 for diagnostic print on stderr		     
 pwr=1.0                 semblance value to the power      		     

 Notes:								     
 Normalized CrossCorrelation sum: sum all possible crosscorrelation	     
 trace pairs in a CMP gather for each trial velocity and zero-offset        
 two-way travel time inside a time window. This coherence measure is        
 normalized by dividing each crosscorrelation trace pair by the geometric   
 mean of the energy, inside the chosen time window, of each trace pair      
 involved in each crosscorrelation. Then, to achieve a maximum amplitude    
 of unity, the result is multiplied by  2/(M(M-1)), which is the inverse    
 of the total number of crosscorrelation. The normalization allows to	     
 bring out weak reflection as long as these reflections have moveouts close 
 to a hyperbola.							     


 
 Credits:  

 CWP:  Valmore Celis, Sept 2002	

 Based on the original code: suvelan.c 
    Colorado School of Mines:  Dave Hale, c. 1989

 Trace header fields accessed:  ns, dt, delrt, offset, cdp, cdpt 
 Trace header fields modified:  ns, dt, offset, cdp

 Reference: Neidell, N.S., and Taner, M.T., 1971, Semblance and 
             other coherency measures for multichannel data: 
             Geophysics, 36, 498-509. 


\end{verbatim}
\pagebreak
\begin{verbatim}
 SUVELAN_NSEL - compute stacking VELocity panel for cdp gathers	     
		using the Normalized Selective CrossCorrelation sum	     

 suvelan_usel <stdin >stdout [optional parameters]			     

 Optional Parameters:							     
 nx=tr.cdpt              number of traces in cdp			     
 dx=tr.d2 	          offset increment				     
 nv=50                   number of velocities				     
 dv=100.0                velocity sampling interval			     
 fv=1500.0               first velocity				     
 tau=0.5                 threshold for significance values                  
 smute=1.5               samples with NMO stretch exceeding smute are zeroed
 dtratio=5               ratio of output to input time sampling intervals   
 nsmooth=dtratio*2+1     length of smoothing window                         
 verbose=0               =1 for diagnostic print on stderr		     
 pwr=1.0                 semblance value to the power      		     

 Notes:								     
 Normalized Selective CrossCorrelation Sum: is based on the coherence       
 measure known as crosscorrelation sum. The difference is that the selective
 approach sum only crosscorrelation pairs with relatively large differential
 moveout, thus increasing the resolving power in the velocity spectra       
 compared to that achieved by conventional methods. The normalization is    
 achieved in much the same way of normalizing the conventional              
 crosscorrelation sum.						             

 Each crosscorrelation is divided by the geometric mean		     
 of the energy of the traces involved, and the multiplying by a constant to 
 achieve maximum amplitude of unity. The constant is just the inverse of the
 total number of crosscorrelations included in the sum.  The selection is   
 made using a parabolic approximation of the differential moveout and       
 imposing a threshold for those differential moveouts.		   	     

 That threshold is the parameter tau in this program, which varies between 0
 to 1.	 A value of tau=0, means conventional crosscorrelation sum is applied
 implying that all crosscorrelations are included in the sum. In contrast,  
 a value of tau=1 (not recomended) means that only the crosscorrelation     
 formed by the trace pair involving the shortest and longest offset is      
 included in the sum. Intermediate values will produce percentages of the   
 crosscorrelations included in the sum that will be shown in the screen     
 before computing the velocity spectra. Typical values for tau are between  
 0.2 and 0.6, producing approximated percentages of crosscorrelations summed
 between 60% and 20%. The higher the value of tau the lower the percentage
 and higher the increase in the resolving power of velocity spectra.        

 Keeping the percentage of crosscorrelations included in the sum between 20%
 and 60% will increase resolution and avoid the precense of artifacts in   
 the results.  In data contaminated by random noise or statics distortions   
 is recomended to mantaing the percentage of crosscorrelations included in   
 the sum above 25%. After computing the velocity spectra one might want to  
 adjust the level  and number of contours before velocity picking.	      

 
 Credits: CWP:  Valmore Celis, Sept 2002	
 
 Based on the original code: suvelan.c 
    Colorado School of Mines:  Dave Hale c. 1989

 References: 
 Neidell, N.S., and Taner, M.T., 1971, Semblance and other 
   coherency measures for multichannel data: Geophysics, 36, 498-509.
 Celis, V. T., 2002, Selective-correlation velocity analysis: CSM thesis.


 Trace header fields accessed:  ns, dt, delrt, offset, cdp
 Trace header fields modified:  ns, dt, offset, cdp

\end{verbatim}
\pagebreak
\begin{verbatim}
 SUVELAN - compute stacking velocity semblance for cdp gathers		     

 suvelan <stdin >stdout [optional parameters]				     

 Optional Parameters:							     
 nv=50                   number of velocities				     
 dv=50.0                 velocity sampling interval			     
 fv=1500.0               first velocity				     
 anis1=0.0               quartic term, numerator of an extended quartic term
 anis2=0.0               in denominator of an extended quartic term         
 smute=1.5               samples with NMO stretch exceeding smute are zeroed
 dtratio=5               ratio of output to input time sampling intervals   
 nsmooth=dtratio*2+1     length of semblance num and den smoothing window   
 verbose=0               =1 for diagnostic print on stderr		     
 pwr=1.0                 semblance value to the power      		     

 Notes:								     
 Velocity analysis is usually a two-dimensional screen for optimal values of
 the vertical two-way traveltime and stacking velocity. But if the travel-  
 time curve is no longer close to a hyperbola, the quartic term of the      
 traveltime series should be considered. In its easiest form (with anis2=0) 
 the optimizion of all parameters requires a three-dimensional screen. This 
 is done by a repetition of the conventional two-dimensional screen with a  
 variation of the quartic term. The extended quartic term is more accurate, 
 though the function is no more a polynomial. When screening for optimal    
 values the theoretical dependencies between these paramters can be taken   
 into account. The traveltime function is defined by                        

                1            anis1                                          
 t^2 = t_0^2 + --- x^2 + ------------- x^4                                  
               v^2       1 + anis2 x^2                                      

 The coefficients anis1, anis2 are assumed to be small, that means the non- 
 hyperbolicity is assumed to be small. Triplications cannot be handled.     

 Semblance is defined by the following quotient:			     

                 n-1                 					     
               [ sum q(t,j) ]^2      					     
                 j=0                 					     
       s(t) = ------------------     					     
                 n-1                 					     
               n sum [q(t,j)]^2      					     
                 j=0                 					     

 where n is the number of non-zero samples after muting.		     
 Smoothing (nsmooth) is applied separately to the numerator and denominator 
 before computing this semblance quotient.				     

 Then, the semblance is set to the power of the parameter pwr. With pwr > 1 
 the difference between semblance values is stretched in the upper half of  
 the range of semblance values [0,1], but compressed in the lower half of   
 it; thus, the few large semblance values are enhanced. With pwr < 1 the    
 many small values are enhanced, thus more discernible against background   
 noise. Of course, always at the expanse of the respective other feature.   

 Input traces should be sorted by cdp - suvelan outputs a group of	     
 semblance traces every time cdp changes.  Therefore, the output will	     
 be useful only if cdp gathers are input.				     

 Credits:
	CWP, Colorado School of Mines:
           Dave Hale (everything except ...)
           Bjoern Rommel (... the quartic term)
      SINTEF, IKU Petroleumsforskning
           Bjoern Rommel (... the power-of-semblance function)

 Trace header fields accessed:  ns, dt, delrt, offset, cdp
 Trace header fields modified:  ns, dt, offset, cdp

\end{verbatim}
\pagebreak
\begin{verbatim}
 SUVELAN_UCCS - compute stacking VELocity panel for cdp gathers	     
		using UnNormalized CrossCorrelation Sum 	             

 suvelan_uccs <stdin >stdout [optional parameters]			     

 Optional Parameters:							     
 nx=tr.cdpt              number of traces in cdp 			     
 nv=50                   number of velocities				     
 dv=50.0                velocity sampling interval			     
 fv=1500.0               first velocity				     
 smute=1.5               samples with NMO stretch exceeding smute are zeroed
 dtratio=5               ratio of output to input time sampling intervals   
 nsmooth=dtratio*2+1     length of smoothing window                         
 verbose=0               =1 for diagnostic print on stderr		     
 pwr=1.0                 semblance value to the power      		     

Notes:									     
 Unnormalized crosscorrelation sum: sum all possible crosscorrelation trace 
 pairs in a CMP gather for each trial velocity and zero-offset two-way      
 travel time inside a time window. This unnormalized coherency measure      
 produces large spectral amplitudes for strong reflections and small        
 spectral amplitudes for weaker ones. If M is the number of traces in the   
 CMP gather M(M-1)/2 is the total number of crosscorrelations for each trial
 velocity and zero-offset two-way traveltime.			 	     

 
 Credits: CWP: Valmore Celis, Sept 2002	
 
 Based on the original code: suvelan.c 
    Colorado School of Mines:  Dave Hale c. 1989


 Reference: Neidell, N.S., and Taner, M.T., 1971, Semblance and other 
            coherency measures for multichannel data: Geophysics, 36, 498-509.

 Trace header fields accessed:  ns, dt, delrt, offset, cdp
 Trace header fields modified:  ns, dt, offset, cdp

\end{verbatim}
\pagebreak
\begin{verbatim}
 SUVELAN_USEL - compute stacking velocity panel for cdp gathers	     
		using the UnNormalized Selective CrossCorrelation Sum	     

 suvelan_usel <stdin >stdout [optional parameters]			     

 Optional Parameters:							     
 nx=tr.cdpt              number of traces in cdp			     
 dx=tr.d2                offset increment				     
 nv=50                   number of velocities				     
 dv=50.0                 velocity sampling interval			     
 fv=1500.0               first velocity				     
 tau=0.5                 threshold for significance values                  
 smute=1.5               samples with NMO stretch exceeding smute are zeroed
 dtratio=5               ratio of output to input time sampling intervals   
 nsmooth=dtratio*2+1     length of smoothing window                         
 verbose=0               =1 for diagnostic print on stderr		     
 pwr=1.0                 semblance value to the power      		     

 Notes:								     
 UnNormalized Selective CrossCorrelation sum: is based on the coherence     
 measure known as crosscorrelation sum. The difference is that the selective
 approach sum only crosscorrelation pairs with relatively large differential
 moveout, thus increasing the resolving power in the velocity spectra       
 compared to that achieved by conventional methods.  The selection is made  
 using a parabolic approximation of the differential moveout and imposing a 
 threshold for those differential moveouts.				     

 That threshold is the parameter tau in this program, which varies between  
 0 to 1.  A value of tau=0, means conventional crosscorrelation sum is      
 applied implying that all crosscorrelations are included in the sum. In    
 contrast, a value of tau=1 (not recomended) means that only the            
 crosscorrelation formed by the trace pair involving the shortest and longest
 offset is included in the sum. Intermediate values will produce percentages
 of the crosscorrelations included in the sum that will be shown in the     
 screen before computing the velocity spectra. Typical values for tau are   
 between 0.2 and 0.6, producing approximated percentages of crosscorrelations
 summed between 60% and 20%. The higher the value of tau the lower the     
 percentage and higher the increase in the resolving power of velocity	      
 spectra.								      

 Keeping the percentage of crosscorrelations included in the sum between 20%
 and 60% will increase resolution and avoid the precense of artifacts in the
 results.  In data contaminated by random noise or statics distortions is    
 recomended to mantaing the percentage of crosscorrelations included in the  
 sum above 25%.  After computing the velocity spectra one might want to     
 adjust the level and number of contours before velocity picking.  	      

 
 Credits:  CWP: Valmore Celis, Sept 2002
 
 Based on the original code: suvelan_.c 
    Colorado School of Mines:  Dave Hale c. 1989


 References: 
 Neidell, N.S., and Taner, M.T., 1971, Semblance and other coherency
             measures for multichannel data: Geophysics, 36, 498-509.
 Celis, V. T., 2002, Selective-correlation velocity analysis: CSM thesis.

 Trace header fields accessed:  ns, dt, delrt, offset, cdp
 Trace header fields modified:  ns, dt, offset, cdp

\end{verbatim}
\pagebreak
\begin{verbatim}
 LAS2SU - convert las2 format well log curves to su traces	

  las2su <stdin nskip= ncurve= >stdout [optional params]	

 Required parameters:						
 none								
 Optional parameters:						
 ncurve=automatic	number of log curves in the las file	
 dz=0.5		input depth sampling (ft)		
 m=0			output (d1,f1) in feet			
			=1 output (d1,f1) in meters		
 ss=0			do not subsample (unless nz > 32767 )	
			=1 pass every other sample		
 verbose=0		=1 to echo las header lines to screen	
 outhdr=las_header.asc	name of file for las headers		

 Notes:							
 1. It is recommended to run LAS_CERTIFY available from CWLS	
    at http://cwls.org.					
 2. First log curve MUST BE depth.				
 3. If number of depth levels > 32767 (segy NT limit)		
    then input is subsampled by factor of 2 or 4 as needed	
 4. Logs may be isolated using tracl header word (1,2,...,ncurve) 
    tracl=1 is depth						

 If the input LAS file contains sonic data as delta t or interval
 transit time and you plan to use these data for generating a 
 reflection coefficient time series in suwellrf, convert the sonic
 trace to velocity using suop with op=s2v (sonic to velocity) 
 before input of the velocity trace to suwellrf.		", 

 Caveat:							", 
 No trap is made for the commonly used null value in LAS files 
 (-999.25). The null value will be output as ?999.25, which	
 really messes up a suxwigb display of density data because the
 ?999.25 skews the more or less 2.5 values of density.		
 The user needs to edit out null values (-999.25) before running
 other programs, such as "suwellrf".				


 Credits:
  *	Chris Liner based on code by Ferhan Ahmed and a2b.c (June 2005)
  *            (Based on code by Ferhan Ahmed and a2b.c)
  *            I gratefully acknowledge Saudi Aramco for permission
  *            to release this code developed while I worked for the
  *            EXPEC-ARC research division.
  *	CWP: John Stockwell 31 Oct 2006, combining lasstrip and
  *	CENPET: lasstrip 2006 by Werner Heigl
  *
  *     Rob Hardy: allow the ncurve parameter to work correctly if set
  *    - change string length to 400 characters to allow more curves
  *    - note nskip in header is totally ignored !
  *
  * Ideas for improvement:
  *	add option to chop off part of logs (often shallow
  *	   portions are not of interest
  *	cross check sampling interval from header against
  *	   values found from first log curve (=depth)
  *
 

\end{verbatim}
\pagebreak
\begin{verbatim}
 SUBACKUSH - calculate Thomsen anisotropy parameters from 	
 	     well log (vp,vs,rho) data and optionally include	
 	     intrinsic VTI shale layers based on gramma ray log	
 	     via BACKUS averaging				
 subackush < vp_vs_rho.su >stdout [options]			
 subackush < vp_vs_rho_gr.su  gr=1 >stdout [options]		

 Required parameters:						
 none								

 Optional parameter:						
 navg=101	number of depth samples in Backus avg window 	

 	Intrinsic anisotropy of shale layers can be included ...
 gr=0		no gamma ray log input for shale 		
		=1 input is vp_vs_rho_gr			
 grs=100	pure shale gamma ray value (API units)		
 grc=10	0% shale gamma ray value (API units)		
 smode=1	include shale anis params prop to shale volume 	
		=0 include shale anis for pure shale only	
 se=0.209	shale epsilon (Thomsen parameter)		
 sd=0.033	shale delta (Thomsen parameter)			
 sg=0.203	shale gamma (Thomsen parameter)			

 Notes:							
 1. Input are (vp,vs,rho) traces in metric units		
 2. Output are  						
    tracl	=(1,2,3,4,5,6)					
    quantity	=(vp0,vs0,<rho>,epsilon,delta,gamma) 		
    units	=(m/s,m/s,kg/m^3,nd,nd,nd) nd=dimensionless	
    tracl	=(7,8)						
    quantity	=(Vsh,shaleEps) Vsh=shale volume fraction	
    units	=(nd,nd) 					
 3. (epsilon,delta,etc.) can be isolated by tracl header field 
 4. (vp0,vs0) are backus averaged vertical wavespeeds		
 5. <rho> is backus averaged density, etc.			

 Example:							
 las2su < logs.las nskip=34 nlog=4 > logs.su 			
 suwind < logs.su  key=tracl min=2 max=3 | suop op=s2vm > v.su	
 suwind < logs.su  key=tracl min=4 max=4 | suop op=d2m > d.su	
 fcat v.su d.su > vp_vs_rho.su					
 subackus < vp_vs_rho.su > vp0_vs0_rho_eps_delta_gamma.su	
 In this example we start with a well las file containing 	
 34 header lines and 4 log tracks (depth,p_son,s_son,den).	
 This is converted to su format by las2su.  Then we pull off	
 the sonic logs and convert them to velocity in metric units.	
 Then the density log is pulled off and converted to metric.	
 All three metric curves are bundled into one su file which 	
 is the input to subackus. 					", 

 Related programs: subackus, sulprime				


 Credits:

	UHouston: Chris Liner 
              I gratefully acknowledge Saudi Aramco for permission
              to release this code developed while I worked for the 
              EXPEC-ARC research division.

 References:
 Anisotropy parameters: Thomsen, 2002, DISC Notes (SEG)
 Backus Method: Berryman, Grechka, and Berge, 1997, SEP94
 Shale params: Wang, 2002, Geophysics, p. 1427	

\end{verbatim}
\pagebreak
\begin{verbatim}
 SUBACKUS - calculate Thomsen anisotropy parameters from 	
 	     well log (vp,vs,rho) data via BACKUS averaging	

 subackus < vp_vs_rho.su >stdout [options]			

 Required parameters:						
 none								

 Optional parameter:						
 navg=201	number of depth samples in Backus avg window 	
 all=0		=1 to output extra parameters 			
		(<vp0>,<vs0>,<rho>,eta,vang,a,f,c,l,A,B,<lam>,<mu>)
 ang=30	angle (deg) for use in vang			

 Notes:							
 1. Input are (vp,vs,rho) traces in metric units		
 2. Output are (epsilon,delta,gamma) dimensionless traces	
    tracl=(1,2,3)=(epsilon,delta,gamma) 			
	(all=1 output optional traces listed below)		
    tracl=(4,5,6,7,8)=(vp0,vs0,<rho>,eta,vang)			
    tracl=(9,10,11,12)=(a,f,c,l)=(c11,c13,c33,c44) backus avg'd
    tracl=(13,14)=(<lam/lamp2mu>^2,4<mu*lampmu/lamp2mu>)=(A,B)	
       used to analyze eps=(a-c)/2c; a=c11=A*x+B;  c=c33=x	
    tracl=(15,16)=(<lambda>,<mu>)				
       for fluid analysis (lambda affected by fluid, mu not)   
    tracl=(17,18,19)=(vp,vs,rho)  orig log values		
    tracl=(20)=(m)=(c66) Backus avg'd 				
    tracl=(21,22,23,24,25)=(a,f,c,l,m)=(c11,c13,c33,c44,c66) orig
 3. (epsilon,delta,etc.) can be isolated by tracl header field 
 4. (vp0,vs0) are backus averaged vertical wavespeeds		
 5. <rho> is backus averaged density, etc.			
 6. eta = (eps - delta) / (1 + 2 delta)			
 7. vang=vp(ang_deg)/vp0  phase velocity ratio			
    The idea being that if vang~vp0 there are small time effects
    (30 deg comes from ~ max ray angle preserved in processing)

 Example:							
 las2su < logs.las nskip=34 nlog=4 > logs.su 			
 suwind < logs.su  key=tracl min=2 max=3 | suop op=s2vm > v.su	
 suwind < logs.su  key=tracl min=4 max=4 | suop op=d2m > d.su	
 fcat v.su d.su > vp_vs_rho.su					
 subackus < vp_vs_rho.su > eps_delta_gamma.su			
 In this example we start with a well las file containing 	
 34 header lines and 4 log tracks (depth,p_son,s_son,den).	
 This is converted to su format by las2su.  Then we pull off	
 the sonic logs and convert them to velocity in metric units.	
 Then the density log is pulled off and converted to metric.	
 All three metric curves are bundled into one su file which 	
 is the input to subackus. 					", 

 Related codes: sulprime subackush				

 Credits:

	UHouston: Chris Liner 
              I gratefully acknowledge Saudi Aramco for permission
              to release this code developed while I worked for the 
              EXPEC-ARC research division.
 References:		
 Anisotropy parameters: Thomsen, 2002, DISC Notes (SEG)
 Backus Method: Berryman, Grechka, and Berge, 1997, SEP94


\end{verbatim}
\pagebreak
\begin{verbatim}
 SUGASSMAN - Model reflectivity change with rock/fluid properties	

	sugassman [optional parameters] > data.su			

 Optional parameters:							
 nt=500 	number of time samples					
 ntr=200	number of traces					
 dt=0.004 	time sampling interval in seconds			
 mode=0	model isolated gassmann refl coefficient		
		=1 embed gassmann RC in random RC series		
		=2 R0 parameter sensitivity output			
 p=0.15 	parameter sensitivity test range (if mode=2)		
 .... Environment variables ...					
 temp=140 	Temperature in degrees C				
 pres=20 	Pressure in megaPascals					
 .... Caprock variables ....						
 v1=37900 	caprock P-wave speed (m/s)				
 r1=44300 	caprock mass density (g/cc)				
 .... Reservoir fluid variables ....					
 g=0.56	Gas specific gravity 0.56 (methane)-1.8 (condensate)	
 api=50 	Gas specific gravity 10 (heavy)-50 (ultra light)	
 s=35		Brine salinity in ppm/(1000 000				
 so=.7 	Oil saturation (0-1)					
 sg=.2 	Gas saturation (0-1)					
 .... Reservoir rock frame variables ....				
 kmin=37900 	Bulk modulus (MPa) of rock frame mineral(s) [default=quartz]
 mumin=44300 	Shear modulus (MPa) of rock frame mineral(s) [default=quartz]
 rmin=2.67 	Mass density (g/cc) of rock frame mineral(s) [default=quartz]
 phi=.24 	Rock frame porosity (0-1)				
 a=1 		Fitting parameters: Mdry/Mmineral ~ 1/(a + b phi^c)	
 b=15 		... where M is P-wave modulus and defaults are for	
 c=1 		... Glenn sandstone [see Liner (2nd ed, table 26.2)]	
	h=20 			Reservoir thickness (m)			

 Notes:								
 Creates a reflection coefficient series based on Gassmann		
 theory of velocity and density for porous elastic media		

 
 Credits: UHouston: Chris Liner	9/23/2009

 trace header fields set: 

\end{verbatim}
\pagebreak
\begin{verbatim}
 SULPRIME - find appropriate Backus average length for  	
 		a given log suite, frequency, and purpose		

 sulprime < vp_vs_rho.su  [options]		or		
 sulprime < vp_vs_rho_gr.su   [options]			

 Required parameters:						
 none								

 Optional parameter:						
 b=2.0		target value of Backus number		 	
		b=2 is transmission limit (ok for proc, mig, etc.)
		b=0.3 is scattering limit (ok for modeling)	
 dz=1		input depth sample interval (ft)		 
 f=60		frequency (Hz)... dominant or max (to be safe) 	
 nmax=301	maximum averaging length (samples)		
 verbose=1	print intermediate results			
		=0 print final result only			

 Notes:							
 1. input is in sync with subackus, but vp and gr not used	
     (gr= gamma ray log)					
 Related codes:  subackus, subackush				

 
 Credits:
	UHouston: Chris Liner Sept 2008
              I gratefully acknowledge Saudi Aramco for permission
              to release this code developed while I worked for the
              EXPEC-ARC research division.
 Reference:			
     The Backus Number (Liner and Fei, 2007, TLE)


\end{verbatim}
\pagebreak
\begin{verbatim}
 SUWELLRF - convert WELL log depth, velocity, density data into a	
	uniformly sampled normal incidence Reflectivity Function of time

 suwellrf [required parameters] [optional parameters] > [stdout]	

 Required Parameters:							
 dvrfile=	file containing depth, velocity, and density values	
 ...or...								
 dvfile=	file containing depth and velocity values		
 drfile=	file containing depth and density values		
 ...or...								
 dfile=	file containing depth values				
 vfile=	file containing velocity log values			
 rhofile=	file containing density log values			
 nval= 	number of triplets of d,v,r values if dvrfile is set,	
 		number of pairs of d,v and d,r values dvfile and drfile	
		are set, or number of values if dfile, vfile, and rhofile
		are set.						

 Optional Parameters:							
 dtout=.004	desired time sampling interval (sec) in output		
 ntr=1         number of traces to output 				

 Notes:								
 The format of the input file(s) is C-style binary float. These files	
 may be constructed from ascii file via:   				

       a2b n1=3 < dvrfile.ascii > dvrfile.bin				
 ...or...								
       a2b n1=2 < dvfile.ascii > dvfile.bin				
       a2b n1=2 < drfile.ascii > drfile.bin				
 ...or...								
       a2b n1=1 < dfile.ascii > dfile.bin				
       a2b n1=1 < vfile.ascii > dfile.bin				
       a2b n1=1 < rhofile.ascii > rhofile.bin				

 A raw normal-incidence impedence reflectivity as a function of time is
 is generated using the smallest two-way traveltime implied by the	
 input velocities as the time sampling interval. This raw reflectivity	
 trace is then resampled to the desired output time sampling interval	
 via 8 point sinc interpolation. If the number of samples on the output
 exceeds SU_NFLTS the output trace will be truncated to that value.	

 Caveat: 								
 This program is really only a first rough attempt at creating a well	
 log utility. User input and modifications are welcome.		

 See also:  suresamp 							



 Author:  CWP: John Stockwell, Summer 2001, updated Summer 2002.
 inspired by a project by GP grad student Leo Brown

\end{verbatim}
\pagebreak
\begin{verbatim}
 SUCOMMAND - pipe traces having the same key header word to command	

     sucommand <stdin >stdout [Optional parameters]			

 Required parameters:							
 	none								

 Optional parameters: 							
 	verbose=0		wordy output				
 	key=cdp			header key word to pipe on		
 	command="suop nop"    command piped into			
	dir=0		0:  change of header key			
			-1: break in monotonic decrease of header key	
			+1: break in monotonic increase of header key	


Notes:									
 This program permits limited parallel processing capability by opening
 pipes for processes, signaled by a change in key header word value.	



 Credits:
	VT: Matthias Imhof

 Note:
	The "valxxx" subroutines are in su/lib/valpkge.c.  In particular,
      "valcmp" shares the annoying attribute of "strcmp" that
		if (valcmp(type, val, valnew) {
			...
		}
	will be performed when val and valnew are different.


\end{verbatim}
\pagebreak
\begin{verbatim}
 SUGETGTHR - Gets su files from a directory and put them               
             throught the unix pipe. This creates continous data flow.	

  sugetgthr  <stdin >sdout   						

 Required parameters:							

 dir=            Name of directory to fetch data from 			
 	          Every file in the directory is treated as an su file	
 Optional parameters:							
 verbose=0		=1 more chatty					
 vt=0			=1 allows gathers with variable length traces	
 			no header checking is done!			
 ns=			must be specified if vt=1; number of samples to read




segy tr;

int
main(int argc, char **argv)
{
	
	cwp_String dir="";	/* input directory containng the gathers
	char *fname=NULL;	
	char *ffname=NULL;
	
	DIR *dp=NULL;
	struct dirent *d=NULL;
	struct stat __st;
	FILE *fp=NULL;
	int fd=0;
	
	int verbose;
	int vt=0;
	ssize_t nread;
	int ns=0;

	
	/* Initialize
	initargs(argc, argv);
	requestdoc(1);

        /* Get parameters
        MUSTGETPARSTRING("dir", &dir);
       	if (!getparint   ("verbose", &verbose)) verbose = 0;
       	if (!getparint   ("vt", &vt)) vt = 0;
	if(vt)MUSTGETPARINT("ns",&ns); 
        checkpars();
	
	/* Open the directory
	if ((dp = opendir(dir)) == NULL)
		err(" %s directory not found\n",dir);
	
	/* For each file in directory
	while (( d = readdir(dp)) !=NULL) {
		
		fname = ealloc1(strlen(d->d_name)+1,sizeof(char));
		strcpy(fname,d->d_name);
		
		/* Skip . and .. directory entries
		if(strcmp(fname,".") && strcmp(fname,"..")) {		
			ffname = ealloc1(strlen(d->d_name)+strlen(dir)+2,sizeof(char));
			
			/* Create full filename
			sprintf(ffname, "%s/%s",dir,fname);
			if(verbose==1) warn("%s",ffname);
			
			/* get some info from the file
			stat(ffname,&__st);
			if(__st.st_size > 0) {
			
				/* Open the file and read traces into stdout*/
 				if(vt) {
					fd = open(ffname,O_RDONLY|CWP_O_LARGEFILE);
				/*	nread=fread(&tr,(size_t) HDRBYTES,1,fp); 
					nread=read(fd,&tr,(size_t) HDRBYTES); 
					memset((void *) &tr.data[tr.ns], (int) '\0' ,MAX(ns-tr.ns,0)*FSIZE);
				/*	nread+=fread(&tr.data[0],(size_t) tr.ns*FSIZE,1,fp);
					nread+=read(fd,&tr.data[0],(size_t) tr.ns*FSIZE);
				} else {
					fp = efopen(ffname, "r");
					nread=fgettr(fp, &tr);
				}
				do {
					if(vt) { 
						tr.ns=ns;
						fwrite(&tr,ns*FSIZE+HDRBYTES,1,stdout);
					} else {
						puttr(&tr);
					}
					if(vt) {
						/* nread=fread(&tr,(size_t) HDRBYTES,1,fp);
						nread=read(fd,&tr,(size_t) HDRBYTES);
						memset((void *) &tr.data[tr.ns], (int) '\0' ,MAX(ns-tr.ns,0)*FSIZE);
						/* nread+=fread(&tr.data[0],(size_t) tr.ns*FSIZE,1,fp);
						nread+=read(fd,&tr.data[0],(size_t) tr.ns*FSIZE);
					} else {
						nread=fgettr(fp, &tr);
					}
				} while(nread);
				if(vt) close(fd);
				else efclose(fp);
			} else {
				warn(" File %s has zero size, skipped.\n",ffname);
			}
			free1(ffname);
		}
		free1(fname);
		
	}
	closedir(dp);
        return(CWP_Exit());
}
\end{verbatim}
\pagebreak
\begin{verbatim}
SUGPRFB - SU program to remove First Breaks from GPR data		

  sugprfb < radar traces >outfile			  		

 nx=51		number of traces to sum to create pilot trace (odd)	
 fbt=60	length of first break in number of samples		

 Notes:								
 The first fbt samples from nx traces are stacked to form a pilot	
 first break trace, this is fitted to the actual traces by shifting	
 and scaling.		 The nx traces long spatial window is		
 slided along the section and a new pilot traces is formed for each	
 position. The scalers in percent and the time shifts are stored in	
 header words trwf and grnors.					  	

   
 Segy data constans
segy 	tr;				/* SEGY trace

void remove_fb(float *data,float *wavelet,int n,short *scaler,short *shft);

int
main( int argc, char *argv[] )
{
 

	int nx;
	int fbt;
	int nt;
	
	float *stacked=NULL;
	int *nnz=NULL;
	int itr=0;
	
 
	initargs(argc, argv);
   	requestdoc(1);
	
	if (!getparint("nx", &nx)) nx = 51;
	if( !ISODD(nx) ) {
		nx++;
		warn(" nx has been changed to %d to be odd.\n",nx);
	}
	
	if (!getparint("fbt", &fbt)) fbt = 60;
        checkpars();
	
	/* Get info from first trace 
	if (!gettr(&tr))  err("can't get first trace");
	nt = tr.ns;
	
	stacked = ealloc1float(fbt);
	nnz = ealloc1int(fbt);
	memset((void *) nnz, (int) '\0', fbt*ISIZE);
	memset((void *) stacked, (int) '\0', fbt*FSIZE);

	/* read nx traces and stack them
	/* The first trace is already read
	
	{ int i,it;
	  float **tr_b;
	  char  **hdr_b;
	  int NXP2=nx/2;
	  short shft,scaler;
	  
		/* ramp on read the first nx traces and create stack
		
	  	tr_b = ealloc2float(nt,nx);
		hdr_b = (char**)ealloc2(HDRBYTES,nx,sizeof(char));
		
		memcpy((void *) hdr_b[0], (const void *) &tr, HDRBYTES);
		memcpy((void *) tr_b[0], (const void *) &tr.data, nt*FSIZE);
		
		for(i=1;i<nx;i++) {
			gettr(&tr);
			memcpy((void *) hdr_b[i], (const void *) &tr, HDRBYTES);
			memcpy((void *) tr_b[i], (const void *) &tr.data, nt*FSIZE);
		}
		
		for(i=0;i<nx;i++) 
			for(it=0;it<fbt;it++) 
				stacked[it] += tr_b[i][it];
		
		
		for(it=0;it<fbt;it++)
			stacked[it] /=(float)nx;
		
			
		/* filter and write out the first nx/2 +1 traces
		for(i=0;i<NXP2+1;i++) {
			memcpy((void *) &tr, (const void *) hdr_b[i], HDRBYTES);
			memcpy((void *) tr.data, (const void *) tr_b[i], nt*FSIZE);
			
			remove_fb(tr.data,stacked,fbt,&scaler,&shft);
			tr.trwf = scaler;
			tr.grnors = shft;

			puttr(&tr);
			++itr;
		}
		
		/* do the rest of the traces
		gettr(&tr);
		
		do {
			
			/* Update the stacked trace  - remove old
			for(it=0;it<fbt;it++) 
				stacked[it] -= tr_b[0][it]/(float)nx;
				
			/* Bump up the storage arrays
			/* This is not very efficient , but good enough
			{int ib;
				for(ib=1;ib<nx;ib++) {
				    memcpy((void *) hdr_b[ib-1],
					(const void *) hdr_b[ib], HDRBYTES);
				memcpy((void *) tr_b[ib-1],
					(const void *) tr_b[ib], nt*FSIZE);
				}
			}
			
			/* Store the new trace
			memcpy((void *) hdr_b[nx-1], (const void *) &tr, HDRBYTES);
			memcpy((void *) tr_b[nx-1], (const void *) &tr.data, nt*FSIZE);
			
			/* Update the stacked array  - add new
			for(it=0;it<fbt;it++) 
				stacked[it] += tr_b[nx-1][it]/(float)nx;
			
			/* Filter and write out the middle one NXP2+1
			memcpy((void *) &tr, (const void *) hdr_b[NXP2], HDRBYTES);
			memcpy((void *) tr.data, (const void *) tr_b[NXP2], nt*FSIZE);
			
			remove_fb(tr.data,stacked,fbt,&scaler,&shft);
			
			tr.trwf = scaler;
			tr.grnors = shft;
			puttr(&tr);
			++itr;
			
			
		} while(gettr(&tr));

		/* Ramp out - write ot the rest of the traces
		/* filter and write out the last nx/2 traces
		for(i=NXP2+1;i<nx;i++) {
			memcpy((void *) &tr, (const void *) hdr_b[i], HDRBYTES);
			memcpy((void *) tr.data, (const void *) tr_b[i], nt*FSIZE);
			
			remove_fb(tr.data,stacked,fbt,&scaler,&shft);
			
			tr.trwf = scaler;
			tr.grnors = shft;
			puttr(&tr);
			itr++;
		
		}
		
		
	}
		
  
	
	free1float(stacked);
	free1int(nnz);
   	return EXIT_SUCCESS;
}

void remove_fb(float *yp,float *ym,int n,short *scaler,short *shft)
 Find Scale and timeshift 

	Yp = data trace
	Ym = wavelet
	
	F=min(Yp(x) - Ym(x)*a)^2 is a function to minimize for scaler
	
	find shift first with xcorrelation then
	
	solve for a - scaler 
/



{

define RAMPR 5

	void find_p(float *ym,float *yp,float *a,int *b,int n);
	
	int it,ir;
	
	float a=1.0;
	int b=0;
	int ramps;
	float *a_ramp;
	
	ramps=NINT(n-n/RAMPR);
	
	find_p(ym,yp,&a,&b,n);
	
	a_ramp = ealloc1float(n);
	
	for(it=0;it<ramps;it++)
		a_ramp[it]=1.0;
	
	for(it=ramps,ir=0;it<n;it++,ir++) {
		a_ramp[it]=1.0-(float)ir/(float)(n-ramps-1);
	}
	
	
	
	if (b<0) {
		for(it=0;it<n+b;it++) 
			yp[it-b] -=ym[it]*a*a_ramp[it-b];
	} else {
		for(it=0;it<n-b;it++) 
			yp[it] -=ym[it-b]*a*a_ramp[it+b];
	}
	
	*scaler = NINT((1.0-a)*100.0);
	*shft = b;
	
	free1float(a_ramp);
}

void find_p(float *ym,float *yp,float *a,int *b,int n)
{
define NP 1

	float *y,**x;
	int n_s;
	
	int k;
	float *res;
	int jpvt[NP];
	float qraux[NP];
	float work[NP]; 
	
														
	x = ealloc2float(n,1);
	y = ealloc1float(n);
	res = ealloc1float(n);
	
	memcpy((void *) &x[0][0], (const void *) &ym[0], n*FSIZE);
	memset((void *) y, (int) '\0', n*FSIZE);	
	
	/* Solve for shift
 	xcor (n,0,ym,n,0,yp,n,-n/2,y);
	/* pick the maximum
	*b = -max_index(n,y,1)+n/2;

	n_s = n-abs(*b);
 	
	if (*b < 0) {
		memcpy((void *) &x[0][0], (const void *) &ym[*b], n_s*FSIZE);
	} else {
		memcpy((void *) &x[0][*b], (const void *) &ym[0], n_s*FSIZE);
													
	}
	/* Solve for scaler
	sqrst(x, n_s, 1,yp,0.0,a,res,&k,&jpvt[0],&qraux[0],&work[0]);
	
	
	
	free2float(x);
	free1float(res);
	free1float(y);
}
\end{verbatim}
\pagebreak
\begin{verbatim}
 SUKILL - zero out traces					

 sukill <stdin >stdout [optional parameters]			

 Optional parameters:						
	key=trid	header name to select traces to kill	
	a=2		header value identifying tracces to kill
 or								
 	min= 		first trace to kill (one-based)		
 	count=1		number of traces to kill 		

 Notes:							
	If min= is set it overrides selecting traces by header.	


 Credits:
	CWP: Chris Liner, Jack K. Cohen
	header-based trace selection: Florian Bleibinhaus

 Trace header fields accessed: ns

\end{verbatim}
\pagebreak
\begin{verbatim}
 SUMIXGATHERS - mix two gathers					

 sumixgathers file1 file2 > stdout [optional parameters]		

 Required Parameters:							
 ntr=tr.ntr	if ntr header field is not set, then ntr is mandatory	

 Optional Parameters: none						

 Notes: Both files have to be sorted by offset				
 Mixes two gathers keeping only the traces of the first file		
 if the offset is the same. The purpose is to substitute only		
 traces non existing in file1 by traces interpolated store in file2. 	", 

 Example. If file1 is original data file and file 2 is obtained by	
 resampling with Radon transform, then the output contains original  	
 traces with gaps filled						



 Credits:
	Daniel Trad. UBC
 Trace header fields accessed: ns, dt, ntr
 Copyright (c) University of British Columbia, 1999.
 All rights reserved.


\end{verbatim}
\pagebreak
\begin{verbatim}
 SUMUTE - MUTE above (or below) a user-defined polygonal curve with	", 
	   the distance along the curve specified by key header word 	

 sumute <stdin >stdout xmute= tmute= [optional parameters]		

 Required parameters:							
 xmute=		array of position values as specified by	
 			the `key' parameter				
 tmute=		array of corresponding time values (sec)	
 			in case of air wave muting, correspond to 	
 			air blast duration				
  ... or input via files:						
 nmute=		number of x,t values defining mute		
 xfile=		file containing position values as specified by	
 			the `key' parameter				
 tfile=		file containing corresponding time values (sec)	
  ... or via header:							
 hmute=		key header word specifying mute time		

 Optional parameters:							
 key=offset		Key header word specifying trace offset 	
 				=tracl  use trace number instead	
 ntaper=0		number of points to taper before hard		
			mute (sine squared taper)			
 mode=0	   mute ABOVE the polygonal curve			
		=1 to zero BELOW the polygonal curve			
		=2 to mute below AND above a straight line. In this case
		 	xmute,tmute describe the total time length of   
			the muted zone as a function of xmute the slope 
			of the line is given by the velocity linvel=	
	 	=3 to mute below AND above a constant velocity hyperbola
			as in mode=2 xmute,tmute describe the total time
			length of the mute zone as a function of xmute,  
			the velocity is given by the value of linvel=	
 		=4 to mute below AND above a user defined polygonal line
			given by xmute, tmute pairs. The widths in time ", 
			of the muted zone are given by the twindow vector
 linvel=330   		constant velocity for linear or hyperbolic mute	
 tm0=0   		time shift of linear or hyperbolic mute at	
			 \'key\'=0					
 twindow=	vector of mute zone widths in time, operative only in mode=4
  ... or input via file:						
 twfile= 								

 Notes: 								
 The tmute interpolant is extrapolated to the left by the smallest time
 sample on the trace and to the right by the last value given in the	
 tmute array.								

 The files tfile and xfile are files of binary (C-style) floats.	

 In the context of this program "above" means earlier time and	
 "below" means later time (above and below as seen on a seismic section.

 The mode=2 option is intended for removing air waves. The mute is	
 is over a narrow window above and below the line specified by the	
 the line specified by the velocity "linvel". Here the values of     
 tmute, xmute or tfile and xfile define the total time width of the mute.

 If data are spatial, such as the (z-x) output of a migration, then    
 depth values are used in place of times in tmute and tfile. The value 
 of the depth sampling interval is given by the d1 header field	
 You must use the option key=tracl in sumute in this case.		

 Caveat: if data are seismic time sections, then tr.dt must be set. If 
 data are seismic depth sections, then tr.trid must be set to the value
 of TRID_DEPTH and the tr.d1 header field must be set.			
 To find the value of TRID_DEPTH:  					
 type: 								
     sukeyword trid							
	and look for the entry for "Depth-Range (z-x) traces


 Credits:

	SEP: Shuki Ronen
	CWP: Jack K. Cohen, Dave Hale, John Stockwell
	DELPHI: Alexander Koek added airwave mute.
      CWP: John Stockwell added modes 3 and 4
	USBG: Florian Bleibinhaus added hmute + some range checks on mute times
 Trace header fields accessed: ns, dt, delrt, key=keyword, trid, d1
 Trace header fields modified: muts or mute

\end{verbatim}
\pagebreak
\begin{verbatim}
 SUPAD - Pad zero traces						

  supad <stdin >stdout min= max= [optional parameters]			

 Required parameters:							
  min=			trace key start					
  max=			trace key end					

 Optional parameters:							
  key1=ep		panel key 					
  key2=tracf		trace key 					
  key3=trid		flag key					
  val3=2		value assigned to padded traces			
  d=1			trace key spacing				

 Notes:								
  In contrast to most SU codes, supad recognizes panels, or ensembles.	
  If the input consists of several panels, each panel will be padded	
  individually.							
  key1 and key2 are the primary and secondary sort key of the data set.
  The sort order of key1 does not matter at all.			
  The sort order of key2 must be monotonous - if key2 is descending,	
	supply a negative value for the spacing d.			
  Traces with a key2-value outside the min/max range will be lost. 	
  Traces with a key2-value that is not a multiple of the spacing from	
	the min-value (the max-value, if the spacing is negative) will	
	not be lost. Instead, they will shift the series of key2-values.
  By default the dead trace flag will be raised for the padded traces.	
  This should make it easy to remove the zero traces later on, if need be.

 Examples:								
	suplane | supad min=1 max=40 key1=offset key2=tracr | ...	
	... appends eight empty traces.					

	suplane | supad min=1 max=32 key1=offset key2=tracr d=0.5 | ...	
	... inserts a zero trace after each trace (even though the	
	header tracr is integer and cannot properly store the floats)	

	suplane | supad min=1 max=32 | ...				
	... produces an error because the panel and trace key are all 0.


 Credits:
	Florian Bleibinhaus, U Salzburg, Austria

\end{verbatim}
\pagebreak
\begin{verbatim}
 SUPUTGTHR - split the stdout flow to gathers on the bases of given	
 		key parameter. 						

	suputgthr <stdin   dir= [Optional parameters]			

 Required parameters:							
 dir=		Name of directory where to put the gathers		
 Optional parameters: 							
 key=ep		header key word to watch   			
 suffix=".hsu"	extension of the output files			
 verbose=0		verbose = 1 echos information			
 numlength=7		Length of numeric part of filename		

 Notes: 			    					
 The name of the file is constructed from the key parameter. Traces	
 are put into a temporary disk file, and renamed when key parameter	
 changes in the input flow to "key.suffix". The result is that the	
 directory "dir" contains separate files by "key" ensemble. 	",	

 Header field modified:  ntr  to be the number of traces in a given 	
 ensemble.								
 Related programs: sugetgthr, susplit 					

 
 Credits: Balazs Nemeth, Potash Corporation, Saskatoon Saskatchewan
 given to CWP in 2008
 Note:
	The "valxxx" subroutines are in su/lib/valpkge.c.  In particular,
	"valcmp" shares the annoying attribute of "strcmp" that
		if (valcmp(type, val, valnew) {
			...
		}
	will be performed when val and valnew are different.



\end{verbatim}
\pagebreak
\begin{verbatim}
 SUSORT - sort on any segy header keywords			

 susort <stdin >stdout [[+-]key1 [+-]key2 ...]			

 Susort supports any number of (secondary) keys with either	
 ascending (+, the default) or descending (-) directions for 	
 each.  The default sort key is cdp.				

 Note:	Only the following types of input/output are supported	
	Disk input --> any output				
	Pipe input --> Disk output				

 Caveat:  On some Linux systems Pipe input and or output often 
		fails						
	Disk input ---> Disk output is recommended		

 Note: If the the CWP_TMPDIR environment variable is set use	
	its value for the path; else use tmpfile()		

 Example:							
 To sort traces by cdp gather and within each gather		
 by offset with both sorts in ascending order:			

 	susort <INDATA >OUTDATA cdp offset			

 Caveat: In the case of Pipe input a temporary file is made	
	to hold the ENTIRE data set.  This temporary is		
	either an actual disk file (usually in /tmp) or in some	
	implementations, a memory buffer.  It is left to the	
	user to be SENSIBLE about how big a file to pipe into	
	susort relative to the user's computer.			


 Credits:
	SEP: Einar Kjartansson , Stew Levin
	CWP: Shuki Ronen,  Jack K. Cohen

 Caveats:
	Since the algorithm depends on sign reversal of the key value
	to obtain a descending sort, the most significant figure may
	be lost for unsigned data types.  The old SEP support for tape
	input was removed in version 1.16---version 1.15 is in the
	Portability directory for those who may want to input SU data
	stored on tape.

 Trace header fields modified: tracl, tracr

\end{verbatim}
\pagebreak
\begin{verbatim}
 SUSORTY - make a small 2-D common shot off-end  		
	    data set in which the data show geometry 		
	    values to help visualize data sorting.		

  susorty [optional parameters] > out_data_file  		

 Optional parameters:						
	nt=100 		number of time samples			
	nshot=10 	number of shots				
	dshot=10 	shot interval (m)			
	noff=20 	number of offsets			
	doff=20 	offset increment (m)			

 Notes:							
 Creates a common shot su data file for sort visualization	
	       time samples           quantity			
	       ----------------      ----------			
	       first   25%           shot coord			
	       second  25%           rec coord			
	       third   25%           offset			
	       fourth  25%           cmp coord			


 1. default is shot ordered (hsv2 cmap looks best to me)	
 susorty | suximage legend=1 units=meters cmap=hsv2		

 2. sort on cmp (note random order within a cmp)		
 susorty | susort cdp > junk.su 				
 suximage < junk.su legend=1 units=meters cmap=hsv2		

 3. sort to cmp and subsort on offset 	 			
 susorty | susort cdp offset > junk.su 			
 suximage < junk.su legend=1 units=meters cmap=hsv2		


 Credits:
	CWP: Chris Liner  10.09.01

 Trace header fields set: ns, dt, sx, gx, offset, cdp, tracl 

\end{verbatim}
\pagebreak
\begin{verbatim}
 SUSPLIT - Split traces into different output files by keyword value	

     susplit <stdin >stdout [options]					

 Required Parameters:							
	none								

 Optional Parameters:							
	key=cdp		Key header word to split on (see segy.h)	
	stem=split_	Stem name for output files			
	middle=key	middle of name of output files			
	suffix=.su	Suffix for output files				
	numlength=7	Length of numeric part of filename		
	verbose=0	=1 to echo filenames, etc.			
	close=1		=1 to close files before opening new ones	

 Notes:								
 The most efficient way to use this program is to presort the input data
 into common keyword gathers, prior to using susplit.			"

 Use "suputgthr" to put SU data into SU data directory format.	

 Credits:
	Geocon: Garry Perratt hacked together from various other codes
 

\end{verbatim}
\pagebreak
\begin{verbatim}
 SUWINDPOLY - WINDow data to extract traces on or within a respective	
	POLYgonal line or POLYgon with coordinates specified by header	
	keyword values 							

  suwindpoly <stdin [Required parameters] [Optional params] file=outfile

 Required parameters:							
 x=x1,x2,...	list of X coordinates for vertices			
 y=y1,y2,...	list of Y coordinates for vertices			
 file=file1,file2,..	output filename(s)				

 Optional parameters							
 xkey=fldr	X coordinate header key					
 ykey=ep	Y coordinate header key					
 pass=0 	polyline mode: pass traces near the polygonal line	
		=1 pass all traces interior to polygon			
		=2 pass all traces exterior to polygon			

 Optional parameters used in polyline pass=0 mode only:		
 The following need to be given if the unit increments in the X & Y	
 directions are not equal.  For example, if fldr increments by 1 and	
 ep increments by 4 to form 25 x 25 m bins specify dx=25.0 & dy=6.25.	
 The output binning key will be converted to integers by the scaling	
 with the smaller of the two values.					

 dx=1.0	unit increment distance in X direction			
 dy=1.0	unit increment distance in Y direction			
 ilkey=tracl	key for resulting inline index in polyline mode		
 xlkey=tracr	key for resulting xline index in polyline mode		
 dw=1.0	distance in X-Y coordinate units of extracted line	
		to pass points to output.  Width of resulting line is	
		2*dw.  Ignored if polygon mode is specified.		
 Notes:								
 In polyline mode (pass=0), a single trace may be output multiple times
 if it meets the acceptance criteria (distance from line segment < dw)	
 for multiple line segments. However, the headers will be distinct	
 and will associate the output trace with a line segment. This		
 behavior facilitates creation of 3D supergathers from polyline	
 output. Use susort after running in polyline mode.			

 x=& y=lists should be repeated for as many polygons as needed when  
 pass=1 or pass=2. 							

 In polygon mode, the polygon closes itself from the last vertex to	
 the first.								

 Example:								
  suwindpoly <input.su x=10,20,50 y=0,30,60 dw=10 pass=0 file=out.su	



 Credits:  Reginald H. Beardsley	rhb@acm.org
	    originally: suxarb.c adapted from the SLT/SU package.


\end{verbatim}
\pagebreak
\begin{verbatim}
 SUWIND - window traces by key word					

  suwind <stdin >stdout [options]					

 Required Parameters:							
  none 								

 Optional Parameters:							
 verbose=0		=1 for verbose					
 key=tracl		Key header word to window on (see segy.h)	
 min=LONG_MIN		min value of key header word to pass		
 max=LONG_MAX		max value of key header word to pass		

 abs=0			=1 to take absolute value of key header word	
 j=1			Pass every j-th trace ...			
 s=0			... based at s  (if ((key - s)%j) == 0)		
 skip=0		skip the initial N traces                       
 count=ULONG_MAX	... up to count traces				
 reject=none		Skip traces with specified key values		
 accept=none		Pass traces with specified key values(see notes)
			processing, but do no window the data		
 ordered=0		=1 if traces sorted in increasing keyword value 
			=-1  if traces are sorted in a decreasing order 

 Options for vertical windowing (time gating):				
 dt=tr.dt (from header) time sampling interval (sec)	(seismic data)	
 			 =tr.d1  (nonseismic)				
 f1=tr.delrt (from header) first sample		(seismic data)	
 			 =tr.f1  (nonseismic)				

 tmin=0.0		min time to pass				
 tmax=(from header)	max time to pass				
 itmin=0		min time sample to pass				
 itmax=(from header)   max time sample to pass				
 nt=itmax-itmin+1	number of time samples to pass			

 Notes:								
 On large data sets, the count parameter should be set if		
 possible.  Otherwise, every trace in the data set will be		
 examined.  However, the count parameter overrides the accept		
 parameter, so you can't specify count if you want true		
 unconditional acceptance.						

 The skip= option allows the user to skip over traces, which helps	
 for selecting traces far from the beginning of the dataset.		
 Caveat: skip only works with disk input.                        	

 The ordered= option will speed up the process if the data are   	
 sorted in according to the key.                                 	

 The accept option is a bit strange--it does NOT mean accept ONLY	
 the traces on the accept list!  It means accept these traces,   	
 even if they would otherwise be rejected (except as noted in the	
 previous paragraph).  To implement accept-only, you can use the 	
 max=0 option (rejecting everything).  For example, to accept    	
 only the tracl values 4, 5 and 6:					
	... | suwind max=0 accept=4,5,6 | ...		   		

 Another example is the case of suppressing nonseismic traces in 	
 a seismic data set. By the SEGY standard header field trace id, 	
 trid=1 designates traces as being seismic traces. Other traces, 	
 such as calibration traces may be designated by another value.  	

 Example:  trid=1 seismic and trid=0 is nonseismic. To reject    	
       nonseismic traces						
       ... | suwind key=trid reject=0 | ...				

 On most 32 bit machines, LONG_MIN, LONG_MAX and ULONG_MAX are   	
 about -2E9,+2E9 and 4E9, they are defined in limits.h.		

 Selecting times beyond the maximum in the data induces		
 zero padding (up to SU_NFLTS).					

 The time gating here is to the nearest neighboring sample or    	
 time value. Gating to the exact temporal value requires	 	
 resampling if the selected times fall between samples on the    	
 trace. Use suresamp to perform the time gating in this case.    	

 It doesn't really make sense to specify both itmin and tmin,		
 but specifying itmin takes precedence over specifying tmin.		
 Similarly, itmax takes precedence over tmax and tmax over nt.		
 If dt in header is not set, then dt is mandatory			


 Credits:
	SEP: Einar Kjartansson
	CWP: Shuki Ronen, Jack Cohen, Chris Liner
	Warnemuende: Toralf Foerster
	CENPET: Werner M. Heigl (modified to include well log data)

 Trace header fields accessed: ns, dt, delrt, keyword
 Trace header fields modified: ns, delrt, ntr

\end{verbatim}
\pagebreak
\begin{verbatim}
 SUPSCONTOUR - PostScript CONTOUR plot of a segy data set		

 supscontour <stdin [optional parameters] | ...			

 Optional parameters:						 	

 n2=tr.ntr or number of traces in the data set (ntr is an alias for n2)

 d1=tr.d1 or tr.dt/10^6	sampling interval in the fast dimension	
   =.004 for seismic 		(if not set)				
   =1.0 for nonseismic		(if not set)				

 d2=tr.d2			sampling interval in the slow dimension	
   =1.0 			(if not set)				

 f1=tr.f1 or tr.delrt/10^3 or 0.0  first sample in the fast dimension	

 f2=tr.f2 or tr.tracr or tr.tracl  first sample in the slow dimension	
   =1.0 for seismic		    (if not set)			
   =d2 for nonseismic		    (if not set)			

 verbose=0              =1 to print some useful information		

 tmpdir=	 	if non-empty, use the value as a directory path	
		 	prefix for storing temporary files; else if the	
	         	the CWP_TMPDIR environment variable is set use	
	         	its value for the path; else use tmpfile()	

 Note that for seismic time domain data, the "fast dimension" is	
 time and the "slow dimension" is usually trace number or range.	
 Also note that "foreign" data tapes may have something unexpected	
 in the d2,f2 fields, use segyclean to clear these if you can afford	
 the processing time or use d2= f2= to override the header values if	
 not.									

 See the pscontour selfdoc for the remaining parameters.		

 On NeXT:	supscontour < infile [optional parameters]  | open	

 Trace header fields accessed: ns, ntr, tracr, tracl, delrt, trid,     
	dt, d1, d2, f1, f2						

 Credits:

	CWP: Dave Hale and Zhiming Li (pscontour, etc.)
	   Jack Cohen and John Stockwell (supscontour, etc.)
      Delphi: Alexander Koek, added support for irregularly spaced traces
      Aarhus University: Morten W. Pedersen copied everything from the xwigb
                         source and just replaced all occurencies of the word

 Notes:
	When the number of traces isn't known, we need to count
	the traces for pscontour.  You can make this value "known"
	either by getparring n2 or by having the ntr field set
	in the trace header.  A getparred value takes precedence
	over the value in the trace header.

	When we do have to count the traces, we use the "tmpfile"
	routine because on many machines it is implemented
	as a memory area instead of a disk file.

	If your system does make a disk file, consider altering
	the code to remove the file on interrupt.  This could be
	done either by trapping the interrupt with "signal"
	or by using the "tmpnam" routine followed by an immediate
	"remove" (aka "unlink" in old unix).

	When we must compute ntr, we don't allocate a 2-d array,
	but just content ourselves with copying trace by trace from
	the data "file" to the pipe into the plotting program.
	Although we could use tr.data, we allocate a trace buffer
	for code clarity.

\end{verbatim}
\pagebreak
\begin{verbatim}
 SUPSCUBECONTOUR - PostScript CUBE plot of a segy data set		

 supscubecontour <stdin [optional parameters] | ...			

 Optional parameters: 							

 n2 is the number of traces per frame.  If not getparred then it	
 is the total number of traces in the data set.  			

 n3 is the number of frames.  If not getparred then it			
 is the total number of frames in the data set measured by ntr/n2	

 d1=tr.d1 or tr.dt/10^6	sampling interval in the fast dimension	
   =.004 for seismic 		(if not set)				
   =1.0 for nonseismic		(if not set)				

 d2=tr.d2			sampling interval in the slow dimension	
   =1.0 			(if not set)				

 f1=tr.f1 or tr.delrt/10^3 or 0.0  first sample in the fast dimension	

 f2=tr.f2 or tr.tracr or tr.tracl  first sample in the slow dimension	
   =1.0 for seismic		    (if not set)			
   =d2 for nonseismic		    (if not set)			

 verbose=0              =1 to print some useful information		

 tmpdir=	 	if non-empty, use the value as a directory path	
		 	prefix for storing temporary files; else if the	
	         	the CWP_TMPDIR environment variable is set use	
	         	its value for the path; else use tmpfile()	

 Note that for seismic time domain data, the "fast dimension" is	
 time and the "slow dimension" is usually trace number or range.	
 Also note that "foreign" data tapes may have something unexpected	
 in the d2,f2 fields, use segyclean to clear these if you can afford	
 the processing time or use d2= f2= to over-ride the header values if	
 not.									

 See the pscubecontour selfdoc for the remaining parameters.		

 example:   supscubecontour < infile [optional parameters]  | gv -	

 Credits:

	CWP: Dave Hale and Zhiming Li (pscube)
	     Jack K. Cohen  (suxmovie)
	     John Stockwell (supscubecontour)

 Notes:
	When n2 isn't getparred, we need to count the traces
	for pscube. Although we compute ntr, we don't allocate a 2-d array
	and content ourselves with copying trace by trace from
	the data "file" to the pipe into the plotting program.
	Although we could use tr.data, we allocate a trace buffer
	for code clarity.

\end{verbatim}
\pagebreak
\begin{verbatim}
 SUPSCUBE - PostScript CUBE plot of a segy data set			

 supscube <stdin [optional parameters] | ...				

 Optional parameters: 							

 n2 is the number of traces per frame.  If not getparred then it	
 is the total number of traces in the data set.  			

 n3 is the number of frames.  If not getparred then it			
 is the total number of frames in the data set measured by ntr/n2	

 d1=tr.d1 or tr.dt/10^6	sampling interval in the fast dimension	
   =.004 for seismic 		(if not set)				
   =1.0 for nonseismic		(if not set)				

 d2=tr.d2			sampling interval in the slow dimension	
   =1.0 			(if not set)				

 f1=tr.f1 or tr.delrt/10^3 or 0.0  first sample in the fast dimension	

 f2=tr.f2 or tr.tracr or tr.tracl  first sample in the slow dimension	
   =1.0 for seismic		    (if not set)			
   =d2 for nonseismic		    (if not set)			

 verbose=0              =1 to print some useful information		

 tmpdir=	 	if non-empty, use the value as a directory path	
		 	prefix for storing temporary files; else if the	
	         	the CWP_TMPDIR environment variable is set use	
	         	its value for the path; else use tmpfile()	

 Note that for seismic time domain data, the "fast dimension" is	
 time and the "slow dimension" is usually trace number or range.	
 Also note that "foreign" data tapes may have something unexpected	
 in the d2,f2 fields, use segyclean to clear these if you can afford	
 the processing time or use d2= f2= to over-ride the header values if	
 not.									

 See the pscube selfdoc for the remaining parameters.			

 On NeXT:     supscube < infile [optional parameters]  | open	       	

 Credits:

	CWP: Dave Hale and Zhiming Li (pscube)
	     Jack K. Cohen  (suxmovie)
	     John Stockwell (supscube)

 Notes:
	When n2 isn't getparred, we need to count the traces
	for pscube. Although we compute ntr, we don't allocate a 2-d array
	and content ourselves with copying trace by trace from
	the data "file" to the pipe into the plotting program.
	Although we could use tr.data, we allocate a trace buffer
	for code clarity.

\end{verbatim}
\pagebreak
\begin{verbatim}
 SUPSGRAPH - PostScript GRAPH plot of a segy data set			

 supsgraph <stdin [optional parameters] | ...				

 Optional parameters: 							
 style=seismic		seismic is default here, =normal is alternate	
			(see psgraph selfdoc for style definitions)	

 nplot is the number of traces (ntr is an acceptable alias for nplot) 	

 d1=tr.d1 or tr.dt/10^6	sampling interval in the fast dimension	
   =.004 for seismic 		(if not set)				
   =1.0 for nonseismic		(if not set)				

 d2=tr.d2			sampling interval in the slow dimension	
   =1.0 			(if not set)				

 f1=tr.f1 or tr.delrt/10^3 or 0.0  first sample in the fast dimension	

 f2=tr.f2 or tr.tracr or tr.tracl  first sample in the slow dimension	
   =1.0 for seismic		    (if not set)			
   =d2 for nonseismic		    (if not set)			

 verbose=0              =1 to print some useful information		

 tmpdir=	 	if non-empty, use the value as a directory path	
		 	prefix for storing temporary files; else if the	
	         	the CWP_TMPDIR environment variable is set use	
	         	its value for the path; else use tmpfile()	

 Note that for seismic time domain data, the "fast dimension" is	
 time and the "slow dimension" is usually trace number or range.	
 Also note that "foreign" data tapes may have something unexpected	
 in the d2,f2 fields, use segyclean to clear these if you can afford	
 the processing time or use d2= f2= to over-ride the header values if	
 not.									

 See the psgraph selfdoc for the remaining parameters.			

 On NeXT:     supsgraph < infile [optional parameters]  | open      	

 Credits:

	CWP: Dave Hale and Zhiming Li (pswigp, etc.)
	   Jack Cohen and John Stockwell (supswigp, etc.)

 Notes:
	When the number of traces isn't known, we need to count
	the traces for pswigp.  You can make this value "known"
	either by getparring nplot or by having the ntr field set
	in the trace header.  A getparred value takes precedence
	over the value in the trace header.

	When we must compute ntr, we don't allocate a 2-d array,
	but just content ourselves with copying trace by trace from
	the data "file" to the pipe into the plotting program.
	Although we could use tr.data, we allocate a trace buffer
	for code clarity.

\end{verbatim}
\pagebreak
\begin{verbatim}
 SUPSIMAGE - PostScript IMAGE plot of a segy data set			

 supsimage <stdin [optional parameters] | ...				

 Optional parameters:						 	

 n2=tr.ntr or number of traces in the data set (ntr is an alias for n2)

 d1=tr.d1 or tr.dt/10^6	sampling interval in the fast dimension	
   =.004 for seismic 		(if not set)				
   =1.0 for nonseismic		(if not set)				

 d2=tr.d2			sampling interval in the slow dimension	
   =1.0 			(if not set)				

 f1=tr.f1 or tr.delrt/10^3 or 0.0  first sample in the fast dimension	

 f2=tr.f2 or tr.tracr or tr.tracl  first sample in the slow dimension	
   =1.0 for seismic		    (if not set)			
   =d2 for nonseismic		    (if not set)			

 verbose=0              =1 to print some useful information		

 tmpdir=	 	if non-empty, use the value as a directory path	
		 	prefix for storing temporary files; else if the	
	         	the CWP_TMPDIR environment variable is set use	
	         	its value for the path; else use tmpfile()	

 Note that for seismic time domain data, the "fast dimension" is	
 time and the "slow dimension" is usually trace number or range.	
 Also note that "foreign" data tapes may have something unexpected	
 in the d2,f2 fields, use segyclean to clear these if you can afford	
 the processing time or use d2= f2= to override the header values if	
 not.									

 See the psimage selfdoc for the remaining parameters.		

 On NeXT:	supsimage < infile [optional parameters]  | open	

 Trace header fields accessed: ns, ntr, tracr, tracl, delrt, trid,     
	dt, d1, d2, f1, f2						

 Credits:

	CWP: Dave Hale and Zhiming Li (psimage, etc.)
	   Jack Cohen and John Stockwell (supsimage, etc.)

 Notes:
	When the number of traces isn't known, we need to count
	the traces for psimage.  You can make this value "known"
	either by getparring n2 or by having the ntr field set
	in the trace header.  A getparred value takes precedence
	over the value in the trace header.

	When we do have to count the traces, we use the "tmpfile"
	routine because on many machines it is implemented
	as a memory area instead of a disk file.
	"remove" (aka "unlink" in old unix).

	When we must compute ntr, we don't allocate a 2-d array,
	but just content ourselves with copying trace by trace from
	the data "file" to the pipe into the plotting program.
	Although we could use tr.data, we allocate a trace buffer
	for code clarity.

\end{verbatim}
\pagebreak
\begin{verbatim}
 SUPSMAX - PostScript of the MAX, min, or absolute max value on each trace
 	   of a SEGY (SU) data	set					

   supsmax <stdin >postscript file [optional parameters]		

 Optional parameters: 							
 mode=max		max value					
 			=min min value					
 			=abs absolute max value				

 n2=tr.ntr or number of traces in the data set (ntr is an alias for n2)

 d1=tr.d1 or tr.dt/10^6	sampling interval in the fast dimension	
   =.004 for seismic 		(if not set)				
   =1.0 for nonseismic		(if not set)				

 d2=tr.d2			sampling interval in the slow dimension	
   =1.0 			(if not set)				

 f1=tr.f1 or tr.delrt/10^3 or 0.0  first sample in the fast dimension	

 f2=tr.f2 or tr.tracr or tr.tracl  first sample in the slow dimension	
   =1.0 for seismic		    (if not set)			
   =d2 for nonseismic		    (if not set)			

 verbose=0              =1 to print some useful information		

 tmpdir=	 	if non-empty, use the value as a directory path	
		 	prefix for storing temporary files; else if the	
	         	the CWP_TMPDIR environment variable is set use	
	         	its value for the path; else use tmpfile()	

 Note that for seismic time domain data, the "fast dimension" is	
 time and the "slow dimension" is usually trace number or range.	
 Also note that "foreign" data tapes may have something unexpected	
 in the d2,f2 fields, use segyclean to clear these if you can afford	
 the processing time or use d2= f2= to over-ride the header values if	
 not.									

 See the sumax selfdoc for additional parameter.			
 See the psgraph selfdoc for the remaining parameters.			


 Credits:

	CWP: John Stockwell, based on Jack Cohen's SU JACKet 

 Notes:
	When the number of traces isn't known, we need to count
	the traces for psgraph.  You can make this value "known"
	either by getparring n2 or by having the ntr field set
	in the trace header.  A getparred value takes precedence
	over the value in the trace header.

	When we do have to count the traces, we use the "tmpfile"
	routine because on many machines it is implemented
	as a memory area instead of a disk file.

\end{verbatim}
\pagebreak
\begin{verbatim}
 SUPSMOVIE - PostScript MOVIE plot of a segy data set			

 supsmovie <stdin [optional parameters] | ...				

 Optional parameters: 							

 n2 is the number of traces per frame.  If not getparred then it	
 is the total number of traces in the data set.  			

 n3 is the number of frames.  If not getparred then it			
 is the total number of frames in the data set measured by ntr/n2	

 d1=tr.d1 or tr.dt/10^6	sampling interval in the fast dimension	
   =.004 for seismic 		(if not set)				
   =1.0 for nonseismic		(if not set)				

 d2=tr.d2			sampling interval in the slow dimension	
   =1.0 			(if not set)				

 f1=tr.f1 or tr.delrt/10^3 or 0.0  first sample in the fast dimension	

 f2=tr.f2 or tr.tracr or tr.tracl  first sample in the slow dimension	
   =1.0 for seismic		    (if not set)			
   =d2 for nonseismic		    (if not set)			

 verbose=0              =1 to print some useful information		

 tmpdir=	 	if non-empty, use the value as a directory path	
		 	prefix for storing temporary files; else if the	
	         	the CWP_TMPDIR environment variable is set use	
	         	its value for the path; else use tmpfile()	

 Note that for seismic time domain data, the "fast dimension" is	
 time and the "slow dimension" is usually trace number or range.	
 Also note that "foreign" data tapes may have something unexpected	
 in the d2,f2 fields, use segyclean to clear these if you can afford	
 the processing time or use d2= f2= to over-ride the header values if	
 not.									

 See the psmovie selfdoc for the remaining parameters.			

 On NeXT:     supsmovie < infile [optional parameters]  | open	       	

 Credits:

	CWP: Dave Hale and Zhiming Li (psmovie)
	     Jack K. Cohen  (suxmovie)
	     John Stockwell (supsmovie)

 Notes:
	When n2 isn't getparred, we need to count the traces
	for psmovie.  In this case:
	we are using tmpfile because on many machines it is
	implemented as a memory area instead of a disk file.
	Although we compute ntr, we don't allocate a 2-d array
	and content ourselves with copying trace by trace from
	the data "file" to the pipe into the plotting program.
	Although we could use tr.data, we allocate a trace buffer
	for code clarity.

\end{verbatim}
\pagebreak
\begin{verbatim}
 SUPSWIGB - PostScript Bit-mapped WIGgle plot of a segy data set	

 supswigb <stdin [optional parameters] | ...				

 Optional parameters:						 	
 key=(keyword)		if set, the values of x2 are set from header field
			specified by keyword				
 n2=tr.ntr or number of traces in the data set	(ntr is an alias for n2)
 d1=tr.d1 or tr.dt/10^6	sampling interval in the fast dimension	
   =.004 for seismic 		(if not set)				
   =1.0 for nonseismic		(if not set)				
 d2=tr.d2			sampling interval in the slow dimension	
   =1.0 			(if not set)				
 f1=tr.f1 or tr.delrt/10^3 or 0.0  first sample in the fast dimension	
 f2=tr.f2 or tr.tracr or tr.tracl  first sample in the slow dimension	
   =1.0 for seismic		    (if not set)			
   =d2 for nonseismic		    (if not set)			

 style=seismic		 normal (axis 1 horizontal, axis 2 vertical) or 
			 vsp (same as normal with axis 2 reversed)	
			 Note: vsp requires use of a keyword		

 verbose=0              =1 to print some useful information		

 tmpdir=	 	if non-empty, use the value as a directory path	
		 	prefix for storing temporary files; else if the	
	         	the CWP_TMPDIR environment variable is set use	
	         	its value for the path; else use tmpfile()	

 Note that for seismic time domain data, the "fast dimension" is	
 time and the "slow dimension" is usually trace number or range.	
 Also note that "foreign" data tapes may have something unexpected	
 in the d2,f2 fields, use segyclean to clear these if you can afford	
 the processing time or use d2= f2= to override the header values if	
 not.									

 If key=keyword is set, then the values of x2 are taken from the header
 field represented by the keyword (for example key=offset, will show   
 traces in true offset). This permit unequally spaced traces to be plotted.
 Type   sukeyword -o   to see the complete list of SU keywords.	

 This program is really just a wrapper for the plotting program: pswigb
 See the pswigb selfdoc for the remaining parameters.			

 Trace header fields accessed: ns, ntr, tracr, tracl, delrt, trid,     
	dt, d1, d2, f1, f2, keyword (if set)				

 Credits:

	CWP: Dave Hale and Zhiming Li (pswigb, etc.)
	   Jack Cohen and John Stockwell (supswigb, etc.)
      Delphi: Alexander Koek, added support for irregularly spaced traces 

	Modified by Brian Zook, Southwest Research Institute, to honor
	 scale factors, added vsp style

 Notes:
	When the number of traces isn't known, we need to count
	the traces for pswigb.  You can make this value "known"
	either by getparring n2 or by having the ntr field set
	in the trace header.  A getparred value takes precedence
	over the value in the trace header.

	When we must compute ntr, we don't allocate a 2-d array,
	but just content ourselves with copying trace by trace from
	the data "file" to the pipe into the plotting program.
	Although we could use tr.data, we allocate a trace buffer
	for code clarity.

\end{verbatim}
\pagebreak
\begin{verbatim}
 SUPSWIGP - PostScript Polygon-filled WIGgle plot of a segy data set	

 supswigp <stdin [optional parameters] | ...				

 Optional parameters:						 	
 key=(keyword)		if set, values of x2 are set from header field	
			specified by keyword				
 n2=tr.ntr or number of traces in the data set	(ntr is an alias for n2)
 d1=tr.d1 or tr.dt/10^6	sampling interval in the fast dimension	
   =.004 for seismic 		(if not set)				
   =1.0 for nonseismic		(if not set)				
 d2=tr.d2			sampling interval in the slow dimension	
   =1.0 			(if not set)				
 f1=tr.f1 or tr.delrt/10^3 or 0.0  first sample in the fast dimension	
 f2=tr.f2 or tr.tracr or tr.tracl  first sample in the slow dimension	
   =1.0 for seismic		    (if not set)			
   =d2 for nonseismic		    (if not set)			

 style=seismic		 normal (axis 1 horizontal, axis 2 vertical) or 
			 vsp (same as normal with axis 2 reversed)	
			 Note: vsp requires use of a keyword		

 verbose=0              =1 to print some useful information		

 tmpdir=	 	if non-empty, use the value as a directory path	
		 	prefix for storing temporary files; else if the	
	         	the CWP_TMPDIR environment variable is set use	
	         	its value for the path; else use tmpfile()	

 Note that for seismic time domain data, the "fast dimension" is	
 time and the "slow dimension" is usually trace number or range.	
 Also note that "foreign" data tapes may have something unexpected	
 in the d2,f2 fields, use segyclean to clear these if you can afford	
 the processing time or use d2= f2= to override the header values if	
 not.									

 If key=keyword is set, then the values of x2 are taken from the header
 field represented by the keyword (for example key=offset, will show   
 traces in true offset). This permit unequally spaced traces to be plotted.
 Type   sukeyword -o   to see the complete list of SU keywords.	

 This program is really just a wrapper for the plotting program: pswigp
 See the pswigp selfdoc for the remaining parameters.			

 On NeXT:	supswigp < infile [optional parameters]  | open		

 Trace header fields accessed: ns, ntr, tracr, tracl, delrt, trid,     
	dt, d1, d2, f1, f2, key specified by key			

 Credits:

	CWP: Dave Hale and Zhiming Li (pswigp, etc.)
	   Jack Cohen and John Stockwell (supswigp, etc.)
	Delphi: Alexander Koek, added support for irregularly spaced traces

	Modified by Brian Zook, Southwest Research Institute, to honor
	 scale factors, added vsp style

 Notes:
	When the number of traces isn't known, we need to count
	the traces for pswigp.  You can make this value "known"
	either by getparring n2 or by having the ntr field set
	in the trace header.  A getparred value takes precedence
	over the value in the trace header.

	When we do have to count the traces, we use the "tmpfile"
	routine because on many machines it is implemented
	as a memory area instead of a disk file.

	If your system does make a disk file, consider altering
	the code to remove the file on interrupt.  This could be
	done either by trapping the interrupt with "signal"
	or by using the "tmpnam" routine followed by an immediate
	"remove" (aka "unlink" in old unix).

	When we must compute ntr, we don't allocate a 2-d array,
	but just content ourselves with copying trace by trace from
	the data "file" to the pipe into the plotting program.
	Although we could use tr.data, we allocate a trace buffer
	for code clarity.

\end{verbatim}
\pagebreak
\begin{verbatim}
 SUXCONTOUR - X CONTOUR plot of Seismic UNIX tracefile via vector plot call

 suxwigb <stdin [optional parameters] | ...				

 Optional parameters:						 	
 key=(keyword)		if set, the values of x2 are set from header field
			specified by keyword				
 n2=tr.ntr or number of traces in the data set	(ntr is an alias for n2)
 d1=tr.d1 or tr.dt/10^6	sampling interval in the fast dimension	
   =.004 for seismic 		(if not set)				
   =1.0 for nonseismic		(if not set)				
 d2=tr.d2			sampling interval in the slow dimension	
   =1.0 			(if not set)				
 f1=tr.f1 or tr.delrt/10^3 or 0.0  first sample in the fast dimension	
 f2=tr.f2 or tr.tracr or tr.tracl  first sample in the slow dimension	
   =1.0 for seismic		    (if not set)			
   =d2 for nonseismic		    (if not set)			

 verbose=0              =1 to print some useful information		

 tmpdir=	 	if non-empty, use the value as a directory path	
		 	prefix for storing temporary files; else if the	
	         	the CWP_TMPDIR environment variable is set use	
	         	its value for the path; else use tmpfile()	

 Note that for seismic time domain data, the "fast dimension" is	
 time and the "slow dimension" is usually trace number or range.	
 Also note that "foreign" data tapes may have something unexpected	
 in the d2,f2 fields, use segyclean to clear these if you can afford	
 the processing time or use d2= f2= to override the header values if	
 not.									

 If key=keyword is set, then the values of x2 are taken from the header
 field represented by the keyword (for example key=offset, will show   
 traces in true offset). This permit unequally spaced traces to be plotted.
 Type   sukeyword -o   to see the complete list of SU keywords.	

 This program is really just a wrapper for the plotting program: xcontour
 See the xcontour selfdoc for the remaining parameters.		



 Credits:

	CWP: Dave Hale and Zhiming Li (xwigb, etc.)
	   Jack Cohen and John Stockwell (suxwigb, etc.)
      Delphi: Alexander Koek, added support for irregularly spaced traces
      Aarhus University: Morten W. Pedersen copied everything from the xwigb
                         source and just replaced all occurencies of the word
                         xwigb with xcountour ;-)

 Notes:
	When the number of traces isn't known, we need to count
	the traces for xcontour.  You can make this value "known"
	either by getparring n2 or by having the ntr field set
	in the trace header.  A getparred value takes precedence
	over the value in the trace header.

	When we must compute ntr, we don't allocate a 2-d array,
	but just content ourselves with copying trace by trace from
	the data "file" to the pipe into the plotting program.
	Although we could use tr.data, we allocate a trace buffer
	for code clarity.

\end{verbatim}
\pagebreak
\begin{verbatim}
 SUXGRAPH - X-windows GRAPH plot of a segy data set			

 suxgraph <stdin [optional parameters] | ...				

 Optional parameters: 							
 (see xgraph selfdoc for optional parametes)				

 nplot= number of traces (ntr is an acceptable alias for nplot) 	

 d1=tr.d1 or tr.dt/10^6	sampling interval in the fast dimension	
   =.004 for seismic 		(if not set)				
   =1.0 for nonseismic		(if not set)				

 d2=tr.d2			sampling interval in the slow dimension	
   =1.0 			(if not set)				

 f1=tr.f1 or tr.delrt/10^3 or 0.0  first sample in the fast dimension	

 f2=tr.f2 or tr.tracr or tr.tracl  first sample in the slow dimension	
   =1.0 for seismic		    (if not set)			
   =d2 for nonseismic		    (if not set)			

 verbose=0              =1 to print some useful information		

 tmpdir=	 	if non-empty, use the value as a directory path	
		 	prefix for storing temporary files; else if the	
	         	the CWP_TMPDIR environment variable is set use	
	         	its value for the path; else use tmpfile()	

 Note that for seismic time domain data, the "fast dimension" is	
 time and the "slow dimension" is usually trace number or range.	
 Also note that "foreign" data tapes may have something unexpected	
 in the d2,f2 fields, use segyclean to clear these if you can afford	
 the processing time or use d2= f2= to over-ride the header values if	
 not.									

 See the xgraph selfdoc for the remaining parameters.			

 On NeXT:     suxgraph < infile [optional parameters]  | open      	

 Credits:

	CWP: Dave Hale and Zhiming Li (pswigp, etc.)
	   Jack Cohen and John Stockwell (supswigp, etc.)

 Notes:
	When the number of traces isn't known, we need to count
	the traces for pswigp.  You can make this value "known"
	either by getparring nplot or by having the ntr field set
	in the trace header.  A getparred value takes precedence
	over the value in the trace header.

	When we do have to count the traces, we use the "tmpfile"
	routine because on many machines it is implemented
	as a memory area instead of a disk file.

	When we must compute ntr, we don't allocate a 2-d array,
	but just content ourselves with copying trace by trace from
	the data "file" to the pipe into the plotting program.
	Although we could use tr.data, we allocate a trace buffer
	for code clarity.

\end{verbatim}
\pagebreak
\begin{verbatim}
 SUXIMAGE - X-windows IMAGE plot of a segy data set	                

 suximage infile= [optional parameters] | ...  (direct I/O)            
  or					                		
 suximage <stdin [optional parameters] | ...	(sequential I/O)        

 Optional parameters:						 	

 infile=NULL SU data to be ploted, default stdin with sequential access
             if 'infile' provided, su data read by (fast) direct access

	      with ftr,dtr and n2 suximage will pass a subset of data   
	      to the plotting program-ximage:                           
 ftr=1       First trace to be plotted                                 
 dtr=1       Trace increment to be plotted                             
 n2=tr.ntr   (Max) number of traces to be plotted (ntr is an alias for n2)
	      Priority: first try to read from parameter line;		
		        if not provided, check trace header tr.ntr;     
		        if still not provided, figure it out using ftello

 d1=tr.d1 or tr.dt/10^6	sampling interval in the fast dimension	
   =.004 for seismic 		(if not set)				
   =1.0 for nonseismic		(if not set)				

 d2=tr.d2		sampling interval in the slow dimension	        
   =1.0 		(if not set or was set to 0)		        

 key=			key for annotating d2 (slow dimension)		
 			If annotation is not at proper increment, try	
 			setting d2; only first trace's key value is read

 f1=tr.f1 or tr.delrt/10^3 or 0.0  first sample in the fast dimension	

 f2=tr.f2 or tr.tracr or tr.tracl  first sample in the slow dimension	
   =1.0 for seismic		    (if not set)			
   =d2 for nonseismic		    (if not set)			

 verbose=0             =1 to print some useful information		

 tmpdir=	 	if non-empty, use the value as a directory path	
		 	prefix for storing temporary files; else if the	
	         	the CWP_TMPDIR environment variable is set use	
	         	its value for the path; else use tmpfile()	

 Note that for seismic time domain data, the "fast dimension" is	
 time and the "slow dimension" is usually trace number or range.	
 Also note that "foreign" data tapes may have something unexpected	
 in the d2,f2 fields, use segyclean to clear these if you can afford	
 the processing time or use d2= f2= to override the header values if	
 not.									

 See the ximage selfdoc for the remaining parameters.		        


 Credits:

	CWP: Dave Hale and Zhiming Li (ximage, etc.)
	   Jack Cohen and John Stockwell (suximage, etc.)
	MTU: David Forel, June 2004, added key for annotating d2
      ConocoPhillips: Zhaobo Meng, Dec 2004, added direct I/O

 Notes:

      When provide ftr and dtr and infile, suximage can be used to plot 
      multi-dimensional volumes efficiently.  For example, for a Offset-CDP
      dataset with 32 offsets, the command line
      suximage infile=volume3d.su ftr=1 dtr=32 ... &
      will display the zero-offset common offset data with ranrom access.  
      It is highly recommend to use infile= to view large datasets, since
      using stdin only allows sequential access, which is very slow for 
      large datasets.

	When the number of traces isn't known, we need to count
	the traces for ximage.  You can make this value "known"
	either by getparring n2 or by having the ntr field set
	in the trace header.  A getparred value takes precedence
	over the value in the trace header.

	When we must compute ntr, we don't allocate a 2-d array,
	but just content ourselves with copying trace by trace from
	the data "file" to the pipe into the plotting program.
	Although we could use tr.data, we allocate a trace buffer
	for code clarity.

\end{verbatim}
\pagebreak
\begin{verbatim}
 SUXMAX - X-windows graph of the MAX, min, or absolute max value on	
	each trace of a SEGY (SU) data set				

   suxmax <stdin [optional parameters]					

 Optional parameters: 							
 mode=max		max value					
 			=min min value					
 			=abs absolute max value				

 n2=tr.ntr or number of traces in the data set (ntr is an alias for n2)

 d1=tr.d1 or tr.dt/10^6	sampling interval in the fast dimension	
   =.004 for seismic 		(if not set)				
   =1.0 for nonseismic		(if not set)				

 d2=tr.d2			sampling interval in the slow dimension	
   =1.0 			(if not set)				

 f1=tr.f1 or tr.delrt/10^3 or 0.0  first sample in the fast dimension	

 f2=tr.f2 or tr.tracr or tr.tracl  first sample in the slow dimension	
   =1.0 for seismic		    (if not set)			
   =d2 for nonseismic		    (if not set)			

 verbose=0              =1 to print some useful information		

 tmpdir=	 	if non-empty, use the value as a directory path	
		 	prefix for storing temporary files; else if the	
	         	the CWP_TMPDIR environment variable is set use	
	         	its value for the path; else use tmpfile()	

 Note that for seismic time domain data, the "fast dimension" is	
 time and the "slow dimension" is usually trace number or range.	
 Also note that "foreign" data tapes may have something unexpected	
 in the d2,f2 fields, use segyclean to clear these if you can afford	
 the processing time or use d2= f2= to over-ride the header values if	
 not.									

 See the sumax selfdoc for additional parameter.			
 See the xgraph selfdoc for the remaining parameters.			


 Credits:

	CWP: John Stockwell, based on Jack Cohen's SU JACKet 

 Notes:
	When the number of traces isn't known, we need to count
	the traces for xgraph.  You can make this value "known"
	either by getparring n2 or by having the ntr field set
	in the trace header.  A getparred value takes precedence
	over the value in the trace header.

	When we do have to count the traces, we use the "tmpfile"
	routine because on many machines it is implemented
	as a memory area instead of a disk file.

\end{verbatim}
\pagebreak
\begin{verbatim}
 SUXMOVIE - X MOVIE plot of a 2D or 3D segy data set 			

 suxmovie <stdin [optional parameters]		 			

 Optional parameters: 							

 n1=tr.ns         	    	number of samples per trace  		
 ntr=tr.ntr     	    	number of traces in the data set	
 n2=tr.shortpad or tr.ntr	number of traces in inline direction 	
 n3=ntr/n2     	    	number of traces in crossline direction	

 d1=tr.d1 or tr.dt/10^6    sampling interval in the fast dimension	
   =.004 for seismic 		(if not set)				
   =1.0 for nonseismic		(if not set)				

 d2=tr.d2		    sampling interval in the slow dimension	
   =1.0 			(if not set)				

 d3=1.0		    sampling interval in the slowest dimension	

 f1=tr.f1 or 0.0  	    first sample in the z dimension		
 f2=tr.f2 or 1.0           first sample in the x dimension		
 f3=1.0 		    						

 mode=0          0= x,z slice movie through y dimension (in line)      
                 1= y,z slice movie through x dimension (cross line)   
                 2= x,y slice movie through z dimension (time slice)   

 verbose=0              =1 to print some useful information		

 tmpdir=	 	if non-empty, use the value as a directory path	
		 	prefix for storing temporary files; else if the	
	         	the CWP_TMPDIR environment variable is set use	
	         	its value for the path; else use tmpfile()	

 Notes:
 For seismic data, the "fast dimension" is either time or		
 depth and the "slow dimension" is usually trace number.	        
 The 3D data set is expected to have n3 sets of n2 traces representing 
 the horizontal coverage of n2*d2 in x  and n3*d3 in y direction.      

 The data is read to memory with and piped to xmovie with the         	
 respective sampling parameters.			        	
 See the xmovie selfdoc for the remaining parameters and X functions.	

\end{verbatim}
\pagebreak
\begin{verbatim}
 SUXPICKER - X-windows  WIGgle plot PICKER of a segy data set		

 suxpicker <stdin [optional parameters] | ...				

 Optional parameters:						 	

 key=(keyword)		if set, the values of x2 are set from header field
			specified by keyword				
			specified by keyword				
 n2=tr.ntr or number of traces in the data set (ntr is an alias for n2)

 d1=tr.d1 or tr.dt/10^6	sampling interval in the fast dimension	
   =.004 for seismic 		(if not set)				
   =1.0 for nonseismic		(if not set)				

 d2=tr.d2			sampling interval in the slow dimension	
   =1.0 			(if not set)				

 f1=tr.f1 or tr.delrt/10^3 or 0.0  first sample in the fast dimension	

 f2=tr.f2 or tr.tracr or tr.tracl  first sample in the slow dimension	
   =1.0 for seismic		    (if not set)			
   =d2 for nonseismic		    (if not set)			

 verbose=0              =1 to print some useful information		


 tmpdir=	 	if non-empty, use the value as a directory path	
		 	prefix for storing temporary files; else if the	
	         	the CWP_TMPDIR environment variable is set use	
	         	its value for the path; else use tmpfile()	

 Note that for seismic time domain data, the "fast dimension" is	
 time and the "slow dimension" is usually trace number or range.	
 Also note that "foreign" data tapes may have something unexpected	
 in the d2,f2 fields, use segyclean to clear these if you can afford	
 the processing time or use d2= f2= to override the header values if	
 not.									

 If key=keyword is set, then the values of x2 are taken from the header
 field represented by the keyword (for example key=offset, will show	
 traces in true offset). This permit unequally spaced traces to be plotted.
 Type	 sukeyword -o	to see the complete list of SU keywords.	

 See the xpicker selfdoc for the remaining parameters.			


 Credits:

	CWP: Dave Hale and Zhiming Li (xpicker, etc.)
	   Jack Cohen and John Stockwell (suxpicker, etc.)

 Notes:
	When the number of traces isn't known, we need to count
	the traces for xpicker.  You can make this value "known"
	either by getparring n2 or by having the ntr field set
	in the trace header.  A getparred value takes precedence
	over the value in the trace header.

	When we do have to count the traces, we use the "tmpfile"
	routine because on many machines it is implemented
	as a memory area instead of a disk file.

	If your system does make a disk file, consider altering
	the code to remove the file on interrupt.  This could be
	done either by trapping the interrupt with "signal"
	or by using the "tmpnam" routine followed by an immediate
	"remove" (aka "unlink" in old unix).

	When we must compute ntr, we don't allocate a 2-d array,
	but just content ourselves with copying trace by trace from
	the data "file" to the pipe into the plotting program.
	Although we could use tr.data, we allocate a trace buffer
	for code clarity.

\end{verbatim}
\pagebreak
\begin{verbatim}
 SUXWIGB - X-windows Bit-mapped WIGgle plot of a segy data set		
 This is a modified suxwigb that uses the depth or coordinate scaling	
 when such values are used as keys.					

 suxwigb <stdin [optional parameters] | ...				

 Optional parameters:							
 key=(keyword)		if set, the values of x2 are set from header field
			specified by keyword				
 n2=tr.ntr or number of traces in the data set (ntr is an alias for n2)
 d1=tr.d1 or tr.dt/10^6	sampling interval in the fast dimension 
   =.004 for seismic		(if not set)				
   =1.0 for nonseismic		(if not set)				
 d2=tr.d2			sampling interval in the slow dimension 
   =1.0			(if not set)				
 f1=tr.f1 or tr.delrt/10^3 or 0.0  first sample in the fast dimension	
 f2=tr.f2 or tr.tracr or tr.tracl  first sample in the slow dimension	
   =1.0 for seismic		    (if not set)			
   =d2 for nonseismic		    (if not set)			

 style=seismic		 normal (axis 1 horizontal, axis 2 vertical) or 
			 vsp (same as normal with axis 2 reversed)	
			 Note: vsp requires use of a keyword		
 verbose=0              =1 to print some useful information		


 tmpdir=	 	if non-empty, use the value as a directory path	
		 	prefix for storing temporary files; else if the	
	         	the CWP_TMPDIR environment variable is set use	
	         	its value for the path; else use tmpfile()	

 Note that for seismic time domain data, the "fast dimension" is	
 time and the "slow dimension" is usually trace number or range.	
 Also note that "foreign" data tapes may have something unexpected	
 in the d2,f2 fields, use segyclean to clear these if you can afford	
 the processing time or use d2= f2= to override the header values if	
 not.									

 If key=keyword is set, then the values of x2 are taken from the header
 field represented by the keyword (for example key=offset, will show	
 traces in true offset). This permit unequally spaced traces to be plotted.
 Type	 sukeyword -o	to see the complete list of SU keywords.	

 This program is really just a wrapper for the plotting program: xwigb	
 See the xwigb selfdoc for the remaining parameters.			


 Credits:

	CWP: Dave Hale and Zhiming Li (xwigb, etc.)
	   Jack Cohen and John Stockwell (suxwigb, etc.)
	Delphi: Alexander Koek, added support for irregularly spaced traces

	Modified by Brian Zook, Southwest Research Institute, to honor
	 scale factors, added vsp style

 Notes:
	When the number of traces isn't known, we need to count
	the traces for xwigb.  You can make this value "known"
	either by getparring n2 or by having the ntr field set
	in the trace header.  A getparred value takes precedence
	over the value in the trace header.

	When we must compute ntr, we don't allocate a 2-d array,
	but just content ourselves with copying trace by trace from
	the data "file" to the pipe into the plotting program.
	Although we could use tr.data, we allocate a trace buffer
	for code clarity.

\end{verbatim}
\pagebreak
\begin{verbatim}
 PSBBOX - change BoundingBOX of existing PostScript file	

 psbbox < PostScriptfile [optional parameters] > PostScriptfile

 Optional Parameters:						
 llx=		new llx						
 lly=		new lly						
 urx=		new urx						
 ury=		new ury						
 verbose=1	=1 for info printed on stderr (0 for no info)	

\end{verbatim}
\pagebreak
\begin{verbatim}
 PSCONTOUR - PostScript CONTOURing of a two-dimensional function f(x1,x2)

 pscontour n1= [optional parameters] <binaryfile >postscriptfile	

 Required Parameters:							
 n1                     number of samples in 1st (fast) dimension	

 Optional Parameters:							
 d1=1.0                 sampling interval in 1st dimension		
 f1=d1                  first sample in 1st dimension			
 x1=f1,f1+d1,...        array of monotonic sampled values in 1st dimension
 n2=all                 number of samples in 2nd (slow) dimension	
 d2=1.0                 sampling interval in 2nd dimension		
 f2=d2                  first sample in 2nd dimension			
 x2=f2,f2+d2,...        array of monotonic sampled values in 2nd dimension
 nc=5                   number of contour values			
 dc=(zmax-zmin)/nc      contour interval				
 fc=min+dc              first contour					
 c=fc,fc+dc,...         array of contour values			
 cwidth=1.0,...         array of contour line widths			
 cgray=0.0,...          array of contour grays (0.0=black to 1.0=white)
 ccolor=none,...        array of contour colors; none means use cgray	
 cdash=0.0,...          array of dash spacings (0.0 for solid)		
 labelcf=1              first labeled contour (1,2,3,...)		
 labelcper=1            label every labelcper-th contour		
 nlabelc=nc             number of labeled contours (0 no contour label)
 nplaces=6              number of decimal places in contour label      
 xbox=1.5               offset in inches of left side of axes box	
 ybox=1.5               offset in inches of bottom side of axes box	
 wbox=6.0               width in inches of axes box			
 hbox=8.0               height in inches of axes box			
 x1beg=x1min            value at which axis 1 begins			
 x1end=x1max            value at which axis 1 ends			
 d1num=0.0              numbered tic interval on axis 1 (0.0 for automatic)
 f1num=x1min            first numbered tic on axis 1 (used if d1num not 0.0)
 n1tic=1                number of tics per numbered tic on axis 1	
 grid1=none             grid lines on axis 1 - none, dot, dash, or solid
 label1=                label on axis 1				
 x2beg=x2min            value at which axis 2 begins			
 x2end=x2max            value at which axis 2 ends			
 d2num=0.0              numbered tic interval on axis 2 (0.0 for automatic)
 f2num=x2min            first numbered tic on axis 2 (used if d2num not 0.0)
 n2tic=1                number of tics per numbered tic on axis 2	
 grid2=none             grid lines on axis 2 - none, dot, dash, or solid
 label2=                label on axis 2				
 labelfont=Helvetica    font name for axes labels			
 labelsize=18           font size for axes labels			
 title=                 title of plot					
 titlefont=Helvetica-Bold font name for title				
 titlesize=24           font size for title				
 labelcfont=Helvetica-Bold font name for contour labels		
 labelcsize=6           font size of contour labels   			
 labelccolor=black      color of contour labels   			
 titlecolor=black       color of title					
 axescolor=black        color of axes					
 gridcolor=black        color of grid					
 axeswidth=1            width (in points) of axes			
 ticwidth=axeswidth     width (in points) of tic marks		
 gridwidth=axeswidth    width (in points) of grid lines		
 style=seismic          normal (axis 1 horizontal, axis 2 vertical) or	
                        seismic (axis 1 vertical, axis 2 horizontal)	

 Note.									
 The line width of unlabeled contours is designed as a quarter of that	
 of labeled contours. 							

 All color specifications may also be made in X Window style Hex format
 example:   axescolor=#255						

 Legal font names are:							
 AvantGarde-Book AvantGarde-BookOblique AvantGarde-Demi AvantGarde-DemiOblique"
 Bookman-Demi Bookman-DemiItalic Bookman-Light Bookman-LightItalic 
 Courier Courier-Bold Courier-BoldOblique Courier-Oblique 
 Helvetica Helvetica-Bold Helvetica-BoldOblique Helvetica-Oblique 
 Helvetica-Narrow Helvetica-Narrow-Bold Helvetica-Narrow-BoldOblique 
 Helvetica-Narrow-Oblique NewCentrySchlbk-Bold"
 NewCenturySchlbk-BoldItalic NewCenturySchlbk-Roman Palatino-Bold  
 Palatino-BoldItalic Palatino-Italics Palatino-Roman 
 SanSerif-Bold SanSerif-BoldItalic SanSerif-Roman 
 Symbol Times-Bold Times-BoldItalic 
 Times-Roman Times-Italic ZapfChancery-MediumItalic 
 type:   sudoc pscontour    for more information			


Notes:

 For nice even-numbered contours, use the parameters  fc and dc

 Example: if the range of the z values of a data set range between
 approximately -1000 and +1000, then use fc=-1000 nc=10 and dc=100
 to get contours spaced by even 100's.



 AUTHOR:  Dave Hale, Colorado School of Mines, 05/29/90
 MODIFIED:  Craig Artley, Colorado School of Mines, 08/30/91
            BoundingBox moved to top of PostScript output
 MODIFIED:  Zhenyue Liu, Colorado School of Mines, 08/26/93
	      Values are labeled on contours  
 MODIFIED:  Craig Artley, Colorado School of Mines, 12/16/93
            Added color options (Courtesy of Dave Hale, Advance Geophysical).
 Modified: Morten Wendell Pedersen, Aarhus University, 23/3-97
           Added ticwidth,axeswidth, gridwidth parameters 


\end{verbatim}
\pagebreak
\begin{verbatim}
 PSCCONTOUR - PostScript Contour plot of a data CUBE		        

 pscubecontour n1= n2= n3= [optional parameters] <binaryfile >postscriptfile	
    or									
 pscubecontour n1= n2= n3= front= side= top= [optional parameters] >postscriptfile

 Data formats supported:						
	1. Entire cube read from stdin (n1*n2*n3 floats) [default format]
	2. Faces read from stdin (n1*n2 floats for front, followed by n1*n3
	   floats for side, and n2*n3 floats for top) [specify faces=1]	
	3. Faces read from separate data files [specify filenames]	

 Required Parameters:							
 n1                     number of samples in 1st (fastest) dimension	
 n2                     number of samples in 2nd dimension		
 n3                     number of samples in 3rd (slowest) dimension	

 Optional Parameters:							
 front                  name of file containing front panel		
 side                   name of file containing side panel		
 top                    name of file containing top panel		
 faces=0                =1 to read faces from stdin (data format 2)	
 d1=1.0                 sampling interval in 1st dimension		
 f1=0.0                 first sample in 1st dimension			
 d2=1.0                 sampling interval in 2nd dimension		
 f2=0.0                 first sample in 2nd dimension			
 d3=1.0                 sampling interval in 3rd dimension		
 f3=0.0                 first sample in 3rd dimension			
 d1s=1.0                factor by which to scale d1 before imaging	
 d2s=1.0                factor by which to scale d2 before imaging	
 d3s=1.0                factor by which to scale d3 before imaging	
 nc=5                   number of contour values			
 dc=(zmax-zmin)/nc      contour interval				
 fc=min+dc              first contour					
 c=fc,fc+dc,...         array of contour values			
 cwidth=1.0,...         array of contour line widths			
 cgray=0.0,...          array of contour grays (0.0=black to 1.0=white)
 ccolor=none,...        array of contour colors; none means use cgray	
 cdash=0.0,...          array of dash spacings (0.0 for solid)		
 labelcf=1              first labeled contour (1,2,3,...)		
 labelcper=1            label every labelcper-th contour		
 nlabelc=nc             number of labeled contours (0 no contour label)
 nplaces=6              number of decimal places in contour label      
 xbox=1.5               offset in inches of left side of axes box	
 ybox=1.5               offset in inches of bottom side of axes box	
 size1=4.0              size in inches of 1st axes (vertical)		
 size2=4.0              size in inches of 2nd axes (horizontal)	
 size3=3.0              size in inches of 3rd axes (projected)		
 angle=45               projection angle of cube in degrees (0<angle<90)
                        (angle between 2nd axis and projected 3rd axis)
 x1end=x1max            value at which axis 1 ends			
 d1num=0.0              numbered tic interval on axis 1 (0.0 for automatic)
 f1num=x1min            first numbered tic on axis 1 (used if d1num not 0.0)
 n1tic=1                number of tics per numbered tic on axis 1	
 grid1=none             grid lines on axis 1 - none, dot, dash, or solid
 label1=                label on axis 1				
 x2beg=x2min            value at which axis 2 begins			
 d2num=0.0              numbered tic interval on axis 2 (0.0 for automatic)
 f2num=x2min            first numbered tic on axis 2 (used if d2num not 0.0)
 n2tic=1                number of tics per numbered tic on axis 2	
 grid2=none             grid lines on axis 2 - none, dot, dash, or solid
 label2=                label on axis 2				
 x3end=x3max            value at which axis 3 ends			
 d3num=0.0              numbered tic interval on axis 3 (0.0 for automatic)
 f3num=x3min            first numbered tic on axis 3 (used if d3num not 0.0)
 n3tic=1                number of tics per numbered tic on axis 3	
 grid3=none             grid lines on axis 3 - none, dot, dash, or solid
 label3=                label on axis 3				
 labelfont=Helvetica    font name for axes labels			
 labelsize=18           font size for axes labels			
 title=                 title of plot					
 titlefont=Helvetica-Bold font name for title				
 titlesize=24           font size for title				
 titlecolor=black       color of title					
 labelcfont=Helvetica-Bold font name for contour labels		
 labelcsize=6           font size of contour labels   			
 labelccolor=black      color of contour labels   			
 axescolor=black        color of axes					
 gridcolor=black        color of grid					

 All color specifications may also be made in X Window style Hex format
 example:   axescolor=#255						

 Note: The values of x1beg=x1min, x2end=x2max and x3beg=x3min cannot   
 be changed.								

 Legal font names are:							
 AvantGarde-Book AvantGarde-BookOblique AvantGarde-Demi AvantGarde-DemiOblique"
 Bookman-Demi Bookman-DemiItalic Bookman-Light Bookman-LightItalic 
 Courier Courier-Bold Courier-BoldOblique Courier-Oblique 
 Helvetica Helvetica-Bold Helvetica-BoldOblique Helvetica-Oblique 
 Helvetica-Narrow Helvetica-Narrow-Bold Helvetica-Narrow-BoldOblique 
 Helvetica-Narrow-Oblique NewCentrySchlbk-Bold"
 NewCenturySchlbk-BoldItalic NewCenturySchlbk-Roman Palatino-Bold  
 Palatino-BoldItalic Palatino-Italics Palatino-Roman 
 SanSerif-Bold SanSerif-BoldItalic SanSerif-Roman 
 Symbol Times-Bold Times-BoldItalic 
 Times-Roman Times-Italic ZapfChancery-MediumItalic 



 (Original codes pscontour and pscube)

 AUTHOR:  Craig Artley, Colorado School of Mines, 03/12/93
 NOTE:  Original written by Zhiming Li & Dave Hale, CSM, 07/01/90
	  Completely rewritten, the code now bears more similarity to
	  psimage than the previous pscube.  Faces of cube now rendered
	  as three separate images, rather than as a single image.  The
	  output no longer suffers from stretching artifacts, and the
	  code is simpler.  -Craig
 MODIFIED:  Craig Artley, Colorado School of Mines, 12/17/93
 	  Added color options.

 PSCCONTOUR: mashed together from pscube and pscontour 
 to generate 3d contour plots by Claudia Vanelle, Institute of Geophysics,
 University of Hamburg, Germany somewhen in 2000

 PSCUBE was "merged" with PSCONTOUR to create PSCUBECONTOUR 
 by Claudia Vanelle, Applied Geophysics Group Hamburg
 somewhen in 2000

\end{verbatim}
\pagebreak
\begin{verbatim}
 PSCUBE - PostScript image plot with Legend of a data CUBE       

 pscube n1= n2= n3= [optional parameters] <binaryfile >postscriptfile	
    or									
 pscube n1= n2= n3= front= side= top= [optional parameters] >postscriptfile

 Data formats supported:						
	1. Entire cube read from stdin (n1*n2*n3 floats) [default format]
	2. Faces read from stdin (n1*n2 floats for front, followed by n1*n3
	   floats for side, and n2*n3 floats for top) [specify faces=1]	
	3. Faces read from separate data files [specify filenames]	

 Required Parameters:							
 n1                     number of samples in 1st (fastest) dimension	
 n2                     number of samples in 2nd dimension		
 n3                     number of samples in 3rd (slowest) dimension	

 Optional Parameters:							
 front                  name of file containing front panel		
 side                   name of file containing side panel		
 top                    name of file containing top panel		
 faces=0                =1 to read faces from stdin (data format 2)	
 d1=1.0                 sampling interval in 1st dimension		
 f1=0.0                 first sample in 1st dimension			
 d2=1.0                 sampling interval in 2nd dimension		
 f2=0.0                 first sample in 2nd dimension			
 d3=1.0                 sampling interval in 3rd dimension		
 f3=0.0                 first sample in 3rd dimension			
 perc=100.0             percentile used to determine clip		
 clip=(perc percentile) clip used to determine bclip and wclip		
 bperc=perc             percentile for determining black clip value	
 wperc=100.0-perc       percentile for determining white clip value	
 bclip=clip             data values outside of [bclip,wclip] are clipped
 wclip=-clip            data values outside of [bclip,wclip] are clipped
 brgb=0.0,0.0,0.0       red, green, blue values corresponding to black	
 wrgb=1.0,1.0,1.0       red, green, blue values corresponding to white	
 bhls=0.0,0.0,0.0       hue, lightness, saturation corresponding to black
 whls=0.0,1.0,0.0       hue, lightness, saturation corresponding to white
 bps=12                 bits per sample for color plots, either 12 or 24
 d1s=1.0                factor by which to scale d1 before imaging	
 d2s=1.0                factor by which to scale d2 before imaging	
 d3s=1.0                factor by which to scale d3 before imaging	
 verbose=1              =1 for info printed on stderr (0 for no info)	
 xbox=1.5               offset in inches of left side of axes box	
 ybox=1.5               offset in inches of bottom side of axes box	
 size1=4.0              size in inches of 1st axes (vertical)		
 size2=4.0              size in inches of 2nd axes (horizontal)	
 size3=3.0              size in inches of 3rd axes (projected)		
 angle=45               projection angle of cube in degrees (0<angle<90)
                        (angle between 2nd axis and projected 3rd axis)
 x1end=x1max            value at which axis 1 ends			
 d1num=0.0              numbered tic interval on axis 1 (0.0 for automatic)
 f1num=x1min            first numbered tic on axis 1 (used if d1num not 0.0)
 n1tic=1                number of tics per numbered tic on axis 1	
 grid1=none             grid lines on axis 1 - none, dot, dash, or solid
 label1=                label on axis 1				
 x2beg=x2min            value at which axis 2 begins			
 d2num=0.0              numbered tic interval on axis 2 (0.0 for automatic)
 f2num=x2min            first numbered tic on axis 2 (used if d2num not 0.0)
 n2tic=1                number of tics per numbered tic on axis 2	
 grid2=none             grid lines on axis 2 - none, dot, dash, or solid
 label2=                label on axis 2				
 x3end=x3max            value at which axis 3 ends			
 d3num=0.0              numbered tic interval on axis 3 (0.0 for automatic)
 f3num=x3min            first numbered tic on axis 3 (used if d3num not 0.0)
 n3tic=1                number of tics per numbered tic on axis 3	
 grid3=none             grid lines on axis 3 - none, dot, dash, or solid
 label3=                label on axis 3				
 labelfont=Helvetica    font name for axes labels			
 labelsize=18           font size for axes labels			
 title=                 title of plot					
 titlefont=Helvetica-Bold font name for title				
 titlesize=24           font size for title				
 titlecolor=black       color of title					
 axescolor=black        color of axes					
 gridcolor=black        color of grid					
 legend=0               =1 display the color scale                     
                        if ==1, resize xbox,ybox,width,height          
 lstyle=vertleft       Vertical, axis label on left side               
                        =vertright (Vertical, axis label on right side)
                        =horibottom (Horizontal, axis label on bottom) 
 units=                 unit label for legend                          
 legendfont=times_roman10    font name for title                       
 following are defaults for lstyle=0. They are changed for other lstyles
 lwidth=1.2             colorscale (legend) width in inches            
 lheight=height/3       colorscale (legend) height in inches           
 lx=1.0                 colorscale (legend) x-position in inches       
 ly=(height-lheight)/2+xybox    colorscale (legend) y-position in pixels
 lbeg= lmin or wclip-5*perc    value at which legend axis begins       
 lend= lmax or bclip+5*perc    value at which legend axis ends         
 ldnum=0.0      numbered tic interval on legend axis (0.0 for automatic)
 lfnum=lmin     first numbered tic on legend axis (used if d1num not 0.0)
 lntic=1        number of tics per numbered tic on legend axis 
 lgrid=none     grid lines on legend axis - none, dot, dash, or solid

 All color specifications may also be made in X Window style Hex format
 example:   axescolor=#255						

 Legal font names are:							
 AvantGarde-Book AvantGarde-BookOblique AvantGarde-Demi AvantGarde-DemiOblique"
 Bookman-Demi Bookman-DemiItalic Bookman-Light Bookman-LightItalic 
 Courier Courier-Bold Courier-BoldOblique Courier-Oblique 
 Helvetica Helvetica-Bold Helvetica-BoldOblique Helvetica-Oblique 
 Helvetica-Narrow Helvetica-Narrow-Bold Helvetica-Narrow-BoldOblique 
 Helvetica-Narrow-Oblique NewCentrySchlbk-Bold"
 NewCenturySchlbk-BoldItalic NewCenturySchlbk-Roman Palatino-Bold  
 Palatino-BoldItalic Palatino-Italics Palatino-Roman 
 SanSerif-Bold SanSerif-BoldItalic SanSerif-Roman 
 Symbol Times-Bold Times-BoldItalic 
 Times-Roman Times-Italic ZapfChancery-MediumItalic 
 Note: The values of x1beg=x1min, x2end=x2max and x3beg=x3min cannot   
 be changed.								

\end{verbatim}
\pagebreak
\begin{verbatim}
 PSEPSI - add an EPSI formatted preview bitmap to an EPS file		

 psepsi <epsfile >epsifile						

 Note:									
 This application requires						
 (1) that gs (the Ghostscript interpreter) exist, and			
 (2) that the input EPS file contain a BoundingBox and EndComments.	
 Ghostscript is used to build the preview bitmap, which is then		
 merged with the input EPS file to make the output EPSI file.		

\end{verbatim}
\pagebreak
\begin{verbatim}
 PSGRAPH - PostScript GRAPHer						
 Graphs n[i] pairs of (x,y) coordinates, for i = 1 to nplot.		

 psgraph n= [optional parameters] <binaryfile >postscriptfile		

 Required Parameters:							
 n                      array containing number of points per plot	

 Data formats supported:						
	1.a. x1,y1,x2,y2,...,xn,yn					
	  b. x1,x2,...,xn,y1,y2,...,yn (must set pairs=0)		
	2.   y1,y2,...,yn (must give non-zero d1[]=)			
	3.   x1,x2,...,xn (must give non-zero d2[]=)			
	4.   nil (must give non-zero d1[]= and non-zero d2[]=)		
  The formats may be repeated and mixed in any order, but if		
  formats 2-4 are used, the d1 and d2 arrays must be specified including
  d1[]=0.0 d2[]=0.0 entries for any internal occurences of format 1.	
  Similarly, the pairs array must contain place-keeping entries for	
  plots of formats 2-4 if they are mixed with both formats 1.a and 1.b.
  Also, if formats 2-4 are used with non-zero f1[] or f2[] entries, then
  the corresponding array(s) must be fully specified including f1[]=0.0
  and/or f2[]=0.0 entries for any internal occurences of format 1 or	
  formats 2-4 where the zero entries are desired.			

  Available colors are all the common ones and many more. The complete	
  list of 68 colors is in the file $CWPROOT/src/psplot/basic.c.	

 Optional Parameters:							
 nplot=number of n's    number of plots				
 d1=0.0,...             x sampling intervals (0.0 if x coordinates input)
 f1=0.0,...             first x values (not used if x coordinates input)
 d2=0.0,...             y sampling intervals (0.0 if y coordinates input)
 f2=0.0,...             first y values (not used if y coordinates input)
 pairs=1,...            =1 for data pairs in format 1.a, =0 for format 1.b
 linewidth=1.0,...      line widths (in points) (0.0 for no lines)	
 linegray=0.0,...       line gray levels (black=0.0 to white=1.0)	
 linecolor=none,...     line colors; none means use linegray		
                        Typical use: linecolor=red,yellow,blue,...	
 lineon=1.0,...         length of line segments for dashed lines (in points)
 lineoff=0.0,...        spacing between dashes (0.0 for solid line)	
 mark=0,1,2,3,...       indices of marks used to represent plotted points
 marksize=0.0,0.0,...   size of marks (0.0 for no marks)		
 xbox=1.5               offset in inches of left side of axes box	
 ybox=1.5               offset in inches of bottom side of axes box	
 wbox=6.0               width in inches of axes box			
 hbox=8.0               height in inches of axes box			
 x1beg=x1min            value at which axis 1 begins			
 x1end=x1max            value at which axis 1 ends			
 d1num=0.0              numbered tic interval on axis 1 (0.0 for automatic)
 f1num=x1min            first numbered tic on axis 1 (used if d1num not 0.0)
 n1tic=1                number of tics per numbered tic on axis 1	
 grid1=none             grid lines on axis 1 - none, dot, dash, or solid
 label1=                label on axis 1				
 x2beg=x2min            value at which axis 2 begins			
 x2end=x2max            value at which axis 2 ends			
 d2num=0.0              numbered tic interval on axis 2 (0.0 for automatic)
 f2num=x2min            first numbered tic on axis 2 (used if d2num not 0.0)
 n2tic=1                number of tics per numbered tic on axis 2	
 grid2=none             grid lines on axis 2 - none, dot, dash, or solid
 label2=                label on axis 2				
 labelfont=Helvetica    font name for axes labels			
 labelsize=18           font size for axes labels			
 title=                 title of plot					
 titlefont=Helvetica-Bold font name for title				
 titlesize=24           font size for title				
 titlecolor=black       color of title					
 axescolor=black        color of axes					
 gridcolor=black        color of grid					
 axeswidth=1            width (in points) of axes			
 ticwidth=axeswidth     width (in points) of tic marks		
 gridwidth=axeswidth    width (in points) of grid lines		
 style=normal           normal (axis 1 horizontal, axis 2 vertical) or	
                        seismic (axis 1 vertical, axis 2 horizontal)	
 reverse=0              =1 to reverse sequence of plotting curves      ",             /* JGHACK
 Note:	n1 and n2 are acceptable aliases for n and nplot, respectively.	

 mark index:                                                           
 1. asterisk                                                           
 2. x-cross                                                            
 3. open triangle                                                      
 4. open square                                                        
 5. open circle                                                        
 6. solid triangle                                                     
 7. solid square                                                       
 8. solid circle                                                       

 All color specifications may also be made in X Window style Hex format
 example:   axescolor=#255						

 Example:								
 psgraph n=50,100,20 d1=2.5,1,0.33 <datafile >psfile			
  plots three curves with equally spaced x values in one plot frame	
  x1-coordinates are x1(i) = f1+i*d1 for i = 1 to n (f1=0 by default)	
  number of x2's and then x2-coordinates for each curve are read	
  sequentially from datafile.						

 Legal font names are:							
 AvantGarde-Book AvantGarde-BookOblique AvantGarde-Demi AvantGarde-DemiOblique"
 Bookman-Demi Bookman-DemiItalic Bookman-Light Bookman-LightItalic 
 Courier Courier-Bold Courier-BoldOblique Courier-Oblique 
 Helvetica Helvetica-Bold Helvetica-BoldOblique Helvetica-Oblique 
 Helvetica-Narrow Helvetica-Narrow-Bold Helvetica-Narrow-BoldOblique 
 Helvetica-Narrow-Oblique NewCentrySchlbk-Bold"
 NewCenturySchlbk-BoldItalic NewCenturySchlbk-Roman Palatino-Bold  
 Palatino-BoldItalic Palatino-Italics Palatino-Roman 
 SanSerif-Bold SanSerif-BoldItalic SanSerif-Roman 
 Symbol Times-Bold Times-BoldItalic 
 Times-Roman Times-Italic ZapfChancery-MediumItalic 
\end{verbatim}
\pagebreak
\begin{verbatim}
 PSIMAGE - PostScript IMAGE plot of a uniformly-sampled function f(x1,x2)
            with the option to display a second attribute		

 psimage n1= [optional parameters] <binaryfile >postscriptfile	

 Required Parameters:							
 n1			 number of samples in 1st (fast) dimension	

 Optional Parameters:							
 d1=1.0		 sampling interval in 1st dimension		
 f1=0.0		 first sample in 1st dimension			
 n2=all		 number of samples in 2nd (slow) dimension	
 d2=1.0		 sampling interval in 2nd dimension		
 f2=0.0		 first sample in 2nd dimension			
 perc=100.0		 percentile used to determine clip		
 clip=(perc percentile) clip used to determine bclip and wclip		
 bperc=perc		 percentile for determining black clip value	
 wperc=100.0-perc	 percentile for determining white clip value	
 bclip=clip		 data values outside of [bclip,wclip] are clipped
 wclip=-clip		 data values outside of [bclip,wclip] are clipped
                        bclip and wclip will be set to be inside       
                        [lbeg,lend] if lbeg and/or lend are supplied   
 threecolor=1		 supply 3 color values instead of only two,	
                        using not only black and white, but f.e. red,	
                        green and blue					
 brgb=0.0,0.0,0.0	 red, green, blue values corresponding to black	
 grgb=1.0,1.0,1.0	 red, green, blue values corresponding to grey	
 wrgb=1.0,1.0,1.0	 red, green, blue values corresponding to white	
 bhls=0.0,0.0,0.0	 hue, lightness, saturation corresponding to black
 ghls=0.0,1.0,0.0	 hue, lightness, saturation corresponding to grey
 whls=0.0,1.0,0.0	 hue, lightness, saturation corresponding to white
 bps=12		 bits per sample for color plots, either 12 or 24
 d1s=1.0		 factor by which to scale d1 before imaging	
 d2s=1.0		 factor by which to scale d2 before imaging	
 verbose=1		 =1 for info printed on stderr (0 for no info)	
 xbox=1.5		 offset in inches of left side of axes box	
 ybox=1.5		 offset in inches of bottom side of axes box	
 width=6.0		 width in inches of axes box			
 height=8.0		 height in inches of axes box			
 x1beg=x1min		 value at which axis 1 begins			
 x1end=x1max		 value at which axis 1 ends			
 d1num=0.0		 numbered tic interval on axis 1 (0.0 for automatic)
 f1num=x1min		 first numbered tic on axis 1 (used if d1num not 0.0)
 n1tic=1		 number of tics per numbered tic on axis 1	
 grid1=none		 grid lines on axis 1 - none, dot, dash, or solid
 label1=		 label on axis 1				
 x2beg=x2min		 value at which axis 2 begins			
 x2end=x2max		 value at which axis 2 ends			
 d2num=0.0		 numbered tic interval on axis 2 (0.0 for automatic)
 f2num=x2min		 first numbered tic on axis 2 (used if d2num not 0.0)
 n2tic=1		 number of tics per numbered tic on axis 2	
 grid2=none		 grid lines on axis 2 - none, dot, dash, or solid
 label2=		 label on axis 2				
 labelfont=Helvetica	 font name for axes labels			
 labelsize=18		 font size for axes labels			
 title=		 title of plot					
 titlefont=Helvetica-Bold font name for title				
 titlesize=24		  font size for title				
 titlecolor=black	 color of title					
 axescolor=black	 color of axes					
 gridcolor=black	 color of grid					
 axeswidth=1            width (in points) of axes                      
 ticwidth=axeswidth     width (in points) of tic marks			
 gridwidth=axeswidth    width (in points) of grid lines		
 style=seismic		 normal (axis 1 horizontal, axis 2 vertical) or	
			 seismic (axis 1 vertical, axis 2 horizontal)	
 legend=0	         =1 display the color scale			
 lnice=0                =1 nice legend arrangement                     
                        (overrides ybox,lx,width,height parameters)    
 lstyle=vertleft 	Vertical, axis label on left side   		
			 =vertright (Vertical, axis label on right side)
			 =horibottom (Horizontal, axis label on bottom)	
 units=		 unit label for legend				
 legendfont=times_roman10    font name for title			
 following are defaults for lstyle=0. They are changed for other lstyles
 lwidth=1.2		 colorscale (legend) width in inches 		
 lheight=height/3     	 colorscale (legend) height in inches		
 lx=1.0		 colorscale (legend) x-position in inches	
 ly=(height-lheight)/2+xybox    colorscale (legend) y-position in pixels
 lbeg= lmin or wclip-5*perc    value at which legend axis begins	
 lend= lmax or bclip+5*perc    value at which legend axis ends        	
 ldnum=0.0	 numbered tic interval on legend axis (0.0 for automatic)
 lfnum=lmin	 first numbered tic on legend axis (used if d1num not 0.0)
 lntic=1	 number of tics per numbered tic on legend axis 
 lgrid=none	 grid lines on legend axis - none, dot, dash, or solid

 curve=curve1,curve2,...  file(s) containing points to draw curve(s)   
 npair=n1,n2,n2,...            number(s) of pairs in each file         
 curvecolor=black,..	 color of curve(s)				
 curvewidth=axeswidth	 width (in points) of curve(s)			
 curvedash=0            solid curve(s), dash indices 1,...,11 produce  
                        curve(s) with various dash styles              

 infile=none            filename of second attribute n1xn2 array       
                        values must be from range 0.0 - 1.0            
                        (plain unformatted C-style file)               
 bckgr=0.5              background gray value				

 NOTES:								
 The curve file is an ascii file with the points specified as x1 x2 	
 pairs, one pair to a line.  A "vector" of curve files and curve	
 colors may be specified as curvefile=file1,file2,etc.			
 and curvecolor=color1,color2,etc, and the number of pairs of values   
 in each file as npair=npair1,npair2,... .				

 You may eliminate the blocky appearance of psimages by adjusting the  
 d1s= and d2s= parameters, so that psimages appear similar to ximages.	

 All color specifications may also be made in X Window style Hex format
 example:   axescolor=#255						

 Some example colormap settings:					
 red white blue: wrgb=1.0,0,0 grgb=1.0,1.0,1.0 brgb=0,0,1.0 		
 white red blue: wrgb=1.0,1.0,1.0 grgb=1.0,0.0,0.0 brgb=0,0,1.0 	
 blue red white: wrgb=0.0,0.0,1.0 grgb=1.0,0.0,0.0 brgb=1.0,1.0,1.0 	
 red green blue: wrgb=1.0,0,0 grgb=0,1.0,0 brgb=0,0,1.0		
 orange light-blue green: wrgb=1.0,.5,0 grgb=0,.7,1.0 brgb=0,1.0,0	
 red light-blue dark blue: wrgb=0.0,0,1.0 grgb=0,1.0,1.0 brgb=0,0,1.0 	

 Legal font names are:							
 AvantGarde-Book AvantGarde-BookOblique AvantGarde-Demi AvantGarde-DemiOblique"
 Bookman-Demi Bookman-DemiItalic Bookman-Light Bookman-LightItalic 
 Courier Courier-Bold Courier-BoldOblique Courier-Oblique 
 Helvetica Helvetica-Bold Helvetica-BoldOblique Helvetica-Oblique 
 Helvetica-Narrow Helvetica-Narrow-Bold Helvetica-Narrow-BoldOblique 
 Helvetica-Narrow-Oblique NewCentrySchlbk-Bold"
 NewCenturySchlbk-BoldItalic NewCenturySchlbk-Roman Palatino-Bold  
 Palatino-BoldItalic Palatino-Italics Palatino-Roman 
 SanSerif-Bold SanSerif-BoldItalic SanSerif-Roman 
 Symbol Times-Bold Times-BoldItalic 
 Times-Roman Times-Italic ZapfChancery-MediumItalic 



 AUTHOR:  Dave Hale, Colorado School of Mines, 05/29/90
 MODIFIED:  Craig Artley, Colorado School of Mines, 08/30/91
	    BoundingBox moved to top of PostScript output
 MODIFIED:  Craig Artley, Colorado School of Mines, 12/16/93
	    Added color options (Courtesy of Dave Hale, Advance Geophysical).
 Modified: Morten Wendell Pedersen, Aarhus University, 23/3-97
           Added ticwidth,axeswidth, gridwidth parameters 
 MODIFIED: Torsten Schoenfelder, Koeln, Germany 006/07/97
          colorbar (legend) (as in ximage (by Berend Scheffers, Delft))
 MODIFIED: Brian K. Macy, Phillips Petroleum, 01/14/99
	    Added curve plotting option
 MODIFIED: Torsten Schoenfelder, Koeln, Germany 02/10/99
          color scale with interpolation of three colors
 MODIFIED: Ekkehart Tessmer, University of Hamburg, Germany, 08/22/2007
          Added dashing option to curve plotting

\end{verbatim}
\pagebreak
\begin{verbatim}
 PSLABEL - output PostScript file consisting of a single TEXT string	
          on a specified background. (Use with psmerge to label plots.)

 pslabel t= [t=] [optional parameters] > epsfile			

Required Parameters (can have multiple specifications to mix fonts):	
  t=                 text string to write to output			

Optional Parameters:							
  f=Times-Bold       font for text string				
                      (multiple specifications for each t)		
  size=30            size of characters in points (72 points/inch)	
  tcolor=black       color of text string				
  bcolor=white       color of background box				
  nsub=0             number of characters to subtract when		
                     computing size of background box (not all		
                     characters are the same size so the		
                     background box may be too big at times.)		

 Example:								
 pslabel t="(a) " f=Times-Bold t="h" f=Symbol t="=0.04" nsub=3 > epsfile

 This example yields the PostScript equivalent of the string		
  (written here in LaTeX notation) $ (a)\\; \\eta=0.04 $		

 Notes:								
 This program produces a (color if desired) PostScript text string that
 can be positioned and pasted on a PostScript plot using   psmerge 	
     see selfdoc of   psmerge for further information.			

 Possible fonts:   Helvetica, Helvetica-Oblique, Helvetica-Bold,	
  Helvetica-BoldOblique,Times-Roman,Times-Italic,Times-Bold,		
  Times-BoldItalic,Courier,Courier-Bold,Courier-Oblique,		
  Courier-BoldOblique,Symbol						

 Possible colors:  greenyellow,yellow,goldenrod,dandelion,apricot,	
  peach,melon,yelloworange,orange,burntorange,bittersweet,redorange,	
  mahogany,maroon,brickred,red,orangered,rubinered,wildstrawberry,	
  salmon,carnationpink,magenta,violetred,rhodamine,mulberry,redviolet,	
  fuchsia,lavender,thistle,orchid,darkorchid,purple,plum,violet,royalpurple,
  blueviolet,periwinkle,cadetblue,cornflowerblue,midnightblue,naveblue,
  royalblue,blue,cerulean,cyan,processblue,skyblue,turquoise,tealblue,	
  aquamarine,bluegreen,emerald,junglegreen,seagreen,green,forestgreen,	
  pinegreen,limegreen,yellowgreen,springgreen,olivegreen,rawsienna,sepia,
  brown,tan,white,black,gray						

 All color specifications may also be made in X Window style Hex format
 example:   tcolor=#255						

 Legal font names are:							
 AvantGarde-Book AvantGarde-BookOblique AvantGarde-Demi AvantGarde-DemiOblique"
 Bookman-Demi Bookman-DemiItalic Bookman-Light Bookman-LightItalic 
 Courier Courier-Bold Courier-BoldOblique Courier-Oblique 
 Helvetica Helvetica-Bold Helvetica-BoldOblique Helvetica-Oblique 
 Helvetica-Narrow Helvetica-Narrow-Bold Helvetica-Narrow-BoldOblique 
 Helvetica-Narrow-Oblique NewCentrySchlbk-Bold"
 NewCenturySchlbk-BoldItalic NewCenturySchlbk-Roman Palatino-Bold  
 Palatino-BoldItalic Palatino-Italics Palatino-Roman 
 SanSerif-Bold SanSerif-BoldItalic SanSerif-Roman 
 Symbol Times-Bold Times-BoldItalic 
 Times-Roman Times-Italic ZapfChancery-MediumItalic 




 AUTHOR:  John E. Anderson, Visiting Scientist from Mobil, 1994

\end{verbatim}
\pagebreak
\begin{verbatim}
 PSMANAGER - printer MANAGER for HP 4MV and HP 5Si Mx Laserjet 
                PostScript printing				

   psmanager < stdin  [optional parameters] > stdout 		

 Required Parameters:						
  none 							
 Optional Parameters:						
 papersize=0	paper size  (US Letter default)			
 		=1       US Legal				
 		=2	 A4					
 		=3     	 11x17					

 orient=0	paper orientation (Portrait default)		
  		=1   	Landscape				

 tray=3        printing tray (Bottom tray default)		
  		=1	tray 1 (multipurpose slot)		
  		=2	tray 2 					

 manual=0	no manual feed 					
  		=1     (Manual Feed)				

 media=0	regular paper					
  		=1     Transparency				
  		=2     Letterhead				
  		=3     Card Stock				
  		=4     Bond					
  		=5     Labels					
  		=6     Prepunched				
  		=7     Recyled					
  		=8     Preprinted				
  		=9     Color (printing on colored paper)	

 Notes: 							
 The option manual=1 implies tray=1. The media options apply	
 only to the HP LaserJet 5Si MX model printer.			

 Examples: 							
   overheads:							
    psmanager <  postscript_file manual=1 media=1 | lpr	
   labels:							
    psmanager <  postscript_file manual=1 media=5 | lpr	



 Notes:  This code was reverse engineered using output from
         the NeXTStep  printer manager.
 
 Author:  John Stockwell, June 1995, October 1997
 
 Reference:   
		PostScript Printer Description File Format Specification,
		version 4.2, Adobe Systems Incorporated

\end{verbatim}
\pagebreak
\begin{verbatim}
 PSMERGE - MERGE PostScript files					

 psmerge in= [optional parameters] >postscriptfile			

 Required Parameters:							
 in=                    postscript file to merge			

 Optional Parameters:							
 origin=0.0,0.0         x,y origin in inches				
 scale=1.0,1.0          x,y scale factors				
 rotate=0.0             rotation angle in degrees			
 translate=0.0,0.0      x,y translation in inches			

 Notes:								
 More than one set of in, origin, scale, rotate, and translate		
 parameters may be specified.  Output x and y coordinates are		
 determined by:							
          x = tx + (x-ox)*sx*cos(d) - (y-oy)*sy*sin(d)			
          y = ty + (x-ox)*sx*sin(d) + (y-oy)*sy*cos(d)			
 where tx,ty are translate coordinates, ox,oy are origin coordinates,	
 sx,sy are scale factors, and d is the rotation angle.  Note that the	
 order of operations is shift (origin), scale, rotate, and translate.	

 If the number of occurrences of a given parameter is less than the number
 of input files, then the last occurrence of that parameter will apply to
 all subsequent files.							

\end{verbatim}
\pagebreak
\begin{verbatim}
 PSMOVIE - PostScript MOVIE plot of a uniformly-sampled function f(x1,x2,x3)

 psmovie n1= [optional parameters] <binaryfile >postscriptfile		

 Required Parameters:							
 n1                     number of samples in 1st (fast) dimension	

 Optional Parameters:							
 d1=1.0                 sampling interval in 1st dimension		
 f1=0.0                 first sample in 1st dimension			
 n2=all                 number of samples in 2nd (slow) dimension	
 d2=1.0                 sampling interval in 2nd dimension		
 f2=0.0                 first sample in 2nd dimension			
 perc=100.0             percentile used to determine clip		
 clip=(perc percentile) clip used to determine bclip and wclip		
 bperc=perc             percentile for determining black clip value	
 wperc=100.0-perc       percentile for determining white clip value	
 bclip=clip             data values outside of [bclip,wclip] are clipped
 wclip=-clip            data values outside of [bclip,wclip] are clipped
 d1s=1.0                factor by which to scale d1 before imaging	
 d2s=1.0                factor by which to scale d2 before imaging	
 verbose=1              =1 for info printed on stderr (0 for no info)	
 xbox=1.0               offset in inches of left side of axes box	
 ybox=1.5               offset in inches of bottom side of axes box	
 wbox=6.0               width in inches of axes box			
 hbox=8.0               height in inches of axes box			
 x1beg=x1min            value at which axis 1 begins			
 x1end=x1max            value at which axis 1 ends			
 d1num=0.0              numbered tic interval on axis 1 (0.0 for automatic)
 f1num=x1min            first numbered tic on axis 1 (used if d1num not 0.0)
 n1tic=1                number of tics per numbered tic on axis 1	
 grid1=none             grid lines on axis 1 - none, dot, dash, or solid
 label1=                label on axis 1				
 x2beg=x2min            value at which axis 2 begins			
 x2end=x2max            value at which axis 2 ends			
 d2num=0.0              numbered tic interval on axis 2 (0.0 for automatic)
 f2num=x2min            first numbered tic on axis 2 (used if d2num not 0.0)
 n2tic=1                number of tics per numbered tic on axis 2	
 grid2=none             grid lines on axis 2 - none, dot, dash, or solid
 label2=                label on axis 2				
 labelfont=Helvetica    font name for axes labels			
 labelsize=18           font size for axes labels			
 title=                 title of plot					
 titlefont=Helvetica-Bold font name for title				
 titlesize=24           font size for title				
 style=seismic          normal (axis 1 horizontal, axis 2 vertical) or	
                        seismic (axis 1 vertical, axis 2 horizontal)	
 n3=1                   number of samples in third dimension		
 title2=                second title to annotate different frames	
 loopdsp=3              display loop type (1=loop over n1; 2=loop over n2;
                                           3=loop over n3)		
 d3=1.0                 sampling interval in 3rd dimension		
 f3=d3                  first sample in 3rd dimension			

 NeXT: view movie via:   psmovie < infile n1= [optional params...] | open
 Note: currently only the Preview Application can handle the multipage  
       PostScript output by this program.				

 All color specifications may also be made in X Window style Hex format
 example:   axescolor=#255						

 Legal font names are:							
 AvantGarde-Book AvantGarde-BookOblique AvantGarde-Demi AvantGarde-DemiOblique"
 Bookman-Demi Bookman-DemiItalic Bookman-Light Bookman-LightItalic 
 Courier Courier-Bold Courier-BoldOblique Courier-Oblique 
 Helvetica Helvetica-Bold Helvetica-BoldOblique Helvetica-Oblique 
 Helvetica-Narrow Helvetica-Narrow-Bold Helvetica-Narrow-BoldOblique 
 Helvetica-Narrow-Oblique NewCentrySchlbk-Bold"
 NewCenturySchlbk-BoldItalic NewCenturySchlbk-Roman Palatino-Bold  
 Palatino-BoldItalic Palatino-Italics Palatino-Roman 
 SanSerif-Bold SanSerif-BoldItalic SanSerif-Roman 
 Symbol Times-Bold Times-BoldItalic 
 Times-Roman Times-Italic ZapfChancery-MediumItalic 

\end{verbatim}
\pagebreak
\begin{verbatim}
 PSWIGB - PostScript WIGgle-trace plot of f(x1,x2) via Bitmap		
 Best for many traces.  Use PSWIGP (Polygon version) for few traces.	

 pswigb n1= [optional parameters] <binaryfile >postscriptfile		

 Required Parameters:							
 n1                     number of samples in 1st (fast) dimension	

 Optional Parameters:							
 d1=1.0                 sampling interval in 1st dimension		
 f1=0.0                 first sample in 1st dimension			
 n2=all                 number of samples in 2nd (slow) dimension	
 d2=1.0                 sampling interval in 2nd dimension		
 f2=0.0                 first sample in 2nd dimension			
 x2=f2,f2+d2,...        array of sampled values in 2nd dimension	
 bias=0.0               data value corresponding to location along axis 2
 perc=100.0             percentile for determining clip		
 clip=(perc percentile) data values < bias+clip and > bias-clip are clipped
 xcur=1.0               wiggle excursion in traces corresponding to clip
 wt=1                   =0 for no wiggle-trace; =1 for wiggle-trace	
 va=1                   =0 for no variable-area; =1 for variable-area fill
                        =2 for variable area, solid/grey fill          
                        SHADING: 2<= va <=5  va=2 lightgrey, va=5 black", 
 nbpi=72                number of bits per inch at which to rasterize	
 verbose=1              =1 for info printed on stderr (0 for no info)	
 xbox=1.5               offset in inches of left side of axes box	
 ybox=1.5               offset in inches of bottom side of axes box	
 wbox=6.0               width in inches of axes box			
 hbox=8.0               height in inches of axes box			
 x1beg=x1min            value at which axis 1 begins			
 x1end=x1max            value at which axis 1 ends			
 d1num=0.0              numbered tic interval on axis 1 (0.0 for automatic)
 f1num=x1min            first numbered tic on axis 1 (used if d1num not 0.0)
 n1tic=1                number of tics per numbered tic on axis 1	
 grid1=none             grid lines on axis 1 - none, dot, dash, or solid
 label1=                label on axis 1				
 x2beg=x2min            value at which axis 2 begins			
 x2end=x2max            value at which axis 2 ends			
 d2num=0.0              numbered tic interval on axis 2 (0.0 for automatic)
 f2num=x2min            first numbered tic on axis 2 (used if d2num not 0.0)
 n2tic=1                number of tics per numbered tic on axis 2	
 grid2=none             grid lines on axis 2 - none, dot, dash, or solid
 label2=                label on axis 2				
 labelfont=Helvetica    font name for axes labels			
 labelsize=18           font size for axes labels			
 title=                 title of plot					
 titlefont=Helvetica-Bold font name for title				
 titlesize=24           font size for title				
 titlecolor=black       color of title					
 axescolor=black        color of axes					
 gridcolor=black        color of grid					
 axeswidth=1            width (in points) of axes			
 ticwidth=axeswidth     width (in points) of tic marks		
 gridwidth=axeswidth    width (in points) of grid lines		
 style=seismic          normal (axis 1 horizontal, axis 2 vertical) or	
                        seismic (axis 1 vertical, axis 2 horizontal)	
 interp=0		 no display interpolation			
			 =1 use 8 point sinc interpolation		
 curve=curve1,curve2,...  file(s) containing points to draw curve(s)   
 npair=n1,n2,n2,...            number(s) of pairs in each file         
 curvecolor=black,..    color of curve(s)                              
 curvewidth=axeswidth   width (in points) of curve(s)                  
 curvedash=0            solid curve(s), dash indices 1,...,11 produce  
                        curve(s) with various dash styles              

 Notes: 								
 The interp option may be useful for high nbpi values, however, it	
 tacitly assumes that the data are purely oscillatory.	Non-oscillatory	
 data will not be represented correctly when this option is set.	

 The curve file is an ascii file with the points specified as x1 x2	
 pairs, one pair to a line.  A "vector" of curve files and curve	
 colors may be specified as curvefile=file1,file2,etc. and 		
 curvecolor=color1,color2,etc, and the number of pairs of values in each
 file as npair=npair1,npair2,... .					

 All color specifications may also be made in X Window style Hex format
 example:   axescolor=#255						

 Legal font names are:							
 AvantGarde-Book AvantGarde-BookOblique AvantGarde-Demi AvantGarde-DemiOblique"
 Bookman-Demi Bookman-DemiItalic Bookman-Light Bookman-LightItalic 
 Courier Courier-Bold Courier-BoldOblique Courier-Oblique 
 Helvetica Helvetica-Bold Helvetica-BoldOblique Helvetica-Oblique 
 Helvetica-Narrow Helvetica-Narrow-Bold Helvetica-Narrow-BoldOblique 
 Helvetica-Narrow-Oblique NewCentrySchlbk-Bold"
 NewCenturySchlbk-BoldItalic NewCenturySchlbk-Roman Palatino-Bold  
 Palatino-BoldItalic Palatino-Italics Palatino-Roman 
 SanSerif-Bold SanSerif-BoldItalic SanSerif-Roman 
 Symbol Times-Bold Times-BoldItalic 
 Times-Roman Times-Italic ZapfChancery-MediumItalic 

\end{verbatim}
\pagebreak
\begin{verbatim}
 PSWIGP - PostScript WIGgle-trace plot of f(x1,x2) via Polygons	
 Best for few traces.  Use PSWIGB (Bitmap version) for many traces.	

 pswigp n1= [optional parameters] <binaryfile >postscriptfile		

 Required Parameters:							
 n1                     number of samples in 1st (fast) dimension	

 Optional Parameters:							
 d1=1.0                 sampling interval in 1st dimension		
 f1=0.0                 first sample in 1st dimension			
 n2=all                 number of samples in 2nd (slow) dimension	
 d2=1.0                 sampling interval in 2nd dimension		
 f2=0.0                 first sample in 2nd dimension			
 x2=f2,f2+d2,...        array of sampled values in 2nd dimension	
 bias=0.0               data value corresponding to location along axis 2
 perc=100.0             percentile for determining clip		
 clip=(perc percentile) data values < bias+clip and > bias-clip are clipped
 xcur=1.0               wiggle excursion in traces corresponding to clip
 fill=1			=0 for no fill;				
				>0 for pos. fill;			
				<0 for neg. fill			
                               =2 for pos. fill solid, neg. fill grey  
                               =-2for neg. fill solid, pos. fill grey  
                       SHADING: 2<=abs(fill)<=5  2=lightgrey 5=black   
 linewidth=1.0         linewidth in points (0.0 for thinest visible line)
 tracecolor=black       color of traces; should contrast with background
 backcolor=none         color of background; none means no background	
 verbose=1              =1 for info printed on stderr (0 for no info)	
 xbox=1.5               offset in inches of left side of axes box	
 ybox=1.5               offset in inches of bottom side of axes box	
 wbox=6.0               width in inches of axes box			
 hbox=8.0               height in inches of axes box			
 x1beg=x1min            value at which axis 1 begins			
 x1end=x1max            value at which axis 1 ends			
 d1num=0.0              numbered tic interval on axis 1 (0.0 for automatic)
 f1num=x1min            first numbered tic on axis 1 (used if d1num not 0.0)
 n1tic=1                number of tics per numbered tic on axis 1	
 grid1=none             grid lines on axis 1 - none, dot, dash, or solid
 label1=                label on axis 1				
 x2beg=x2min            value at which axis 2 begins			
 x2end=x2max            value at which axis 2 ends			
 d2num=0.0              numbered tic interval on axis 2 (0.0 for automatic)
 f2num=x2min            first numbered tic on axis 2 (used if d2num not 0.0)
 n2tic=1                number of tics per numbered tic on axis 2	
 grid2=none             grid lines on axis 2 - none, dot, dash, or solid
 label2=                label on axis 2				
 labelfont=Helvetica    font name for axes labels			
 labelsize=18           font size for axes labels			
 title=                 title of plot					
 titlefont=Helvetica-Bold font name for title				
 titlesize=24           font size for title				
 titlecolor=black       color of title					
 axescolor=black        color of axes					
 gridcolor=black        color of grid					
 axeswidth=1            width (in points) of axes			
 ticwidth=axeswidth     width (in points) of tic marks		
 gridwidth=axeswidth    width (in points) of grid lines		
 frame=1		 =0 wiggle traces only, no frame		
 style=seismic          normal (axis 1 horizontal, axis 2 vertical) or	
                        seismic (axis 1 vertical, axis 2 horizontal)	

 curve=curve1,curve2,...  file(s) containing points to draw curve(s)   
 npair=n1,n2,n2,...            number(s) of pairs in each file         
 curvecolor=black,..    color of curve(s)                              
 curvewidth=axeswidth   width (in points) of curve(s)                  
 curvedash=0            solid curve(s), dash indices 1,...,11 produce  
                        curve(s) with various dash styles              

 Note:  linewidth=0.0 produces the thinest possible line on the output.	
 device.  Thus the result is device-dependent, put generally looks the	
 best for seismic traces.						

 The curve file is an ascii file with the points specified as x1 x2 pairs,
 one pair to a line.  A "vector" of curve files and curve colors may 
 be specified as curvefile=file1,file2,etc. and similarly		
 curvecolor=color1,color2,etc, and the number of pairs of values	
 in each file as npair=npair1,npair2,... .                             

 All color specifications may also be made in X Window style Hex format
 example:   axescolor=#255						

 Legal font names are:							
 AvantGarde-Book AvantGarde-BookOblique AvantGarde-Demi AvantGarde-DemiOblique"
 Bookman-Demi Bookman-DemiItalic Bookman-Light Bookman-LightItalic 
 Courier Courier-Bold Courier-BoldOblique Courier-Oblique 
 Helvetica Helvetica-Bold Helvetica-BoldOblique Helvetica-Oblique 
 Helvetica-Narrow Helvetica-Narrow-Bold Helvetica-Narrow-BoldOblique 
 Helvetica-Narrow-Oblique NewCentrySchlbk-Bold"
 NewCenturySchlbk-BoldItalic NewCenturySchlbk-Roman Palatino-Bold  
 Palatino-BoldItalic Palatino-Italics Palatino-Roman 
 SanSerif-Bold SanSerif-BoldItalic SanSerif-Roman 
 Symbol Times-Bold Times-BoldItalic 
 Times-Roman Times-Italic ZapfChancery-MediumItalic 

\end{verbatim}
\pagebreak
\begin{verbatim}
 LCMAP - List Color Map of root window of default screen 
 
 Usage:   lcmap 
 
 
\end{verbatim}
\pagebreak
\begin{verbatim}
 LPROP - List PROPerties of root window of default screen of display 

 Usage:  lprop


\end{verbatim}
\pagebreak
\begin{verbatim}
 SCMAP - set default standard color map (RGB_DEFAULT_MAP)

 Usage:   scmap


\end{verbatim}
\pagebreak
\begin{verbatim}
 XCONTOUR - X CONTOUR plot of f(x1,x2) via vector plot call		

 xcontour n1= [optional parameters] <binaryfile [>psplotfile]		

 X Functionality:							
 Button 1	Zoom with rubberband box				
 Button 2	Show mouse (x1,x2) coordinates while pressed		
 q or Q key	Quit 							
 s key		Save current mouse (x1,x2) location to file		
 p or P key	Plot current window with pswigb	(only from disk files)	

 Required Parameters:							
 n1                     number of samples in 1st (fast) dimension	

 Optional Parameters:							
 d1=1.0                 sampling interval in 1st dimension		
 f1=0.0                 first sample in 1st dimension			
 x1=f1,f1+d1,...        array of sampled values in 1nd dimension	
 n2=all                 number of samples in 2nd (slow) dimension	
 d2=1.0                 sampling interval in 2nd dimension		
 f2=0.0                 first sample in 2nd dimension			
 x2=f2,f2+d2,...        array of sampled values in 2nd dimension	
 mpicks=/dev/tty        file to save mouse picks in			
 verbose=1              =1 for info printed on stderr (0 for no info)	
 nc=5                   number of contour values                       
 dc=(zmax-zmin)/nc      contour interval                               
 fc=min+dc              first contour                                  
 c=fc,fc+dc,...         array of contour values                        
 cwidth=1.0,...         array of contour line widths                   
 ccolor=none,...        array of contour colors; none means use cgray  
 cdash=0.0,...          array of dash spacings (0.0 for solid)         
 labelcf=1              first labeled contour (1,2,3,...)              
 labelcper=1            label every labelcper-th contour               
 nlabelc=nc             number of labeled contours (0 no contour label)
 nplaces=6              number of decimal places in contour labeling	
 xbox=50                x in pixels of upper left corner of window	
 ybox=50                y in pixels of upper left corner of window	
 wbox=550               width in pixels of window			
 hbox=700               height in pixels of window			
 x1beg=x1min            value at which axis 1 begins			
 x1end=x1max            value at which axis 1 ends			
 d1num=0.0              numbered tic interval on axis 1 (0.0 for automatic)
 f1num=x1min            first numbered tic on axis 1 (used if d1num not 0.0)
 n1tic=1                number of tics per numbered tic on axis 1	
 grid1=none             grid lines on axis 1 - none, dot, dash, or solid
 x2beg=x2min            value at which axis 2 begins			
 x2end=x2max            value at which axis 2 ends			
 d2num=0.0              numbered tic interval on axis 2 (0.0 for automatic)
 f2num=x2min            first numbered tic on axis 2 (used if d2num not 0.0)
 n2tic=1                number of tics per numbered tic on axis 2	
 grid2=none             grid lines on axis 2 - none, dot, dash, or solid
 label2=                label on axis 2				
 labelfont=Erg14        font name for axes labels			
 title=                 title of plot					
 titlefont=Rom22        font name for title				
 windowtitle=xwigb      title on window				
 labelcolor=blue        color for axes labels				
 titlecolor=red         color for title				
 gridcolor=blue         color for grid lines				
 labelccolor=black      color of contour labels                        ",   
 labelcfont=fixed       font name for contour labels                   
 style=seismic		 normal (axis 1 horizontal, axis 2 vertical) or	
			 seismic (axis 1 vertical, axis 2 horizontal)	


 Notes:								
 For some reason the contour might slight differ from ones generated   
 by pscontour (propably due to the pixel nature of the plot            
 coordinates)                                                          

 The line width of unlabeled contours is designed as a quarter of that	
 of labeled contours. 							


 AUTHOR: Morten Wendell Pedersen, Aarhus University 

 All the coding is based on snippets taken from xwigb, ximage and pscontour
 All I have done is put the parts together and put in some bugs ;-)

 So credits should go to the authors of these packages... 

 Caveats and Notes:
 The code has been developed under Linux 1.3.20/Xfree 3.1.2E (X11 6.1)
 with gcc-2.7.0 But hopefully it should work on other platforms as well

 Since all the contours are drawn by Vector plot call's everytime the
 Window is exposed, the exposing can be darn slow 
 OOPS This should be history... Now I keep my window content with backing
 store so I won't have to redraw my window unless I really have to...

 Portability Question: I guess I should check if the display supports
 backingstore and redraw if it doesn't (see DoesBackingStore(3) )
 I have to be able to use CWBackingStore==Always (other values can be
 NonUseful and WhenMapped

 Since I put the contour labels everytime I draw one contour level the area 
 that contains the label will be crossed by the the next contour lines...
 (this bug also seems to be present in pscontour)
 To fix this I have to redraw all the labels after been through all
 the contour calls
 Right now I can't see a way to fix this without actually to through
 the entire label positioning again....Overkill I would say

 
 The relative short length of the contour segments will propably mask the
 cdash settings
 which means it is disposable (but I know how to draw dashed lines :)
 A way of fixing this could be to get all connected point and then use
 XDrawlines or XDrawSegments... just an idea...No idea if it'll work. 

 I think there is a bug in xContour since my plot coordinates increase
 North and west ward instead of south and eastward

   I need to check the Self Doc so if the right parameters are described
   (I have been through it a couple of times but....)

   All functions need a heavy cleanup for unused variables
   I suppose there is a couple of memory leaks due to missing free'ing of
 numerous pointers (especially fonts,GC's & colors could be a problem...

   I have to browse through the internal pscontour call... basically what
 I have done is just putting pscontour instead of pswigb... Instead of
 repositioning the input file  pointer (which doesnt work with pipes) one
 should consider the use of temporary file
   or write your zoombox to pscontour (...how one does that?)

  Wish List:
   The use of cgray's unused until now... I guess I'll need to allocate
 a gray Colormap  -> meaning that the code not will run at other display
 than 8 bit Pseudocolor :( (with the use of present version of the colormap
 library (code in $CWPROOT/src/xplot/lib ) )

  The format of contour label should be open for the user.. 
  
  It could be nice if one could choose to have a pixmap (like ximage )
 underlying  the contours... this should be defined either by the input
 data  or by a seperate file
  eg useful for viewing traveltime contours on top a plot of the velocity
 field

\end{verbatim}
\pagebreak
\begin{verbatim}
 XESPB - X windows display of Encapsulated PostScript as a single Bitmap

 Usage:   xepsb < stdin

 Caveat: your system must have Display PostScript Graphics

 NOTE: This program is included as a demo of EPS -> X programming.
 See:  xepsp and xpsp these are more advanced versions


\end{verbatim}
\pagebreak
\begin{verbatim}
 XEPSP - X windows display of Encapsulated PostScript

 Usage:   xepsp < stdin

 Caveat: your system must have Display PostScript Graphics
 Note:  
 this program is a more advanced version of   xepsb. See also:  xepsp


\end{verbatim}
\pagebreak
\begin{verbatim}
 XIMAGE - X IMAGE plot of a uniformly-sampled function f(x1,x2)     	

 ximage n1= [optional parameters] <binaryfile			        

 X Functionality:							
 Button 1	Zoom with rubberband box				
 Button 2	Show mouse (x1,x2) coordinates while pressed		
 q or Q key	Quit							
 s key		Save current mouse (x1,x2) location to file		
 p or P key	Plot current window with pswigb (only from disk files)	
 a or page up keys		enhance clipping by 10%			
 c or page down keys		reduce clipping by 10%			
 up,down,left,right keys	move zoom window by half width/height	
 i or +(keypad) 		zoom in by factor 2 			
 o or -(keypad) 		zoom out by factor 2 			

 ... change colormap interactively					
 r	     install next RGB - colormap				
 R	     install previous RGB - colormap				
 h	     install next HSV - colormap				
 H	     install previous HSV - colormap				
 H	     install previous HSV - colormap				
 (Move mouse cursor out and back into window for r,R,h,H to take effect)

 Required Parameters:							
 n1			 number of samples in 1st (fast) dimension	

 Optional Parameters:							
 d1=1.0		 sampling interval in 1st dimension		
 f1=0.0		 first sample in 1st dimension			
 n2=all		 number of samples in 2nd (slow) dimension	
 d2=1.0		 sampling interval in 2nd dimension		
 f2=0.0		 first sample in 2nd dimension			
 mpicks=/dev/tty	 file to save mouse picks in			
 perc=100.0		 percentile used to determine clip		
 clip=(perc percentile) clip used to determine bclip and wclip		
 bperc=perc		 percentile for determining black clip value	
 wperc=100.0-perc	 percentile for determining white clip value	
 bclip=clip		 data values outside of [bclip,wclip] are clipped
 wclip=-clip		 data values outside of [bclip,wclip] are clipped
 balance=0		 bclip & wclip individually			
			 =1 set them to the same abs value		
			   if specified via perc (avoids colorbar skew)	
 cmap=hsv\'n\' or rgb\'m\'	\'n\' is a number from 0 to 13		
				\'m\' is a number from 0 to 11		
				cmap=rgb0 is equal to cmap=gray		
				cmap=hsv1 is equal to cmap=hue		
				(compatibility to older versions)	
 legend=0	        =1 display the color scale			
 units=		unit label for legend				
 legendfont=times_roman10    font name for title			
 verbose=1		=1 for info printed on stderr (0 for no info)	
 xbox=50		x in pixels of upper left corner of window	
 ybox=50		y in pixels of upper left corner of window	
 wbox=550		width in pixels of window			
 hbox=700		height in pixels of window			
 lwidth=16		colorscale (legend) width in pixels		
 lheight=hbox/3	colorscale (legend) height in pixels		
 lx=3			colorscale (legend) x-position in pixels	
 ly=(hbox-lheight)/3   colorscale (legend) y-position in pixels	
 x1beg=x1min		value at which axis 1 begins			
 x1end=x1max		value at which axis 1 ends			
 d1num=0.0		numbered tic interval on axis 1 (0.0 for automatic)
 f1num=x1min		first numbered tic on axis 1 (used if d1num not 0.0)
 n1tic=1		number of tics per numbered tic on axis 1	
 grid1=none		grid lines on axis 1 - none, dot, dash, or solid
 label1=		label on axis 1					
 x2beg=x2min		value at which axis 2 begins			
 x2end=x2max		value at which axis 2 ends			
 d2num=0.0		numbered tic interval on axis 2 (0.0 for automatic)
 f2num=x2min		first numbered tic on axis 2 (used if d2num not 0.0)
 n2tic=1		number of tics per numbered tic on axis 2	
 grid2=none		grid lines on axis 2 - none, dot, dash, or solid
 label2=		label on axis 2					
 labelfont=Erg14	font name for axes labels			
 title=		title of plot					
 titlefont=Rom22	font name for title				
 windowtitle=ximage	title on window					
 labelcolor=blue	color for axes labels				
 titlecolor=red	color for title					
 gridcolor=blue	color for grid lines				
 style=seismic	        normal (axis 1 horizontal, axis 2 vertical) or  
			seismic (axis 1 vertical, axis 2 horizontal)	
 blank=0		This indicates what portion of the lower range  
			to blank out (make the background color).  The  
			value should range from 0 to 1.			
 plotfile=plotfile.ps  filename for interactive ploting (P)  		
 curve=curve1,curve2,...  file(s) containing points to draw curve(s)   
 npair=n1,n2,n2,...            number(s) of pairs in each file         
 curvecolor=color1,color2,...  color(s) for curve(s)                   
 blockinterp=0       whether to use block interpolation (0=no, 1=yes)  


 NOTES:								
 The curve file is an ascii file with the points  specified as x1 x2	
 pairs separated by a space, one pair to a line.  A "vector" of curve
 files and curve colors may be specified as curvefile=file1,file2,etc. 
 and curvecolor=color1,color2,etc, and the number of pairs of values   
 in each file as npair=npair1,npair2,... .                             


 AUTHOR:  Dave Hale, Colorado School of Mines, 08/09/90

 Stewart A. Levin, Mobil - Added ps print option

 Brian Zook, Southwest Research Institute, 6/27/96, added blank option

 Toralf Foerster, Baltic Sea Research Institute, 9/15/96, new colormaps

 Berend Scheffers, Delft, colorbar (legend)

 Brian K. Macy, Phillips Petroleum, 11/27/98, added curve plotting option
 
 G.Klein, GEOMAR Kiel, 2004-03-12, added cursor scrolling and
                                   interactive change of zoom and clipping.
 
 Zhaobo Meng, ConocoPhillips, 12/02/04, added amplitude display
 
 Garry Perratt, Geocon, 08/04/05, modified perc handling to center colorbar if balance==1.

INTL2B_block - blocky interpolation of a 2-D array of bytes

intl2b_block		blocky interpolation of a 2-D array of bytes

Function Prototype:
void intl2b_block(int nxin, float dxin, float fxin,
	int nyin, float dyin, float fyin, unsigned char *zin,
	int nxout, float dxout, float fxout,
	int nyout, float dyout, float fyout, unsigned char *zout);

Input:
nxin		number of x samples input (fast dimension of zin)
dxin		x sampling interval input
fxin		first x sample input
nyin		number of y samples input (slow dimension of zin)
dyin		y sampling interval input
fyin		first y sample input
zin		array[nyin][nxin] of input samples (see notes)
nxout		number of x samples output (fast dimension of zout)
dxout		x sampling interval output
fxout		first x sample output
nyout		number of y samples output (slow dimension of zout)
dyout		y sampling interval output
fyout		first y sample output

Output:
zout		array[nyout][nxout] of output samples (see notes)

Notes:
The arrays zin and zout must passed as pointers to the first element of
a two-dimensional contiguous array of unsigned char values.

Constant extrapolation of zin is used to compute zout for
output x and y outside the range of input x and y.

Author:  James Gunning, CSIRO Petroleum 1999. Hacked from
intl2b() by Dave Hale, Colorado School of Mines, c. 1989-1991
\end{verbatim}
\pagebreak
\begin{verbatim}
 XPICKER - X wiggle-trace plot of f(x1,x2) via Bitmap with PICKing	

 xpicker n1= [optional parameters] <binaryfile				

 X Menu functionality:							
    Pick Filename Window	default is pick_file			
    Load		load an existing Pick Filename			
    Save		save to Pick Filename				
    View only/Pick	default is View, click to enable Picking	
    Add/Delete		default is Add, click to delete picks		
    Cross off/on	default is Cross off, click to enable Crosshairs

 In View mode:								
    a or page up keys          enhance clipping by 10%                 
    c or page down keys        reduce clipping by 10%                  
    up,down,left,right keys    move zoom window by half width/height   
    i or +(keypad)             zoom in by factor 2                     
    o or -(keypad)             zoom out by factor 2                    
    l 				lock the zoom while moving the coursor	
    u 				unlock the zoom 			

 Notes:								
	Menu selections and toggles ("clicks") are made with button 1	
	Pick selections are made with button 3				
	Edit a pick selection by dragging it with button 3 down	or	
	by making a new pick on that trace				
	Reaching the window limits while moving within changes the zoom	
	factor in this direction. The use of zoom locking(l) disables it

 Other X Mouse functionality:						
 Mouse Button 1	Zoom with rubberbox				
 Mouse Button 2	Show mouse (x1,x2) coordinates while pressed	

 The following keys are active in View Only mode:			

 Required Parameters:							
 n1=		number of samples in 1st (fast) dimension		

 Optional Parameters:							
 mpicks=pick_file	name of output (input) pick file		
 d1=1.0		sampling interval in 1st dimension		
 f1=d1		  first sample in 1st dimension				
 n2=all		 number of samples in 2nd (slow) dimension	
 d2=1.0		 sampling interval in 2nd dimension		
 f2=d2		  first sample in 2nd dimension				
 x2=f2,f2+d2,...	array of sampled values in 2nd dimension	
 bias=0.0	       data value corresponding to location along axis 2
 perc=100.0	     percentile for determining clip			
 clip=(perc percentile) data values < bias+clip and > bias-clip are clipped
 xcur=1.0	       wiggle excursion in traces corresponding to clip	
 wt=1		   =0 for no wiggle-trace; =1 for wiggle-trace		
 va=1		   =0 for no variable-area; =1 for variable-area fill	
                        =2 for variable area, solid/grey fill          
                        SHADING: 2<=va<=5  va=2 light grey, va=5 black 
 verbose=1	      =1 for info printed on stderr (0 for no info)	
 xbox=50		x in pixels of upper left corner of window	
 ybox=50		y in pixels of upper left corner of window	
 wbox=550	      	width in pixels of window			
 hbox=700		height in pixels of window			
 x1beg=x1min		value at which axis 1 begins			
 x1end=x1max		value at which axis 1 ends			
 d1num=0.0		 numbered tic interval on axis 1 (0.0 for automatic)
 f1num=x1min		first numbered tic on axis 1 (used if d1num not 0.0)
 n1tic=1		number of tics per numbered tic on axis 1	
 grid1=none		grid lines on axis 1 - none, dot, dash, or solid
 label1=		label on axis 1					
 x2beg=x2min		value at which axis 2 begins			
 x2end=x2max		value at which axis 2 ends			
 d2num=0.0		 numbered tic interval on axis 2 (0.0 for automatic)
 f2num=x2min		first numbered tic on axis 2 (used if d2num not 0.0)
 n2tic=1		number of tics per numbered tic on axis 2	
 grid2=none		grid lines on axis 2 - none, dot, dash, or solid 
 label2=		label on axis 2					
 labelfont=Erg14	font name for axes labels			
 title=		title of plot					
 titlefont=Rom22	font name for title				
 labelcolor=blue	color for axes labels				
 titlecolor=red	color for title					
 gridcolor=blue	color for grid lines				
 style=seismic		normal (axis 1 horizontal, axis 2 vertical) or	
 		    	seismic (axis 1 vertical, axis 2 horizontal)	
 endian=		=0 little endian, =1 big endian			
 interp=0		no sinc interpolation				
			=1 perform sinc interpolation			
 x2file=		file of "acceptable" x2 values		
 x1x2=1		save picks in the order (x1,x2) 		
 			=0 save picks in the order (x2,x1) 		

 Notes:								
 Xpicker will try to detect the endian value of the X-display and will	
 set it to the right value. If it gets obviously wrong information the 
 endian value will be set to the endian value of the machine that is	
 given at compile time as the value of CWPENDIAN defined in cwp.h	
 and set via the compile time flag ENDIANFLAG in Makefile.config.	

 The only time that you might want to change the value of the endian   
 variable is if you are viewing traces on a machine with a different   
 byte order than the machine you are creating the traces on AND if for 
 some reason the automaic detection of the display byte order fails.   
 Set endian to that of the machine you	are viewing the traces on.	

 The interp flag is useful for making better quality wiggle trace for	
 making plots from screen dumps. However, this flag assumes that the	
 data are purely oscillatory. This option may not be appropriate for all
 data sets.								

 If the x2file=  option is set, then the values from the specified file
 will define the set of "acceptable" values of x2 for xpicker to	
 output. The format is a single column of ASCII values. The number of  
 specified values is arbitrary.					

 Such a file can be built from an SU data set via:			
     sugethw < sudata key=offset output=geom > x2example 		

 If the value of x2file= is not set, then				
 xpicker will use the values specified via: x2=.,.,.,. or those that are", 
 computed from the values of f2=  and d2= as being the "acceptible
 values.								

 See the selfdoc of  suxpicker  for information on using key fields from
 the SU trace headers directly. 					



 AUTHOR:  Dave Hale, Colorado School of Mines, 08/09/90
 with picking by Wenying Cai of University of Utah.
 Endian stuff by Morten Pedersen and John Stockwell of CWP.
 Interp stuff by Tony Kocurko of Memorial University of Newfoundland
 Modified to include acceptable values by Bill Lutter of the
     Department of Geology, University of Wisconsin 10/96
 MODIFIED:  P. Michaels, Boise State Univeristy  29 December 2000
            Added solid/grey color scheme for peaks/troughs
 
 G.Klein, IFG Kiel University, 2003-09-29, added cursor scrolling and
            interactive change of zoom and clipping.

 NOTES:
 Interactive picker improved to allow x-axis of picks to be
 coordinated with "key=header" parameter set in driver routine
 suxpicker. Multiple picks per trace are now allowed.

  Input:
  The command line of suxpicker is unchanged.  The parameter"key=header"
  set in  suxpicker controls a) trace x-axis  displayed via xpicker and
  b) the header values in the first column of a pick file either read in
     or written out from xpicker c) header values expected in optional file
      or written out from xpicker c) header values expected in optional file
     x2file= which reads into xpicker allowable trace x-axis values.

   a) example command line:  suxpicker key=offset < shot10.plotpik

   b) pick file format:
	x-axis_value_1 time_1
	x-axis_value_2 time_2  
	x-axis_value_3 time_3
	etc.
	x-axis_value_n time_n

	pick file example:
         1000.000000 0.500000
         2000.000000 1.000000
         3000.000000 1.500000
         4000.000000 2.000000
         5000.000000 2.500000

  c)  format of optional file x2file=:
    	   header_value_1
 	   header_value_2
	   etc.
	   header_val_m

       If file "x2file=" exists in directory from which suxpicker is
      invoked, then these trace header x-axis values are the only allowable
      x-axis pick values used in the pick "add" or "delete" menu operation.
      Header values do not need to be sorted or 1 to 1 with input traces.
      Further, pick file x-axis values can be read into xpicker via load
      operation without having to match key_pickx1_val x-axis values and
      can also be rewritten out an output pickfile.  As indicated, only
      the "add" and "delete" pick operations are influenced by existence
       of this file.

      Offset header values for "x2file=" can be generated by the
      command line:

      sugethw < su_segyfile key=offset output=geom >  x2examplefile=

      Output: Only change is in format of pick_file (format described above).
      If x2file= file exists then x-axis value of added picks
      will be forced to nearest allowable trace x-axis value (input values
      of x2file= file). If x2file= is not set, then the values of x2 
      that are either assigned uniformly to the traces via f2 and d2,
      or by the vector of values of x2=.,.,.,.    will be the "acceptable"
      values.

    Strategy:
   a) malloc() and realloc() used to dynamically allocate memory
	  for array of x-axis value read in from optional file
	  x2file=.  This is done in function read_keyval().

	  b) The pick file dimensions are set in main program via malloc()
	  and then initialized (*apick)[i].picked = FALSE) in function
	  init_picks().  The pick file is declared as pick_t **apick, in
	  order to use realloc() as needed in functions load_picks where the
	  pick file is read in and edit_picks where picks are added.  The
	  call to realloc() and further initializing is performed in
	  function realloc_picks().

	  c) If x2file= file exists the mouse derived x-axis value
	  for a pick to be added is checked against allowable x-axis values
	  to find the closest match via function add_pick called from
	  edit_picks.  If the pick is to be deleted, first a search is done
	  to find the closest x-axis value, then the existing pick values
	  are searched to find the closest radial value (x**2 + t**2) via
	  function del_pick() invoked from edit_picks.

	  d) Code modifications are limited to above mentioned functions,
	  except for additional parameters passed to functions edit_picks,
	  load_picks, save_picks, and check_buttons.
 

\end{verbatim}
\pagebreak
\begin{verbatim}
 XPSP - Display conforming PostScript in an X-window

 xpsp < stdin

 Note: this is the most advanced version of xepsb and xepsp. 
 Caveat: your  system must have Display PostScript Graphics


\end{verbatim}
\pagebreak
\begin{verbatim}
 XWIGB - X WIGgle-trace plot of f(x1,x2) via Bitmap			

 xwigb n1= [optional parameters] <binaryfile   	   

 X Functionality:							
 Button 1	Zoom with rubberband box				
 Button 2	Show mouse (x1,x2) coordinates while pressed		
 q or Q key	Quit							
 s key		Save current mouse (x1,x2) location to file		
 p or P key	Plot current window with pswigb (only from disk files)	
 a or page up keys		enhance clipping by 10%			
 c or page down keys		reduce clipping by 10%			
 up,down,left,right keys	move zoom window by half width/height	
 i or +(keypad) 		zoom in by factor 2 			
 o or -(keypad) 		zoom out by factor 2 			
    l 				lock the zoom while moving the coursor	
    u 				unlock the zoom 			
 1,2,...,9	Zoom/Move factor of the window size			

 Notes:								
	Reaching the window limits while moving within changes the zoom	
	factor in this direction. The use of zoom locking(l) disables it

 Required Parameters:							
 n1			 number of samples in 1st (fast) dimension	

 Optional Parameters:							
 d1=1.0		 sampling interval in 1st dimension		
 f1=0.0		 first sample in 1st dimension			
 n2=all		 number of samples in 2nd (slow) dimension	
 d2=1.0		 sampling interval in 2nd dimension		
 f2=0.0		 first sample in 2nd dimension			
 x2=f2,f2+d2,...	 array of sampled values in 2nd dimension	
 mpicks=/dev/tty	 file to save mouse picks in			
 bias=0.0		 data value corresponding to location along axis 2
 perc=100.0		 percentile for determining clip		
 clip=(perc percentile) data values < bias+clip and > bias-clip are clipped
 xcur=1.0		 wiggle excursion in traces corresponding to clip
 wt=1			 =0 for no wiggle-trace; =1 for wiggle-trace	
 va=1			 =0 for no variable-area; =1 for variable-area fill
                        =2 for variable area, solid/grey fill          
                        SHADING: 2<=va<=5  va=2 light grey, va=5 black 
 verbose=0		 =1 for info printed on stderr (0 for no info)	
 xbox=50		 x in pixels of upper left corner of window	
 ybox=50		 y in pixels of upper left corner of window	
 wbox=550		 width in pixels of window			
 hbox=700		 height in pixels of window			
 x1beg=x1min		 value at which axis 1 begins			
 x1end=x1max		 value at which axis 1 ends			
 d1num=0.0		 numbered tic interval on axis 1 (0.0 for automatic)
 f1num=x1min		 first numbered tic on axis 1 (used if d1num not 0.0)
 n1tic=1		 number of tics per numbered tic on axis 1	
 grid1=none		 grid lines on axis 1 - none, dot, dash, or solid
 x2beg=x2min		 value at which axis 2 begins			
 x2end=x2max		 value at which axis 2 ends			
 d2num=0.0		 numbered tic interval on axis 2 (0.0 for automatic)
 f2num=x2min		 first numbered tic on axis 2 (used if d2num not 0.0)
 n2tic=1		 number of tics per numbered tic on axis 2	
 grid2=none		 grid lines on axis 2 - none, dot, dash, or solid
 label2=		 label on axis 2				
 labelfont=Erg14	 font name for axes labels			
 title=		 title of plot					
 titlefont=Rom22	 font name for title				
 windowtitle=xwigb	 title on window				
 labelcolor=blue	 color for axes labels				
 titlecolor=red	 color for title				
 gridcolor=blue	 color for grid lines				
 style=seismic		 normal (axis 1 horizontal, axis 2 vertical) or 
			 seismic (axis 1 vertical, axis 2 horizontal)	
 endian=		 =0 little endian =1 big endian			
 interp=0		 no interpolation in display			
			 =1 use 8 point sinc interpolation		
 wigclip=0		 If 0, the plot box is expanded to accommodate	
			 the larger wiggles created by xcur>1.	If this 
			 flag is non-zero, the extra-large wiggles are	
			 are clipped at the boundary of the plot box.	
 plotfile=plotfile.ps   filename for interactive ploting (P)  		
 curve=curve1,curve2,...  file(s) containing points to draw curve(s)   
 npair=n1,n2,n2,...            number(s) of pairs in each file         
 curvecolor=color1,color2,...  color(s) for curve(s)                   

 Notes:								
 Xwigb will try to detect the endian value of the X-display and will	
 set it to the right value. If it gets obviously wrong information the 
 endian value will be set to the endian value of the machine that is	
 given at compile time as the value of CWPENDIAN defined in cwp.h	
 and set via the compile time flag ENDIANFLAG in Makefile.config.	

 The only time that you might want to change the value of the endian	
 variable is if you are viewing traces on a machine with a different	
 byte order than the machine you are creating the traces on AND if for 
 some reason the automaic detection of the display byte order fails.	
 Set endian to that of the machine you are viewing the traces on.	

 The interp flag is useful for making better quality wiggle trace for	
 making plots from screen dumps. However, this flag assumes that the	
 data are purely oscillatory. This option may not be appropriate for all
 data sets.								

 The curve file is an ascii file with the points specified as x1 x2    
 pairs, separated by a space, one pair to a line.  A "vector" of curve
 files and curve colors may be specified as curvefile=file1,file2,etc. 
 and curvecolor=color1,color2,etc, and the number of pairs of values   
 in each file as npair=npair1,npair2,... .                             


 AUTHOR:  Dave Hale, Colorado School of Mines, 08/09/90

 Endian stuff by: 
    Morten Wendell Pedersen, Aarhus University (visiting CSM, June 1995)
  & John Stockwell, Colorado School of Mines, 5 June 1995

 Stewart A. Levin, Mobil - Added ps print option
 John Stockwell - Added optional sinc interpolation
 Stewart A. Levin, Mobil - protect title, labels in pswigb call

 Brian J. Zook, SwRI - Added style=normal and wigclip flag

 Brian K. Macy, Phillips Petroleum, 11/27/98, added curve plotting option
 Curve plotting notes:
 MODIFIED:  P. Michaels, Boise State Univeristy  29 December 2000
            Added solid/grey color scheme for peaks/troughs
 
 G.Klein, IFG Kiel University, 2002-09-29, added cursor scrolling and
            interactive change of zoom and clipping.
          IFM-GEOMAR Kiel, 2004-03-12, added zoom locking 
          IFM-GEOMAR Kiel, 2004-03-25, interactive plotting fixed 

\end{verbatim}
\pagebreak
\begin{verbatim}
 XGRAPH - X GRAPHer							
 Graphs n[i] pairs of (x,y) coordinates, for i = 1 to nplot.		

 xgraph n= [optional parameters] <binaryfile 				

 X Functionality:                                                      
 q or Q key    Quit                                                    

 Required Parameters:							
 n                      array containing number of points per plot	

 Optional Parameters:							
 nplot=number of n's    number of plots				
 d1=0.0,...             x sampling intervals (0.0 if x coordinates input)
 f1=0.0,...             first x values (not used if x coordinates input)
 d2=0.0,...             y sampling intervals (0.0 if y coordinates input)
 f2=0.0,...             first y values (not used if y coordinates input)
 pairs=1,...            =1 for data pairs in format 1.a, =0 for format 1.b
 linewidth=1,1,...      line widths in pixels (0 for no lines)		
 linecolor=2,3,...      line colors (black=0, white=1, 2,3,4 = RGB, ...)
 mark=0,1,2,3,...       indices of marks used to represent plotted points
 marksize=0,0,...       size of marks in pixels (0 for no marks)	
 x1beg=x1min            value at which axis 1 begins			
 x1end=x1max            value at which axis 1 ends			
 x2beg=x2min            value at which axis 2 begins			
 x2end=x2max            value at which axis 2 ends			
 reverse=0              =1 to reverse sequence of plotting curves	", 

 Optional resource parameters (defaults taken from resource database):	
 windowtitle=      	 title on window				
 width=                 width in pixels of window			
 height=                height in pixels of window			
 nTic1=                 number of tics per numbered tic on axis 1	
 grid1=                 grid lines on axis 1 - none, dot, dash, or solid
 label1=                label on axis 1				
 nTic2=                 number of tics per numbered tic on axis 2	
 grid2=                 grid lines on axis 2 - none, dot, dash, or solid
 label2=                label on axis 2				
 labelFont=             font name for axes labels			
 title=                 title of plot					
 titleFont=             font name for title				
 titleColor=            color for title				
 axesColor=             color for axes					
 gridColor=             color for grid lines				
 style=                 normal (axis 1 horizontal, axis 2 vertical) or	
                        seismic (axis 1 vertical, axis 2 horizontal)	

 Data formats supported:						
 	1.a. x1,y1,x2,y2,...,xn,yn					
 	  b. x1,x2,...,xn,y1,y2,...,yn					
 	2. y1,y2,...,yn (must give non-zero d1[]=)			
 	3. x1,x2,...,xn (must give non-zero d2[]=)			
 	4. nil (must give non-zero d1[]= and non-zero d2[]=)		
   The formats may be repeated and mixed in any order, but if		
   formats 2-4 are used, the d1 and d2 arrays must be specified including
   d1[]=0.0 d2[]=0.0 entries for any internal occurences of format 1.	
   Similarly, the pairs array must contain place-keeping entries for	
   plots of formats 2-4 if they are mixed with both formats 1.a and 1.b.
   Also, if formats 2-4 are used with non-zero f1[] or f2[] entries, then
   the corresponding array(s) must be fully specified including f1[]=0.0
   and/or f2[]=0.0 entries for any internal occurences of format 1 or	
   formats 2-4 where the zero entries are desired.			
 mark index:                                                           
 1. asterisk                                                           
 2. x-cross                                                            
 3. open triangle                                                      
 4. open square                                                        
 5. open circle                                                        
 6. solid triangle                                                     
 7. solid square                                                       
 8. solid circle                                                       

 Note:	n1 and n2 are acceptable aliases for n and nplot, respectively.	

 Example:								
 xgraph n=50,100,20 d1=2.5,1,0.33 <datafile				
   plots three curves with equally spaced x values in one plot frame	
   x1-coordinates are x1(i) = f1+i*d1 for i = 1 to n (f1=0 by default)	
   number of x2's and then x2-coordinates for each curve are read	
   sequentially from datafile.						

\end{verbatim}
\pagebreak
\begin{verbatim}
 XMOVIE - image one or more frames of a uniformly sampled function f(x1,x2)

 xmovie n1= n2= [optional parameters] <fileoffloats			

 X Functionality:							
 Button 1	Zoom with rubberband box				
 Button 2 	reverse the direction of the movie.			
 Button 3 	stop and start the movie.				
 q or Q key	Quit 							
 s or S key	stop display and switch to Step mode		    
 b or B key	set frame direction to Backward			 
 f or F key	set frame direction to Forward			  
 n or N key	same as 'f'					     
 c or C key	set display mode to Continuous mode		     

 Required Parameters:							
 n1=		    number of samples in 1st (fast) dimension	
 n2=		    number of samples in 2nd (slow) dimension	

 Optional Parameters:							
 d1=1.0		 sampling interval in 1st dimension		
 f1=0.0		 first sample in 1st dimension			
 d2=1.0		 sampling interval in 2nd dimension		
 f2=0.0		 first sample in 2nd dimension			
 perc=100.0	     percentile used to determine clip		
 clip=(perc percentile) clip used to determine bclip and wclip		
 bperc=perc	     percentile for determining black clip value	
 wperc=100.0-perc       percentile for determining white clip value	
 bclip=clip	     data values outside of [bclip,wclip] are clipped
 wclip=-clip	    data values outside of [bclip,wclip] are clipped
 x1beg=x1min	    value at which axis 1 begins			
 x1end=x1max	    value at which axis 1 ends			
 x2beg=x2min	    value at which axis 2 begins			
 x2end=x2max	    value at which axis 2 ends			
 fframe=1	       value corresponding to first frame		
 dframe=1	       frame sampling interval			
 loop=0		 =1 to loop over frames after last frame is input
			=2 to run movie back and forth		 
 interp=1	       =0 for a non-interpolated, blocky image	
 verbose=1	      =1 for info printed on stderr (0 for no info)	
 idm=0		  =1 to set initial display mode to stepmode

 Optional resource parameters (defaults taken from resource database):	
 windowtitle=      	 title on window and icon			
 width=		 width in pixels of window			
 height=		height in pixels of window			
 nTic1=		 number of tics per numbered tic on axis 1	
 grid1=		 grid lines on axis 1 - none, dot, dash, or solid
 label1=		label on axis 1				
 nTic2=		 number of tics per numbered tic on axis 2	
 grid2=		 grid lines on axis 2 - none, dot, dash, or solid
 label2=		label on axis 2				
 labelFont=	     font name for axes labels			
 title=		 title of plot					
 titleFont=	     font name for title				
 titleColor=	    color for title				
 axesColor=	     color for axes					
 gridColor=	     color for grid lines				
 style=		 normal (axis 1 horizontal, axis 2 vertical) or	
			seismic (axis 1 vertical, axis 2 horizontal)	
 sleep=		 delay between frames in seconds (integer)	

 Color options:							
 cmap=gray     gray, hue, saturation, or default colormaps may be specified
 bhue=0	hue mapped to bclip (hue and saturation maps)		
 whue=240      hue mapped to wclip (hue and saturation maps)		
 sat=1	 saturation (hue map only)				
 bright=1      brightness (hue and saturation maps)			
 white=(bclip+wclip)/2  data value mapped to white (saturation map only)

 Notes:								
 Colors are specified using the HSV color wheel model:			
   Hue:  0=360=red, 60=yellow, 120=green, 180=cyan, 240=blue, 300=magenta
   Saturation:  0=white, 1=pure color					
   Value (brightness):  0=black, 1=maximum intensity			
 For the saturation mapping (cmap=sat), data values between white and bclip
   are mapped to bhue, with saturation varying from white to the pure color.
   Values between wclip and white are similarly mapped to whue.	
 For the hue mapping (cmap=hue), data values between wclip and bclip are
   mapped to hues between whue and bhue.  Intermediate hues are found by
   moving counterclockwise around the circle from bhue to whue.  To reverse
   the polarity of the image, exchange bhue and whue.  Equivalently,	
   exchange bclip and wclip (setting perc=0 is an easy way to do this).
   Hues in excess of 360 degrees can be specified in order to reach the
   opposite side of the color circle, or to wrap around the circle more
   than once.								

 The title string may contain a C printf format string containing a	
   conversion character for the frame number.  The frame number is	
   computed from dframe and fframe.  E.g., try setting title="Frame %g".

\end{verbatim}
\pagebreak
\begin{verbatim}
 XRECTS - plot rectangles on a two-dimensional grid			

 xrects x1min= x1max= x2min= x2max= [optional parameters] <rectangles 	

 Required Parameters:							
 x1min                  minimum x1 coordinate				
 x1max                  maximum x1 coordinate				
 x2min                  minimum x2 coordinate				
 x2max                  maximum x2 coordinate				

 Optional Parameters:							
 color=red              color used for rectangules			

 Optional resource parameters (defaults taken from resource database):	
 width=                 width in pixels of window			
 height=                height in pixels of window			
 nTic1=                 number of tics per numbered tic on axis 1	
 grid1=                 grid lines on axis 1 - none, dot, dash, or solid
 label1=                label on axis 1				
 nTic2=                 number of tics per numbered tic on axis 2	
 grid2=                 grid lines on axis 2 - none, dot, dash, or solid
 label2=                label on axis 2				
 labelFont=             font name for axes labels			
 title=                 title of plot					
 titleFont=             font name for title				
 titleColor=            color for title				
 axesColor=             color for axes					
 gridColor=             color for grid lines				
 style=                 normal (axis 1 horizontal, axis 2 vertical) or	
                        seismic (axis 1 vertical, axis 2 horizontal)	

\end{verbatim}
\pagebreak
\begin{verbatim}
 FFTLAB - Motif-X based graphical 1D Fourier Transform

 Usage:  fftlab

 Caveat: you must have the Motif Developer's package to install this code

\end{verbatim}
\pagebreak
\begin{verbatim}
 SUPSCONTOUR - PostScript CONTOUR plot of a segy data set		

 supscontour <stdin [optional parameters] | ...			

 Optional parameters:						 	

 n2=tr.ntr or number of traces in the data set (ntr is an alias for n2)

 d1=tr.d1 or tr.dt/10^6	sampling interval in the fast dimension	
   =.004 for seismic 		(if not set)				
   =1.0 for nonseismic		(if not set)				

 d2=tr.d2			sampling interval in the slow dimension	
   =1.0 			(if not set)				

 f1=tr.f1 or tr.delrt/10^3 or 0.0  first sample in the fast dimension	

 f2=tr.f2 or tr.tracr or tr.tracl  first sample in the slow dimension	
   =1.0 for seismic		    (if not set)			
   =d2 for nonseismic		    (if not set)			

 verbose=0              =1 to print some useful information		

 tmpdir=	 	if non-empty, use the value as a directory path	
		 	prefix for storing temporary files; else if the	
	         	the CWP_TMPDIR environment variable is set use	
	         	its value for the path; else use tmpfile()	

 Note that for seismic time domain data, the "fast dimension" is	
 time and the "slow dimension" is usually trace number or range.	
 Also note that "foreign" data tapes may have something unexpected	
 in the d2,f2 fields, use segyclean to clear these if you can afford	
 the processing time or use d2= f2= to override the header values if	
 not.									

 See the pscontour selfdoc for the remaining parameters.		

 On NeXT:	supscontour < infile [optional parameters]  | open	

 Trace header fields accessed: ns, ntr, tracr, tracl, delrt, trid,     
	dt, d1, d2, f1, f2						

 Credits:

	CWP: Dave Hale and Zhiming Li (pscontour, etc.)
	   Jack Cohen and John Stockwell (supscontour, etc.)
      Delphi: Alexander Koek, added support for irregularly spaced traces
      Aarhus University: Morten W. Pedersen copied everything from the xwigb
                         source and just replaced all occurencies of the word

 Notes:
	When the number of traces isn't known, we need to count
	the traces for pscontour.  You can make this value "known"
	either by getparring n2 or by having the ntr field set
	in the trace header.  A getparred value takes precedence
	over the value in the trace header.

	When we do have to count the traces, we use the "tmpfile"
	routine because on many machines it is implemented
	as a memory area instead of a disk file.

	If your system does make a disk file, consider altering
	the code to remove the file on interrupt.  This could be
	done either by trapping the interrupt with "signal"
	or by using the "tmpnam" routine followed by an immediate
	"remove" (aka "unlink" in old unix).

	When we must compute ntr, we don't allocate a 2-d array,
	but just content ourselves with copying trace by trace from
	the data "file" to the pipe into the plotting program.
	Although we could use tr.data, we allocate a trace buffer
	for code clarity.

\end{verbatim}
\pagebreak
\begin{verbatim}
 SUPSCUBECONTOUR - PostScript CUBE plot of a segy data set		

 supscubecontour <stdin [optional parameters] | ...			

 Optional parameters: 							

 n2 is the number of traces per frame.  If not getparred then it	
 is the total number of traces in the data set.  			

 n3 is the number of frames.  If not getparred then it			
 is the total number of frames in the data set measured by ntr/n2	

 d1=tr.d1 or tr.dt/10^6	sampling interval in the fast dimension	
   =.004 for seismic 		(if not set)				
   =1.0 for nonseismic		(if not set)				

 d2=tr.d2			sampling interval in the slow dimension	
   =1.0 			(if not set)				

 f1=tr.f1 or tr.delrt/10^3 or 0.0  first sample in the fast dimension	

 f2=tr.f2 or tr.tracr or tr.tracl  first sample in the slow dimension	
   =1.0 for seismic		    (if not set)			
   =d2 for nonseismic		    (if not set)			

 verbose=0              =1 to print some useful information		

 tmpdir=	 	if non-empty, use the value as a directory path	
		 	prefix for storing temporary files; else if the	
	         	the CWP_TMPDIR environment variable is set use	
	         	its value for the path; else use tmpfile()	

 Note that for seismic time domain data, the "fast dimension" is	
 time and the "slow dimension" is usually trace number or range.	
 Also note that "foreign" data tapes may have something unexpected	
 in the d2,f2 fields, use segyclean to clear these if you can afford	
 the processing time or use d2= f2= to over-ride the header values if	
 not.									

 See the pscubecontour selfdoc for the remaining parameters.		

 example:   supscubecontour < infile [optional parameters]  | gv -	

 Credits:

	CWP: Dave Hale and Zhiming Li (pscube)
	     Jack K. Cohen  (suxmovie)
	     John Stockwell (supscubecontour)

 Notes:
	When n2 isn't getparred, we need to count the traces
	for pscube. Although we compute ntr, we don't allocate a 2-d array
	and content ourselves with copying trace by trace from
	the data "file" to the pipe into the plotting program.
	Although we could use tr.data, we allocate a trace buffer
	for code clarity.

\end{verbatim}
\pagebreak
\begin{verbatim}
 SUPSCUBE - PostScript CUBE plot of a segy data set			

 supscube <stdin [optional parameters] | ...				

 Optional parameters: 							

 n2 is the number of traces per frame.  If not getparred then it	
 is the total number of traces in the data set.  			

 n3 is the number of frames.  If not getparred then it			
 is the total number of frames in the data set measured by ntr/n2	

 d1=tr.d1 or tr.dt/10^6	sampling interval in the fast dimension	
   =.004 for seismic 		(if not set)				
   =1.0 for nonseismic		(if not set)				

 d2=tr.d2			sampling interval in the slow dimension	
   =1.0 			(if not set)				

 f1=tr.f1 or tr.delrt/10^3 or 0.0  first sample in the fast dimension	

 f2=tr.f2 or tr.tracr or tr.tracl  first sample in the slow dimension	
   =1.0 for seismic		    (if not set)			
   =d2 for nonseismic		    (if not set)			

 verbose=0              =1 to print some useful information		

 tmpdir=	 	if non-empty, use the value as a directory path	
		 	prefix for storing temporary files; else if the	
	         	the CWP_TMPDIR environment variable is set use	
	         	its value for the path; else use tmpfile()	

 Note that for seismic time domain data, the "fast dimension" is	
 time and the "slow dimension" is usually trace number or range.	
 Also note that "foreign" data tapes may have something unexpected	
 in the d2,f2 fields, use segyclean to clear these if you can afford	
 the processing time or use d2= f2= to over-ride the header values if	
 not.									

 See the pscube selfdoc for the remaining parameters.			

 On NeXT:     supscube < infile [optional parameters]  | open	       	

 Credits:

	CWP: Dave Hale and Zhiming Li (pscube)
	     Jack K. Cohen  (suxmovie)
	     John Stockwell (supscube)

 Notes:
	When n2 isn't getparred, we need to count the traces
	for pscube. Although we compute ntr, we don't allocate a 2-d array
	and content ourselves with copying trace by trace from
	the data "file" to the pipe into the plotting program.
	Although we could use tr.data, we allocate a trace buffer
	for code clarity.

\end{verbatim}
\pagebreak
\begin{verbatim}
 SUPSGRAPH - PostScript GRAPH plot of a segy data set			

 supsgraph <stdin [optional parameters] | ...				

 Optional parameters: 							
 style=seismic		seismic is default here, =normal is alternate	
			(see psgraph selfdoc for style definitions)	

 nplot is the number of traces (ntr is an acceptable alias for nplot) 	

 d1=tr.d1 or tr.dt/10^6	sampling interval in the fast dimension	
   =.004 for seismic 		(if not set)				
   =1.0 for nonseismic		(if not set)				

 d2=tr.d2			sampling interval in the slow dimension	
   =1.0 			(if not set)				

 f1=tr.f1 or tr.delrt/10^3 or 0.0  first sample in the fast dimension	

 f2=tr.f2 or tr.tracr or tr.tracl  first sample in the slow dimension	
   =1.0 for seismic		    (if not set)			
   =d2 for nonseismic		    (if not set)			

 verbose=0              =1 to print some useful information		

 tmpdir=	 	if non-empty, use the value as a directory path	
		 	prefix for storing temporary files; else if the	
	         	the CWP_TMPDIR environment variable is set use	
	         	its value for the path; else use tmpfile()	

 Note that for seismic time domain data, the "fast dimension" is	
 time and the "slow dimension" is usually trace number or range.	
 Also note that "foreign" data tapes may have something unexpected	
 in the d2,f2 fields, use segyclean to clear these if you can afford	
 the processing time or use d2= f2= to over-ride the header values if	
 not.									

 See the psgraph selfdoc for the remaining parameters.			

 On NeXT:     supsgraph < infile [optional parameters]  | open      	

 Credits:

	CWP: Dave Hale and Zhiming Li (pswigp, etc.)
	   Jack Cohen and John Stockwell (supswigp, etc.)

 Notes:
	When the number of traces isn't known, we need to count
	the traces for pswigp.  You can make this value "known"
	either by getparring nplot or by having the ntr field set
	in the trace header.  A getparred value takes precedence
	over the value in the trace header.

	When we must compute ntr, we don't allocate a 2-d array,
	but just content ourselves with copying trace by trace from
	the data "file" to the pipe into the plotting program.
	Although we could use tr.data, we allocate a trace buffer
	for code clarity.

\end{verbatim}
\pagebreak
\begin{verbatim}
 SUPSIMAGE - PostScript IMAGE plot of a segy data set			

 supsimage <stdin [optional parameters] | ...				

 Optional parameters:						 	

 n2=tr.ntr or number of traces in the data set (ntr is an alias for n2)

 d1=tr.d1 or tr.dt/10^6	sampling interval in the fast dimension	
   =.004 for seismic 		(if not set)				
   =1.0 for nonseismic		(if not set)				

 d2=tr.d2			sampling interval in the slow dimension	
   =1.0 			(if not set)				

 f1=tr.f1 or tr.delrt/10^3 or 0.0  first sample in the fast dimension	

 f2=tr.f2 or tr.tracr or tr.tracl  first sample in the slow dimension	
   =1.0 for seismic		    (if not set)			
   =d2 for nonseismic		    (if not set)			

 verbose=0              =1 to print some useful information		

 tmpdir=	 	if non-empty, use the value as a directory path	
		 	prefix for storing temporary files; else if the	
	         	the CWP_TMPDIR environment variable is set use	
	         	its value for the path; else use tmpfile()	

 Note that for seismic time domain data, the "fast dimension" is	
 time and the "slow dimension" is usually trace number or range.	
 Also note that "foreign" data tapes may have something unexpected	
 in the d2,f2 fields, use segyclean to clear these if you can afford	
 the processing time or use d2= f2= to override the header values if	
 not.									

 See the psimage selfdoc for the remaining parameters.		

 On NeXT:	supsimage < infile [optional parameters]  | open	

 Trace header fields accessed: ns, ntr, tracr, tracl, delrt, trid,     
	dt, d1, d2, f1, f2						

 Credits:

	CWP: Dave Hale and Zhiming Li (psimage, etc.)
	   Jack Cohen and John Stockwell (supsimage, etc.)

 Notes:
	When the number of traces isn't known, we need to count
	the traces for psimage.  You can make this value "known"
	either by getparring n2 or by having the ntr field set
	in the trace header.  A getparred value takes precedence
	over the value in the trace header.

	When we do have to count the traces, we use the "tmpfile"
	routine because on many machines it is implemented
	as a memory area instead of a disk file.
	"remove" (aka "unlink" in old unix).

	When we must compute ntr, we don't allocate a 2-d array,
	but just content ourselves with copying trace by trace from
	the data "file" to the pipe into the plotting program.
	Although we could use tr.data, we allocate a trace buffer
	for code clarity.

\end{verbatim}
\pagebreak
\begin{verbatim}
 SUPSMAX - PostScript of the MAX, min, or absolute max value on each trace
 	   of a SEGY (SU) data	set					

   supsmax <stdin >postscript file [optional parameters]		

 Optional parameters: 							
 mode=max		max value					
 			=min min value					
 			=abs absolute max value				

 n2=tr.ntr or number of traces in the data set (ntr is an alias for n2)

 d1=tr.d1 or tr.dt/10^6	sampling interval in the fast dimension	
   =.004 for seismic 		(if not set)				
   =1.0 for nonseismic		(if not set)				

 d2=tr.d2			sampling interval in the slow dimension	
   =1.0 			(if not set)				

 f1=tr.f1 or tr.delrt/10^3 or 0.0  first sample in the fast dimension	

 f2=tr.f2 or tr.tracr or tr.tracl  first sample in the slow dimension	
   =1.0 for seismic		    (if not set)			
   =d2 for nonseismic		    (if not set)			

 verbose=0              =1 to print some useful information		

 tmpdir=	 	if non-empty, use the value as a directory path	
		 	prefix for storing temporary files; else if the	
	         	the CWP_TMPDIR environment variable is set use	
	         	its value for the path; else use tmpfile()	

 Note that for seismic time domain data, the "fast dimension" is	
 time and the "slow dimension" is usually trace number or range.	
 Also note that "foreign" data tapes may have something unexpected	
 in the d2,f2 fields, use segyclean to clear these if you can afford	
 the processing time or use d2= f2= to over-ride the header values if	
 not.									

 See the sumax selfdoc for additional parameter.			
 See the psgraph selfdoc for the remaining parameters.			


 Credits:

	CWP: John Stockwell, based on Jack Cohen's SU JACKet 

 Notes:
	When the number of traces isn't known, we need to count
	the traces for psgraph.  You can make this value "known"
	either by getparring n2 or by having the ntr field set
	in the trace header.  A getparred value takes precedence
	over the value in the trace header.

	When we do have to count the traces, we use the "tmpfile"
	routine because on many machines it is implemented
	as a memory area instead of a disk file.

\end{verbatim}
\pagebreak
\begin{verbatim}
 SUPSMOVIE - PostScript MOVIE plot of a segy data set			

 supsmovie <stdin [optional parameters] | ...				

 Optional parameters: 							

 n2 is the number of traces per frame.  If not getparred then it	
 is the total number of traces in the data set.  			

 n3 is the number of frames.  If not getparred then it			
 is the total number of frames in the data set measured by ntr/n2	

 d1=tr.d1 or tr.dt/10^6	sampling interval in the fast dimension	
   =.004 for seismic 		(if not set)				
   =1.0 for nonseismic		(if not set)				

 d2=tr.d2			sampling interval in the slow dimension	
   =1.0 			(if not set)				

 f1=tr.f1 or tr.delrt/10^3 or 0.0  first sample in the fast dimension	

 f2=tr.f2 or tr.tracr or tr.tracl  first sample in the slow dimension	
   =1.0 for seismic		    (if not set)			
   =d2 for nonseismic		    (if not set)			

 verbose=0              =1 to print some useful information		

 tmpdir=	 	if non-empty, use the value as a directory path	
		 	prefix for storing temporary files; else if the	
	         	the CWP_TMPDIR environment variable is set use	
	         	its value for the path; else use tmpfile()	

 Note that for seismic time domain data, the "fast dimension" is	
 time and the "slow dimension" is usually trace number or range.	
 Also note that "foreign" data tapes may have something unexpected	
 in the d2,f2 fields, use segyclean to clear these if you can afford	
 the processing time or use d2= f2= to over-ride the header values if	
 not.									

 See the psmovie selfdoc for the remaining parameters.			

 On NeXT:     supsmovie < infile [optional parameters]  | open	       	

 Credits:

	CWP: Dave Hale and Zhiming Li (psmovie)
	     Jack K. Cohen  (suxmovie)
	     John Stockwell (supsmovie)

 Notes:
	When n2 isn't getparred, we need to count the traces
	for psmovie.  In this case:
	we are using tmpfile because on many machines it is
	implemented as a memory area instead of a disk file.
	Although we compute ntr, we don't allocate a 2-d array
	and content ourselves with copying trace by trace from
	the data "file" to the pipe into the plotting program.
	Although we could use tr.data, we allocate a trace buffer
	for code clarity.

\end{verbatim}
\pagebreak
\begin{verbatim}
 SUPSWIGB - PostScript Bit-mapped WIGgle plot of a segy data set	

 supswigb <stdin [optional parameters] | ...				

 Optional parameters:						 	
 key=(keyword)		if set, the values of x2 are set from header field
			specified by keyword				
 n2=tr.ntr or number of traces in the data set	(ntr is an alias for n2)
 d1=tr.d1 or tr.dt/10^6	sampling interval in the fast dimension	
   =.004 for seismic 		(if not set)				
   =1.0 for nonseismic		(if not set)				
 d2=tr.d2			sampling interval in the slow dimension	
   =1.0 			(if not set)				
 f1=tr.f1 or tr.delrt/10^3 or 0.0  first sample in the fast dimension	
 f2=tr.f2 or tr.tracr or tr.tracl  first sample in the slow dimension	
   =1.0 for seismic		    (if not set)			
   =d2 for nonseismic		    (if not set)			

 style=seismic		 normal (axis 1 horizontal, axis 2 vertical) or 
			 vsp (same as normal with axis 2 reversed)	
			 Note: vsp requires use of a keyword		

 verbose=0              =1 to print some useful information		

 tmpdir=	 	if non-empty, use the value as a directory path	
		 	prefix for storing temporary files; else if the	
	         	the CWP_TMPDIR environment variable is set use	
	         	its value for the path; else use tmpfile()	

 Note that for seismic time domain data, the "fast dimension" is	
 time and the "slow dimension" is usually trace number or range.	
 Also note that "foreign" data tapes may have something unexpected	
 in the d2,f2 fields, use segyclean to clear these if you can afford	
 the processing time or use d2= f2= to override the header values if	
 not.									

 If key=keyword is set, then the values of x2 are taken from the header
 field represented by the keyword (for example key=offset, will show   
 traces in true offset). This permit unequally spaced traces to be plotted.
 Type   sukeyword -o   to see the complete list of SU keywords.	

 This program is really just a wrapper for the plotting program: pswigb
 See the pswigb selfdoc for the remaining parameters.			

 Trace header fields accessed: ns, ntr, tracr, tracl, delrt, trid,     
	dt, d1, d2, f1, f2, keyword (if set)				

 Credits:

	CWP: Dave Hale and Zhiming Li (pswigb, etc.)
	   Jack Cohen and John Stockwell (supswigb, etc.)
      Delphi: Alexander Koek, added support for irregularly spaced traces 

	Modified by Brian Zook, Southwest Research Institute, to honor
	 scale factors, added vsp style

 Notes:
	When the number of traces isn't known, we need to count
	the traces for pswigb.  You can make this value "known"
	either by getparring n2 or by having the ntr field set
	in the trace header.  A getparred value takes precedence
	over the value in the trace header.

	When we must compute ntr, we don't allocate a 2-d array,
	but just content ourselves with copying trace by trace from
	the data "file" to the pipe into the plotting program.
	Although we could use tr.data, we allocate a trace buffer
	for code clarity.

\end{verbatim}
\pagebreak
\begin{verbatim}
 SUPSWIGP - PostScript Polygon-filled WIGgle plot of a segy data set	

 supswigp <stdin [optional parameters] | ...				

 Optional parameters:						 	
 key=(keyword)		if set, values of x2 are set from header field	
			specified by keyword				
 n2=tr.ntr or number of traces in the data set	(ntr is an alias for n2)
 d1=tr.d1 or tr.dt/10^6	sampling interval in the fast dimension	
   =.004 for seismic 		(if not set)				
   =1.0 for nonseismic		(if not set)				
 d2=tr.d2			sampling interval in the slow dimension	
   =1.0 			(if not set)				
 f1=tr.f1 or tr.delrt/10^3 or 0.0  first sample in the fast dimension	
 f2=tr.f2 or tr.tracr or tr.tracl  first sample in the slow dimension	
   =1.0 for seismic		    (if not set)			
   =d2 for nonseismic		    (if not set)			

 style=seismic		 normal (axis 1 horizontal, axis 2 vertical) or 
			 vsp (same as normal with axis 2 reversed)	
			 Note: vsp requires use of a keyword		

 verbose=0              =1 to print some useful information		

 tmpdir=	 	if non-empty, use the value as a directory path	
		 	prefix for storing temporary files; else if the	
	         	the CWP_TMPDIR environment variable is set use	
	         	its value for the path; else use tmpfile()	

 Note that for seismic time domain data, the "fast dimension" is	
 time and the "slow dimension" is usually trace number or range.	
 Also note that "foreign" data tapes may have something unexpected	
 in the d2,f2 fields, use segyclean to clear these if you can afford	
 the processing time or use d2= f2= to override the header values if	
 not.									

 If key=keyword is set, then the values of x2 are taken from the header
 field represented by the keyword (for example key=offset, will show   
 traces in true offset). This permit unequally spaced traces to be plotted.
 Type   sukeyword -o   to see the complete list of SU keywords.	

 This program is really just a wrapper for the plotting program: pswigp
 See the pswigp selfdoc for the remaining parameters.			

 On NeXT:	supswigp < infile [optional parameters]  | open		

 Trace header fields accessed: ns, ntr, tracr, tracl, delrt, trid,     
	dt, d1, d2, f1, f2, key specified by key			

 Credits:

	CWP: Dave Hale and Zhiming Li (pswigp, etc.)
	   Jack Cohen and John Stockwell (supswigp, etc.)
	Delphi: Alexander Koek, added support for irregularly spaced traces

	Modified by Brian Zook, Southwest Research Institute, to honor
	 scale factors, added vsp style

 Notes:
	When the number of traces isn't known, we need to count
	the traces for pswigp.  You can make this value "known"
	either by getparring n2 or by having the ntr field set
	in the trace header.  A getparred value takes precedence
	over the value in the trace header.

	When we do have to count the traces, we use the "tmpfile"
	routine because on many machines it is implemented
	as a memory area instead of a disk file.

	If your system does make a disk file, consider altering
	the code to remove the file on interrupt.  This could be
	done either by trapping the interrupt with "signal"
	or by using the "tmpnam" routine followed by an immediate
	"remove" (aka "unlink" in old unix).

	When we must compute ntr, we don't allocate a 2-d array,
	but just content ourselves with copying trace by trace from
	the data "file" to the pipe into the plotting program.
	Although we could use tr.data, we allocate a trace buffer
	for code clarity.

\end{verbatim}
\pagebreak
\begin{verbatim}
 SUXCONTOUR - X CONTOUR plot of Seismic UNIX tracefile via vector plot call

 suxwigb <stdin [optional parameters] | ...				

 Optional parameters:						 	
 key=(keyword)		if set, the values of x2 are set from header field
			specified by keyword				
 n2=tr.ntr or number of traces in the data set	(ntr is an alias for n2)
 d1=tr.d1 or tr.dt/10^6	sampling interval in the fast dimension	
   =.004 for seismic 		(if not set)				
   =1.0 for nonseismic		(if not set)				
 d2=tr.d2			sampling interval in the slow dimension	
   =1.0 			(if not set)				
 f1=tr.f1 or tr.delrt/10^3 or 0.0  first sample in the fast dimension	
 f2=tr.f2 or tr.tracr or tr.tracl  first sample in the slow dimension	
   =1.0 for seismic		    (if not set)			
   =d2 for nonseismic		    (if not set)			

 verbose=0              =1 to print some useful information		

 tmpdir=	 	if non-empty, use the value as a directory path	
		 	prefix for storing temporary files; else if the	
	         	the CWP_TMPDIR environment variable is set use	
	         	its value for the path; else use tmpfile()	

 Note that for seismic time domain data, the "fast dimension" is	
 time and the "slow dimension" is usually trace number or range.	
 Also note that "foreign" data tapes may have something unexpected	
 in the d2,f2 fields, use segyclean to clear these if you can afford	
 the processing time or use d2= f2= to override the header values if	
 not.									

 If key=keyword is set, then the values of x2 are taken from the header
 field represented by the keyword (for example key=offset, will show   
 traces in true offset). This permit unequally spaced traces to be plotted.
 Type   sukeyword -o   to see the complete list of SU keywords.	

 This program is really just a wrapper for the plotting program: xcontour
 See the xcontour selfdoc for the remaining parameters.		



 Credits:

	CWP: Dave Hale and Zhiming Li (xwigb, etc.)
	   Jack Cohen and John Stockwell (suxwigb, etc.)
      Delphi: Alexander Koek, added support for irregularly spaced traces
      Aarhus University: Morten W. Pedersen copied everything from the xwigb
                         source and just replaced all occurencies of the word
                         xwigb with xcountour ;-)

 Notes:
	When the number of traces isn't known, we need to count
	the traces for xcontour.  You can make this value "known"
	either by getparring n2 or by having the ntr field set
	in the trace header.  A getparred value takes precedence
	over the value in the trace header.

	When we must compute ntr, we don't allocate a 2-d array,
	but just content ourselves with copying trace by trace from
	the data "file" to the pipe into the plotting program.
	Although we could use tr.data, we allocate a trace buffer
	for code clarity.

\end{verbatim}
\pagebreak
\begin{verbatim}
 SUXGRAPH - X-windows GRAPH plot of a segy data set			

 suxgraph <stdin [optional parameters] | ...				

 Optional parameters: 							
 (see xgraph selfdoc for optional parametes)				

 nplot= number of traces (ntr is an acceptable alias for nplot) 	

 d1=tr.d1 or tr.dt/10^6	sampling interval in the fast dimension	
   =.004 for seismic 		(if not set)				
   =1.0 for nonseismic		(if not set)				

 d2=tr.d2			sampling interval in the slow dimension	
   =1.0 			(if not set)				

 f1=tr.f1 or tr.delrt/10^3 or 0.0  first sample in the fast dimension	

 f2=tr.f2 or tr.tracr or tr.tracl  first sample in the slow dimension	
   =1.0 for seismic		    (if not set)			
   =d2 for nonseismic		    (if not set)			

 verbose=0              =1 to print some useful information		

 tmpdir=	 	if non-empty, use the value as a directory path	
		 	prefix for storing temporary files; else if the	
	         	the CWP_TMPDIR environment variable is set use	
	         	its value for the path; else use tmpfile()	

 Note that for seismic time domain data, the "fast dimension" is	
 time and the "slow dimension" is usually trace number or range.	
 Also note that "foreign" data tapes may have something unexpected	
 in the d2,f2 fields, use segyclean to clear these if you can afford	
 the processing time or use d2= f2= to over-ride the header values if	
 not.									

 See the xgraph selfdoc for the remaining parameters.			

 On NeXT:     suxgraph < infile [optional parameters]  | open      	

 Credits:

	CWP: Dave Hale and Zhiming Li (pswigp, etc.)
	   Jack Cohen and John Stockwell (supswigp, etc.)

 Notes:
	When the number of traces isn't known, we need to count
	the traces for pswigp.  You can make this value "known"
	either by getparring nplot or by having the ntr field set
	in the trace header.  A getparred value takes precedence
	over the value in the trace header.

	When we do have to count the traces, we use the "tmpfile"
	routine because on many machines it is implemented
	as a memory area instead of a disk file.

	When we must compute ntr, we don't allocate a 2-d array,
	but just content ourselves with copying trace by trace from
	the data "file" to the pipe into the plotting program.
	Although we could use tr.data, we allocate a trace buffer
	for code clarity.

\end{verbatim}
\pagebreak
\begin{verbatim}
 SUXIMAGE - X-windows IMAGE plot of a segy data set	                

 suximage infile= [optional parameters] | ...  (direct I/O)            
  or					                		
 suximage <stdin [optional parameters] | ...	(sequential I/O)        

 Optional parameters:						 	

 infile=NULL SU data to be ploted, default stdin with sequential access
             if 'infile' provided, su data read by (fast) direct access

	      with ftr,dtr and n2 suximage will pass a subset of data   
	      to the plotting program-ximage:                           
 ftr=1       First trace to be plotted                                 
 dtr=1       Trace increment to be plotted                             
 n2=tr.ntr   (Max) number of traces to be plotted (ntr is an alias for n2)
	      Priority: first try to read from parameter line;		
		        if not provided, check trace header tr.ntr;     
		        if still not provided, figure it out using ftello

 d1=tr.d1 or tr.dt/10^6	sampling interval in the fast dimension	
   =.004 for seismic 		(if not set)				
   =1.0 for nonseismic		(if not set)				

 d2=tr.d2		sampling interval in the slow dimension	        
   =1.0 		(if not set or was set to 0)		        

 key=			key for annotating d2 (slow dimension)		
 			If annotation is not at proper increment, try	
 			setting d2; only first trace's key value is read

 f1=tr.f1 or tr.delrt/10^3 or 0.0  first sample in the fast dimension	

 f2=tr.f2 or tr.tracr or tr.tracl  first sample in the slow dimension	
   =1.0 for seismic		    (if not set)			
   =d2 for nonseismic		    (if not set)			

 verbose=0             =1 to print some useful information		

 tmpdir=	 	if non-empty, use the value as a directory path	
		 	prefix for storing temporary files; else if the	
	         	the CWP_TMPDIR environment variable is set use	
	         	its value for the path; else use tmpfile()	

 Note that for seismic time domain data, the "fast dimension" is	
 time and the "slow dimension" is usually trace number or range.	
 Also note that "foreign" data tapes may have something unexpected	
 in the d2,f2 fields, use segyclean to clear these if you can afford	
 the processing time or use d2= f2= to override the header values if	
 not.									

 See the ximage selfdoc for the remaining parameters.		        


 Credits:

	CWP: Dave Hale and Zhiming Li (ximage, etc.)
	   Jack Cohen and John Stockwell (suximage, etc.)
	MTU: David Forel, June 2004, added key for annotating d2
      ConocoPhillips: Zhaobo Meng, Dec 2004, added direct I/O

 Notes:

      When provide ftr and dtr and infile, suximage can be used to plot 
      multi-dimensional volumes efficiently.  For example, for a Offset-CDP
      dataset with 32 offsets, the command line
      suximage infile=volume3d.su ftr=1 dtr=32 ... &
      will display the zero-offset common offset data with ranrom access.  
      It is highly recommend to use infile= to view large datasets, since
      using stdin only allows sequential access, which is very slow for 
      large datasets.

	When the number of traces isn't known, we need to count
	the traces for ximage.  You can make this value "known"
	either by getparring n2 or by having the ntr field set
	in the trace header.  A getparred value takes precedence
	over the value in the trace header.

	When we must compute ntr, we don't allocate a 2-d array,
	but just content ourselves with copying trace by trace from
	the data "file" to the pipe into the plotting program.
	Although we could use tr.data, we allocate a trace buffer
	for code clarity.

\end{verbatim}
\pagebreak
\begin{verbatim}
 SUXMAX - X-windows graph of the MAX, min, or absolute max value on	
	each trace of a SEGY (SU) data set				

   suxmax <stdin [optional parameters]					

 Optional parameters: 							
 mode=max		max value					
 			=min min value					
 			=abs absolute max value				

 n2=tr.ntr or number of traces in the data set (ntr is an alias for n2)

 d1=tr.d1 or tr.dt/10^6	sampling interval in the fast dimension	
   =.004 for seismic 		(if not set)				
   =1.0 for nonseismic		(if not set)				

 d2=tr.d2			sampling interval in the slow dimension	
   =1.0 			(if not set)				

 f1=tr.f1 or tr.delrt/10^3 or 0.0  first sample in the fast dimension	

 f2=tr.f2 or tr.tracr or tr.tracl  first sample in the slow dimension	
   =1.0 for seismic		    (if not set)			
   =d2 for nonseismic		    (if not set)			

 verbose=0              =1 to print some useful information		

 tmpdir=	 	if non-empty, use the value as a directory path	
		 	prefix for storing temporary files; else if the	
	         	the CWP_TMPDIR environment variable is set use	
	         	its value for the path; else use tmpfile()	

 Note that for seismic time domain data, the "fast dimension" is	
 time and the "slow dimension" is usually trace number or range.	
 Also note that "foreign" data tapes may have something unexpected	
 in the d2,f2 fields, use segyclean to clear these if you can afford	
 the processing time or use d2= f2= to over-ride the header values if	
 not.									

 See the sumax selfdoc for additional parameter.			
 See the xgraph selfdoc for the remaining parameters.			


 Credits:

	CWP: John Stockwell, based on Jack Cohen's SU JACKet 

 Notes:
	When the number of traces isn't known, we need to count
	the traces for xgraph.  You can make this value "known"
	either by getparring n2 or by having the ntr field set
	in the trace header.  A getparred value takes precedence
	over the value in the trace header.

	When we do have to count the traces, we use the "tmpfile"
	routine because on many machines it is implemented
	as a memory area instead of a disk file.

\end{verbatim}
\pagebreak
\begin{verbatim}
 SUXMOVIE - X MOVIE plot of a 2D or 3D segy data set 			

 suxmovie <stdin [optional parameters]		 			

 Optional parameters: 							

 n1=tr.ns         	    	number of samples per trace  		
 ntr=tr.ntr     	    	number of traces in the data set	
 n2=tr.shortpad or tr.ntr	number of traces in inline direction 	
 n3=ntr/n2     	    	number of traces in crossline direction	

 d1=tr.d1 or tr.dt/10^6    sampling interval in the fast dimension	
   =.004 for seismic 		(if not set)				
   =1.0 for nonseismic		(if not set)				

 d2=tr.d2		    sampling interval in the slow dimension	
   =1.0 			(if not set)				

 d3=1.0		    sampling interval in the slowest dimension	

 f1=tr.f1 or 0.0  	    first sample in the z dimension		
 f2=tr.f2 or 1.0           first sample in the x dimension		
 f3=1.0 		    						

 mode=0          0= x,z slice movie through y dimension (in line)      
                 1= y,z slice movie through x dimension (cross line)   
                 2= x,y slice movie through z dimension (time slice)   

 verbose=0              =1 to print some useful information		

 tmpdir=	 	if non-empty, use the value as a directory path	
		 	prefix for storing temporary files; else if the	
	         	the CWP_TMPDIR environment variable is set use	
	         	its value for the path; else use tmpfile()	

 Notes:
 For seismic data, the "fast dimension" is either time or		
 depth and the "slow dimension" is usually trace number.	        
 The 3D data set is expected to have n3 sets of n2 traces representing 
 the horizontal coverage of n2*d2 in x  and n3*d3 in y direction.      

 The data is read to memory with and piped to xmovie with the         	
 respective sampling parameters.			        	
 See the xmovie selfdoc for the remaining parameters and X functions.	

\end{verbatim}
\pagebreak
\begin{verbatim}
 SUXPICKER - X-windows  WIGgle plot PICKER of a segy data set		

 suxpicker <stdin [optional parameters] | ...				

 Optional parameters:						 	

 key=(keyword)		if set, the values of x2 are set from header field
			specified by keyword				
			specified by keyword				
 n2=tr.ntr or number of traces in the data set (ntr is an alias for n2)

 d1=tr.d1 or tr.dt/10^6	sampling interval in the fast dimension	
   =.004 for seismic 		(if not set)				
   =1.0 for nonseismic		(if not set)				

 d2=tr.d2			sampling interval in the slow dimension	
   =1.0 			(if not set)				

 f1=tr.f1 or tr.delrt/10^3 or 0.0  first sample in the fast dimension	

 f2=tr.f2 or tr.tracr or tr.tracl  first sample in the slow dimension	
   =1.0 for seismic		    (if not set)			
   =d2 for nonseismic		    (if not set)			

 verbose=0              =1 to print some useful information		


 tmpdir=	 	if non-empty, use the value as a directory path	
		 	prefix for storing temporary files; else if the	
	         	the CWP_TMPDIR environment variable is set use	
	         	its value for the path; else use tmpfile()	

 Note that for seismic time domain data, the "fast dimension" is	
 time and the "slow dimension" is usually trace number or range.	
 Also note that "foreign" data tapes may have something unexpected	
 in the d2,f2 fields, use segyclean to clear these if you can afford	
 the processing time or use d2= f2= to override the header values if	
 not.									

 If key=keyword is set, then the values of x2 are taken from the header
 field represented by the keyword (for example key=offset, will show	
 traces in true offset). This permit unequally spaced traces to be plotted.
 Type	 sukeyword -o	to see the complete list of SU keywords.	

 See the xpicker selfdoc for the remaining parameters.			


 Credits:

	CWP: Dave Hale and Zhiming Li (xpicker, etc.)
	   Jack Cohen and John Stockwell (suxpicker, etc.)

 Notes:
	When the number of traces isn't known, we need to count
	the traces for xpicker.  You can make this value "known"
	either by getparring n2 or by having the ntr field set
	in the trace header.  A getparred value takes precedence
	over the value in the trace header.

	When we do have to count the traces, we use the "tmpfile"
	routine because on many machines it is implemented
	as a memory area instead of a disk file.

	If your system does make a disk file, consider altering
	the code to remove the file on interrupt.  This could be
	done either by trapping the interrupt with "signal"
	or by using the "tmpnam" routine followed by an immediate
	"remove" (aka "unlink" in old unix).

	When we must compute ntr, we don't allocate a 2-d array,
	but just content ourselves with copying trace by trace from
	the data "file" to the pipe into the plotting program.
	Although we could use tr.data, we allocate a trace buffer
	for code clarity.

\end{verbatim}
\pagebreak
\begin{verbatim}
 SUXWIGB - X-windows Bit-mapped WIGgle plot of a segy data set		
 This is a modified suxwigb that uses the depth or coordinate scaling	
 when such values are used as keys.					

 suxwigb <stdin [optional parameters] | ...				

 Optional parameters:							
 key=(keyword)		if set, the values of x2 are set from header field
			specified by keyword				
 n2=tr.ntr or number of traces in the data set (ntr is an alias for n2)
 d1=tr.d1 or tr.dt/10^6	sampling interval in the fast dimension 
   =.004 for seismic		(if not set)				
   =1.0 for nonseismic		(if not set)				
 d2=tr.d2			sampling interval in the slow dimension 
   =1.0			(if not set)				
 f1=tr.f1 or tr.delrt/10^3 or 0.0  first sample in the fast dimension	
 f2=tr.f2 or tr.tracr or tr.tracl  first sample in the slow dimension	
   =1.0 for seismic		    (if not set)			
   =d2 for nonseismic		    (if not set)			

 style=seismic		 normal (axis 1 horizontal, axis 2 vertical) or 
			 vsp (same as normal with axis 2 reversed)	
			 Note: vsp requires use of a keyword		
 verbose=0              =1 to print some useful information		


 tmpdir=	 	if non-empty, use the value as a directory path	
		 	prefix for storing temporary files; else if the	
	         	the CWP_TMPDIR environment variable is set use	
	         	its value for the path; else use tmpfile()	

 Note that for seismic time domain data, the "fast dimension" is	
 time and the "slow dimension" is usually trace number or range.	
 Also note that "foreign" data tapes may have something unexpected	
 in the d2,f2 fields, use segyclean to clear these if you can afford	
 the processing time or use d2= f2= to override the header values if	
 not.									

 If key=keyword is set, then the values of x2 are taken from the header
 field represented by the keyword (for example key=offset, will show	
 traces in true offset). This permit unequally spaced traces to be plotted.
 Type	 sukeyword -o	to see the complete list of SU keywords.	

 This program is really just a wrapper for the plotting program: xwigb	
 See the xwigb selfdoc for the remaining parameters.			


 Credits:

	CWP: Dave Hale and Zhiming Li (xwigb, etc.)
	   Jack Cohen and John Stockwell (suxwigb, etc.)
	Delphi: Alexander Koek, added support for irregularly spaced traces

	Modified by Brian Zook, Southwest Research Institute, to honor
	 scale factors, added vsp style

 Notes:
	When the number of traces isn't known, we need to count
	the traces for xwigb.  You can make this value "known"
	either by getparring n2 or by having the ntr field set
	in the trace header.  A getparred value takes precedence
	over the value in the trace header.

	When we must compute ntr, we don't allocate a 2-d array,
	but just content ourselves with copying trace by trace from
	the data "file" to the pipe into the plotting program.
	Although we could use tr.data, we allocate a trace buffer
	for code clarity.

\end{verbatim}
\pagebreak
\begin{verbatim}
 SXPLOT - X Window plot a triangulated sloth function s(x1,x2)		

 sxplot <modelfile [optional parameters]				

 Optional Parameters:							
 edgecolor=cyan         color to draw fixed edges			
 tricolor=yellow        color to draw non-fixed edges of triangles	
	 =none		non-fixed edges of triangles are not shown      
 bclip=minimum sloth    sloth value corresponding to black		
 wclip=maximum sloth    sloth value corresponding to white		
 x1beg=x1min            value at which x1 axis begins			
 x1end=x1max            value at which x1 axis ends			
 x2beg=x2min            value at which x2 axis begins			
 x2end=x2max            value at which x2 axis ends			
 cmap=gray              gray, hue, or default colormaps may be specified

 Optional resource parameters (defaults taken from resource database):	
 width=                 width in pixels of window			
 height=                height in pixels of window			
 nTic1=                 number of tics per numbered tic on axis 1	
 grid1=                 grid lines on axis 1 - none, dot, dash, or solid
 label1=                label on axis 1				
 nTic2=                 number of tics per numbered tic on axis 2	
 grid2=                 grid lines on axis 2 - none, dot, dash, or solid
 label2=                label on axis 2				
 labelFont=             font name for axes labels			
 title=                 title of plot					
 titleFont=             font name for title				
 titleColor=            color for title				
 axesColor=             color for axes					
 gridColor=             color for grid lines				
 style=                 normal (axis 1 horizontal, axis 2 vertical) or	
                       seismic (axis 1 vertical, axis 2 horizontal)	



 AUTHOR:  Dave Hale, Colorado School of Mines, 05/17/91

\end{verbatim}
\pagebreak
\begin{verbatim}
 GBBEAM - Gaussian beam synthetic seismograms for a sloth model 	

 gbbeam <rayends >syntraces xg= zg= [optional parameters]		

 Required Parameters:							
 xg=              x coordinates of receiver surface			
 zg=              z coordinates of receiver surface			

 Optional Parameters:							
 ng=101           number of receivers (uniform distributed along surface)
 krecord=1        integer index of receiver surface (see notes below)	
 ang=0.0          array of angles corresponding to amplitudes in amp	
 amp=1.0          array of amplitudes corresponding to angles in ang	
 bw=0             beamwidth at peak frequency				
 nt=251           number of time samples				
 dt=0.004         time sampling interval				
 ft=0.0           first time sample					
 reftrans=0       =1 complex refl/transm. coefficients considered	
 prim             =1, only single-reflected rays are considered	",     
                  =0, only direct hits are considered			
 atten=0          =1 add noncausal attenuation				
                  =2 add causal attenuation				
 lscale=          if defined restricts range of extrapolation		
 aperture=        maximum angle of receiver aperture			
 fpeak=0.1/dt     peak frequency of ricker wavelet			
 infofile         ASCII-file to store useful information		
 NOTES:								
 Only rays that terminate with index krecord will contribute to the	
 synthetic seismograms at the receiver (xg,zg) locations.  The		
 receiver locations are determined by cubic spline interpolation	
 of the specified (xg,zg) coordinates.					



 AUTHOR:  Dave Hale, Colorado School of Mines, 02/09/91
 MODIFIED:  Andreas Rueger, Colorado School of Mines, 08/18/93
	Modifications include: 2.5-D amplitudes, computation of reflection/
			transmission losses, attenuation,
			timewindow, lscale, aperture, beam width, etc.

\end{verbatim}
\pagebreak
\begin{verbatim}
 NORMRAY - dynamic ray tracing for normal incidence rays in a sloth model

    normray <modelfile >rayends [optional parameters]			

 Optional Parameters:							
 caustic	 0: show all rays 1: show only caustic rays		
 nonsurface	 0: show rays which reach surface 1: show all rays      
 surface	 0: shot ray from subsurface 1: from surface               
 nrays 	 number of location to shoot rays                       
 dangle 	 increment of ray angle for one location                
 nangle 	 number of rays shot from one location                  
 ashift 	 shift first taking off angle                           
 xs1 	         x of shooting location                                 
 zs1 	         z of shooting location                                 
 nangle=101     number of takeoff angles				
 fangle=-45     first takeoff angle (in degrees)			
 rayfile        file of ray x,z coordinates of ray-edge intersections	
 nxz=101        number of (x,z) in optional rayfile (see notes below)	
 wavefile       file of ray x,z coordinates uniformly sampled in time	
 nt=101         number of (x,z) in optional wavefile (see notes below)	
 infofile       ASCII-file to store useful information 		
 fresnelfile    used if you want to plot the fresnel volumes. 		
                default is <fresnelfile.bin> 				
 outparfile     contains parameters for the plotting software. 	
                default is <outpar> 					
 krecord        if specified, only rays incident at interface with index
                krecord are displayed and stored			
 prim           =1, only single-reflected rays are plotted 		",     
                =0, only direct hits are displayed  			
 ffreq=-1       FresnelVolume frequency 				
 refseq=1,0,0   index of reflector followed by sequence of reflection (1)
                transmission(0) or ray stops(-1).			
                The default rayend is at the model boundary.		
                NOTE:refseq must be defined for each reflector		
 NOTES:								
 The rayends file contains ray parameters for the locations at which	
 the rays terminate.  							

 The rayfile is useful for making plots of ray paths.			
 nxz should be larger than twice the number of triangles intersected	
 by the rays.								

 The wavefile is useful for making plots of wavefronts.		
 The time sampling interval in the wavefile is tmax/(nt-1),		
 where tmax is the maximum time for all rays.				

 The infofile is useful for collecting information along the		
 individual rays. The fresnelfile contains data used to plot 		
 the Fresnel volumes. The outparfile stores information used 		
 for the plotting software.						



 AUTHOR:  Dave Hale, Colorado School of Mines, 02/16/91
 MODIFIED:  Andreas Rueger, Colorado School of Mines, 08/12/93
	Modifications include: functions writeFresnel, checkIfSourceIsOnEdge;
		options refseq=, krecord=, prim=, infofile=;
		computation of reflection/transmission losses, attenuation.
 MODIFIED: Boyi Ou, Colorado School of Mines, 4/14/95

 Notes:
 This code can shoot rays from specified interface by users, normally you
need to use gbmodel2 to generate interface parameters for this code, both
code have a parameter named nrays, it should be same. If you just want to
shoot rays from one specified location, you need to specify xs1,zs1,
otherwise, leave them alone. If you want to shoot rays from surface, you need
to define surface equal to 1. The rays from one location will be
approximately symetric with direction Normal_direction - ashift.(if nangle is
odd, it is symetric, even, almost symetric. The formula for the first take
off angle is: angle=normal_direction-nangle/2*dangle-ashift. If you only want to
see caustics, you specify caustic=1, if you want to see rays which does not
reach surface, you specify nonsurface=1. 
/
\end{verbatim}
\pagebreak
\begin{verbatim}
 TRI2UNI - convert a TRIangulated model to UNIformly sampled model	

 tri2uni <triangfile >uniformfile n2= n1= [optional parameters]	

 Required Parameters:							
 n1=                     number of samples in the first (fast) dimension
 n2=                     number of samples in the second dimension	

 Optional Parameters:							
 d1=1.0                 sampling interval in first (fast) dimension	
 d2=1.0                 sampling interval in second dimension		
 f1=0.0                 first value in dimension 1 sampled		
 f2=0.0                 first value in dimension 2 sampled		

 Note:									
 The triangulated/uniformly-sampled quantity is assumed to be sloth=1/v^2



 AUTHOR:  Dave Hale, Colorado School of Mines, 04/23/91


\end{verbatim}
\pagebreak
\begin{verbatim}
 TRIMODEL - make a triangulated sloth (1/velocity^2) model                  		

 trimodel >modelfile [optional parameters] 				

 Optional Parameters:							
 xmin=0.0               minimum horizontal coordinate (x)		
 xmax=1.0               maximum horizontal coordinate (x)		
 zmin=0.0               minimum vertical coordinate (z)		
 zmax=1.0               maximum vertical coordinate (z)		
 xedge=                 x coordinates of an edge			
 zedge=                 z coordinates of an edge			
 sedge=                 sloth along an edge				
 kedge=                 array of indices used to identify edges	
 normray               0:do not generate parameters 1: does it   	
 normface              specify which interface to shoot rays   	
 nrays                 number of locations to shoot rays      	        
 sfill=                 x, z, x0, z0, s00, dsdx, dsdz to fill a region	
 densfill=              x, z, dens to fill a region 			
 qfill=                 x, z, Q-factor to fill a region 		
 maxangle=5.0           maximum angle (deg) between adjacent edge segments

 Notes: 								
 More than set of xedge, zedge, and sedge parameters may be 		
 specified, but the numbers of these parameters must be equal. 	

 Within each set, vertices will be connected by fixed edges. 		

 Edge indices in the k array are used to identify edges 		
 specified by the x and z parameters.  The first k index 		
 corresponds to the first set of x and z parameters, the 		
 second k index corresponds to the second set, and so on. 		

 After all vertices and their corresponding sloth values have 		
 been inserted into the model, the optional sfill parameters 		
 are used to fill closed regions bounded by fixed edges. 		
 Let (x,z) denote any point inside a closed region.  Sloth inside 	
 this region is determined by s(x,z) = s00+(x-x0)*dsdx+(z-z0)*dsdz.  	
 The (x,z) component of the sfill parameter is used to identify a 	
 closed region. 							




 AUTHOR:  Dave Hale, Colorado School of Mines, 02/12/91
 MODIFIED: Andreas Rueger, Colorado School of Mines, 01/18/93
    Fill regions with attenuation Q-factors and density values.
 MODIFIED: Craig Artley, Colorado School of Mines, 03/27/94
    Corrected bug in computing s00 in makeSlothForTri() function.
 MODIFIED: Boyi Ou, Colorado School of Mines, 4/14/95
     Make code to generate interface parameters for shooting rays 
     from specified interface

 NOTE:
 When you use normface to specify interface, the number of interface might
 not be the number of interface in the picture, for example, you build a one
 interface model, this interface is very long, so in the shell, you use three
 part of xedge,zedge,sedge to make this interface, so when you use normface to
 specify interface, this interface is just part of whole interface. If you
 want see the normal rays from entire interface, you need to maek normal ray
 picture few times, and merge them together.
 

\end{verbatim}
\pagebreak
\begin{verbatim}
 TRIRAY - dynamic RAY tracing for a TRIangulated sloth model		

  triray <modelfile >rayends [optional parameters]			

 Optional Parameters:							
 xs=(max-min)/2 x coordinate of source (default is halfway across model)
 zs=min	 z coordinate of source (default is at top of model)	
 nangle=101	number of takeoff angles				
 fangle=-45	first takeoff angle (in degrees)			
 langle=45	last takeoff angle (in degrees)				
 rayfile=	file of ray x,z coordinates of ray-edge intersections	
 nxz=101	number of (x,z) in optional rayfile (see notes below)	
 wavefile=	file of ray x,z coordinates uniformly sampled in time	
 nt=101	number of (x,z) in optional wavefile (see notes below)	
 infofile=	ASCII-file to store useful information 		
 fresnelfile=  used if you want to plot the fresnel volumes.		
		default is <fresnelfile.bin> 				
 outparfile=	contains parameters for the plotting software.		
		default is <outpar> 					
 krecord=	if specified, only rays incident at interface with index
		krecord are displayed and stored			
 prim=	   =1, only single-reflected rays are plotted			
		=0, only direct hits are displayed  			
 ffreq=-1	FresnelVolume frequency 				
 refseq=1,0,0  index of reflector followed by sequence of reflection (1)
		transmission(0) or ray stops(-1).			
		The default rayend is at the model boundary.		
		NOTE:refseq must be defined for each reflector		
 NOTES:								
 The rayends file contains ray parameters for the locations at which	
 the rays terminate.  							

 The rayfile is useful for making plots of ray paths.			
 nxz should be larger than twice the number of triangles intersected	
 by the rays.								

 The wavefile is useful for making plots of wavefronts.		
 The time sampling interval in the wavefile is tmax/(nt-1),		
 where tmax is the maximum time for all rays.				

 The infofile is useful for collecting information along the		
 individual rays. The fresnelfile contains data used to plot 		
 the Fresnel volumes. The outparfile stores information used 		
 for the plotting software.						



 AUTHOR:  Dave Hale, Colorado School of Mines, 02/16/91
 MODIFIED:  Andreas Rueger, Colorado School of Mines, 08/12/93
	Modifications include: functions writeFresnel, checkIfSourceIsOnEdge;
		options refseq=, krecord=, prim=, infofile=;
		computation of reflection/transmission losses, attenuation.

\end{verbatim}
\pagebreak
\begin{verbatim}
 TRISEIS - Gaussian beam synthetic seismograms for a sloth model	

  triseis <modelfile >seisfile xs= zs= xg= zg= [optional parameters]	

 Required Parameters:							
 xs=            x coordinates of source surface			
 zs=            z coordinates of source surface			
 xg=            x coordinates of receiver surface			
 zg=            z coordinates of receiver surface			

 Optional Parameters:							
 ns=1           number of sources uniformly distributed along s surface
 ds=            increment between source locations (see notes below)	
 fs=0.0         first source location (relative to start of s surface)	
 ng=101         number of receivers uniformly distributed along g surface
 dg=            increment between receiver locations (see notes below)	
 fg=0.0         first receiver location (relative to start of g surface)
 dgds=0.0       change in receiver location with respect to source location
 krecord=1      integer index of receiver surface (see notes below)	
 kreflect=-1    integer index of reflecting surface (see notes below)	
 prim           =1, only single-reflected rays are considered 		",     
                =0, only direct hits are considered  			
 bw=0           beamwidth at peak frequency				
 nt=251         number of time samples					
 dt=0.004       time sampling interval					
 ft=0.0         first time sample					
 nangle=101     number of ray takeoff angles				
 fangle=-45     first ray takeoff angle (in degrees)			
 langle=45      last ray takeoff angle (in degrees)			
 reftrans=0     =1 complex refl/transm. coefficients considered 	
 atten=0        =1 add noncausal attenuation 				
                =2 add causal attenuation 				
 lscale=        if defined restricts range of extrapolation		
 fpeak=0.1/dt   peak frequency of ricker wavelet 			
 aperture=      maximum angle of receiver aperture 			

 NOTES:								
 Only rays that terminate with index krecord will contribute to the	
 synthetic seismograms at the receiver (xg,zg) locations.  The		
 source and receiver locations are determined by cubic spline		
 interpolation of the specified (xs,zs) and (xg,zg) coordinates.	
 The default source location increment (ds) is determined to span	
 the source surface defined by (xs,zs).  Likewise for dg.		



 AUTHOR:  Dave Hale, Colorado School of Mines, 02/09/91
 MODIFIED:  Andreas Rueger, Colorado School of Mines, 08/18/93
	Modifications include: 2.5-D amplitudes, correction for ref/transm,
			timewindow, lscale, aperture, beam width, etc.

\end{verbatim}
\pagebreak
\begin{verbatim}
 UNI2TRI - convert UNIformly sampled model to a TRIangulated model	

 uni2tri <slothfile >modelfile n2= n1= [optional parameters]		

 Required Parameters:							
 n1=                    number of samples in first (fast) dimension	
 n2=                    number of samples in second dimension		

 Optional Parameters:							
 d1=1.0                 sampling interval in dimension 1		
 d2=1.0                 sampling interval in dimension 2		
 f1=0.0                 first value in dimension 1			
 f2=0.0                 first value in dimesion 2			
 ifile=                 triangulated model file used as initial model	
 errmax=                maximum sloth error (see notes below)		
 verbose=1              =0 for silence					
                        =1 to report maximum error at each stage to stderr
                        =2 to also write the normalized error to efile	
 efile=emax.dat         filename for error file (for verbose=2)	
 mm=0			output every mm-th intermediate model (0 for none)
 mfile=intmodel        intermediate models written to intmodel%d	
 method=3              =1 add 1 vertex at maximum error		
                       =2 add vertex to every triangle that exceeds errmax
                       =3 method 2, but avoid closely spaced vertices	
 tol=10                closeness criterion for (in samples)		
 sfill=                 x, z, x0, z0, s00, dsdx, dsdz to fill a region	

 Notes:								
 Triangles are constructed until the maximum error is			
 not greater than the user-specified errmax.  The default errmax	
 is 1% of the maximum value in the sampled input file.			

 After the uniform values have been triangulated, the optional sfill	
 parameters are used to fill closed regions bounded by fixed edges.	
 Let (x,z) denote any point inside a closed region.  Values inside	
 this region is determined by s(x,z) = s00+(x-x0)*dsdx+(z-z0)*dsdz.	
 The (x,z) component of the sfill parameter is used to identify a	
 closed region.							

 The uniformly sampled quantity is assumed to be sloth=1/v^2.		



 AUTHOR:  Craig Artley, Colorado School of Mines, 03/31/94
 NOTE:  After a program outlined by Dave Hale, 12/27/90.


\end{verbatim}
\pagebreak
\begin{verbatim}
 SPSPLOT - plot a triangulated sloth function s(x,z) via PostScript	

 spsplot <modelfile >postscriptfile [optional parameters]		

 Optional Parameters:							
 gedge=0.0             gray to draw fixed edges (in interval [0.0,1.0])
 gtri=1.0              gray to draw non-fixed edges of triangles 	
 gmin=0.0              min gray to shade triangles (in interval [0.0,1.0])
 gmax=1.0              max gray to shade triangles (in interval [0.0,1.0])
 sgmin=minimum s(x,z)  s(x,y) corresponding to gmin 			
 sgmax=maximum s(x,z)  s(x,y) corresponding to gmax 			
 xbox=1.5              offset in inches of left side of axes box 	
 ybox=1.5              offset in inches of bottom side of axes box	
 wbox=6.0              width in inches of axes box			
 hbox=8.0              height in inches of axes box			
 xbeg=xmin             value at which x axis begins			
 xend=xmax             value at which x axis ends			
 dxnum=0.0             numbered tic interval on x axis (0.0 for automatic)
 fxnum=xmin            first numbered tic on x axis (used if dxnum not 0.0)
 nxtic=1               number of tics per numbered tic on x axis	
 gridx=none            grid lines on x axis - none, dot, dash, or solid
 labelx=               label on x axis					
 zbeg=zmin             value at which z axis begins			
 zend=zmax             value at which z axis ends			
 dznum=0.0             numbered tic interval on z axis (0.0 for automatic)
 fznum=zmin            first numbered tic on z axis (used if dynum not 0.0)
 nztic=1               number of tics per numbered tic on z axis	
 gridz=none            grid lines on z axis - none, dot, dash, or solid
 labelz=               label on z axis					
 labelfont=Helvetica   font name for axes labels			
 labelsize=12          font size for axes labels			
 title=                title of plot					
 titlefont=Helvetica-Bold  font name for title				
 titlesize=24          font size for title				
 titlecolor=black      color of title					
 axescolor=black       color of axes					
 gridcolor=black       color of grid					
 style=seismic         normal (z axis horizontal, x axis vertical) or	
                       seismic (z axis vertical, x axis horizontal)	

 Note:  A value of gedge or gtri outside the interval [0.0,1.0]	
 results in that class of edge not being drawn.			



 AUTHOR:  Dave Hale, Colorado School of Mines, 10/18/90
 MODIFIED: Craig Artley, Colorado School of Mines, 03/27/94
    Tweaks to improve PostScript header, add basic color support.

 NOTE:  Have observed errors in output when compiled with optimization
    under NEXTSTEP 3.1.  Caveat Emptor.

 Modified: Morten Wendell Pedersen, Aarhus University, 23/3-97
           Added ticwidth,axeswidth, gridwidth parameters 

\end{verbatim}
\pagebreak
Shells: 
\begin{verbatim}
______
 ARGV - give examples of dereferencing char **argv

 Usage: argv

\end{verbatim}
\begin{verbatim}
______
 COPYRIGHT - insert CSM COPYRIGHT lines at top of files in working directory

 Usage: copyright file(s)

\end{verbatim}
\begin{verbatim}
______
 CPALL , RCPALL - for local and remote directory tree/file transfer

 Usage: cpall sourcedir destinationdir 
 Caveat: destinationdir must exist and be writeable by the user

 Usage: rcpall sourcedir remotemachine  destinationdir 

 If user name is different on the remote machine, then second"
 entry is "remotemachine -l remoteusername"
 Caveats: rsh, copy, and write permissions required
         You must be on the source machine,
	 destinationdir must exist and be writeable by the user.

 Notes: Both of these shell scripts use tar to do the transfer.

\end{verbatim}
\begin{verbatim}
______
 CWPFIND - look for files with patterns in CWPROOT/src/cwp/lib

 Usage: cwpfind  pattern_fragment
        cwpfind -e exact_pattern

\end{verbatim}
\begin{verbatim}
______
 Grep  - recursively call egrep in pwd

 Usage: Grep [-egrep_options] pattern

 Caution:  Do NOT redirect into file in pwd, either use something
	like  >../Grep.out or perhaps pipe output into mail to yourself.

 Author: Jack, 7/95

\end{verbatim}
\begin{verbatim}
______
 DIRTREE - show DIRectory TREE

 Usage: dirtree

\end{verbatim}
\begin{verbatim}
______
 FILETYPE - list all files of given type

 Usage: filetype string_from_file_output

 Examples:
	filetype text      - list printable files
       filetype stripped  - list unstripped files
\end{verbatim}
\begin{verbatim}
______
 NEWCASE - Changes the case of all the filenames in a directory, dir

 Usage: newcase -l dir  change all filenames to lower case 
                -u dir  change all filenames to upper case

 Notes: Useful for files downloaded from VAX.

\end{verbatim}
\begin{verbatim}
______
 OVERWRITE - copy stdin to stdout after EOF

 This shell is called from the shell script:    replace

\end{verbatim}
\begin{verbatim}
______
 PRECEDENCE - give table of C precedences from Kernighan and Ritchie

 Usage: precedence

\end{verbatim}
\begin{verbatim}
______
 REPLACE - REPLACE string1 with string2  in files

 Usage:  replace string1 string2 files

\end{verbatim}
\begin{verbatim}
______
 THIS_YEAR - print the current year

 Usage: this_year

 NOTES - useful for building dated filenames, etc.

\end{verbatim}
\begin{verbatim}
______
 TIME_NOW - prints time in ZULU format with no spaces 

 Usage: time_now

 Note: Useful for building dated filenames

\end{verbatim}
\begin{verbatim}
______
 TODAYS_DATE - prints today's date in ZULU format with no spaces 

 Usage: todays_date

 Note: Useful for building dated filenames

\end{verbatim}
\begin{verbatim}
______
 USERNAMES - get list of all login names

 Usage: usernames

\end{verbatim}
\begin{verbatim}
______
 VARLIST - list variables used in a Fortran program

 Usage: varlist file.f ... 
 
 Output is in the file: vars.file

\end{verbatim}
\begin{verbatim}
______
 WEEKDAY - prints today's WEEKDAY designation

 Usage: weekday

 Note: Useful for building dated filenames

\end{verbatim}
\begin{verbatim}
______
 ZAP - kill processes by name

 Typical usages:
	zap ximage
	zap 'xmovie|xgraph'

 Zap accepts full pattern matching for the process names

 Caveat: zap assumes that the FIRST field produced by the
	Unix "ps" command is the pid (process identifier) number.
	If not, change the number in the awk print statement to
	the appropriate field.

 Author: Jack, 6/95 -- after Kernighan and Pike's zap

\end{verbatim}
\pagebreak
\begin{verbatim}
______
 GENDOCS - generate complete list of selfdocs in latex form

 Usage: gendocs -o  output filename is:  selfdocs.tex

\end{verbatim}
\begin{verbatim}
______
 STRIPTOTXT -  put files from $CWPROOT/src/doc/Stripped into a new
               directory in the form $CWPROOT/src/TXT/NAME.txt

 Usage: striptotxt

 Author: John Stockwell,  Sept 2001
\end{verbatim}
\begin{verbatim}
______
 UPDATEDOCALL - put self-docs in ../doc/Stripped

 Usage: updatedocall  

 Note: this shell uses updatedoc to update the  database used by
       suname and gendocs 

\end{verbatim}
\begin{verbatim}
______
 UPDATEDOC - put self-docs in ../doc/Stripped and ../doc/Headers

 Usage: updatedoc  path

 Notes:
 Paths include: cwp/main cwp/lib cwp/shell par/main par/lib par/shell
     xplot/main xplot/lib psplot/main psplot/lib psplot/shell
     Xtcwp/main Xtcwp/lib Sfio/main 
      su/main/amplitudes su/main/attributes_parameter_estimation
      su/main/convolution_correlation /su/main/data_compression
      su/main/data_conversion su/main/datuming su/main/decon_shaping
      su/main/dip_moveout su/main/filters su/main/headers su/main/interp_extrap
      su/main/migration_inversion su/main/multicomponent su/main/noise
      su/main/operations su/main/picking su/main/stacking su/main/statics
      su/main/stretching_moveout_resamp su/main/supromax 
      su/main/synthetics_waveforms_testpatterns su/main/tapering
      su/main/transforms su/main/velocity_analysis su/main/well_logs 
      su/main/windowing_sorting_muting
     su/lib su/shell su/graphics/psplot
     su/graphics/xplot tri/main tri/lib xtri tri/graphics/psplot
     tetra/lib tetra/main
     comp/dct/lib comp/dct/main comp/dct/libutil comp/dwpt/1d/lib
     comp/dwpt/1d/main comp/dwpt/2d/lib comp/dwpt/2d/main
     
 Use: updatedocall to update full directory, use updatehead to
      to update the master header file.

 This shell builds the database used by  suname and gendocs 
\end{verbatim}
\begin{verbatim}
______
 UPDATEHEAD - update ../doc/Headers/Headers.all

 Usage: updatehead

 Notes:
      
 This file builds the database used by  suname 
\end{verbatim}
\pagebreak
\begin{verbatim}
______
 LOOKPAR - show getpar lines in SU code with defines evaluated

 Usage: lookpar filename ...

\end{verbatim}
\begin{verbatim}
______
 MAXDIFF - find absolute maximum difference in two segy data sets

 Usage: maxdiff file1 file2

\end{verbatim}
\begin{verbatim}
______
 RECIP - sum opposing (reciprocal) offsets in cdp sorted data

 Usage: recip <stdin >stdout

\end{verbatim}
\begin{verbatim}
______
 RMAXDIFF - find percentage maximum difference in two segy data sets

 Usage: rmaxdiff file1 file2

\end{verbatim}
\begin{verbatim}
______
 SUAGC - perform agc on SU data 

 Note: this is an interface to sugain for backward compatibility
 See selfdoc of:   sugain   for more information
\end{verbatim}
\begin{verbatim}
______
 SUBAND - Trapezoid-like Sin squared tapered Bandpass filter via  SUFILTER
 
 Usage:   suband < stdin > stdout 
 
 Note: this shell mimmics the old program SUBAND, supersceded by SUFILTER
 See selfdoc of:   sufilter   for more information
\end{verbatim}
\begin{verbatim}
______
 SUDIFF, SUSUM, SUPROD, SUQUO, SUPTDIFF, SUPTSUM,
 SUPTPROD, SUPTQUO - difference, sum, product, quotient of two SU data
                     sets via suop2

 Usage:
 sudiff file1 file2 > stdout
 susum file1 file2 > stdout
 ...etc

 Note: uses   suop2  to perform the computation
\end{verbatim}
\begin{verbatim}
______
 SUDOC - get DOC listing for code

 Usage: sudoc name

 Note: Use this shell script to get selfdoc information for
 codes labeled with and asterisk (*) or pound sign (#) in suname list      
\end{verbatim}
\begin{verbatim}
______
 SUENV - Instantaneous amplitude, frequency, and phase via: SUATTRIBUTES
 
 Usage:   suenv < stdin > stdout 
 
 Note: this shell mimmics the old program SUENV, supersceded by SUATTRIBUTES
 See selfdoc of:   suattributes   for more information
\end{verbatim}
\begin{verbatim}
______
 SUFIND - get info from self-docs

 Usage: sufind [-v -n] string

 sufind string    gives a brief synopsis
 sufind -v string  is a verbose hunt for relevant items
 sufind -n name_fragment      searches for command name

\end{verbatim}
\begin{verbatim}
______
 SUFIND - get info from self-docs

 Usage: sufind [-v -n] string

 sufind string    gives a brief synopsis
 sufind -v string  is a verbose hunt for relevant items
 sufind -n name_fragment      searches for command name

 Author: CWP: Jack K. Cohen,  1992
 Modified by: CWP: S. Narahara, 04/11/1998.

\end{verbatim}
\begin{verbatim}
______
 SUGENDOCS - generate complete list of selfdocs in latex form

 Usage: sugendocs -o  output filename is: selfdocs.tex
 Note: this shell simply calls    gendocs

\end{verbatim}
\begin{verbatim}
______
 SUHELP - list the CWP/SU programs and shells

 Usage:   suhelp

\end{verbatim}
\begin{verbatim}
______
 SUKEYWORD -- guide to SU keywords in segy.h

 Usage: sukeyword -o            to begin at the top of segy.h
        sukeyword [string]      to find [string]

 Note:  keyword=  occurs in many SU programs.
\end{verbatim}
\begin{verbatim}
______
 SUNAME - get name line from self-docs

 Usage: suname [name]

 Note: dummy selfdocs have been included in all cwp and shell programs
       that don't have automatic selfdocs.
\end{verbatim}
\begin{verbatim}
______
 UNGLITCH - zonk outliers in data

 Usage: unglitch < stdin

 Note: this shell just invokes:  sugain < stdin qclip=.99 > stdout
 See selfdoc of:   sugain   for further information
\end{verbatim}
\pagebreak
\begin{verbatim}
______
 MERGE2 - put 2 standard size PostScript figures on one page

 Usage: merge2 fig1 fig2

 Notes: Translation values are hard-coded numbers that work well for 
	standard size (8.5 x 11) figures. 
 See selfdoc of:   psmerge for details
\end{verbatim}
\begin{verbatim}
______
 MERGE4 - put 4 standard size PostScript plots on one page

 Usage: merge4 ulfig urfig llfig lrfig

 Note: Translation values are hard-coded numbers that work well for 
	standard size (8.5 x 11) figures.
 See selfdoc of:   psmerge   for further information
\end{verbatim}
\pagebreak
Libs: 
\begin{verbatim}
BASIC - Basic C function interface to PostScript

beginps		write PostScript prolog (including %%Pages comment)
endps		write PostScript trailer (including %%Pages comment)
begineps	write encapsulated PostScript prolog (no %%Pages comment)
endeps		write encapsulated PostScript trailer (no %%Pages comment)
boundingbox	set BoundingBox to llx lly urx ury
newpage		print "%%%%Page: label ordinal" to stdout
showpage	print "showpage" to stdout
gsave		print "GS" to stdout
grestore	print "GR" to stdout
newpath		print "NP" to stdout
closepath	print "CP" to stdout
clip		print "clip" to stdout
translate	print "tx ty TR" to stdout, tx,ty = translation in x,y
scale		print "sx sy SC" to stdout, sx,sy = scaling in x,y
rotate		print "angle RO" to stdout, angle = rotation angle
concat		print "m[0] m[1] m[2] m[3] m[4] m[5] CAT" to stdout
setgray		print "gray setgray" to stdout, gray is 0-255 gray level
setrgbcolor	print "red green blue setrgbcolor" to stdout
			red,green,blue = 0-255 red,green,blue levels
setcolor	set color by name based on definition in color structure
setlinewidth	print "width SLW" to stdout, width = desired line width
setlinejoin	print "code setlinejoin"
setdash		print "[ dash ] offset setdash" to stdout
			dash = array defining dash, offset = dash offset
moveto		print "x y M" to stdout,   move to x,y
rmoveto		print "x y RM" to stdout,  move to x,y
lineto		print "x y L" to stdout, draw a line to x,y
rlineto		print "x y RL" to stdout, draw a line to x,y
arc		print "x y r ang1 ang2 arc" to stdout, draw an arc
			x,y = vertex  r = radius from ang1 to ang2
stroke		print "S" to stdout
fill		print "F" to stdout
show 		print "str SH" to stdout, show a string
justshow	justify and show a string
image		write a sampled gray-scale image
rgbimage	write sampled color (rgb) image
setfont		execute findfont, scalefont, and setfont for specified font
			 and size
fontbbox	determine font bounding box for specified font and size
fontheight	return maximum height for specified font and size
fontwidth	return maximum width for specified font and size
fontcapheight	return maximum capheight for specified font and size
fontxheight	return maximum xheight for specified font and size
fontdescender	return maximum descender for specified font and size
polyline	draw a segmented line
markto		draw a mark at specified location
rectclip	set a rectangular clipping path
rectfill	draw a filled rectangle
strokerect	stroke a rectangle

Function Prototypes:
void beginps (void);
void endps (void);
void begineps (void);
void endeps (void);
void newpage (const char *label, int ordinal);
void boundingbox (int llx, int lly, int urx, int ury);
void showpage (void);
void gsave (void);
void grestore (void);
void newpath (void);
void closepath (void);
void clip(void);
void translate (float tx, float ty);
void scale (float sx, float sy);
void rotate (float angle);
void concat (float m[]);
void setgray (float gray);
void setrgbcolor (float red, float green, float blue);
void setcolor (const char *name);
void setlinewidth (float width);
void setlinejoin (int code);
void setdash (float dash[], int ndash, float offset);
void moveto (float x, float y);
void rmoveto (float x, float y);
void lineto (float x, float y);
void rlineto (float x, float y);
void arc (float x, float y, float r, float ang1, float ang2);
void stroke (void);
void fill (void);
void show (const char *str);
void justshow (float just, const char *str);
void image (int w, int h, int bps, float m[], unsigned char *samples);
void rgbimage (int w, int h, int bpc, float m[], unsigned char *samples);
void cymkimage (int w, int h, int bpc, float m[], unsigned char *samples);
void setfont (const char *fontname, float fontsize);
void fontbbox (const char *fontname, float fontsize, float bbox[]);
float fontheight (const char *fontname, float fontsize);
float fontwidth (const char *fontname, float fontsize);
float fontcapheight (const char *fontname, float fontsize);
float fontxheight (const char *fontname, float fontsize);
float fontdescender (const char *fontname, float fontsize);
float fontascender (const char *fontname, float fontsize);
void polyline (const float *x, const float *y, int n);
void markto (float x, float y, int index, float size);
void rectclip (float x, float y, float width, float height);
void rectfill (float x, float y, float width, float height);
void rectstroke (float x, float y, float width, float height);

justshow:
Input:
just		justification factor
str		string

image:
Input:
w		width of image (in samples)
h		height of image (in samples)
bps		number of bits per sample
m		array[6] containing image matrix
samples		array[w*h] of sample values

rgbimage:
Input:
w		width of image (in samples)
h		height of image (in samples)
bpc		number of bits per component
m		array[6] containing image matrix
samples		array[3*w*h] of sample values

cymkimage:
Input:
w		width of image (in samples)
h		height of image (in samples)
bpc		number of bits per component
m		array[6] containing image matrix
samples		array[4*w*h] of sample values


polyline:
Input:
x		array[n] of x-coordinates
y		array[n] of y-coordinates
n		number of points

markto:
Input:
x		x-coordinate of mark
y		y-coordinate of mark
index		type of mark to draw
size		size of mark

rectclip:
Input:
x		x-coordinate of clipping path origin
y		y-coordinate of clipping path origin
width		width of clipping path
height		height of clipping path

rectfill:
Input:
x		x-coordinate of rectangle origin
y		y-coordinate of rectangle origin
width		width of rectangle
height		height of rectangle

strokerect:
Input:
x		x-coordinate of rectangle origin
y		y-coordinate of rectangle origin
width		width of rectangle
height		height of rectangle

Notes:
The majority of these routines are self explanatory. They are just
C wrappers that echo PostScript graphics commands.

justshow:
The justification factor positions the string relative to the current point.
just" may assume any value, but the common uses are:
	-1.0	right-justify the string
	-0.5	center the string on the current point
	 0.0	left-justify the string (like using "show")

image:
Level 1 PostScript implementations support 1, 2, 4, and 8 bits per
sample.  Level 2 adds support for 12 bits per sample.
Samples are hex-encoded, and output lines are limited to 78 characters.

rgbimage:
In general, Level 1 PostScript implementations do not support rgbimage.
Level 2 supports 1, 2, 4, 8, and 12 bits per color component.  The
samples array should contain three color components (in R,G,B... order)
for each sample value.  Samples are hex-encoded, and output lines are
limited to 78 characters.

polyline:
The path is stroked every 200 points.
References:

Author:  Dave Hale, Colorado School of Mines, 1989
with modifications by Craig Artley, Colorado School of Mines, 1991, and
additions by Dave Hale, Advance Geophysical, 1992.
\end{verbatim}
\pagebreak
\begin{verbatim}
PSAXESBOX3 -  Functions draw an axes box via PostScript, estimate bounding box
               these are versions of psAxesBox and psAxesBox3 for psmovie.

psAxesBox3	draw an axes box via PostScript
psAxesBBox3	estimate bounding box for an axes box drawn via psAxesBox3

Function Prototypes:
void psAxesBox3(
	float x, float y, float width, float height,
	float x1Beg, float x1End, float p1Beg, float p1End, 
	float d1Num, float f1Num, int n1Tic, int grid1, char *label1,
	float x2Beg, float x2End, float p2Beg, float p2End,
	float d2Num, float f2Num, int n2Tic, int grid2, char *label2,
	char *labelFont, float labelSize,
	char *title, char *titleFont, float titleSize,
	int style, char *title2);
void psAxesBBox3(
	float x, float y, float width, float height,
	char *labelFont, float labelSize,
	char *titleFont, float titleSize,
	int style, int bbox[]);

Input:
x		x coordinate of lower left corner of box
y		y coordinate of lower left corner of box
width		width of box
height		height of box
x1Beg  		axis value at beginning of axis 1
x1End		axis value at end of axis 1
p1Beg  		pad value at beginning of axis 1
p1End		pad value at end of axis 1
d1Num		numbered tic increment for axis 1 (0.0 for automatic)
f1Num		first numbered tic for axis 1
n1Tic		number of horizontal tics per numbered tic for axis 1
grid1		grid code for axis 1:  NONE, DOT, DASH, or SOLID
label1		label for axis 1
x2Beg  		axis value at beginning of axis 2
x2End		axis value at end of axis 2
p2Beg  		pad value at beginning of axis 2
p2End		pad value at end of axis 2
d2Num		vertical numbered tic increment (0.0 for automatic)
f2Num		first numbered vertical tic
n2Tic		number of vertical tics per numbered tic
grid2		grid code for vertical axis:  NONE, DOT, DASH, or SOLID
label2		vertical axis label
labelFont	name of font to use for axes labels
labelSize	size of font to use for axes labels
title		axes box title
titleFont	name of font to use for title
titleSize	size of font to use for title
style		NORMAL (axis 1 on bottom, axis 2 on left) 
		  SEISMIC (axis 1 on left, axis 2 on top)
title2		second title

psAxesBBox3:
Input:
x		x coordinate of lower left corner of box
y		y coordinate of lower left corner of box
width		width of box
height		height of box
labelFont	name of font to use for axes labels
labelSize	size of font to use for axes labels
titleFont	name of font to use for title
titleSize	size of font to use for title
style		NORMAL (axis 1 on bottom, axis 2 on left) 
			  SEISMIC (axis 1 on left, axis 2 on top)
Output:
bbox		bounding box (bbox[0:3] = llx, lly, ulx, uly)

Notes:
psAxesBox3:
psAxesBox will determine the numbered tic increment and first
numbered tic automatically, if the specified increment is zero.

Pad values must be specified in the same units as the corresponding 
axes values.  These pads are useful when the contents of the axes box 
requires more space than implied by the axes values.  For example, 
the first and last seismic wiggle traces plotted inside an axes box 
will typically extend beyond the axes values corresponding to the 
first and last traces.  However, all tics will lie with the limits 
specified in the axes values (x1Beg, x1End, x2Beg, x2End).

psAxesBBox3:
psAxesBBox uses font sizes to estimate the bounding box for
an axes box drawn with psAxesBox.  To be on the safe side, 
psAxesBBox overestimates.

psAxesBBox assumes that the axes labels and titles do not extend
beyond the corresponding edges of the axes box.

References: 
(see references in basic.c)
Author:  Dave Hale, Colorado School of Mines, 06/27/89
	 modified by Zhiming Li, CSM, 7/1/90
\end{verbatim}
\pagebreak
\begin{verbatim}
PSAXESBOX - Functions to draw PostScript axes and estimate bounding box

psAxesBox	Draw an axes box via PostScript
psAxesBBox	estimate bounding box for an axes box drawn via psAxesBox

Function Prototypes:
void psAxesBox(
	float x, float y, float width, float height,
	float x1Beg, float x1End, float p1Beg, float p1End, 
	float d1Num, float f1Num, int n1Tic, int grid1, char *label1,
	float x2Beg, float x2End, float p2Beg, float p2End,
	float d2Num, float f2Num, int n2Tic, int grid2, char *label2,
	char *labelFont, float labelSize,
	char *title, char *titleFont, float titleSize,
	char *titleColor, char *axesColor, char *gridColor,
	float ticwidth, float axeswidth, float gridwidth,
	int style);

void psAxesBBox(
	float x, float y, float width, float height,
	char *labelFont, float labelSize,
	char *titleFont, float titleSize,
	int style, int bbox[]);

psAxesBox:
Input:
x		x coordinate of lower left corner of box
y		y coordinate of lower left corner of box
width		width of box
height		height of box
x1Beg  		axis value at beginning of axis 1
x1End		axis value at end of axis 1
p1Beg  		pad value at beginning of axis 1
p1End		pad value at end of axis 1
d1Num		numbered tic increment for axis 1 (0.0 for automatic)
f1Num		first numbered tic for axis 1
n1Tic		number of horizontal tics per numbered tic for axis 1
grid1		grid code for axis 1:  NONE, DOT, DASH, or SOLID
label1		label for axis 1
x2Beg  		axis value at beginning of axis 2
x2End		axis value at end of axis 2
p2Beg  		pad value at beginning of axis 2
p2End		pad value at end of axis 2
d2Num		vertical numbered tic increment (0.0 for automatic)
f2Num		first numbered vertical tic
n2Tic		number of vertical tics per numbered tic
grid2		grid code for vertical axis:  NONE, DOT, DASH, or SOLID
label2		vertical axis label
labelFont	name of font to use for axes labels
labelSize	size of font to use for axes labels
title		axes box title
titleFont	name of font to use for title
titleSize	size of font to use for title
titleColor	color to use for title
axesColor	color to use for axes and axes labels
gridColor	color to use for grid lines
axeswidth   width (in points) of axes
ticwidth    width (in points) of tic marks
gridwidth   width (in points) of grid lines
style		NORMAL (axis 1 on bottom, axis 2 on left) 
		SEISMIC (axis 1 on left, axis 2 on top)

psAxesBBox:
Input:
x		x coordinate of lower left corner of box
y		y coordinate of lower left corner of box
width		width of box
height		height of box
labelFont	name of font to use for axes labels
labelSize	size of font to use for axes labels
titleFont	name of font to use for title
titleSize	size of font to use for title
style		NORMAL (axis 1 on bottom, axis 2 on left) 
			  SEISMIC (axis 1 on left, axis 2 on top)
Output:
bbox		bounding box (bbox[0:3] = llx, lly, ulx, uly)

Notes:
psAxesBox:
psAxesBox will determine the numbered tic increment and first numbered
tic automatically, if the specified increment is zero.  Axis numbering
is in scientific notation, if necessary and is plotted to four
significant digits.

Pad values must be specified in the same units as the corresponding
axes values.  These pads are useful when the contents of the axes box
requires more space than implied by the axes values.  For example, the
first and last seismic wiggle traces plotted inside an axes box will
typically extend beyond the axes values corresponding to the first and
last traces.  However, all tics will lie with the limits specified in
the axes values (x1Beg, x1End, x2Beg, x2End).

psAxesBBox:
psAxesBBox uses font sizes to estimate the bounding box for
an axes box drawn with psAxesBox.  To be on the safe side, 
psAxesBBox overestimates.

psAxesBBox assumes that the axes labels and titles do not extend
beyond the corresponding edges of the axes box.

References:
(see References for basic.c)
Author:   Dave Hale,  Colorado School of Mines, 06/27/89
Modified: Ken Larner, Colorado School of Mines, 08/30/90
Modified: Dave Hale, Advance Geophysical, 10/18/92
Added color parameters for title, axes, and grid.
Modified: Morten Wendell Pedersen, Aarhus University, 23/3-97
Added ticwidth,axeswidth, gridwidth parameters 
\end{verbatim}
\pagebreak
\begin{verbatim}
PSCAXESBOX - Draw an axes box for cube via PostScript

psCubeAxesBox	Draw an axes box for cube via PostScript

Function Prototype:
void psCubeAxesBox(
	float x, float y, float size1, float size2, float size3, float angle,
	float x1Beg, float x1End, float p1Beg, float p1End, 
	float d1Num, float f1Num, int n1Tic, int grid1, char *label1,
	float x2Beg, float x2End, float p2Beg, float p2End,
	float d2Num, float f2Num, int n2Tic, int grid2, char *label2,
	float x3Beg, float x3End, float p3Beg, float p3End,
	float d3Num, float f3Num, int n3Tic, int grid3, char *label3,
	char *labelFont, float labelSize,
	char *title, char *titleFont, float titleSize,
	char *titleColor, char *axesColor, char *gridColor);

Input:
x		x coordinate of lower left corner of box
y		y coordinate of lower left corner of box
size1		size of 1st dimension of the input cube
size2		size of 2nd dimension of the input cube
size3		size of 3rd dimension of the input cube
angle		projection angle of the cube	 
x1Beg  		axis value at beginning of axis 1
x1End		axis value at end of axis 1
p1Beg  		pad value at beginning of axis 1
p1End		pad value at end of axis 1
d1Num		numbered tic increment for axis 1 (0.0 for automatic)
f1Num		first numbered tic for axis 1
n1Tic		number of tics for axis 1
grid1		grid code for axis 1:  NONE, DOT, DASH, or SOLID
label1		label for axis 1
x2Beg  		axis value at beginning of axis 2
x2End		axis value at end of axis 2
p2Beg  		pad value at beginning of axis 2
p2End		pad value at end of axis 2
d2Num		numbered tic increment for axis 2 (0.0 for automatic)
f2Num		first numbered tic for axis 2
n2Tic		number of tics for axis 2
grid2		grid code for axis 2:  NONE, DOT, DASH, or SOLID
label2		label for axis 2
x3Beg  		axis value at beginning of axis 3
x3End		axis value at end of axis 3
p3Beg  		pad value at beginning of axis 3
p3End		pad value at end of axis 3
d3Num		numbered tic increment for axis 3 (0.0 for automatic)
f3Num		first numbered tic for axis 3
n3Tic		number of tics for axis 3
grid3		grid code for axis 3:  NONE, DOT, DASH, or SOLID
label3		label for axis 3
labelFont	name of font to use for axes labels
labelSize	size of font to use for axes labels
title		axes box title
titleFont	name of font to use for title
titleSize	size of font to use for title
titleColor	color to use for title
axesColor	color to use for axes and axes labels
gridColor	color to use for grid lines
Authors:   Zhiming Li & Dave Hale,  Colorado School of Mines, 6/90
Modified: Craig Artley, Colorado School of Mines, 3/12/93
	Changed name to psCubeAxesBox (from psAxes3), fixed minor bugs.
Modified: Craig Artley, Colorado School of Mines, 12/16/93
	Added color parameters for title, axes, and grid.
\end{verbatim}
\pagebreak
\begin{verbatim}
PSCONTOUR - draw contour of a two-dimensional array via PostScript

psContour	draw contour of a two-dimensional array via PostScript

Function Prototype:
void psContour (float c, int nx, float x[], int ny, float y[], float z[],
	float lcs, char *lcf, char *lcc, float *w, int nplaces);

Input:
c			contour value
nx			number of x-coordinates
x			array of x-coordinates (see notes below)
ny			number of y-coordinates
y			array of y-coordinates (see notes below)
lcs			font size of contour label
lcf			font name of contour label
lcc			color of contour label
LSB flag arrays (see Notes):
z			array of nx*ny z(x,y) values (see notes below)
w			array of nx*ny z(x,y) values (see notes below)

Notes:
The two-dimensional array z is actually passed as a one-dimensional
array containing nx*ny values, stored with nx fast and ny slow.

The x and y arrays define a grid that is not necessarily 
uniformly-sampled.  Linear interpolation of z values on the 
grid defined by the x and y arrays is used to determine z values 
between the gridpoints.
		 
The two least significant bits of z are used to mark intersections
of the contour with the x,y grid; therefore, the z values will
almost always be altered (slightly) by contour.

pscontour isolates the use of PostScript to four internal functions:

void coninit(void) - called before any contour drawing
void conmove(float x, float y) - moves current position to x,y
void condraw(float x, float y) - draws from current position to x,y
void condone(void) - called when contour drawing is done

These functions can usually be replaced with equivalent functions in
other graphics environments.

The w array is used to restrict the range of contour labeling that 
occurs only if w<1. 

As suggested in the reference, the following scheme is used to refer
to a cell of the two-dimensional array z:

                north (0)
      (ix,iy+1)	--------- (ix+1,iy+1)
                | cell  |
       west (3)	| ix,iy	| east (1)
                |       |
        (ix,iy) --------- (ix+1,iy)
                south (2)

Reference:
Cottafava, G. and Le Moli, G., 1969, Automatic contour map:
Commuincations of the ACM, v. 12, n. 7, July, 1969.

Author:  Dave Hale, Colorado School of Mines, 06/28/89
contour labeling added by: Zhenyue Liu, August 1993
\end{verbatim}
\pagebreak
\begin{verbatim}
PSDRAWCURVE - Functions to draw a curve from a set of points

psDrawCurve	Draw a curve from a set of points via PostScript
Function Prototypes:
void psDrawCurve(
	float x, float y, float width, float height,
	float x1Beg, float x1End, float p1Beg, float p1End, 
	float x2Beg, float x2End, float p2Beg, float p2End,
	float *x1curve, float *x2curve, int ncurve,
	char *curveColor, float curvewidth, int curvedash, int style);
psDrawCurve:
Input:
x		x coordinate of lower left corner of box
y		y coordinate of lower left corner of box
width		width of box
height		height of box
x1Beg  		axis value at beginning of axis 1
x1End		axis value at end of axis 1
p1Beg  		pad value at beginning of axis 1
p1End		pad value at end of axis 1
x2Beg  		axis value at beginning of axis 2
x2End		axis value at end of axis 2
p2Beg  		pad value at beginning of axis 2
p2End		pad value at end of axis 2
x1curve		vector of x1 coordinates for points along curve
x2curve		vector of x2 coordinates for points along curve
ncurve		number of points along curve
curveColor	color to use for curve
curvewidth	width (in points) of curve
style		NORMAL (axis 1 on bottom, axis 2 on left) 
		SEISMIC (axis 1 on left, axis 2 on top)

Author:		Brian Macy, Phillips Petroleum Co., 11/20/98
		(Adapted after Dave Hale and other's psAxesBox routine)
\end{verbatim}
\pagebreak
\begin{verbatim}
PSLEGENDBOX - Functions to draw PostScript axes and estimate bounding box

psLegendBox	Draw an legend box via PostScript
psLegendBBox	estimate bounding box for an legend box drawn via psLegendBox

Function Prototypes:
void psLegendBox(
	float x, float y, float width, float height,
	float x1Beg, float x1End, float p1Beg, float p1End, 
	float d1Num, float f1Num, int n1Tic, int grid1, char *label1,
	char *labelFont, float labelSize,
	char *axesColor, char *gridColor,
	int style);

void psLegendBBox(
	float x, float y, float width, float height,
	char *labelFont, float labelSize,
	int style, int bbox[]);


psLegendBox:
Input:
x		x coordinate of lower left corner of box
y		y coordinate of lower left corner of box
width		width of box
height		height of box
x1Beg  		axis value at beginning of axis 1
x1End		axis value at end of axis 1
p1Beg  		pad value at beginning of axis 1
p1End		pad value at end of axis 1
d1Num		numbered tic increment for axis 1 (0.0 for automatic)
f1Num		first numbered tic for axis 1
n1Tic		number of horizontal tics per numbered tic for axis 1
grid1		grid code for axis 1:  NONE, DOT, DASH, or SOLID
label1		label for axis 1
labelFont	name of font to use for axes labels
labelSize	size of font to use for axes labels
axesColor	color to use for axes and axes labels
gridColor	color to use for grid lines
style		VERTLEFT (Vertical, axis label on left side) 
		VERTRIGHT (Vertical, axis label on right side) 
		HORIBOTTOM (Horizontal, axis label on bottom)

psLegendBBox:
Input:
x		x coordinate of lower left corner of box
y		y coordinate of lower left corner of box
width		width of box
height		height of box
labelFont	name of font to use for axes labels
labelSize	size of font to use for axes labels
style		VERTLEFT (Vertical, axis label on left side) 
		VERTRIGHT (Vertical, axis label on right side) 
		HORIBOTTOM (Horizontal, axis label on bottom)
Output:
bbox		bounding box (bbox[0:3] = llx, lly, ulx, uly)

Notes:
psLegendBox:
psLegendBox will determine the numbered tic increment and first numbered
tic automatically, if the specified increment is zero.  Axis numbering
is in scientific notation, if necessary and is plotted to four
significant digits.

Pad values must be specified in the same units as the corresponding
Legend values.  These pads are useful when the contents of the Legend box
requires more space than implied by the Legend values.  For example, the
first and last seismic wiggle traces plotted inside an Legend box will
typically extend beyond the Legend values corresponding to the first and
last traces.  However, all tics will lie with the limits specified in
the Legend values (x1Beg, x1End, x2Beg, x2End).

psLegendBBox:
psLegendBBox uses font sizes to estimate the bounding box for
an Legend box drawn with psLegendBox.  To be on the safe side, 
psLegendBBox overestimates.

psLegendBBox assumes that the Legend labels and titles do not extend
beyond the corresponding edges of the Legend box.

References:
(see References for basic.c)
Author:   Dave Hale,  Colorado School of Mines, 06/27/89
Modified: Ken Larner, Colorado School of Mines, 08/30/90
Modified: Dave Hale, Advance Geophysical, 10/18/92
	Added color parameters for title, axes, and grid.
Modified: Torsten Schoenfelder, Koeln, Germany, 07/06/97
        Display a legend for ps file, move axis from left to right
Modified: Torsten Schoenfelder, Koeln, Germany, 10/02/98
        Corrected width of bbox to include legend title
\end{verbatim}
\pagebreak
\begin{verbatim}
PSWIGGLE - draw wiggle-trace with (optional) area-fill via PostScript

psWiggle  draw wiggle-trace with (optional) area-fill via PostScript

Function Prototype:
void psWiggle (int n, float z[], float zmin, float zmax, float zbase,
	float yzmin, float yzmax, float xfirst, float xlast, int fill);

Inputs:
n		number of samples to draw
z		array to draw
zmin		z values below zmin will be clipped
zmax		z values above zmax will be clipped
zbase		z values between zbase and either zmin or zmax will be filled
yzmin		y-coordinate corresponding to zmin
yzmax		y-coordinate corresponding to zmax
xfirst		x-coordinate corresponding to z[0]
xlast		x-coordinate corresponding to z[n-1]
fill		= 0 for no fill
			  > 0 for fill between zbase and zmax
			  < 0 for fill between zbase and zmin
		+2 for fill solid between zbase and zmax grey between zbase and zmin
		-2 for fill solid between zbase and zmin grey between zbase and zmax
		SHADING: 2<= abs(fill) <=5   abs(fill)=2 light grey  abs(fill)=5 black

NOTES:
psWiggle reduces PostScript output by eliminating linetos when
z values are essentially constant.  The tolerance for detecting 
constant" z values is ZEPS*(zmax-zmin), where ZEPS is a fraction 
defined below.

A more complete optimization would eliminate all connected line 
segments that are essentially colinear.

psWiggle breaks up the wiggle into segments that can be drawn
without exceeding the PostScript pathlimit.

Author:  Dave Hale, Colorado School of Mines, 07/03/89
Modified:  Craig Artley, Colorado School of Mines, 04/13/92
           Corrected dead trace bug.  Now the last point of each segment
           is guaranteed to be drawn.
MODIFIED: Paul Michaels, Boise State University, 29 December 2000
           added fill=+/-2 option of solid/grey color scheme
\end{verbatim}
\pagebreak
\begin{verbatim}
AXESBOX - Functions to draw axes in X-windows graphics

xDrawAxesBox	draw a labeled axes box
xSizeAxesBox	determine optimal origin and size for a labeled axes box

Function Prototypes:
void xDrawAxesBox (Display *dpy, Window win,
	int x, int y, int width, int height,
	float x1beg, float x1end, float p1beg, float p1end,
	float d1num, float f1num, int n1tic, int grid1, char *label1,
	float x2beg, float x2end, float p2beg, float p2end,
	float d2num, float f2num, int n2tic, int grid2, char *label2,
	char *labelfont, char *title, char *titlefont, 
	char *axescolor, char *titlecolor, char *gridcolor,
	int style);
void xSizeAxesBox (Display *dpy, Window win, 
	char *labelfont, char *titlefont, int style,
	int *x, int *y, int *width, int *height);

xDrawAxesBox:
Input:
dpy		display pointer
win		window
x		x coordinate of upper left corner of box
y		y coordinate of upper left corner of box
width		width of box
height		height of box
x1beg		axis value at beginning of axis 1
x1end		axis value at end of axis 1
p1beg		pad value at beginning of axis 1
p1end		pad value at end of axis 1
d1num		numbered tic increment for axis 1 (0.0 for automatic)
f1num		first numbered tic for axis 1
n1tic		number of tics per numbered tic for axis 1
grid1		grid code for axis 1:  NONE, DOT, DASH, or SOLID
label1		label for axis 1
x2beg		axis value at beginning of axis 2
x2end		axis value at end of axis 2
p2beg		pad value at beginning of axis 2
p2end		pad value at end of axis 2
d2num		numbered tic increment for axis 2 (0.0 for automatic)
f2num		first numbered tic for axis 2
n2tic		number of tics per numbered tic for axis 2
grid2		grid code for axis 2:  NONE, DOT, DASH, or SOLID
label2		label for axis 2
labelfont	name of font to use for axes labels
title		axes box title
titlefont	name of font to use for title
axescolor	name of color to use for axes
titlecolor	name of color to use for title
gridcolor	name of color to use for grid
int style	NORMAL (axis 1 on bottom, axis 2 on left)
		SEISMIC (axis 1 on left, axis 2 on top)

xSizeAxesBox:
Input:
dpy		display pointer
win		window
labelfont	name of font to use for axes labels
titlefont	name of font to use for title
int style	NORMAL (axis 1 on bottom, axis 2 on left)
		SEISMIC (axis 1 on left, axis 2 on top)

Output:
x		x coordinate of upper left corner of box
y		y coordinate of upper left corner of box
width		width of box
height		height of box
{
	XFontStruct *fa,*ft;
Notes:
xDrawAxesBox:
will determine the numbered tic incremenet and first
numbered tic automatically, if the specified increment is zero.

Pad values must be specified in the same units as the corresponding
axes values.  These pads are useful when the contents of the axes box
requires more space than implied by the axes values.  For example,
the first and last seismic wiggle traces plotted inside an axes box
will typically extend beyond the axes values corresponding to the
first and last traces.  However, all tics will lie within the limits
specified in the axes values (x1beg, x1end, x2beg, x2end).

xSizeAxesBox:
is intended to be used prior to xDrawAxesBox.

An "optimal" axes box is one that more or less fills the window, 
with little wasted space around the edges of the window.

Author:		Dave Hale, Colorado School of Mines, 01/27/90
\end{verbatim}
\pagebreak
\begin{verbatim}
COLORMAP - Functions to manipulate X colormaps:

xCreateRGBDefaultMap	create XA_RGB_DEFAULT_MAP property of root window if
			it does not already exist
xGetFirstPixel		return first pixel in range of contiguous pixels in
			XA_RGB_DEFAULT_MAP
xGetLastPixel		return last pixel in range of contiguous pixels in
			XA_RGB_DEFAULT_MAP
xCreateHSVColormap	create a 2 ramp colormap (HSV - Model)
xCreateRGBColormap	create a 2 ramp colormap (RGB - Model)

Function Prototypes:
Status xCreateRGBDefaultMap (Display *dpy, XStandardColormap *scmap);
unsigned long xGetFirstPixel (Display *dpy);
unsigned long xGetLastPixel (Display *dpy);
Colormap xCreateRGBColormap (Display *dpy, Window win,
	char *str_cmap, int verbose)
Colormap xCreateHSVColormap (Display *dpy, Window win,
	char *str_cmap, int verbose)

xCreateRGBDefaultMap:
Input:
dpy		display

Output:
scmap		the standard colormap structure

xGetFirstPixel, xGetLastPixel:
Input:
dpy		display

Notes:
PROBLEM
-------

Most mid-range display devices today support what X calls
the "PseudoColor visual".  Typically, only 256 colors (or gray
levels) may be displayed simultaneously.  Although these 256 colors
may be chosen from a much larger (4096 or more) set of available
colors, only 256 colors can appear on a display at one time.

These 256 colors are indexed by pixel values in a table called
the colormap.  Each window can have its own colormap, but only
one colormap can be installed in the display hardware at a time.
(Again, only 256 colors may be displayed at one time.)  The window 
manager is responsible for installing a window's colormap when that 
window becomes the key window.

Many of the applications we are likely to write require a large,
contiguous range of pixels (entries in the colormap).  In this
range, we must be able to:
(1) given a color (or gray), determine the corresponding pixel.
(2) given a pixel, determine the corresponding color (or gray).
An example would be an imaging application that uses a gray scale
to display images in shades of gray between black and white.
Such applications are also likely to require a few additional colors
for drawing axes, text, etc.

The problem is to coordinate the use of the limited number of
256 simultaneous colors so that windows for different applications 
appear reasonable, even when their particular colormaps are not
installed in the display hardware.  For example, we might expect 
an analog xclock's hands to be visible even when xclock's window
is not the key window, when its colormap is not installed.

We should ensure that the range of contiguous pixels used by one
application (perhaps for imaging) does not conflict with the pixels
used by other applications to draw text, clock hands, etc.


SOLUTION
--------

Applications that do not require special colormaps should simply
use the default colormap inherited from the root window when new
top-level windows are created.

Applications that do require a special colormap MUST create their
own colormap.  They must not assume that space will be available
in the default colormap for a contiguous range of read/write pixels,
because the server or window manager may have already allocated
these pixels as read-only.  Even if sufficient pixels are available
in the default colormap, they should not be allocated by a single
application.  The default colormap should be used only for windows
requiring a limited number of typical colors, such as red, yellow, etc.

Applications that require a contiguous range of read/write pixels
should allocate these pixels in their window's private colormaps.
They should determine which contiguous pixels to allocate from 
parameters in the standard colormap XA_RGB_DEFAULT_MAP.  In particular,
the first pixel in the range of contiguous pixels should be 
	base_pixel
and the last pixel in the range should be 
	base_pixel+red_max*red_mult+green_max*green_mult+blue_max*blue_mult,
where base_pixel, red_max, etc. are members in the XStandardColormap
structure.  On an 8-bit display, this range will typically provide 216
contiguous pixels, which may be set to a gray scale, color scale, or
whatever.  This leaves 40 colors for drawing text, axes, etc.

If the XA_RGB_DEFAULT_MAP does not exist, it should be created to 
consist of various colors composed of an equal number of reds, 
greens, and blues.  For example, if 216 colors are to be allocated,
then red_max=green_max=blue_max=5, red_mult=36, green_mult=6, and
blue_mult=1.  Because of the difficulty in forcing a particular 
pixel to correspond to a particular color in read-only color cells,
these 216 colors will likely be read/write color cells unless
created by the X server.  In any case, these 216 colors should not
be modified by any application.  In creating custom colormaps, the
only use of XA_RGB_DEFAULT_MAP should be in determining which 216
pixels to allocate for contiguous pixels.

In creating a custom colormap for a window, the application should
initialize this colormap to the colors already contained in the
window's colormap, which was inherited initially from its parent.
This will ensure that typical colors already allocated by other
applications will be consistent with pixels used by the application
requiring the custom colormap.  Ideally, windows might have
different colormaps, but the only differences would be in the
range of contiguous colors used for imaging, rendering, etc.
Ideally, the pixels corresponding to colors used to draw text, 
axes, etc. would be consistent for all windows.

Unfortunately, it is impractical to maintain complete consistency 
among various private colormaps.  For example, suppose a custom
colormap is created for a window before other applications have
had the opportunity to allocate their colors from the default
colormap.  Then, when the window with the custom colormap becomes
the key window, the windows of the other applications may be 
displayed with false colors, since the colormap of the key window
may not contain the true colors.  The colors used by the other 
applications did not exist when the custom colormap was created.
One solution to this problem might be to initially allocate a
set of "common" colors in the default colormap before launching
any applications.  This will increase the likelihood that typical
colors will be consistent among various colormaps.

Functions are provided below to
(1) create the standard colormap XA_RGB_DEFAULT_MAP, if it does not exist,
(2) determine the first and last pixels in the contiguous range of pixels,
(3) create some common private colormaps 

xCreateRGBDefaultMap:
This function returns 0 if the XA_RGB_DEFAULT_MAP property does not exist
and cannot be created.  At least 8 contiguous color cells must be free
in the default colormap to create the XA_RGB_DEFAULT_MAP.  If created, the
red_max, green_max, and blue_max values returned in scmap will be equal.

xGetFirstPixel, xGetLastPixel:
If it does not already exist, XA_RGB_DEFAULT_MAP will be created.
If XA_RGB_DEFAULT_MAP does not exist and cannot be created, then
this function returns 0.

xCreateRGBColormap, xCreateHSVColormap:
The returned colormap is only created; the window's colormap attribute
is not changed, and the colormap is not installed by this function.
The returned colormap is a copy of the window's current colormap, but 
with an RGB color scale allocated in the range of contiguous cells
determined by XA_RGB_DEFAULT_MAP.  If it does not already exist,
XA_RGB_DEFAULT_MAP will be created.

Author:  Dave Hale, Colorado School of Mines, 09/30/90
\end{verbatim}
\pagebreak
\begin{verbatim}
DRAWCURVE - Functions to draw a curve from a set of points

xDrawCurve	draw a curve from a set of points
Function Prototypes:
void xDrawCurve(Display *dpy, Window win,
		int x, int y, int width, int height,
		float x1beg, float x1end, float p1beg, float p1end,
		float x2beg, float x2end, float p2beg, float p2end,
		float *x1curve, float *x2curve, int ncurve,
		char *curvecolor, int style);
xDrawCurve:
Input:
dpy		display pointer
win		window
x		x coordinate of upper left corner of box
y		y coordinate of upper left corner of box
width		width of box
height		height of box
x1beg		axis value at beginning of axis 1
x1end		axis value at end of axis 1
p1beg		pad value at beginning of axis 1
p1end		pad value at end of axis 1
x2beg		axis value at beginning of axis 2
x2end		axis value at end of axis 2
p2beg		pad value at beginning of axis 2
p2end		pad value at end of axis 2
x1curve		vector of x1 coordinates for points along curve
x2curve		vector of x2 coordinates for points along curve
ncurve		number of points along curve
curvecolor	name of color to use for axes
int style	NORMAL (axis 1 on bottom, axis 2 on left)
		SEISMIC (axis 1 on left, axis 2 on top)

Author:		Brian Macy, Phillips Petroleum Co., 11/14/98
		(Adapted after Dave Hale's xDrawAxesBox routine)
\end{verbatim}
\pagebreak
\begin{verbatim}
IMAGE - Function for making the image in an X-windows image plot

xNewImage	make a new image of pixels from bytes

Function Prototype:
XImage *xNewImage (Display *dpy, unsigned long pmin, unsigned long pmax,
	int width, int height, float blank, unsigned char *bytes);

Input:
dpy		display pointer
pmin		minimum pixel value (corresponding to byte=0)
pmax		maximum pixel value (corresponding to byte=255)
width		number of bytes in x dimension
height		number of bytes in y dimension
blank		portion for blanking (0 to 1)
bytes		unsigned bytes to be mapped to an image

Author:	  Dave Hale, Colorado School of Mines, 06/08/90
Revision: Brian Zook, Southwest Research Institute, 6/27/96 added blank option
	  This allows replacing the low end by the background.
\end{verbatim}
\pagebreak
\begin{verbatim}
LEGENDBOX - draw a labeled axes box for a legend (i.e. colorscale)

Function Prototype:
void xDrawLegendBox (Display *dpy, Window win,
	int x, int y, int width, int height,
	float bclip, float wclip, char *units, char *legendfont,
	char *labelfont, char *title, char *titlefont,
	char *axescolor, char *titlecolor, char *gridcolor,
	int style);

Input:
dpy		display pointer
win		window
x		x coordinate of upper left corner of box
y		y coordinate of upper left corner of box
width		width of box
height		height of box
units		label for legend
legendfont	name of font to use for legend labels
labelfont	name of font to use for axes labels
title		axes box title
titlefont	name of font to use for title
axescolor	name of color to use for axes
titlecolor	name of color to use for title
gridcolor	name of color to use for grid
int style	NORMAL (axis 1 on bottom, axis 2 on left)
		SEISMIC (axis 1 on left, axis 2 on top)
Notes:
xDrawLegendBox will determine the numbered tic incremenet and first
numbered tic automatically, if the specified increment is zero.

Pad values must be specified in the same units as the corresponding
axes values.  These pads are useful when the contents of the axes box
requires more space than implied by the axes values.  For example,
the first and last seismic wiggle traces plotted inside an axes box
will typically extend beyond the axes values corresponding to the
first and last traces.  However, all tics will lie within the limits
specified in the axes values (x1beg, x1end, x2beg, x2end).
Author:		Dave Hale, Colorado School of Mines, 01/27/90
Author:		Berend Scheffers , TNO Delft, 06/11/92
\end{verbatim}
\pagebreak
\begin{verbatim}
RUBBERBOX -  Function to draw a rubberband box in X-windows plots

xRubberBox	Track pointer with rubberband box

Function Prototype:
void xRubberBox (Display *dpy, Window win, XEvent event,
	int *x, int *y, int *width, int *height);

Input:
dpy		display pointer
win		window ID
event		event of type ButtonPress

Output:
x		x of upper left hand corner of box in pixels
y		y of upper left hand corner of box in pixels
width		width of box in pixels
height		height of box in pixels

Notes:
xRubberBox assumes that event is a ButtonPress event for the 1st button;
i.e., it tracks motion of the pointer while the 1st button is down, and
it sets x, y, w, and h and returns after a ButtonRelease event for the
1st button.

Before calling xRubberBox, both ButtonRelease and Button1Motion events 
must be enabled.

This is the same rubberbox.c as in Xtcwp/lib, only difference is
that xRubberBox here is XtcwpRubberBox there, and a shift has been
added to make the rubberbox more visible.

Author:		Dave Hale, Colorado School of Mines, 01/27/90
\end{verbatim}
\pagebreak
\begin{verbatim}
WINDOW - Function to create a window in X-windows graphics

xNewWindow	Create a new window and return the window ID

Function Prototype:
Window xNewWindow (Display *dpy, int x, int y, int width, int height,
	int border, int background, char *name);

Input:
dpy		display pointer
x		x in pixels of upper left corner
y		y in pixels of upper left corner
width		width in pixels
height		height in pixels
border		border pixel
background	background pixel
name		name of window (also used for icon)

Notes:
The parent window is the root window.
The border_width is 4 pixels.

Author:		Dave Hale, Colorado School of Mines, 01/06/90
\end{verbatim}
\pagebreak
\begin{verbatim}
XCONTOUR - draw contour of a two-dimensional array via X vectorplot calls

xContour	draw contour of a two-dimensional array via X vector plot calls

Function Prototype:
void xContour(Display *dpy, Window win,GC gcc, GC gcl, 
	       float *cp,int nx, float x[], int ny, float y[], float z[], 
	       char lcflag,char *lcf,char *lcc, float *w, int nplaces)
Input:
c			contour value
nx			number of x-coordinates
x			array of x-coordinates (see notes below)
ny			number of y-coordinates
y			array of y-coordinates (see notes below)
lcf			font name of contour label
lcc			color of contour label
Least Significat Bits:
z			array of nx*ny z(x,y) values (see notes below)
w			array of nx*ny z(x,y) values (see notes below)
Notes:
The two-dimensional array z is actually passed as a one-dimensional
array containing nx*ny values, stored with nx fast and ny slow.

The x and y arrays define a grid that is not necessarily 
uniformly-sampled.  Linear interpolation of z values on the 
grid defined by the x and y arrays is used to determine z values 
between the gridpoints.
		 
The two least significant bits of z are used to mark intersections
of the contour with the x,y grid; therefore, the z values will
almost always be altered (slightly) by contour.

xContour is a modified version of psContour where the use of conmove
and condraw call have been changed to match X-Windows.

Since XDrawLine requires a start and end point, the use of a manually update
of the position variables x0 and y0 is used instead of conmove.

The w array is used to restrict the range of contour labeling that 
occurs only if w<1. 

As suggested in the reference, the following scheme is used to refer
to a cell of the two-dimensional array z:

                north (0)
      (ix,iy+1)	--------- (ix+1,iy+1)
                | cell  |
       west (3)	| ix,iy	| east (1)
                |       |
        (ix,iy) --------- (ix+1,iy)
                south (2)

Reference:
Cottafava, G. and Le Moli, G., 1969, Automatic contour map:
Commuincations of the ACM, v. 12, n. 7, July, 1969.

Author:  Morten Wendell Pedersen Aarhus University 07/20/96
Heavily based on psContour by
  Dave Hale, Colorado School of Mines, 06/28/89
  and with contour labeling added by: Zhenyue Liu, June 1993
(actually most of the credit should go to these two guys)
\end{verbatim}
\pagebreak
\begin{verbatim}
AXES - the Axes Widget

XtcwpPointInAxesRectangle	returns TRUE if point is inside axes
				rectangle, otherwise FALSE
XtcwpSetAxesValues	set axes values
XtcwpSetAxesPads	set axes pads

Function Prototype:
Boolean XtcwpPointInAxesRectangle (Widget w, Position x, Position y);
void XtcwpSetAxesValues (Widget w, float x1beg, float x1end, float x2beg,
				float x2end);

void XtcwpSetAxesPads (Widget w, float p1beg, float p1end, float p2beg,
				float p2end);
XtcwpPointInAxesRectangle:
Input:
w		axes widget
x		x coordinate of point
y		y coordinate of point

XtcwpSetAxesValues:
Input:
w		axes widget
x1beg		axis value at beginning of axis 1
x1end		axis value at end of axis 1
x2beg		axis value at beginning of axis 2
x2end		axis value at end of axis 2

XtcwpSetAxesPads:
Input:
w		axes widget
p1beg		axis pad at beginning of axis 1
p1end		axis pad at end of axis 1
p2beg		axis pad at beginning of axis 2
p2end		axis pad at end of axis 2

Notes:
XtcwpPointInAxesRectangle:
This function is useful for determining whether or not input events
occured with the pointer inside the axes rectangle.  I.e., the input
callback function will typically call this function.

XtcwpSetAxesPads:
Pad values must be specified in the same units as the corresponding 
axes values.  These pads are useful when the contents of the axes box
require more space than implied by the axes values.  For example, the
first and last seismic wiggle traces plotted inside an axes box
will typically extend beyond the axes values corresponding to the
first and last traces.  However, all tics will lie within the limits
specified in the axes values (x1beg, x1end, x2beg, and x2end).

Author:  Dave Hale, Colorado School of Mines, 08/28/90
Modified:  Craig Artley, Colorado School of Mines, 06/03/93, Rotate label for
	   vertical axis (Courtesy Dave Hale, Advance Geophysical).
\end{verbatim}
\pagebreak
\begin{verbatim}
COLORMAP - Functions to manipulate X colormaps:

XtcwpCreateRGBDefaultMap	create XA_RGB_DEFAULT_MAP property of root
				window if it does not already exist
XtcwpGetFirstPixel		return first pixel in range of contiguous
				pixels in XA_RGB_DEFAULT_MAP
XtcwpGetLastPixel		return last pixel in range of contiguous
				pixels in XA_RGB_DEFAULT_MAP
XtcwpCreateRGBColormap	create a colormap with an RGB color scale in
			contiguous cells
XtcwpCreateGrayColormap	create a colormap with a gray scale in contiguous cells
XtcwpCreateHueColormap	create a colormap with varying hues (blue to red)
			in contiguous cells
XtcwpCreateSatColormap	create a colormap with varying saturations in
			contiguous cells

Function Prototypes:
Status XtcwpCreateRGBDefaultMap (Display *dpy, XStandardColormap *scmap);
unsigned long XtcwpGetFirstPixel (Display *dpy);
unsigned long XtcwpGetLastPixel (Display *dpy);
Colormap XtcwpCreateRGBColormap (Display *dpy, Window win);
Colormap XtcwpCreateGrayColormap (Display *dpy, Window win);
Colormap XtcwpCreateHueColormap (Display *dpy, Window win);
Colormap XtcwpCreateSatColormap (Display *dpy, Window win,
	float fhue, float lhue, float wfrac, float bright)

XtcwpCreateRGBDefaultMap:
Input:
dpy		display

Output:
scmap		the standard colormap structure

XtcwpGetFirstPixel, XtcwpGetLastPixel:
Input:
dpy		display

XtcwpCreateRGBColormap, XtcwpCreateGrayColormap, XtcwpCreateHueColormap:
Input:
dpy		display
win		window

XtcwpCreateSatColormap:
Input:
dpy		display
win		window
fhue		first hue in colormap (saturation=1)
lhue		last hue in colormap (saturation=1)
wfrac		fractional position of white within the colormap (saturation=0)
bright		brightness

Notes:
PROBLEM
-------

Most mid-range display devices today support what X calls
the "PseudoColor visual".  Typically, only 256 colors (or gray
levels) may be displayed simultaneously.  Although these 256 colors
may be chosen from a much larger (4096 or more) set of available
colors, only 256 colors can appear on a display at one time.

These 256 colors are indexed by pixel values in a table called
the colormap.  Each window can have its own colormap, but only
one colormap can be installed in the display hardware at a time.
(Again, only 256 colors may be displayed at one time.)  The window 
manager is responsible for installing a window's colormap when that 
window becomes the key window.

Many of the applications we are likely to write require a large,
contiguous range of pixels (entries in the colormap).  In this
range, we must be able to:
(1) given a color (or gray), determine the corresponding pixel.
(2) given a pixel, determine the corresponding color (or gray).
An example would be an imaging application that uses a gray scale
to display images in shades of gray between black and white.
Such applications are also likely to require a few additional colors
for drawing axes, text, etc.

The problem is to coordinate the use of the limited number of
256 simultaneous colors so that windows for different applications 
appear reasonable, even when their particular colormaps are not
installed in the display hardware.  For example, we might expect 
an analog xclock's hands to be visible even when xclock's window
is not the key window, when its colormap is not installed.

We should ensure that the range of contiguous pixels used by one
application (perhaps for imaging) does not conflict with the pixels
used by other applications to draw text, clock hands, etc.


SOLUTION
--------

Applications that do not require special colormaps should simply
use the default colormap inherited from the root window when new
top-level windows are created.

Applications that do require a special colormap MUST create their
own colormap.  They must not assume that space will be available
in the default colormap for a contiguous range of read/write pixels,
because the server or window manager may have already allocated
these pixels as read-only.  Even if sufficient pixels are available
in the default colormap, they should not be allocated by a single
application.  The default colormap should be used only for windows
requiring a limited number of typical colors, such as red, yellow, etc.

Applications that require a contiguous range of read/write pixels
should allocate these pixels in their window's private colormaps.
They should determine which contiguous pixels to allocate from 
parameters in the standard colormap XA_RGB_DEFAULT_MAP.  In particular,
the first pixel in the range of contiguous pixels should be 
	base_pixel
and the last pixel in the range should be 
	base_pixel+red_max*red_mult+green_max*green_mult+blue_max*blue_mult,
where base_pixel, red_max, etc. are members in the XStandardColormap
structure.  On an 8-bit display, this range will typically provide 216
contiguous pixels, which may be set to a gray scale, color scale, or
whatever.  This leaves 40 colors for drawing text, axes, etc.

If the XA_RGB_DEFAULT_MAP does not exist, it should be created to 
consist of various colors composed of an equal number of reds, 
greens, and blues.  For example, if 216 colors are to be allocated,
then red_max=green_max=blue_max=5, red_mult=36, green_mult=6, and
blue_mult=1.  Because of the difficulty in forcing a particular 
pixel to correspond to a particular color in read-only color cells,
these 216 colors will likely be read/write color cells unless
created by the X server.  In any case, these 216 colors should not
be modified by any application.  In creating custom colormaps, the
only use of XA_RGB_DEFAULT_MAP should be in determining which 216
pixels to allocate for contiguous pixels.

In creating a custom colormap for a window, the application should
initialize this colormap to the colors already contained in the
window's colormap, which was inherited initially from its parent.
This will ensure that typical colors already allocated by other
applications will be consistent with pixels used by the application
requiring the custom colormap.  Ideally, windows might have
different colormaps, but the only differences would be in the
range of contiguous colors used for imaging, rendering, etc.
Ideally, the pixels corresponding to colors used to draw text, 
axes, etc. would be consistent for all windows.

Unfortunately, it is impractical to maintain complete consistency 
among various private colormaps.  For example, suppose a custom
colormap is created for a window before other applications have
had the opportunity to allocate their colors from the default
colormap.  Then, when the window with the custom colormap becomes
the key window, the windows of the other applications may be 
displayed with false colors, since the colormap of the key window
may not contain the true colors.  The colors used by the other 
applications did not exist when the custom colormap was created.
One solution to this problem might be to initially allocate a
set of "common" colors in the default colormap before launching
any applications.  This will increase the likelihood that typical
colors will be consistent among various colormaps.

Functions are provided below to
(1) create the standard colormap XA_RGB_DEFAULT_MAP, if it does not exist,
(2) determine the first and last pixels in the contiguous range of pixels,
(3) create some common private colormaps - gray scale, hue scale, etc.

XtcwpCreateRGBDefaultMap:
This function returns 0 if the XA_RGB_DEFAULT_MAP property does not exist
and cannot be created.  At least 8 contiguous color cells must be free
in the default colormap to create the XA_RGB_DEFAULT_MAP.  If created, the
red_max, green_max, and blue_max values returned in scmap will be equal.

XtcwpGetFirstPixel, XtcwpGetLastPixel:
If it does not already exist, XA_RGB_DEFAULT_MAP will be created.
If XA_RGB_DEFAULT_MAP does not exist and cannot be created, then
this function returns 0.

XtcwpCreateRGBColormap, XtcwpCreateGrayColormap, XtcwpCreateHueColormap,
XtcwpCreateSatColormap:
The returned colormap is only created; the window's colormap attribute
is not changed, and the colormap is not installed by this function.
The returned colormap is a copy of the window's current colormap, but 
with an RGB color scale allocated in the range of contiguous cells
determined by XA_RGB_DEFAULT_MAP.  If it does not already exist,
XA_RGB_DEFAULT_MAP will be created.

Author:  Dave Hale, Colorado School of Mines, 09/30/90
\end{verbatim}
\pagebreak
\begin{verbatim}
FX - Functions to support floating point coordinates in X

FMapFX			map float x to x
FMapFY			map float y to y
FMapFWidth		map float width to width
FMapFHeight		map float height to height
FMapFAngle		map float angle to angle
FMapFPoint		map float x,y to x,y
FMapFPoints		map float points to points
FMapX			inverse map x to float x
FMapY			inverse map y to float y
FMapWidth		inverse map width to float width
FMapHeight		inverse map height to float height
FMapAngle		inverse map angle to float angle
FMapPoint		map x,y to float x,y
FMapPoints		map points to float points
FSetGC			set graphics context
FSetMap			set map (scales and shifts)
FSetClipRectangle	set clip rectangle
FClipOn			turn clip on
FClipOff		turn clip off
FClipPoint		clip point
FClipLine		clip line
FClipRectangle		clip rectangle
FXCreateFGC		create float graphics context
FXFreeFGC		free float graphic context
FXDrawPoint		draw point at float x,y (with clipping)
FXDrawPoints		draw float points (with clipping)
FXDrawLine		draw line from float x1,y1 to float x2,y2
				(with clipping)
FXDrawLines		draw lines between float points (with clipping)
FXDrawRectangle		draw rectangle with float x,y,width,height
				(with clipping)
FXDrawArc		draw arc with float x,y,width,height,angle1,angle2
FXDrawString		draw string at float x,y
FXFillRectangle		fill rectangle with float x,y,width,height (with
				clipping)

Function Prototypes:
int FMapFX (FGC fgc, float fx);
int FMapFY (FGC fgc, float fy);
int FMapFWidth (FGC fgc, float fwidth);
int FMapFHeight (FGC fgc, float fheight);
int FMapFAngle (FGC fgc, float fangle);
void FMapFPoint (FGC fgc, float fx, float fy, int *x_return, int *y_return);
void FMapFPoints (FGC fgc, FXPoint fpoints[], int npoints, 
	XPoint points_return[]);
float FMapX (FGC fgc, int x);
float FMapY (FGC fgc, int y);
float FMapWidth (FGC fgc, int width);
float FMapHeight (FGC fgc, int height);
float FMapAngle (FGC fgc, int angle);
void FMapPoint (FGC fgc, int x, int y, float *fx_return, float *fy_return);
void FMapPoints (FGC fgc, XPoint points[], int npoints, 
	FXPoint fpoints_return[]);
void FSetGC (FGC fgc, GC gc);
void FSetMap (FGC fgc, int x, int y, int width, int height,
	float fx, float fy, float fwidth, float fheight);
void FSetClipRectangle(FGC fgc, float fxa, float fya, float fxb, float fyb);
void FClipOn (FGC fgc);
void FClipOff (FGC fgc);
int FClipPoint (FGC fgc, float fx, float fy);
int FClipLine (FGC fgc, float fx1, float fy1, float fx2, float fy2,
	float *fx1c, float *fy1c, float *fx2c, float *fy2c);
int FClipRectangle (FGC fgc, float fx, float fy, float fwidth, float fheight,
	float *fxc, float *fyc, float *fwidthc, float *fheightc);
FGC FXCreateFGC (GC gc, int x, int y, int width, int height,
	float fx, float fy, float fwidth, float fheight);
void FXFreeFGC (FGC fgc);
void FXDrawPoint (Display *display, Drawable d, FGC fgc, float fx, float fy);
void FXDrawPoints (Display *display, Drawable d, FGC fgc, 
	FXPoint fpoints[], int npoints, int mode);
void FXDrawLine (Display *display, Drawable d, FGC fgc,
	float fx1, float fy1, float fx2, float fy2);
void FXDrawLines (Display *display, Drawable d, FGC fgc,
	FXPoint fpoints[], int npoints, int mode);
void FXDrawRectangle (Display *display, Drawable d, FGC fgc, 
	float fx, float fy, float fwidth, float fheight);
void FXDrawArc (Display *display, Drawable d, FGC fgc,
	float fx, float fy, float fwidth, float fheight, 
	float fangle1, float fangle2);
void FXDrawString (Display *display, Drawable d, FGC fgc, 
	float fx, float fy, char *string, int length);
void FXFillRectangle (Display *display, Drawable d, FGC fgc, 
	float fx, float fy, float fwidth, float fheight);

Notes:
The functions defined below are designed to resemble the equivalent 
X functions.  For example, FXDrawLine() is analogous to XDrawLine.
Each of the FXDraw<xxx>() functions requires an FGC instead of a GC
(graphics context).  An FGC contains a GC, along with the information 
required to transform floating point coordinates to integer (pixel) 
coordinates.

Additional functions are provided to transform floating point coordinates
to integer coordinates and vice versa.  Where feasible, macros are also
provided to perform these coordinate transformations.

Clipping of floating point coordinates is supported, because clipping
after mapping to integer coordinates is not valid when the mapped
integer coordinates overflow the range of short integers.  By clipping
the floating point coordinates before mapping to integers, this overflow
can be avoided.  By default, clipping is turned off until a clip rectangle
is specified or until clipping is explicitly turned on.  Clipping is
not currently supported for all FXDraw functions.

Author:  Dave Hale, Colorado School of Mines, 07/24/90
Modified:  Dave Hale, Colorado School of Mines, 05/18/91
	Added floating point clipping capability to some FXDraw functions.
\end{verbatim}
\pagebreak
\begin{verbatim}
MISC - Miscellaneous X-Toolkit functions

XtcwpDrawString90	Draw a string rotated 90 degrees counter-clockwise

Function Prototype:
void XtcwpDrawString90 (Display *dpy, Drawable d, GC gc,
	int x, int y, char *string, int count);

Input:
dpy		X display
d		X drawable
gc		X graphics context
x,y		coordinates of baseline starting position of the string
string		array[count] of characters to be drawn
count		number of characters in string

Author:  Dave Hale, Advance Geophysical, 06/03/93
\end{verbatim}
\pagebreak
\begin{verbatim}
RESCONV - general purpose resource type converters

XtcwpStringToFloat	convert  string to float in resource

Function Prototype:
void XtcwpStringToFloat (XrmValue *args, int *nargs, 
	XrmValue *fromVal, XrmValue *toVal);

Author:  Dave Hale, Colorado School of Mines, 08/28/90
\end{verbatim}
\pagebreak
\begin{verbatim}
RUBBERBOX -  Function to draw a rubberband box in X-windows plots

XtcwpRubberbox	Track pointer with rubberband box

Function Prototype:
void XtcwpRubberbox (Display *dpy, Window win, XEvent event,
	int *x, int *y, int *width, int *height);

Input:
dpy		display pointer
win		window ID
event		event of type ButtonPress

Output:
x		x of upper left hand corner of box in pixels
y		y of upper left hand corner of box in pixels
width		width of box in pixels
height		height of box in pixels

Notes:
XtcwpRubberbox assumes that event is a ButtonPress event for the 1st button;
i.e., it tracks motion of the pointer while the 1st button is down, and
it sets x, y, w, and h and returns after a ButtonRelease event for the
1st button.

Before calling XtcwpRubberbox, both ButtonRelease and Button1Motion events 
must be enabled.

Author:		Dave Hale, Colorado School of Mines, 01/27/90
\end{verbatim}
\pagebreak
\begin{verbatim}
RADIOBUTTONS -  convenience functions creating and using radio buttons

XtcwpCreateStringRadioButtons		create an XmFrame containing radio
						buttons labeled with strings

Function Prototypes:
Widget XtcwpCreateStringRadioButtons (Widget parent, char *label,
	int nstrings, char **strings, int first,
	void (*callback)(int selected, void *clientdata), void *clientdata);

Input:
parent		parent widget
label		label for this collection of radio bottons
nstrings	number of strings
strings		array[nstrings] of character strings, one per button
first		index of button to be initially selected
callback	function called when radio buttons change state
clientdata	pointer to client data to be passed to callback

Notes:
This code depends on the Motif Developer's Package.

An integer index of the selected button (along with the clientdata pointer)
is passed to the callback function.

The returned XmFrame is not managed.

Author:  Dave Hale, Colorado School of Mines, 08/28/90
\end{verbatim}
\pagebreak
\begin{verbatim}
SAMPLES - Motif-based Graphics Functions

samplesCreate
samplesDraw
samplesSetN
samplesSetData
samplesSetPlotValue
samplesSetEditMode
samplesSetOrigin

Function Prototypes:
Samples *samplesCreate (Widget parent, char *title,
	void (*editDone)(Samples *s));	
void samplesDraw (Samples *s);
void samplesSetN (Samples *s, int n);
void samplesSetData (Samples *s, float *d);
void samplesSetPlotValue (Samples *s, float pv);
void samplesSetEditMode (Samples *s, EditMode m);
void samplesSetOrigin (Samples *s, int i);

Notes:
Watch this space.

Author: Dave Hale, Colorado School of Mines
\end{verbatim}
\pagebreak
\begin{verbatim}
FGETGTHR - get gathers from SU datafiles

fget_gather - get a gather from a file
get_gather - get a gather from stdin


Function Prototypes:
segy **fget_gather(FILE *fp, cwp_String *key,cwp_String *type,Value *n_val,
			int *nt,int *ntr, float *dt,int *first);
segy **get_gather(cwp_String *key,cwp_String *type,Value *n_val,
			int *nt,int *ntr, float *dt,int *first)

fget_gather - get a gather from a file
Input:
fp		file pointer of input file
key		header key of ensemble
type		header value type
value		value of ensemble key word
nt		number of time samples
ntr		number of traces in ensemble
dt		time sampling interval
first		flag is this the first ensemble being read?
Notes: 
The input seismic dataset must be sorted into ensembles defined by key
Author: Potash Corporation Sascatchewan, Balasz Nemeth
given to CWP in 2008.

get_gather - get a gather from stdin
Input:
key		header key of ensemble
type		header value type
value		value of ensemble key word
nt		number of time samples
ntr		number of traces in ensemble
dt		time sampling interval
first		flag is this the first ensemble being read?
Notes: 
The input seismic dataset must be sorted into ensembles defined by key
Author: Potash Corporation Sascatchewan, Balasz Nemeth
given to CWP in 2008.

segy tr;

segy **fget_gather(FILE *fp, cwp_String *key,cwp_String *type,Value *n_val,
			int *nt,int *ntr, float *dt,int *first)
fget_gather - get a gather from a file
Input:
fp		file pointer of input file
key		header key of ensemble
type		header value type
value		value of ensemble key word
nt		number of time samples
ntr		number of traces in ensemble
dt		time sampling interval
first		flag is this the first ensemble being read?
Notes: 
The input seismic dataset must be sorted into ensembles defined by key
Author: Potash Corporation Sascatchewan, Balasz Nemeth
given to CWP in 2008.
{
	int nsegy;
	FILE *tracefp=NULL;
	FILE *headerfp=NULL;
	static FILE *ntracefp=NULL; /* first different trace pointer
	static FILE *nheaderfp=NULL;
	segy **rec=NULL;
	int ntrr=0;
	int indx=0;
	static Value val;
	
	
	*type = hdtype(*key);
	indx = getindex(*key);
        *ntr = 0;
        	
	
	if(*first==0) {
	       	/* get info from first trace
		nsegy = fvgettr(fp,&tr);
		if (nsegy==0)  err("can't get first trace");
		*nt = tr.ns;
        	*dt = (float) tr.dt/1000000.0;
        	++ntrr;
		gethval(&tr, indx, n_val);
		ntracefp = etmpfile();
		nheaderfp = etmpfile();
		*first=1;
	} else {
		/* This is the first trace of the nex gather
		erewind(nheaderfp);
        	erewind(ntracefp);
                fread (&tr,HDRBYTES, 1, nheaderfp);
                fread (tr.data,FSIZE, *nt, ntracefp);
		gethval(&tr, indx, n_val);
	}
		
	

        /* Store traces in tmpfile while getting a count
	tracefp = etmpfile();
	headerfp = etmpfile();
        do {
                efwrite(&tr, 1, HDRBYTES, headerfp);
                efwrite(tr.data, FSIZE, *nt, tracefp);
		
		/* read the next trace
		*ntr+=1;
		val=*n_val;
		nsegy = fvgettr(fp,&tr);
		if(nsegy) ntrr++;
		gethval(&tr, indx, n_val);
	} while (nsegy && !valcmp(*type,val,*n_val));

	
	/* If there are no more traces then return
	if(nsegy==0 && ntrr==0 ) {
		 *ntr=0;
		 efclose(nheaderfp);
		 efclose(ntracefp);
		 return(rec=NULL);
	} else {
		/* store the first trace of the next gather
		erewind(nheaderfp);
        	erewind(ntracefp);
                efwrite(&tr, 1, HDRBYTES, nheaderfp);
                efwrite(tr.data, FSIZE, *nt, ntracefp);
	}

	/* allocate memory for the record
	{ register int i;
		rec = ealloc1(*ntr,sizeof(segy *));
		for(i=0;i<*ntr;i++)
			rec[i] = (segy *)ealloc1((*nt*FSIZE+HDRBYTES),sizeof(char));
	}
	
	
	/* load traces into an array and close temp file
	erewind(headerfp);
        erewind(tracefp);
	{ register int ix;
        	for (ix=0; ix<*ntr; ix++)
                	fread (rec[ix],HDRBYTES, 1, headerfp);
        	efclose (headerfp);
		for(ix=0; ix<*ntr; ix++)
                	fread ((*rec[ix]).data,FSIZE, *nt, tracefp);
        	efclose (tracefp);
	}
	
	return(rec);	
}

segy tr;

segy **get_gather(cwp_String *key,cwp_String *type,Value *n_val,
			int *nt,int *ntr, float *dt,int *first)
get_gather - get a gather from stdin
Input:
key		header key of ensemble
type		header value type
value		value of ensemble key word
nt		number of time samples
ntr		number of traces in ensemble
dt		time sampling interval
first		flag is this the first ensemble being read?
Notes: 
The input seismic dataset must be sorted into ensembles defined by key
Author: Potash Corporation Sascatchewan, Balasz Nemeth
given to CWP in 2008.
{
	int nsegy;
	FILE *tracefp=NULL;
	FILE *headerfp=NULL;
	static FILE *ntracefp=NULL; /* first different trace pointer
	static FILE *nheaderfp=NULL;
	segy **rec=NULL;
	int ntrr=0;
	int indx=0;
	static Value val;
	
	
	*type = hdtype(*key);
	indx = getindex(*key);
        *ntr = 0;
        	
	
	if(*first==0) {
	       	/* get info from first trace
		nsegy = vgettr(&tr);
		if (nsegy==0)  err("can't get first trace");
		*nt = tr.ns;
        	*dt = (float) tr.dt/1000000.0;
        	++ntrr;
		gethval(&tr, indx, n_val);
		ntracefp = etmpfile();
		nheaderfp = etmpfile();
		*first=1;
	} else {
		/* This is the first trace of the nex gather
		erewind(nheaderfp);
        	erewind(ntracefp);
                fread (&tr,HDRBYTES, 1, nheaderfp);
                fread (tr.data,FSIZE, *nt, ntracefp);
		gethval(&tr, indx, n_val);
	}
		
	

        /* Store traces in tmpfile while getting a count
	tracefp = etmpfile();
	headerfp = etmpfile();
        do {
                efwrite(&tr, 1, HDRBYTES, headerfp);
                efwrite(tr.data, FSIZE, *nt, tracefp);
		
		/* read the next trace
		*ntr+=1;
		val=*n_val;
		nsegy = vgettr(&tr);
		if(nsegy) ntrr++;
		gethval(&tr, indx, n_val);
	} while (nsegy && !valcmp(*type,val,*n_val));

	
	/* If there are no more traces then return
	if(nsegy==0 && ntrr==0 ) {
		 *ntr=0;
		 efclose(nheaderfp);
		 efclose(ntracefp);
		 return(rec=NULL);
	} else {
		/* store the first trace of the next gather
		erewind(nheaderfp);
        	erewind(ntracefp);
                efwrite(&tr, 1, HDRBYTES, nheaderfp);
                efwrite(tr.data, FSIZE, *nt, ntracefp);
	}

	/* allocate memory for the record
	{ register int i;
		rec = ealloc1(*ntr,sizeof(segy *));
		for(i=0;i<*ntr;i++)
			rec[i] = (segy *)ealloc1((*nt*FSIZE+HDRBYTES),sizeof(char));
	}
	
	
	/* load traces into an array and close temp file
	erewind(headerfp);
        erewind(tracefp);
	{ register int ix;
        	for (ix=0; ix<*ntr; ix++)
                	fread (rec[ix],HDRBYTES, 1, headerfp);
        	efclose (headerfp);
		for(ix=0; ix<*ntr; ix++)
                	fread ((*rec[ix]).data,FSIZE, *nt, tracefp);
        	efclose (tracefp);
	}
	
	return(rec);	
}
\end{verbatim}
\pagebreak
\begin{verbatim}
fgethdr - get segy tape identification headers from the file by file pointer
 
Input:
fp         file pointer

Output:
chdr       3200 bytes of segy character header
bhdr        400 bytes of segy binary header
Authors:  zhiming li  and j. dulac ,   unocal
 modified for CWP/SU: R. Beardsley
\end{verbatim}
\pagebreak
\begin{verbatim}
FGETTR - Routines to get an SU trace from a file 

fgettr		get a fixed-length segy trace from a file by file pointer
fvgettr		get a variable-length segy trace from a file by file pointer
fgettra		get a fixed-length trace from disk file by trace number
gettr		macro using fgettr to get a trace from stdin
vgettr		macro using vfgettr to get a trace from stdin
gettra		macro using fgettra to get a trace from stdin by trace number
 
Function Prototype:
int fgettr(FILE *fp, segy *tp);
int fvgettr(FILE *fp, segy *tp);
int fgettra(FILE *fp, segy *tp, int itr);

Returns:
fgettr, fvgettr:
int: number of bytes read on current trace (0 after last trace)

fgettra:
int: number of traces in disk file
 
Macros defined in segy.h
define gettr(x)	fgettr(stdin, (x))
define vgettr(x)	fgettr(stdin, (x))

Usage example:
 	segy tr;
 	...
 	while (gettr(&tr)) {
 		tr.offset = abs(tr.offset);
 		puttr(&tr);
 	}
 	...

Authors: SEP: Einar Kjartansson, Stew Levin CWP: Shuki Ronen, Jack Cohen

 Revised: 7/2/95 Stewart A. Levin   Mobil
     Major rewrite:  Use xdr library for portable su output file
     format.   Merge fgettr and fgettra into same source file.
     Make input from multiple streams work (at long last!).
 Revised: 11/22/95 Stewart A. Levin  Mobil
     Always set ntr for DISK input.  This fixes susort failure.
 Revised: 1/9/96  jkc CWP
     Set lastfp on nread <=0 return, too.
 Revised: 28 Mar, 2006 Stewart A. Levin   Landmark Graphics
     Reworked XDR to support random seeks on > 2GB files
     and to read big endian SHORTPACK data on little endian machines.

\end{verbatim}
\pagebreak
\begin{verbatim}
FPUTGTHR - put gathers to a file

fput_gather - put a gather to a file
put_gather - put a gather to stdout

Function Prototypes:
segy **fput_gather(FILE *fp, segy **rec,int *nt, int *ntr);
segy **put_gather(segy **rec,int *nt, int *ntr)

fput_gather - put a gather to a file
Input:
fp		pointer to output file
rec		array of segy traces
nt		number of time samples per trace
ntr		number of traces in ensemble
Author: Potash Corporation Sascatchewan, Balasz Nemeth
given to CWP in 2008.

put_gather - put a gather to stdout
Input:
rec		array of segy traces
nt		number of time samples per trace
ntr		number of traces in ensemble
Author: Potash Corporation Sascatchewan, Balasz Nemeth
given to CWP in 2008.

include "su.h"
include "segy.h"
include "header.h"


segy tr;

segy **fput_gather(FILE *fp, segy **rec,int *nt, int *ntr)
fput_gather - put a gather to a file
Input:
fp		pointer to output file
rec		array of segy traces
nt		number of time samples per trace
ntr		number of traces in ensemble
Author: Potash Corporation Sascatchewan, Balasz Nemeth
given to CWP in 2008.
{


	segy tr;
	
	{ register int i;
		for(i=0;i<*ntr;i++) {
			memcpy( (void *) &tr, (const void *) rec[i],
					*nt*FSIZE+HDRBYTES);
			fvputtr(fp,&tr);
			free1((void *)rec[i]);
		}
	}
	return(rec=NULL);
}


segy **put_gather(segy **rec,int *nt, int *ntr)
put_gather - put a gather to stdout
Input:
rec		array of segy traces
nt		number of time samples per trace
ntr		number of traces in ensemble
Author: Potash Corporation Sascatchewan, Balasz Nemeth
given to CWP in 2008.
{
	segy tr;
	
	{ register int i;
		for(i=0;i<*ntr;i++) {
			memcpy( (void *) &tr, (const void *) rec[i],
					*nt*FSIZE+HDRBYTES);
			vputtr(&tr);
			free1((void *)rec[i]);
		}
	}
	return(rec=NULL);
}
\end{verbatim}
\pagebreak
\begin{verbatim}
FPUTTR - Routines to put an SU trace to a file 

fputtr		put a segy trace to a file by file pointer
fvputtr		put a segy trace to a file by file pointer (variable ns)
puttr		macro using fputtr to put a trace to stdin
vputtr		macro using fputtr to put a trace to stdin (variable ns)
 
Function Prototype:
void fputtr(FILE *fp, segy *tp);
void fvputtr(FILE *fp, segy *tp);

Returns:

	void
 
Notes:

The functions puttr(x) vputtr(x) are macros defined in segy.h
define puttr(x)	fputtr(stdin, (x))
define vputtr(x)	fvputtr(stdin, (x))

Usage example:
 	segy tr;
 	...
 	while (gettr(&tr)) {
 		tr.offset = abs(tr.offset);
 		puttr(&tr);
 	}
 	...

Authors: SEP: Einar Kjartansson, Stew Levin CWP: Shuki Ronen, Jack Cohen
\end{verbatim}
\pagebreak
\begin{verbatim}
HDRPKGE - routines to access the SEGY header via the hdr structure.
 
gethval		get a trace header word by index
puthval		put a trace header word by index
getbhval	get a binary header word by index
putbhval	put a binary header word by index
gethdval	get a trace header word by name
puthdval	put a trace header word by name


hdtype		get the data type of a trace header word by name
getkey		get the name of a trace header word from its index

getindex	get the index of a trace header word from the name

swaphval	swap the trace header words by index
swapbhval	swap the binary header words by index
gettapehval	get a tape trace header word by index
puttapehval	put a tape trace header word by index
gettapebhval	get a tape binary header word by index
puttapebhval	put a tape binary header word by index
printheader	display non-null header field values

Function Prototypes:
void gethval(const segy *tr, int index, Value *valp);
void puthval(segy *tr, int index, Value *valp);
void putbhval(bhed *bh, int index, Value *valp);
void getbhval(const bhed *bh, int index, Value *valp);
void gethdval(const segy *tr, char *key, Value *valp);
void puthdval(segy *tr, char *key, Value *valp);
char *hdtype(const char *key);
char *getkey(const int index);
int getindex(const char *key);
void swaphval(segy *tr, int index);
void swapbhval(bhed *bh, int index);
void gettapehval(tapesegy *tapetr, int index, Value *valp);
void puttapehval(tapesegy *tapetr, int index, Value *valp);
void gettapebhval(tapebhed *tapetr, int index, Value *valp);
void puttapebhval(tapebhed *tapetr, int index, Value *valp);
void printheader(const segy *tp);

Notes:
This package includes only those routines that directly access
the "hdr" or "bhdr" structures.  It does not include routines
such as printfval, printftype, printfhead that use the routines
in this package to indirectly access these structures.

Note that while gethdval and puthdval are more convenient to use
than gethval and puthval, they incur an inefficiency in the
common case of iterating code over a set of traces with a fixed
key or keys.  In such cases, it is advisable to set the index
or indices outside the loop using getindex.

swaphval:
Byte-swapping is needed for converting SU data from big-endian to little-
endian formats, and vice versa. The swap_.... subroutines are based
on subroutines provided by Jens Hartmann of the Institut fur Geophysik
in Hamburg. These are found in .../cwp/lib/swapbyte.c.
Authors: SEP: Einar Kjartansson	CWP: Jack Cohen, Shuki Ronen
swaphval: CWP: John Stockwell
\end{verbatim}
\pagebreak
\begin{verbatim}
TABPLOT - TABPLOT selected sample points on selected trace

tabplot		tabplot selected sample points on selected trace

Function Prototype:
void tabplot(segy *tp, int itmin, int itmax);

Input:
tp		pointer to a segy
itmin		minimum time sample printed
itmax		maximum time sample printed

Authors: CWP: Brian Sumner, Jack K. Cohen
\end{verbatim}
\pagebreak
\begin{verbatim}
VALPKGE - routines to handle variables of type Value

vtoi		cast Value variable as an int
vtol		cast Value variable as a long
vtof		cast Value variable as a float
vtod		cast Value variable as a double
atoval		convert ascii to Value
valtoabs	take absolute value of a Value variable
valcmp		compare Value variables
printfval	printf a Value variable
fprintfval	fprintf a Value variable
scanfval	scanf a Value variable
printftype	printf for the type of a segy header word

Function Prototypes:
int vtoi(register cwp_String type, Value val);
long vtol(register cwp_String type, Value val);
float vtof(register cwp_String type, Value val);
double vtod(register cwp_String type, Value val);
void atoval(cwp_String type, cwp_String keyval, Value *valp);
Value valtoabs(cwp_String type, Value val);
int valcmp(register cwp_String type, Value val1, Value val2);
void printfval(register cwp_String type, Value val);
void fprintfval(FILE *stream, register cwp_String type, Value val);
void scanfval(register cwp_String type, Value *valp);

Notes:
A Value is defined by the following in .../su/include/su.h:

typedef union { * storage for arbitrary type *
	char s[8];
	short h;
	unsigned short u;
	long l;
	unsigned long v;
	int i;
	unsigned int p;
	float f;
	double d;
	unsigned int U:16;
	unsigned int P:32;
} Value;

The use of the valpkge routines, as well as the hdrpkge routines,
permits the user to change the definition of the types of the 
various fields of the segy data type, without breaking codes that
look at part or all of these fields.

Authors: CWP: Jack K. Cohen, Shuki Ronen
\end{verbatim}
\pagebreak
\begin{verbatim}
ABEL - Functions to compute the discrete ABEL transform:

abelalloc	allocate and return a pointer to an Abel transformer
abelfree 	free an Abel transformer
abel		compute the Abel transform

Function prototypes:
void *abelalloc (int n);
void abelfree (void *at);
void abel (void *at, float f[], float g[]);

Input:
ns		number of samples in the data to be transformed
f[]		array of floats, the function being transformed

Output:
at		pointer to Abel transformer returned by abelalloc(int n)
g[]		array of floats, the transformed data returned by 
		abel(*at,f[],g[])

Notes:
The Abel transform is defined by:

	         Infinity
	g(y) = 2 Integral dx f(x)/sqrt(1-(y/x)^2)
		   |y|

Linear interpolation is used to define the continuous function f(x)
corresponding to the samples in f[].  The first sample f[0] corresponds
to f(x=0) and the sampling interval is assumed to be 1.  Therefore, the
input samples correspond to 0 <= x <= n-1.  Samples of f(x) for x > n-1
are assumed to be zero.  These conventions imply that 

	g[0] = f[0] + 2*f[1] + 2*f[2] + ... + 2*f[n-1]

References:
Hansen, E. W., 1985, Fast Hankel transform algorithm:  IEEE Trans. on
Acoustics, Speech and Signal Processing, v. ASSP-33, n. 3, p. 666-671.
(Beware of several errors in the equations in this paper!)

Authors:  Dave Hale and Lydia Deng, Colorado School of Mines, 06/01/90
\end{verbatim}
\pagebreak
\begin{verbatim}
AIRY - Approximate the Airy functions  Ai(x), Bi(x) and their respective
       derivatives  Ai'(x), Bi'(x)

airya		return approximation of Ai(x)
airypa		return approximation of Ai'(x)
airyb		return approximation of Bi(x)
airybp		return approximation of Bi'(x)

Function Prototypes:
float airya (float x);
float airyap (float x);
float airyb (float x);
float airybp (float x);

Input:
x		value at which to evaluate Ai(x)

Returned:
airya		Ai(x)
airypa		Ai'(x)
airyb		Bi(x)
airybp		Bi'(x)

Reference:
The approximation is derived from tables and formulas in Abramowitz
and Stegun, p. 475-477.

Author:  Dave Hale, Colorado School of Mines, 06/06/89
\end{verbatim}
\pagebreak
\begin{verbatim}
ALLOC - Allocate and free multi-dimensional arrays

alloc1		allocate a 1-d array
realloc1	re-allocate a 1-d array
free1		free a 1-d array
alloc2		allocate a 2-d array
free2		free a 2-d array
alloc3		allocate a 3-d array
free3		free a 3-d array
alloc4		allocate a 4-d array
free4		free a 4-d array
alloc5		allocate a 5-d array
free5		free a 5-d array
alloc6		allocate a 6-d array
free6		free a 6-d arrayalloc1int	
allocate a 1-d array of ints
realloc1int	re-allocate a 1-d array of ints
free1int	free a 1-d array of ints
alloc2int	allocate a 2-d array of ints
free2int	free a 2-d array of ints
alloc3int	allocate a 3-d array of ints
free3int	free a 3-d array of ints
alloc1float	allocate a 1-d array of floats
realloc1float	re-allocate a 1-d array of floats
free1float	free a 1-d array of floats
alloc2float	allocate a 2-d array of floats
free2float	free a 2-d array of floats
alloc3float	allocate a 3-d array of floats
free3float	free a 3-d array of floats
alloc4float	allocate a 4-d array of floats 
free4float      free a 4-d array of floats 
alloc5float     allocate a 5-d array of floats 
free5float      free a 5-d array of floats 
alloc6float     allocate a 6-d array of floats 
free6float      free a 6-d array of floats 
alloc4int       allocate a 4-d array of ints 
free4int        free a 4-d array of ints 
alloc5int       allocate a 5-d array of ints 
free5int        free a 5-d array of ints 
alloc5uchar	allocate a 5-d array of unsigned chars 
free5uchar	free a 5-d array of unsiged chars 
alloc5ushort    allocate a 5-d array of unsigned shorts 
free5ushort     free a 5-d array of unsiged shorts
alloc6ushort    allocate a 6-d array of unsigned shorts 
free6ushort     free a 6-d array of unsiged shorts
alloc1double	allocate a 1-d array of doubles
realloc1double	re-allocate a 1-d array of doubles
free1double	free a 1-d array of doubles
alloc2double	allocate a 2-d array of doubles
free2double	free a 2-d array of doubles
alloc3double	allocate a 3-d array of doubles
free3double	free a 3-d array of doubles
alloc1complex	allocate a 1-d array of complexs
realloc1complex	re-allocate a 1-d array of complexs
free1complex	free a 1-d array of complexs
alloc2complex	allocate a 2-d array of complexs
free2complex	free a 2-d array of complexs
alloc3complex	allocate a 3-d array of complexs
free3complex	free a 3-d array of complexs

alloc1dcomplex   allocate a 1-d array of complexs
realloc1dcomplex re-allocate a 1-d array of complexs
free1dcomplex    free a 1-d array of complexs
alloc2dcomplex   allocate a 2-d array of complexs 
free2dcomplex    free a 2-d array of complexs
alloc3dcomplex   allocate a 3-d array of complexs  
free3dcomplex    free a 3-d array of complexs

Function Prototypes:
void *alloc1 (size_t n1, size_t size);
void *realloc1 (void *v, size_t n1, size_t size);
void free1 (void *p);
void **alloc2 (size_t n1, size_t n2, size_t size);
void free2 (void **p);
void ***alloc3 (size_t n1, size_t n2, size_t n3, size_t size);
void free3 (void ***p);
void ****alloc4 (size_t n1, size_t n2, size_t n3, size_t n4, size_t size);
void free4 (void ****p);
                  size_t size);
int *alloc1int (size_t n1);
int *realloc1int (int *v, size_t n1);
void free1int (int *p);
int **alloc2int (size_t n1, size_t n2);
void free2int (int **p);
int ***alloc3int (size_t n1, size_t n2, size_t n3);
void free3int (int ***p);
float *alloc1float (size_t n1);
float *realloc1float (float *v, size_t n1);
void free1float (float *p);
float **alloc2float (size_t n1, size_t n2);
void free2float (float **p);
float ***alloc3float (size_t n1, size_t n2, size_t n3);
void free3float (float ***p);
float ****alloc4float (size_t n1, size_t n2, size_t n3, size_t n4);
void free4float (float ****p);
                         size_t n6);
int ****alloc4int (size_t n1, size_t n2, size_t n3, size_t n4);
void free4int (int ****p);
	size_t n5);
        size_t n5);
        size_t n5,size_t n6);
double *alloc1double (size_t n1);
double *realloc1double (double *v, size_t n1);
void free1double (double *p);
double **alloc2double (size_t n1, size_t n2);
void free2double (double **p);
double ***alloc3double (size_t n1, size_t n2, size_t n3);
void free3double (double ***p);
complex *alloc1complex (size_t n1);
complex *realloc1complex (complex *v, size_t n1);
void free1complex (complex *p);
complex **alloc2complex (size_t n1, size_t n2);
void free2complex (complex **p);
complex ***alloc3complex (size_t n1, size_t n2, size_t n3);
void free3complex (complex ***p);

complex *alloc1dcomplex (size_t n1);
complex *realloc1dcomplex (dcomplex *v, size_t n1);
void free1dcomplex (dcomplex *p);
complex **alloc2dcomplex (size_t n1, size_t n2);
void free2dcomplex (dcomplex **p);
complex ***alloc3dcomplex (size_t n1, size_t n2, size_t n3);
void free3dcomplex (dcomplex ***p);


Notes:
The functions defined below are intended to simplify manipulation
of multi-dimensional arrays in scientific programming in C.  These
functions are useful only because true multi-dimensional arrays
in C cannot have variable dimensions (as in FORTRAN).  For example,
the following function IS NOT valid in C:
	void badFunc(a,n1,n2)
	float a[n2][n1];
	{
		a[n2-1][n1-1] = 1.0;
	}
However, the following function IS valid in C:
	void goodFunc(a,n1,n2)
	float **a;
	{
		a[n2-1][n1-1] = 1.0;
	}
Therefore, the functions defined below do not allocate true
multi-dimensional arrays, as described in the C specification.
Instead, they allocate and initialize pointers (and pointers to 
pointers) so that, for example, a[i2][i1] behaves like a 2-D array.

The array dimensions are numbered, which makes it easy to add 
functions for arrays of higher dimensions.  In particular,
the 1st dimension of length n1 is always the fastest dimension,
the 2nd dimension of length n2 is the next fastest dimension,
and so on.  Note that the 1st (fastest) dimension n1 is the 
first argument to the allocation functions defined below, but 
that the 1st dimension is the last subscript in a[i2][i1].
(This is another important difference between C and Fortran.)

The allocation of pointers to pointers implies that more storage
is required than is necessary to hold a true multi-dimensional array.
The fraction of the total storage allocated that is used to hold 
pointers is approximately 1/(n1+1).  This extra storage is unlikely
to represent a significant waste for large n1.

The functions defined below are significantly different from similar 
functions described by Press et al, 1988, NR in C.
In particular, the functions defined below:
	(1) Allocate arrays of arbitrary size elements.
	(2) Allocate contiguous storage for arrays.
	(3) Return NULL if allocation fails (just like malloc).
	(4) Do not provide arbitrary lower and upper bounds for arrays.

Contiguous storage enables an allocated multi-dimensional array to
be passed to a C function that expects a one-dimensional array.
For example, to allocate and zero an n1 by n2 two-dimensional array
of floats, one could use
	a = alloc2(n1,n2,sizeof(float));
	zeroFloatArray(n1*n2,a[0]);
where zeroFloatArray is a function defined as
	void zeroFloatArray(int n, float *a)
	{
		int i;
		for (i=0; i<n; i++)
			a[i] = 0.0;
	}

Internal error handling and arbitrary array bounds, if desired,
should be implemented in functions that call the functions defined 
below, with the understanding that these enhancements may limit 
portability.

Author:    	Dave Hale, Colorado School of Mines, 12/31/89
                Zhaobo Meng, added 4D, 5D and 6D functions, 1996
\end{verbatim}
\pagebreak
\begin{verbatim}
ANTIALIAS - Butterworth anti-aliasing filter

antialias	use before increasing the sampling interval of data
		 i.e. subsampling 

Function Prototype:
void antialias (float frac, int phase, int n, float p[], float q[]);

Input:
frac		current sampling interval / future interval (should be <= 1)
phase		=0 for zero-phase filter; =1 for minimum-phase filter
n		number of samples
p		array[n] of input samples

Output:
q		array[n] of output (anti-alias filtered) samples		

Notes:
The anti-alias filter is a recursive (Butterworth) filter.  For zero-phase
anti-alias filtering, the recursive filter is applied forwards and backwards.

Author:  Dave Hale, Colorado School of Mines, 06/06/90
\end{verbatim}
\pagebreak
\begin{verbatim}
AXB - Functions to solve a linear system of equations Ax=b by LU
	decomposition, invert a square matrix or directly multiply an
	inverse matrix by another matrix (without explicitely computing
	the inverse).

LU_decomposition	Decompose a matrix (A) into a lower triangular (L)
			and an upper triangular (U) such that A=LU

backward_substitution	Apply backward substitution to an LU decomposed
			matrix to solve the linear system of equations Ax=b

inverse_matrix		compute the inverse of a square non-singular matrix

inverse_matrix_multiply	computes the product A^(-1)*B without explicitely
			computing the inverse matrix

Function prototypes:
void LU_decomposition (int nrows, float **matrix, int *idx, float *d);
void backward_substitution (int nrows, float **matrix, int *idx, float *b);
void inverse_matrix (int nrows, float **matrix);
void inverse_matrix_multiply (int nrows1, float **matrix1, int ncols2,
        int nrows2, float **matrix2, float **out_matrix);

LU_decomposition:
Input:
nrows		number of rows of matrix to invert
matrix		matrix of coefficients in linear system Ax=b 

Output:
matrix		matrix containing LU decomposition (original matrix destroyed)
idx		vector recording the row permutations effected by partial
		pivoting
d		+/- 1 depending on whether the number of row interchanges
		was even or odd
backward_substitution
Input:
nrows		number of rows (and columns) of input matrix
matrix		matrix of coefficients (after LU decomposition)
idx		permutation vector obtained from routine LU_decomposition 
b		right hand side vector in equation Ax=b

Output:
b		vector with the solution
inverse_matrix
Input:
nrows		number of rows (and columns) of input matrix
matrix		matrix to invert

Output:
matrix		inverse of input matrix 
inverse_matrix_multiply
nrows1          number of rows (and columns) of matrix to invert
matrix1         square matrix to invert
ncols2          number of coulmns of second matrix
nrows2		number of rows of second matrix
matrix          second matrix (multiplicator)

Output Parameters:
out_matrix      matrix containing the product of the inverse of the first
                matrix by the second one.
Note:
matrix1 and matrix2 are not destroyed, (not clobbered)
Notes:
To solve the set of linear equations Ax=b, first do the LU decomposition of
A (which will clobber A with its LU decomposition) and then do the backward 
substitution with this new matrix and the right-hand side vector b. The vector
b will be clobbered with the solution. Both, the original matrix and vector B,
will have been destroyed.

The LU decomposition is carried out with the Crout's method with implicit
partial pivoting that guaratees that the maximum pivot is used in every
step of the algorithm.

The operation count to solve a linear system of equations via LU decomposition
is 1/3N^3 and is a factor of 3 better than the standard Gauss-Jordan algorithm
To invert a matrix the count is the same with both algorithms: N^3.

Once a linear system Ax=b has been solved, to solve another linear system
with the same matrix A but with different vetor b, ONLY the back substitution 
has to be repeated with the new b (remember that the matrix in backsubstitution
is not the original matrix but its LU decomposition)

If you want to compute A^(-1)*B from matrices A and B, it is better to 
use the subroutine inverse_matrix_multiply rather than explicitely computing
the inverse. This saves a whole martix multiplication and is also more accurate.

Refferences:
Press, Teukolsky, Vettering and Flannery, Numerical Recipes in C: 
	The art of scientific computing. Cambridge University Press.
	second edition. (1992).
Golub and Van Loan, Matrix Computations. John Hopkins University Press.
	Second Edition. (1989). 
Horn and Johnson, Matrix Analysis. Cambridge University Press. (1985).
Credits:
Adapted from discussions in Numerical Recipes, by Gabriel Alvarez (1995)
\end{verbatim}
\pagebreak
\begin{verbatim}
BIGMATRIX - Functions to manipulate 2-dimensional matrices that are too big 
	    to fit in real memory, but that are small enough to fit in
		virtual memory:

bmalloc		allocate a big matrix
bmfree		free a big matrix 
bmread		read a vector from a big matrix
bmwrite		write a vector to a big matrix

Function Prototypes:
void *bmalloc (int nbpe, int n1, int n2);
void bmfree (void *bm);
void bmread (void *bm, int dir, int k1, int k2, int n, void *v);
void bmwrite (void *bm, int dir, int k1, int k2, int n, void *v);

bmalloc:
Input:
nbpe		number of bytes per matrix element
n1		number of elements in 1st (fastest) dimension
n2		number of elements in 2nd (slowest) dimension

Returned:
bm		pointer to big matrix

bmfree:
Input:
bm		pointer to big matrix state (returned by bmalloc)

bmread:
Input:
bm    		pointer to big matrix state (returned by bmalloc)
d  		= 1 or 2:  direction in which to read matrix elements
k1		1st dimension index of first matrix element to read
k2		2nd dimension index of first matrix element to read
n		number of elements to read

Output:
v		array[n] to contain matrix elements read

bmwrite:
Input:
bm    		pointer to big matrix state (returned by bmalloc)
d  		= 1 or 2:  direction in which to write matrix elements
k1		1st dimension index of first matrix element to write
k2		2nd dimension index of first matrix element to write
n		number of elements to write
v		array[n] containing matrix elements to write

Notes:
The bm functions provide access to a big 2-dimensional matrix along
either the 1st or 2nd dimensions.  Although, the matrix must be small
enough to fit in virtual memory, it may be too large to fit in real memory.
These functions provide equally efficient (or equally inefficient) access 
to vectors in a big matrix along either the 1st or 2nd dimensions.

For example, the following algorithm will efficiently transpose an
n1 by n2 array of (n1*n2) floats stored in a file:

	void *bm;
	float *v;
	bm = bmalloc(sizeof(float),n1,n2);
	for (i2=0; i2<n2; i2++) {
		(read n1 floats from input file into array v);
		bmwrite(bm,1,0,i2,n1,(char*)v);
	}
	for (i1=0; i1<n1; i1++) {
		bmread(bm,2,i1,0,n2,(char*)v);
		(write n2 floats in array v to output file);
	}
	bmfree(bm);

Author:  Dave Hale, Colorado School of Mines, 05/17/89
\end{verbatim}
\pagebreak
\begin{verbatim}
BUTTERWORTH - Functions to design and apply Butterworth filters:

bfdesign	design a Butterworth filter
bfhighpass	apply a high-pass Butterworth filter 
bflowpass	apply a low-pass Butterworth filter 

Function Prototypes:
void bfhighpass (int npoles, float f3db, int n, float p[], float q[]);
void bflowpass (int npoles, float f3db, int n, float p[], float q[]);
void bfdesign (float fpass, float apass, float fstop, float astop,
	int *npoles, float *f3db);

bfdesign:
Input:
fpass		frequency in pass band at which amplitude is >= apass
apass		amplitude in pass band corresponding to frequency fpass
fstop 		frequency in stop band at which amplitude is <= astop
astop		amplitude in stop band corresponding to frequency fstop

Output:
npoles		number of poles
f3db		frequency at which amplitude is sqrt(0.5) (-3 db)

bfhighpass and bflowpass:
Input:
npoles		number of poles (and zeros); npoles>=0 is required
f3db		3 db frequency; nyquist = 0.5; 0.0<=f3db<=0.5 is required
n		length of p and q
p		array[n] to be filtered

Output:
q		filtered array[n] (may be equivalent to p)

Notes:
(1) Nyquist frequency equals 0.5

(2) The following conditions must be true:
	(0.0<fpass && fpass<0.5) &&
	(0.0<fstop && fstop<0.5) &&
	(fpass!=fstop) &&
	(0.0<astop && astop<apass && apass<1.0)

(3) if (fpass<fstop)

bfdesign:
Butterworth filter:  compute number of poles and -3 db frequency
for a low-pass or high-pass filter, given a frequency response
constrained at two frequencies.

Author:  Dave Hale, Colorado School of Mines, 06/02/89
\end{verbatim}
\pagebreak
\begin{verbatim}
COMPLEX - Functions to manipulate complex numbers

cadd	add two complex numbers
csub	subtract two complex numbers
cmul	multiply two complex numbers
cdiv	divide two complex numbers
cmplx	make a complex number from two real numbers
conjg	complex conjugate of a complex number 
cneg	negate a complex number
cinv	invert a complex number
cwp_csqrt	complex square root of a complex number
cwp_cexp	complex exponential of a complex number
crmul	multiply a complex number by a real number 
rcabs	real magnitude of a complex number

Structure:
typedef struct _complexStruct {  complex number
	float r,i;
} complex;

Function Prototypes:
complex cadd (complex a, complex b);
complex csub (complex a, complex b);
complex cmul (complex a, complex b);
complex cdiv (complex a, complex b);
float rcabs (complex z);
complex cmplx (float re, float im);
complex conjg (complex z);
complex cneg (complex z);
complex cinv (complex z);
complex cwp_csqrt (complex z);
complex cwp_cexp (complex z);
complex crmul (complex a, float x);

Notes:
The function "rcabs" was originally called "fcabs". This produced
a collision on some systems so a new name was chosen.

Reference:
Adapted from Press et al, 1988, Numerical Recipes in C (Appendix E).

Author:  Dave Hale, Colorado School of Mines, 06/02/89
Modified:  Dave Hale, Colorado School of Mines, 04/26/90
	Added function cinv().
\end{verbatim}
\pagebreak
\begin{verbatim}
COMPLEXD - Functions to manipulate double-precision complex numbers

dcadd	add two dcomplex numbers
dcsub	subtract two dcomplex numbers
dcmul	multiply two dcomplex numbers
dcdiv	divide two dcomplex numbers
dcmplx	make a dcomplex number from two real numbers
dconjg	dcomplex conjugate of a dcomplex number 
dcneg	negate a dcomplex number
dcinv	invert a dcomplex number
dcsqrt	dcomplex square root of a dcomplex number
dcexp	dcomplex exponential of a dcomplex number
dcrmul	multiply a dcomplex number by a real number 
drcabs	real magnitude of a dcomplex number

Structure:
typedef struct _dcomplexStruct {  dcomplex number
	double r,i;
} dcomplex;

Function Prototypes:
dcomplex dcadd (dcomplex a, dcomplex b);
dcomplex dcsub (dcomplex a, dcomplex b);
dcomplex dcmul (dcomplex a, dcomplex b);
dcomplex dcdiv (dcomplex a, dcomplex b);
double drcabs (dcomplex z);
dcomplex dcmplx (double re, double im);
dcomplex dconjg (dcomplex z);
dcomplex dcneg (dcomplex z);
dcomplex dcinv (dcomplex z);
dcomplex dcsqrt (dcomplex z);
dcomplex dcexp (dcomplex z);
dcomplex dcrmul (dcomplex a, double x);

Notes:
The function "drcabs" was originally called "fcabs". This produced
a collision on some systems so a new name was chosen.

Reference:
Adapted from Press et al, 1988, Numerical Recipes in C (Appendix E).

Author:  Dave Hale, Colorado School of Mines, 06/02/89
Modified:  Dave Hale, Colorado School of Mines, 04/26/90
	Added function dcinv().
\end{verbatim}
\pagebreak
\begin{verbatim}
COMPLEXF  - Subroutines to perform operations on complex numbers.
		This set of functions complement the one in complex.c
		of the CWP library

cipow		raise a complex number to an integer power
crpow		raise a complex number to a real power
rcpow		raise a real number to a complex power
ccpow		raise a complex number to a complex power
cwp_ccos		compute the complex cosine of a complex angle
cwp_csin		compute the complex sine of a complex angle
cwp_ccosh		compute the complex hyperbolic cosine of a complex angle
cwp_csinh		compute the complex hyperbolic sine of a complex angle
cwp_cexp1		compute the complex exponential of a complex number
cwp_clog		compute the complex logarithm of a complex number
Function Prototypes:
complex cipow(complex a, int p);
complex crpow(complex a, float p);
complex rcpow(float a, complex p);
complex ccpow (complex a, complex p)
complex cwp_ccos(complex a);
complex cwp_csin(complex a);
complex cwp_ccosh(complex a);
complex cwp_csinh(complex a);
complex cwp_cexp1(complex a);
complex cwp_clog(complex a);
Credits:
	Dave Hale, original version in C++
	Gabriel Alvarez, translation to C
\end{verbatim}
\pagebreak
\begin{verbatim}
COMPLEXFD  - Subroutines to perform operations on double complex numbers.
		This set of functions complement the one in complexd.c
		of the CWP library

dcipow		raise a double complex number to an integer power
dcrpow		raise a double complex number to a real power
rdcpow		raise a real number to a double complex power
dcdcpow		raise a double complex number to a double complex power
dccos		compute the double complex cosine of a double complex angle
dcsin		compute the double complex sine of a double complex angle
dccosh		compute the double complex hyperbolic cosine of a double complex angle
dcsinh		compute the double complex hyperbolic sine of a double complex angle
dcexp1		compute the double complex exponential of a double complex number
dclog		compute the double complex logarithm of a double complex number
Function Prototypes:
dcomplex dcipow(dcomplex a, int p);
dcomplex dcrpow(dcomplex a, float p);
dcomplex rdcpow(float a, dcomplex p);
dcomplex dcdcpow (dcomplex a, dcomplex p)
dcomplex dccos(dcomplex a);
dcomplex dcsin(dcomplex a);
dcomplex dccosh(dcomplex a);
dcomplex dcsinh(dcomplex a);
dcomplex dcexp1(dcomplex a);
dcomplex dclog(dcomplex a);
Credits:
	Dave Hale, original version in C++
	Gabriel Alvarez, translation to C
\end{verbatim}
\pagebreak
\begin{verbatim}
Conjugate Gradient routines -

simple_conj_gradient -  simple conjugate gradient routine.


Function Prototypes:
void simple_conj_gradient(int n, float *x, int m, float *b, float **a, int niter);


References:
Claerbout, J. F. (1985), Conjugate gradients for beginners, SEP 44,
Stanford Exploration Project.
Shewchuk, Jonathan Richard (1994), An introduction to the conjugate 
gradient method without the agonizing pain,
edition 1 1/4, School of Computer Science Carnegie Mellon University
Credits:
CWP: John Stockwell,  May 2014
Based on the subroutine cg() which appeared in (Claerbout, 1985)

\end{verbatim}
\pagebreak
\begin{verbatim}
CONVOLUTION - Compute z = x convolved with y

convolve_cwp	compute the convolution of two input vector arrays

Input:
lx		length of x array
ifx		sample index of first x
x		array[lx] to be convolved with y
ly		length of y array
ify		sample index of first y
y		array[ly] with which x is to be convolved
lz		length of z array
ifz		sample index of first z

Output:
z		array[lz] containing x convolved with y

Function Prototype:
void convolve_cwp (int lx, int ifx, float *x, int ly, int ify, float *y,
	int lz, int ifz, float *z);

Notes:
The operation z = x convolved with y is defined to be
           ifx+lx-1
    z[i] =   sum    x[j]*y[i-j]  ;  i = ifz,...,ifz+lz-1
            j=ifx
The x samples are contained in x[0], x[1], ..., x[lx-1]; likewise for
the y and z samples.  The sample indices of the first x, y, and z values
determine the location of the origin for each array.  For example, if
z is to be a weighted average of the nearest 5 samples of y, one might
use 
	...
	x[0] = x[1] = x[2] = x[3] = x[4] = 1.0/5.0;
	conv(5,-2,x,lx,0,y,ly,0,z);
	...
In this example, the filter x is symmetric, with index of first sample = -2.

This function is optimized for architectures that can simultaneously perform
a multiply, add, and one load from memory; e.g., the IBM RISC System/6000.
Because, for each value of i, it accumulates the convolution sum z[i] in a
scalar, this function is not likely to be optimal for vector architectures.

Author:  Dave Hale, Colorado School of Mines, 11/23/91
\end{verbatim}
\pagebreak
\begin{verbatim}
CUBICSPLINE - Functions to compute CUBIC SPLINE interpolation coefficients

cakima		compute cubic spline coefficients via Akima's method
		  (continuous 1st derivatives, only)
cmonot		compute cubic spline coefficients via the Fritsch-Carlson method
		  (continuous 1st derivatives, only)
csplin		compute cubic spline coefficients for interpolation 
		  (continuous 1st and 2nd derivatives)

chermite	compute cubic spline coefficients via Hermite Polynomial
		  (continuous 1st derivatives only)
Function Prototypes:
void cakima   (int n, float x[], float y[], float yd[][4]);
void cmonot   (int n, float x[], float y[], float yd[][4]);
void csplin   (int n, float x[], float y[], float yd[][4]);
void chermite (int n, float x[], float y[], float yd[][4]);

Input:
n		number of samples
x  		array[n] of monotonically increasing or decreasing abscissae
y		array[n] of ordinates

Output:
yd		array[n][4] of cubic interpolation coefficients (see notes)

Notes:
The computed cubic spline coefficients are as follows:
yd[i][0] = y(x[i])    (the value of y at x = x[i])
yd[i][1] = y'(x[i])   (the 1st derivative of y at x = x[i])
yd[i][2] = y''(x[i])  (the 2nd derivative of y at x = x[i])
yd[i][3] = y'''(x[i]) (the 3rd derivative of y at x = x[i])

To evaluate y(x) for x between x[i] and x[i+1] and h = x-x[i],
use the computed coefficients as follows:
y(x) = yd[i][0]+h*(yd[i][1]+h*(yd[i][2]/2.0+h*yd[i][3]/6.0))

Akima's method provides continuous 1st derivatives, but 2nd and
3rd derivatives are discontinuous.  Akima's method is not linear, 
in that the interpolation of the sum of two functions is not the 
same as the sum of the interpolations.

The Fritsch-Carlson method yields continuous 1st derivatives, but 2nd
and 3rd derivatives are discontinuous.  The method will yield a 
monotonic interpolant for monotonic data.  1st derivatives are set 
to zero wherever first divided differences change sign.

The method used by "csplin" yields continuous 1st and 2nd derivatives.

References:
See Akima, H., 1970, A new method for 
interpolation and smooth curve fitting based on local procedures,
Journal of the ACM, v. 17, n. 4, p. 589-602.

For more information, see Fritsch, F. N., and Carlson, R. E., 1980, 
Monotone piecewise cubic interpolation:  SIAM J. Numer. Anal., v. 17,
n. 2, p. 238-246.
Also, see the book by Kahaner, D., Moler, C., and Nash, S., 1989, 
Numerical Methods and Software, Prentice Hall.  This function was 
derived from SUBROUTINE PCHEZ contained on the diskette that comes 
with the book.

For more general information on spline functions of all types see the book by:
Greville, T.N.E, 1969, Theory and Applications of Spline Functions,
Academic Press.

Author:  Dave Hale, Colorado School of Mines c. 1989, 1990, 1991
\end{verbatim}
\pagebreak
\begin{verbatim}
DBLAS - Double precision Basic Linear Algebra subroutines
		(adapted from LINPACK FORTRAN):

idamax	return index of element with maximum absolute value
dasum	return sum of absolute values
daxpy	compute y[i] = a*x[i]+y[i]
dcopy	copy x[i] to y[i] (i.e., set y[i] = x[i])
ddot	return sum of x[i]*y[i] (i.e., return the dot product of x and y)
dnrm2	return square root of sum of squares of x[i]
dscal	compute x[i] = a*x[i]
dswap	swap x[i] and y[i]

Function Prototypes:
int idamax (int n, double *sx, int incx);
double dasum (int n, double *sx, int incx);
void daxpy (int n, double sa, double *sx, int incx, double *sy, int incy);
void dcopy (int n, double *sx, int incx, double *sy, int incy);
double ddot (int n, double *sx, int incx, double *sy, int incy);
double dnrm2 (int n, double *sx, int incx);
void dscal (int n, double sa, double *sx, int incx);
void dswap (int n, double *sx, int incx, double *sy, int incy);

idmax:
Input: 
n		number of elements in array
sx		array[n] of elements
incx		increment between elements 

Returned:
index of element with maximum absolute value (idamax)

dasum:
Input: 
n		number of elements in array
sx		array[n] of elements
incx		increment between elements 

Returned:
sum of absolute values (dasum)

daxpy:
Input: 
n		number of elements in arrays
sa		the scalar multiplier
sx		array[n] of elements to be scaled and added
incx		increment between elements of sx
sy		array[n] of elements to be added
incy		increment between elements of sy

Output:
sy		array[n] of accumulated elements

dcopy:
Input: 
n		number of elements in arrays
sx		array[n] of elements to be copied
incx		increment between elements of sx
incy		increment between elements of sy

Output:
sy		array[n] of copied elements

ddot:
Input:
n		number of elements in arrays
sx		array[n] of elements
incx		increment between elements of sx
sy		array[n] of elements
incy		increment between elements of sy

Returned:	dot product of the two arrays

dnrm2:
Input:
n		number of elements in array
sx		array[n] of elements
incx		increment between elements 

Returned:	square root of sum of squares of x[i]

dscal:
Input:
n		number of elements in array
sa		the scalar multiplier
sx		array[n] of elements
incx		increment between elements 

Output:
sx		array[n] of scaled elements

Author:  Dave Hale, Colorado School of Mines, 10/01/89
\end{verbatim}
\pagebreak
\begin{verbatim}
DGE - Double precision Gaussian Elimination matrix subroutines  adapted
	from LINPACK FORTRAN:

dgefa	Gaussian elimination to obtain the LU factorization of a matrix.
dgeco	Gaussian elimination to obtain the LU factorization and condition
		number of a matrix.
dgesl	Solve linear system Ax = b or A'x = b after LU factorization.

Function Prototypes:
void dgefa (double **a, int n, int *ipvt, int *info);
void dgeco (double **a, int n, int *ipvt, double *rcond, double *z);
void dgesl (double **a, int n, int *ipvt, double *b, int job);

dgfa:
Input:
a		matrix[n][n] to be factored (see notes below)
n		dimension of a

Output:
a		matrix[n][n] factored (see notes below)
ipvt		indices of pivot permutations (see notes below)
info		index of last zero pivot (or -1 if no zero pivots)

dgeco:
Input:
a		matrix[n][n] to be factored (see notes below)
n		dimension of a

Output:
a		matrix[n][n] factored (see notes below)
ipvt		indices of pivot permutations (see notes below)
rcond		reciprocal of condition number (see notes below)

Workspace:
z		array[n]

dgesl:
Input:
a		matrix[n][n] that has been LU factored (see notes below)
n		dimension of a
ipvt		indices of pivot permutations (see notes below)
b		right-hand-side vector[n]
job		=0 to solve Ax = b
		=1 to solve A'x = b

Output:
b		solution vector[n]

Notes:
These functions were adapted from LINPACK FORTRAN.  Because two-dimensional 
arrays cannot be declared with variable dimensions in C, the matrix a
is actually a pointer to an array of pointers to floats, as declared
above and used below.

Elements of a are stored as follows:
a[0][0]    a[1][0]    a[2][0]   ... a[n-1][0]
a[0][1]    a[1][1]    a[2][1]   ... a[n-1][1]
a[0][2]    a[1][2]    a[2][2]   ... a[n-1][2]
.                                       .
.             .                         .
.                        .              .
.                                       .
a[0][n-1]  a[1][n-1]  a[2][n-1] ... a[n-1][n-1]

Both the factored matrix a and the pivot indices ipvt are required
to solve linear systems of equations via dgesl.


dgeco:
Given the reciprocal of the condition number, rcond, and the double
epsilon, DBL_EPSILON, the number of significant decimal digits, nsdd,
in the solution of a linear system of equations may be estimated by:
	nsdd = (int)log10(rcond/DBL_EPSILON)

Author:  Dave Hale, Colorado School of Mines, 1989
\end{verbatim}
\pagebreak
\begin{verbatim}
DIFFERENTIATE - simple DIFFERENTIATOR codes

differentiate - 1D two point centered difference based derivative

Function Prototype:
void differentiate(int n, float h, float *f, float *fprime);
void ddifferentiate(int n, double h, double *f, double *fprime);

differentiate:
Input:
n		number of samples
h		sample rate
f		array[n] of input values
Output:
fprime		array[n], the derivative of f

fprime:
Notes:
This is a simple 2 point centered-difference differentiator.
The derivatives at the endpoints are computed via 2 point leading and
lagging differences. 

Author: John Stockwell, CWP, 1994
\end{verbatim}
\pagebreak
\begin{verbatim}
PFAFFT - Functions to perform Prime Factor (PFA) FFT's, in place

npfa_d		return valid n for complex-to-complex PFA
npfar_d		return valid n for real-to-complex/complex-to-real PFA
npfao_d		return optimal n for complex-to-complex PFA
npfaro_d		return optimal n for real-to-complex/complex-to-real PFA
pfacc_d		1D PFA complex to complex
pfacr_d		1D PFA complex to real
pfarc_d		1D PFA real to complex
pfamcc_d		multiple PFA complex to real
pfa2cc_d		2D PFA complex to complex
pfa2cr_d		2D PFA complex to real
pfa2rc_d		2D PFA real to complex

Function Prototypes:
int npfa_d (int nmin);
int npfao_d (int nmin, int nmax);
int npfar_d (int nmin);
int npfaro_d (int nmin, int nmax);
void pfacc_d (int isign, int n, real_complex z[]);
void pfacr_d (int isign, int n, real_complex cz[], real rz[]);
void pfarc_d (int isign, int n, real rz[], real_complex cz[]);
void pfamcc_d (int isign, int n, int nt, int k, int kt, real_complex z[]);
void pfa2cc_d (int isign, int idim, int n1, int n2, real_complex z[]);
void pfa2cr_d (int isign, int idim, int n1, int n2, real_complex cz[], real rz[]);
void pfa2rc_d (int isign, int idim, int n1, int n2, real rz[], real_complex cz[]);

npfa_d:
Input:
nmin		lower bound on returned value (see notes below)

Returned:	valid n for prime factor fft

npfao_d
Input:
nmin		lower bound on returned value (see notes below)
nmax		desired (but not guaranteed) upper bound on returned value

Returned:	valid n for prime factor fft

npfar_d
Input:
nmin		lower bound on returned value

Returned:	valid n for real-to-real_complex/real_complex-to-real prime factor fft

npfaro_d:
Input:
nmin		lower bound on returned value
nmax		desired (but not guaranteed) upper bound on returned value

Returned:	valid n for real-to-real_complex/real_complex-to-real prime factor fft

pfacc_d:
Input:
isign		sign of isign is the sign of exponent in fourier kernel
n		length of transform (see notes below)
z		array[n] of real_complex numbers to be transformed in place

Output:
z		array[n] of real_complex numbers transformed

pfacr_d:
Input:
isign       sign of isign is the sign of exponent in fourier kernel
n           length of transform (see notes below)
cz          array[n/2+1] of real_complex values (may be equivalenced to rz)

Output:
rz          array[n] of real values (may be equivalenced to cz)

pfarc_d:
Input:
isign       sign of isign is the sign of exponent in fourier kernel
n           length of transform; must be even (see notes below)
rz          array[n] of real values (may be equivalenced to cz)

Output:
cz          array[n/2+1] of real_complex values (may be equivalenced to rz)

pfamcc_d:
Input:
isign       	sign of isign is the sign of exponent in fourier kernel
n           	number of real_complex elements per transform (see notes below)
nt          	number of transforms
k           	stride in real_complex elements within transforms
kt          	stride in real_complex elements between transforms
z           	array of real_complex elements to be transformed in place

Output:
z		array of real_complex elements transformed

pfa2cc_d:
Input:
isign       	sign of isign is the sign of exponent in fourier kernel
idim        	dimension to transform, either 1 or 2 (see notes)
n1          	1st (fast) dimension of array to be transformed (see notes)
n2          	2nd (slow) dimension of array to be transformed (see notes)
z           	array[n2][n1] of real_complex elements to be transformed in place

Output:
z		array[n2][n1] of real_complex elements transformed

pfa2cr_d:
Input:
isign       sign of isign is the sign of exponent in fourier kernel
idim        dimension to transform, which must be either 1 or 2 (see notes)
n1          1st (fast) dimension of array to be transformed (see notes)
n2          2nd (slow) dimension of array to be transformed (see notes)
cz          array of real_complex values (may be equivalenced to rz)

Output:
rz          array of real values (may be equivalenced to cz)

pfa2rc_d:
Input:
isign       sign of isign is the sign of exponent in fourier kernel
idim        dimension to transform, which must be either 1 or 2 (see notes)
n1          1st (fast) dimension of array to be transformed (see notes)
n2          2nd (slow) dimension of array to be transformed (see notes)
rz          array of real values (may be equivalenced to cz)

Output:
cz          array of real_complex values (may be equivalenced to rz)

Notes:
Table of valid n and cost for prime factor fft.  For each n, cost
was estimated to be the inverse of the number of ffts done in 1 sec
on an IBM RISC System/6000 Model 320H, by Dave Hale, 08/04/91.
(Redone by Jack Cohen for 15 sec to rebuild NTAB table on advice of
David and Gregory Chudnovsky, 05/03/94).
Cost estimates are least accurate for very small n.  An alternative method
for estimating cost would be to count multiplies and adds, but this
method fails to account for the overlapping of multiplies and adds
that is possible on some computers, such as the IBM RS/6000 family.

npfa_d:
The returned n will be composed of mutually prime factors from
the set {2,3,4,5,7,8,9,11,13,16}.  Because n cannot exceed
720720 = 5*7*9*11*13*16, 720720 is returned if nmin exceeds 720720.

npfao_d:
The returned n will be composed of mutually prime factors from
the set {2,3,4,5,7,8,9,11,13,16}.  Because n cannot exceed
720720 = 5*7*9*11*13*16, 720720 is returned if nmin exceeds 720720.
If nmin does not exceed 720720, then the returned n will not be 
less than nmin.  The optimal n is chosen to minimize the estimated
cost of performing the fft, while satisfying the constraint, if
possible, that n not exceed nmax.

npfar and npfaro:
Current implemenations of real-to-real_complex and real_complex-to-real prime 
factor ffts require that the transform length n be even and that n/2 
be a valid length for a real_complex-to-real_complex prime factor fft.  The 
value returned by npfar satisfies these conditions.  Also, see notes 
for npfa.

pfacc:
n must be factorable into mutually prime factors taken 
from the set {2,3,4,5,7,8,9,11,13,16}.  in other words,
	n = 2**p * 3**q * 5**r * 7**s * 11**t * 13**u
where
	0 <= p <= 4,  0 <= q <= 2,  0 <= r,s,t,u <= 1
is required for pfa to yield meaningful results.  this
restriction implies that n is restricted to the range
	1 <= n <= 720720 (= 5*7*9*11*13*16)

pfacr:
Because pfacr uses pfacc to do most of the work, n must be even 
and n/2 must be a valid length for pfacc.  The simplest way to
obtain a valid n is via n = npfar(nmin).  A more optimal n can be 
obtained with npfaro.

pfarc:
Because pfarc uses pfacc to do most of the work, n must be even 
and n/2 must be a valid length for pfacc.  The simplest way to
obtain a valid n is via n = npfar(nmin).  A more optimal n can be 
obtained with npfaro.

pfamcc:
To perform a two-dimensional transform of an n1 by n2 real_complex array 
(assuming that both n1 and n2 are valid "n"), stored with n1 fast 
and n2 slow:
    pfamcc(isign,n1,n2,1,n1,z); (to transform 1st dimension)
    pfamcc(isign,n2,n1,n1,1,z); (to transform 2nd dimension)

pfa2cc:
Only one (either the 1st or 2nd) dimension of the 2-D array is transformed.

If idim equals 1, then n2 transforms of n1 real_complex elements are performed; 
else, if idim equals 2, then n1 transforms of n2 real_complex elements are 
performed.

Although z appears in the argument list as a one-dimensional array,
z may be viewed as an n1 by n2 two-dimensional array:  z[n2][n1].

Valid n is computed via the "np" subroutines.

To perform a two-dimensional transform of an n1 by n2 real_complex array 
(assuming that both n1 and n2 are valid "n"), stored with n1 fast 
and n2 slow:  pfa2cc(isign,1,n1,n2,z);  pfa2cc(isign,2,n1,n2,z);

pfa2cr:
If idim equals 1, then n2 transforms of n1/2+1 real_complex elements to n1 real 
elements are performed; else, if idim equals 2, then n1 transforms of n2/2+1 
real_complex elements to n2 real elements are performed.

Although rz appears in the argument list as a one-dimensional array,
rz may be viewed as an n1 by n2 two-dimensional array:  rz[n2][n1].  
Likewise, depending on idim, cz may be viewed as either an n1/2+1 by 
n2 or an n1 by n2/2+1 two-dimensional array of real_complex elements.

Let n denote the transform length, either n1 or n2, depending on idim.
Because pfa2rc uses pfa2cc to do most of the work, n must be even 
and n/2 must be a valid length for pfa2cc.  The simplest way to
obtain a valid n is via n = npfar(nmin).  A more optimal n can be 
obtained with npfaro.

pfa2rc:
If idim equals 1, then n2 transforms of n1 real elements to n1/2+1 real_complex 
elements are performed; else, if idim equals 2, then n1 transforms of n2 
real elements to n2/2+1 real_complex elements are performed.

Although rz appears in the argument list as a one-dimensional array,
rz may be viewed as an n1 by n2 two-dimensional array:  rz[n2][n1].  
Likewise, depending on idim, cz may be viewed as either an n1/2+1 by 
n2 or an n1 by n2/2+1 two-dimensional array of real_complex elements.

Let n denote the transform length, either n1 or n2, depending on idim.
Because pfa2rc uses pfa2cc to do most of the work, n must be even 
and n/2 must be a valid length for pfa2cc.  The simplest way to
obtain a valid n is via n = npfar(nmin).  A more optimal n can be 
obtained with npfaro.

References:  
Temperton, C., 1985, Implementation of a self-sorting
in-place prime factor fft algorithm:  Journal of
Computational Physics, v. 58, p. 283-299.

Temperton, C., 1988, A new set of minimum-add rotated
rotated dft modules: Journal of Computational Physics,
v. 75, p. 190-198.

Press et al, 1988, Numerical Recipes in C, p. 417.

Author:  Dave Hale, Colorado School of Mines, 04/27/89
Revised by Baoniu Han to handle double precision. 12/14/98
\end{verbatim}
\pagebreak
\begin{verbatim}
CWP_Exit - exit subroutine for CWP/SU codes



\end{verbatim}
\pagebreak
\begin{verbatim}
FRANNOR - functions to generate a pseudo-random float normally distributed
		with N(0,1); i.e., with zero mean and unit variance.

frannor		return a normally distributed random float
srannor		seed random number generator for normal distribution

Function Prototypes:
float frannor (void);
void srannor (int seed);

frannor:
Input:		(none)
Returned:	normally distributed random float

srannor:
Input:
seed		different seeds yield different sequences of random numbers.

Notes:
Adapted from subroutine rnor in Kahaner,  Moler and Nash (1988)
which in turn was based on  an algorithm by 
Marsaglia and Tsang (1984).

References:
Numerical Methods and Software", D.
Kahaner, C. Moler, S. Nash, Prentice Hall, 1988.

Marsaglia G. and Tsang, W. W., 1984,
A fast, easily implemented method for sampling from decreasing or symmetric
unimodal density functions:  SIAM J. Sci. Stat. Comput., v. 5, no. 2,
p. 349-359.

Author:  Dave Hale, Colorado School of Mines, 01/21/89
\end{verbatim}
\pagebreak
\begin{verbatim}
FRANUNI - Functions to generate a pseudo-random float uniformly distributed
	on [0,1); i.e., between 0.0 (inclusive) and 1.0 (exclusive)

franuni		return a random float
sranuni		seed random number generator

Function Prototypes:
float franuni (void);
void sranuni (int seed);

franuni:
Input:		(none)
Returned:	pseudo-random float

sranuni:
seed		different seeds yield different sequences of random numbers.

Notes:
Adapted from subroutine uni in Kahaner, Moler, and Nash (1988). 
This book references a set of unpublished notes by
Marsaglia.

According to the reference, this random
number generator "passes all known tests and has a period that is ...
approximately 10^19".

References:
Numerical Methods and Software", D. Kahaner, C. Moler, S. Nash,
Prentice Hall, 1988. 

Marsaglia G., "Comments on the perfect uniform random number generator
Unpublished notes, Wash S. U.

Author:  Dave Hale, Colorado School of Mines, 12/30/89
\end{verbatim}
\pagebreak
\begin{verbatim}
HANKEL - Functions to compute discrete Hankel transforms

hankelalloc	allocate and return a pointer to a Hankel transformer
hankelfree 	free a Hankel transformer
hankel0		compute the zeroth-order Hankel transform
hankel1		compute the first-order Hankel transform

Function Prototypes:
void *hankelalloc (int nfft);
void hankelfree (void *ht);
void hankel0 (void *ht, float f[], float h[]);
void hankel1 (void *ht, float f[], float h[]);

hankelalloc:
Input:
nfft		valid length for real to complex fft  (see notes below)

Returned:
pointer to Hankel transformer

hankelfree:
Input:
ht		pointer to Hankel transformer (as returned by hankelalloc)

hankel0:
Input:
ht		pointer to Hankel transformer (as returned by hankelalloc)
f		array[nfft/2+1] to be transformed

Output:
h		array[nfft/2+1] transformed

hankel1:
Input:
ht		pointer to Hankel transformer (as returned by hankelalloc)
f		array[nfft/2+1] to be transformed

Output:
h		array[nfft/2+1] transformed

Notes:
The zeroth-order Hankel transform is defined by:

	        Infinity
	h0(k) = Integral dr r j0(k*r) f(r)
		   0

where j0 denotes the zeroth-order Bessel function.

The first-order Hankel transform is defined by:

	        Infinity
	h1(k) = Integral dr r j1(k*r) f(r)
		   0

where j1 denotes the first-order Bessel function.

The Hankel transform is its own inverse.

The Hankel transform is computed by an Abel transform followed by
a Fourier transform.

References:
Hansen, E. W., 1985, Fast Hankel transform algorithm:  IEEE Trans. on
Acoustics, Speech and Signal Processing, v. ASSP-33, n. 3, p. 666-671.
(Beware of several errors in the equations in this paper!)

Authors:  Dave Hale, Colorado School of Mines, 06/04/90
\end{verbatim}
\pagebreak
\begin{verbatim}
Hartley - routines for fast Hartley transform

srfht -  in-place FHT using the split-radix algorithm
dsrfht - in-place FHT using the split-radix algorithm (double precision)
r4fht -	in-place FHT using the radix-4 algorithm
nextpow2 - find m such that n=2^m via lookup table
nextpow4 - find m such that n=4^m via lookup table
srfht - split-radix in-place FHT

Function Prototypes:

void srfht(int *n, int *m, float *f);
void dsrfht(int *n, int *m, double *f);
void r4fht(int n, int m, float *f);
int nextpow2(int n);
int nextpow4(int n);



Notes:
srfht -  in-place FHT using the split-radix algorithm
		(single precision)
		does not include division by n
		n and m are not modified
		usage:  srfht(int *n, int *m, float *f)
			n  length of input sequence n=2^m
			m  exponent such that n=2^m
	  		f  input sequence to FHT  
		
dsrfht - in-place FHT using the split-radix algorithm
		(double precision)
		does not include division by n
		n and m are not modified
		usage:  srfht(int *n, int *m, double *f)
			n  length of input sequence n=2^m
			m  exponent such that n=2^m
	  		f  input sequence to FHT  

r4fht -	in-place FHT using the radix-4 algorithm
		does not include division by n
		usage:  r4fht(int n, int m, float *f)
			n  length of input sequence n=4^m
			m  exponent such that n=4^m
	  		f  input sequence to FHT  

nextpow2 - find m such that n=2^m via lookup table
			max(n)=2^24
		
nextpow4 - find m such that n=4^m via lookup table
			max(n)=4^12

srfht - split-radix in-place FHT
	  n	length of input sequence n=2^m
	  m	exponent such that n=2^m
 	  f	input sequence to FHT  
   Reference:
       Sorensen, H. V., Jones, D. L., Burrus, C. S, & Heideman, M. T.,
       1985, On computing the Hartley transform: IEEE Trans. Acoust.,
       Speech, and Signal Proc., v. ASSP-33, no. 4, p. 1231-1238.
			
	possible improvements:
		find optimum length for FHT (nfhto)
		use bit shift operators in r4fht
  Credits:	CENPET: Werner M. Heigl, April 2006

\end{verbatim}
\pagebreak
\begin{verbatim}
HILBERT - Compute Hilbert transform y of x

hilbert		compute the Hilbert transform

Function Prototype:
void hilbert (int n, float x[], float y[]);

Input:
n		length of x and y
x		array[n] to be Hilbert transformed

Output:
y		array[n] containing Hilbert transform of x

Notes:
The Hilbert transform is computed by convolving x with a
windowed (approximate) version of the ideal Hilbert transformer.

Author:  Dave Hale, Colorado School of Mines, 06/02/89
\end{verbatim}
\pagebreak
\begin{verbatim}
HOLBERG1D - Compute coefficients of Holberg's 1st derivative filter

holberg1d	comput the coefficients of Holberg's 1st derivative filter

Function Prototype:
void holbergd1 (float e, int n, float d[]);

Input:
e		maximum relative error in group velocity
n		number of coefficients in filter (must be 2, 4, 6, or 8)

Output:
d		array[n] of coefficients

Notes:
Coefficients are output in a form suitable for convolution.  The
derivative is centered halfway between coefficients d[n/2-1] and d[n/2].

Coefficients are computed via the power series method of Kindelan et al.,
1990, On the construction and efficiency of staggered numerical
differentiators for the wave equation:  Geophysics 55, 107-110.
See also, Holberg, 1987, Computational aspects of the choice of
operator and sampling interval for numerical differentiation in
large-scale simulation of wave phenomena:  Geophys. Prosp., 35, 629-655

Reference:
Kindelan et al., 1990, 
On the construction and efficiency of staggered numerical
differentiators for the wave equation:  Geophysics 55, 107-110.

See also, Holberg, 1987, Computational aspects of the choice of
operator and sampling interval for numerical differentiation in
large-scale simulation of wave phenomena:  Geophys. Prosp., 35, 629-655

Author:  Dave Hale, Colorado School of Mines, 06/06/91
\end{verbatim}
\pagebreak
\begin{verbatim}
INTCUB - evaluate y(x), y'(x), y''(x), ... via piecewise cubic interpolation

intcub		evaluate y(x), y'(x), y''(x), ... via piecewise spline
			interpolation

Function Prototype:
void intcub (int ideriv, int nin, float xin[], float ydin[][4],

Input:
ideriv		=0 if y(x) desired; =1 if y'(x) desired, ...
nin		length of xin and ydin arrays
xin		array[nin] of monotonically increasing or decreasing x values
ydin		array[nin][4] of y(x), y'(x), y''(x), and y'''(x)
nout		length of xout and yout arrays
xout		array[nout] of x values at which to evaluate y(x), y'(x), ...

Output:
yout		array[nout] of y(x), y'(x), ... values

Notes:
xin values must be monotonically increasing or decreasing.

Extrapolation of the function y(x) for xout values outside the range
spanned by the xin values is performed using the derivatives in 
ydin[0][0:3] or ydin[nin-1][0:3], depending on whether xout is closest
to xin[0] or xin[nin-1], respectively.

Reference:
See: Greville, T. N., Theory and Application of Spline Functions, Academic Press
for a general discussion of spline functions.

Author:  Dave Hale, Colorado School of Mines, 06/02/89
\end{verbatim}
\pagebreak
\begin{verbatim}
INTL2B - bilinear interpolation of a 2-D array of bytes

intl2b		bilinear interpolation of a 2-D array of bytes

Function Prototype:
void intl2b (int nxin, float dxin, float fxin,
	int nyin, float dyin, float fyin, unsigned char *zin,
	int nxout, float dxout, float fxout,
	int nyout, float dyout, float fyout, unsigned char *zout);

Input:
nxin		number of x samples input (fast dimension of zin)
dxin		x sampling interval input
fxin		first x sample input
nyin		number of y samples input (slow dimension of zin)
dyin		y sampling interval input
fyin		first y sample input
zin		array[nyin][nxin] of input samples (see notes)
nxout		number of x samples output (fast dimension of zout)
dxout		x sampling interval output
fxout		first x sample output
nyout		number of y samples output (slow dimension of zout)
dyout		y sampling interval output
fyout		first y sample output

Output:
zout		array[nyout][nxout] of output samples (see notes)

Notes:
The arrays zin and zout must passed as pointers to the first element of
a two-dimensional contiguous array of unsigned char values.

Constant extrapolation of zin is used to compute zout for
output x and y outside the range of input x and y.
 
For efficiency, this function builds a table of interpolation
coefficents pre-multiplied by byte values.  To keep the table
reasonably small, the interpolation does not distinguish
between x and y values that differ by less than dxin/ICMAX
and dyin/ICMAX, respectively, where ICMAX is a parameter
defined above.

Author:  Dave Hale, Colorado School of Mines, c. 1989-1991.
\end{verbatim}
\pagebreak
\begin{verbatim}
INTLINC - evaluate complex y(x) via linear interpolation of y(x[0]), y(x[1]), ...

Function Prototype:
void intlinc (int nin, float xin[], complex yin[], complex yinl, complex yinr,
	int nout, float xout[], complex yout[]);

Input:
nin		length of xin and yin arrays
xin		array[nin] of monotonically increasing or decreasing x values
yin		array[nin] of input y(x) values
yinl		value used to extraplate y(x) to left of input yin values
yinr		value used to extraplate y(x) to right of input yin values
nout		length of xout and yout arrays
xout		array[nout] of x values at which to evaluate y(x)

Output:
yout		array[nout] of linearly interpolated y(x) values

Notes:
xin values must be monotonically increasing or decreasing.

Extrapolation of the function y(x) for xout values outside the range
spanned by the xin values in performed as follows:

	For monotonically increasing xin values,
		yout=yinl if xout<xin[0], and yout=yinr if xout>xin[nin-1].

	For monotonically decreasing xin values, 
		yout=yinl if xout>xin[0], and yout=yinr if xout<xin[nin-1].

If nin==1, then the monotonically increasing case is used.
/

\end{verbatim}
\pagebreak
\begin{verbatim}
INTLIN - evaluate y(x) via linear interpolation of y(x[0]), y(x[1]), ...

intlin		evaluate y(x) via linear interpolation of y(x[0]), y(x[1]), ...

Function Prototype:
void intlin (int nin, float xin[], float yin[], float yinl, float yinr,
	int nout, float xout[], float yout[]);

Input:
nin		length of xin and yin arrays
xin		array[nin] of monotonically increasing or decreasing x values
yin		array[nin] of input y(x) values
yinl		value used to extraplate y(x) to left of input yin values
yinr		value used to extraplate y(x) to right of input yin values
nout		length of xout and yout arrays
xout		array[nout] of x values at which to evaluate y(x)

Output:
yout		array[nout] of linearly interpolated y(x) values

Notes:
xin values must be monotonically increasing or decreasing.

Extrapolation of the function y(x) for xout values outside the range
spanned by the xin values in performed as follows:

	For monotonically increasing xin values,
		yout=yinl if xout<xin[0], and yout=yinr if xout>xin[nin-1].

	For monotonically decreasing xin values, 
		yout=yinl if xout>xin[0], and yout=yinr if xout<xin[nin-1].

If nin==1, then the monotonically increasing case is used.

Author:  Dave Hale, Colorado School of Mines, 06/02/89
\end{verbatim}
\pagebreak
\begin{verbatim}
INTLIRR2B - bilinear interpolation of a 2-D array of bytes

intlirr2b	bilinear interpolation of a 2-D array of bytes
                where x spacings are irregular

Function Prototype:
void intlirr2b (int nxin, float *xin,
	int nyin, float dyin, float fyin, unsigned char *zin,
	int nxout, float dxout, float fxout,
	int nyout, float dyout, float fyout, unsigned char *zout);

Input:
nxin		number of x samples input (fast dimension of zin)
xin		array of (irregular) input x-coordinates
nyin		number of y samples input (slow dimension of zin)
dyin		y sampling interval input
fyin		first y sample input
zin		array[nyin][nxin] of input samples (see notes)
nxout		number of x samples output (fast dimension of zout)
dxout		x sampling interval output
fxout		first x sample output
nyout		number of y samples output (slow dimension of zout)
dyout		y sampling interval output
fyout		first y sample output

Output:
zout		array[nyout][nxout] of output samples (see notes)

Notes:
The arrays zin and zout must passed as pointers to the first element of
a two-dimensional contiguous array of unsigned char values.

Constant extrapolation of zin is used to compute zout for
output x and y outside the range of input x and y.
 
For efficiency, this function builds a table of interpolation
coefficents pre-multiplied by byte values.  To keep the table
reasonably small, the interpolation does not distinguish
between x and y values that differ by less than dxin/ICMAX
and dyin/ICMAX, respectively, where ICMAX is a parameter
defined above.

Author:  Dave Hale, Colorado School of Mines, c. 1989-1991.
\end{verbatim}
\pagebreak
\begin{verbatim}
INTSINC8 - Functions to interpolate uniformly-sampled data via 8-coeff. sinc
		approximations:

ints8c	interpolation of a uniformly-sampled complex function y(x) via an
	 8-coefficient sinc approximation.
ints8r	Interpolation of a uniformly-sampled real function y(x) via a
		table of 8-coefficient sinc approximations

Function Prototypes:
void ints8c (int nxin, float dxin, float fxin, complex yin[], 
	complex yinl, complex yinr, int nxout, float xout[], complex yout[]);
void ints8r (int nxin, float dxin, float fxin, float yin[], 
	float yinl, float yinr, int nxout, float xout[], float yout[]);

Input:
nxin		number of x values at which y(x) is input
dxin		x sampling interval for input y(x)
fxin		x value of first sample input
yin		array[nxin] of input y(x) values:  yin[0] = y(fxin), etc.
yinl		value used to extrapolate yin values to left of yin[0]
yinr		value used to extrapolate yin values to right of yin[nxin-1]
nxout		number of x values a which y(x) is output
xout		array[nxout] of x values at which y(x) is output

Output:
yout		array[nxout] of output y(x):  yout[0] = y(xout[0]), etc.

Notes:
Because extrapolation of the input function y(x) is defined by the
left and right values yinl and yinr, the xout values are not restricted
to lie within the range of sample locations defined by nxin, dxin, and
fxin.

The maximum error for frequiencies less than 0.6 nyquist is less than
one percent.

Author:  Dave Hale, Colorado School of Mines, 06/02/89
\end{verbatim}
\pagebreak
\begin{verbatim}
INTTABLE8 -  Interpolation of a uniformly-sampled complex function y(x)
		via a table of 8-coefficient interpolators

intt8c	interpolation of a uniformly-sampled complex function y(x)
		via a table of 8-coefficient interpolators
intt8r	interpolation of a uniformly-sampled real function y(x) via a
		table of 8-coefficient interpolators

Function Prototype:
void intt8c (int ntable, float table[][8],
	int nxin, float dxin, float fxin, complex yin[], 
	complex yinl, complex yinr, int nxout, float xout[], complex yout[]);
void intt8r (int ntable, float table[][8],
	int nxin, float dxin, float fxin, float yin[], 
	float yinl, float yinr, int nxout, float xout[], float yout[]);

Input:
ntable		number of tabulated interpolation operators; ntable>=2
table		array of tabulated 8-point interpolation operators
nxin		number of x values at which y(x) is input
dxin		x sampling interval for input y(x)
fxin		x value of first sample input
yin		array of input y(x) values:  yin[0] = y(fxin), etc.
yinl		value used to extrapolate yin values to left of yin[0]
yinr		value used to extrapolate yin values to right of yin[nxin-1]
nxout		number of x values a which y(x) is output
xout		array of x values at which y(x) is output

Output:
yout		array of output y(x) values:  yout[0] = y(xout[0]), etc.

NOTES:
ntable must not be less than 2.

The table of interpolation operators must be as follows:

Let d be the distance, expressed as a fraction of dxin, from a particular
xout value to the sampled location xin just to the left of xout.  Then,
for d = 0.0,

table[0][0:7] = 0.0, 0.0, 0.0, 1.0, 0.0, 0.0, 0.0, 0.0

are the weights applied to the 8 input samples nearest xout.
Likewise, for d = 1.0,

table[ntable-1][0:7] = 0.0, 0.0, 0.0, 0.0, 1.0, 0.0, 0.0, 0.0

are the weights applied to the 8 input samples nearest xout.  In general,
for d = (float)itable/(float)(ntable-1), table[itable][0:7] are the
weights applied to the 8 input samples nearest xout.  If the actual sample
distance d does not exactly equal one of the values for which interpolators
are tabulated, then the interpolator corresponding to the nearest value of
d is used.

Because extrapolation of the input function y(x) is defined by the left
and right values yinl and yinr, the xout values are not restricted to lie
within the range of sample locations defined by nxin, dxin, and fxin.

AUTHOR:  Dave Hale, Colorado School of Mines, 06/02/89
\end{verbatim}
\pagebreak
\begin{verbatim}
LINEAR_REGRESSION - Compute linear regression of (y1,y2,...,yn) against 
(x1,x2,...,xn)
Function Prototype:
void linear_regression(float *y, float *x, int n, float coeff[4]);
Input:
y		array of y values
x		array of x values
n		length of y and x arrays
Output:
coeff[4] where:

coeff[0]	slope of best fit line
coeff[1]	intercept of best fit line
coeff[2]	correlation coefficient of fit (1 = perfect) [dimensionless]
coeff[3]	standard error of fit (0 = perfect) [dimensions of y]
Notes: 

y(x) 
    |      *  .    fit is  y(x) = a x + b
    |       .          
    |     .  *
    | * .    
    | . *         
     ------------------- x
     
         n Sum[x*y] - Sum[x]*Sum[y]
     a = --------------------------
         n Sum[x*x] - Sum[x]*Sum[x]
         
         Sum[y] - a*Sum[x]
     b = -----------------
                n
    
     cc = std definition
     
     se = std definition
    
Author:  Chris Liner, UTulsa, 11/16/03
                
\end{verbatim}
\pagebreak
\begin{verbatim}
maxmin - subroutines that pertain to maximum and minimum values
min_index - find the value of the index where an array is a minimum

function prototypes:
int max_index(int n, float *a,int inc);
int min_index(int n, float *a,int inc);

Author: Balasz Nemeth: Potash Corporation, given to CWP in 2008
\end{verbatim}
\pagebreak
\begin{verbatim}
MKDIFF - Make an n-th order DIFFerentiator via Taylor's series method.

mkdiff		make discrete Taylor series approximation to n'th derivative.

Function Prototype:
void mkdiff (int n, float a, float h, int l, int m, float d[]);

Input:
n		order of desired derivative (n>=0 && n<=m-l is required)
a		fractional distance from integer sampling index (see notes)
h		sampling interval
l		sampling index of first coefficient (see notes below)
m		sampling index of last coefficient (see notes below)

Output:
d		array[m-l+1] of coefficients for n'th order differentiator

Notes:
The abscissae x of a sampled function f(x) can always be expressed as
x = (j+a)*h, where j is an integer, a is a fraction, and h is the
sampling interval.  To approximate the n'th order derivative fn(x)
of the sampled function f(x) at x = (j+a)*h, use the m-l+1 coefficients
in the output array d[] as follows:

	fn(x) = d[0]*f(j-l) + d[1]*f(j-l-1) +...+ d[m-l]*f(j-m)

i.e., convolve the coefficients in d with the samples in f.

m-l+1 (the number of coefficients) must not be greater than the
NCMAX parameter specified below.

For best approximations,
when n is even, use a = 0.0, l = -m
when n is odd, use a = 0.5, l = -m-1

Author:  Dave Hale, Colorado School of Mines, 06/02/89
\end{verbatim}
\pagebreak
\begin{verbatim}
MKHDIFF - Compute filter approximating the bandlimited HalF-DIFFerentiator.

mkhdiff - Compute filter approximating the bandlimited half-differentiator.

Function Prototype:
void mkhdiff (float h, int l, float d[]);

Input:
h		sampling interval
l		half-length of half-differentiator (length = 1+2*l is odd)

Output:
d		array[1+2*l] of coefficients for half-differentiator

Notes:
The half-differentiator is defined by

				  pi
    d[l+j] = sqrt(1/h)/(2pi) * integral dw sqrt(-iw)*exp(-iwj)
				 -pi

				  pi
           = sqrt(2/h)/(2pi) * integral dw sqrt(w)*(cos(wj)-sin(wj))
				  0 

    for j = -l, -l+1, ... , l.

An alternative definition is that f'(j) = d(j)*d(j)*f(j), where
f'(j) denotes the derivative of a sampled function f(j) and *
denotes a convolution sum.

The half-derivative g(j) of f(j) may be computed by the following sum:

	g(j) = d[0]*f(j+l) + d[1]*f(j+l-1) + ... + d[2*l]*f(j-l)

The integral over frequency is evaluated numerically using Simpson's
method.  Although the Filon method of numerical integration is more
appropriate for this integral, the truncation of d[l+j] for |j| > l
is probably the greatest source of error.  In any case, d[l+j] is 
cosine-tapered to reduce these truncation errors.

Author:  Dave Hale, Colorado School of Mines, 06/02/89
\end{verbatim}
\pagebreak
\begin{verbatim}
MKSINC - Compute least-squares optimal sinc interpolation coefficients.

mksinc		Compute least-squares optimal sinc interpolation coefficients.

Function Prototype:
void mksinc (float d, int lsinc, float sinc[]);

Input:
d		fractional distance to interpolation point; 0.0<=d<=1.0
lsinc		length of sinc approximation; lsinc%2==0 and lsinc<=20

Output:
sinc		array[lsinc] containing interpolation coefficients

Notes:
The coefficients are a least-squares-best approximation to the ideal
sinc function for frequencies from zero up to a computed maximum
frequency.  For a given interpolator length, lsinc, mksinc computes
the maximum frequency, fmax (expressed as a fraction of the nyquist
frequency), using the following empirically derived relation (from
a Western Geophysical Technical Memorandum by Ken Larner):

	fmax = min(0.066+0.265*log(lsinc),1.0)

Note that fmax increases as lsinc increases, up to a maximum of 1.0.
Use the coefficients to interpolate a uniformly-sampled function y(i) 
as follows:

            lsinc-1
    y(i+d) =  sum  sinc[j]*y(i+j+1-lsinc/2)
              j=0

Interpolation error is greatest for d=0.5, but for frequencies less
than fmax, the error should be less than 1.0 percent.

Author:  Dave Hale, Colorado School of Mines, 06/02/89
\end{verbatim}
\pagebreak
\begin{verbatim}
MNEWT - Solve non-linear system of equations f(x) = 0 via Newton's method

mnewt	Solve non-linear system of equations f(x) = 0 via Newton's method

Function Prototype:
int mnewt (int maxiter, float ftol, float dxtol, int n, float *x, void *aux,
	void (*fdfdx)(int n, float *x, float *f, float **dfdx, void *aux));

Input:
maxiter		maximum number of iterations
ftol		converged when sum of absolute values of f less than ftol
dxtol		converged when sum of absolute values of dx less than dxtol
n		number of equations
x		array[n] containing initial guess of solution
aux		pointer to auxiliary parameters to be passed to fdfdx
fdfdx		pointer to function to evaluate f(x) and f'(x)

Output:
x		array[n] containing solution

Returned:	number of iterations; -1 if failed to converge in maxiter

Input to the user-supplied function fdfdx:
n		number of equations
x		array[n] of x0, x1, ...
aux		pointer to auxiliary variables required by fdfdx.

Output from the user-supplied function fdfdx:
f		array[n] of f0(x), f1(x), ...
dfdx		array[n][n] of f'(x);  dfdx[j][i] = dfi/dxj

Author:  Dave Hale, Colorado School of Mines, 06/06/91
\end{verbatim}
\pagebreak
\begin{verbatim}
ORTHPOLYNOMIALS - compute ORTHogonal POLYNOMIALS

hermite_n_polynomial:  Compute n-th order generalized Hermite polynomial.

hermite_n_polynomial: 

Input:
h0		array of size nt holding H_{n-1}
h1		array of size nt holding H_{n}
t		array of size nt holding time vector
nt		size of arrays, no. of samples
n		order of polynomial
sigma		variance
Output:
h		array of size nt holding H_{n+1}
Notes:	Note that n in the function call is the order of the derivative and
        j in the code below is the n in the recurrence relation
Copyright (c) 2007 by the Society of Exploration Geophysicists.
For more information, go to http://software.seg.org/2007/0004 .
You must read and accept usage terms at:
http://software.seg.org/disclaimer.txt before use.
Author: Werner M. Heigl, Apache Corporation, E&P Technology, December 2006



include "cwp.h"



void
hermite_n_polynomial(double *h, double *h0, double *h1, double *t, int nt, int n, double sigma)
  hermite_n_polynomial:  Compute n-th order generalized Hermite polynomial.
Input:
h0		array of size nt holding H_{n-1}
h1		array of size nt holding H_{n}
t		array of size nt holding time vector
nt		size of arrays, no. of samples
n		order of polynomial
sigma		variance
Output:
h		array of size nt holding H_{n+1}
Note:	Note that n in the function call is the order of the derivative and
        j in the code below is the n in the recurrence relation

Copyright (c) 2007 by the Society of Exploration Geophysicists.
For more information, go to http://software.seg.org/2007/0004 .
You must read and accept usage terms at:
http://software.seg.org/disclaimer.txt before use.
Author: Werner M. Heigl, Apache Corporation, E&P Technology, December 2006
{
	int i;		/* loop variable
	int j=1;	/* recurrence counter

	/* as long as necessary use recurrence relation
	do {
		/* current instance of recurrence relation
		for (i = 0; i < nt; ++i) 		
			h[i] = ( t[i] * h1[i] - j * h0[i] ) / sigma;
		
		/* update inputs to recurrence relation
		memcpy((void *) h0, (const void *) h1, DSIZE * nt);
		memcpy((void *) h1, (const void *) h , DSIZE * nt);
		
		/* update counter
		++j;
		
	} while (j < n);
	
}
\end{verbatim}
\pagebreak
\begin{verbatim}
PFAFFT - Functions to perform Prime Factor (PFA) FFT's, in place

npfa		return valid n for complex-to-complex PFA
npfar		return valid n for real-to-complex/complex-to-real PFA
npfao		return optimal n for complex-to-complex PFA
npfaro		return optimal n for real-to-complex/complex-to-real PFA
pfacc		1D PFA complex to complex
pfacr		1D PFA complex to real
pfarc		1D PFA real to complex
pfamcc		multiple PFA complex to real
pfa2cc		2D PFA complex to complex
pfa2cr		2D PFA complex to real
pfa2rc		2D PFA real to complex

Function Prototypes:
int npfa (int nmin);
int npfao (int nmin, int nmax);
int npfar (int nmin);
int npfaro (int nmin, int nmax);
void pfacc (int isign, int n, complex z[]);
void pfacr (int isign, int n, complex cz[], float rz[]);
void pfarc (int isign, int n, float rz[], complex cz[]);
void pfamcc (int isign, int n, int nt, int k, int kt, complex z[]);
void pfa2cc (int isign, int idim, int n1, int n2, complex z[]);
void pfa2cr (int isign, int idim, int n1, int n2, complex cz[], float rz[]);
void pfa2rc (int isign, int idim, int n1, int n2, float rz[], complex cz[]);

npfa:
Input:
nmin		lower bound on returned value (see notes below)

Returned:	valid n for prime factor fft

npfao
Input:
nmin		lower bound on returned value (see notes below)
nmax		desired (but not guaranteed) upper bound on returned value

Returned:	valid n for prime factor fft

npfar
Input:
nmin		lower bound on returned value

Returned:	valid n for real-to-complex/complex-to-real prime factor fft

npfaro:
Input:
nmin		lower bound on returned value
nmax		desired (but not guaranteed) upper bound on returned value

Returned:	valid n for real-to-complex/complex-to-real prime factor fft

pfacc:
Input:
isign		sign of isign is the sign of exponent in fourier kernel
n		length of transform (see notes below)
z		array[n] of complex numbers to be transformed in place

Output:
z		array[n] of complex numbers transformed

pfacr:
Input:
isign       sign of isign is the sign of exponent in fourier kernel
n           length of transform (see notes below)
cz          array[n/2+1] of complex values (may be equivalenced to rz)

Output:
rz          array[n] of real values (may be equivalenced to cz)

pfarc:
Input:
isign       sign of isign is the sign of exponent in fourier kernel
n           length of transform; must be even (see notes below)
rz          array[n] of real values (may be equivalenced to cz)

Output:
cz          array[n/2+1] of complex values (may be equivalenced to rz)

pfamcc:
Input:
isign       	sign of isign is the sign of exponent in fourier kernel
n           	number of complex elements per transform (see notes below)
nt          	number of transforms
k           	stride in complex elements within transforms
kt          	stride in complex elements between transforms
z           	array of complex elements to be transformed in place

Output:
z		array of complex elements transformed

pfa2cc:
Input:
isign       	sign of isign is the sign of exponent in fourier kernel
idim        	dimension to transform, either 1 or 2 (see notes)
n1          	1st (fast) dimension of array to be transformed (see notes)
n2          	2nd (slow) dimension of array to be transformed (see notes)
z           	array[n2][n1] of complex elements to be transformed in place

Output:
z		array[n2][n1] of complex elements transformed

pfa2cr:
Input:
isign       sign of isign is the sign of exponent in fourier kernel
idim        dimension to transform, which must be either 1 or 2 (see notes)
n1          1st (fast) dimension of array to be transformed (see notes)
n2          2nd (slow) dimension of array to be transformed (see notes)
cz          array of complex values (may be equivalenced to rz)

Output:
rz          array of real values (may be equivalenced to cz)

pfa2rc:
Input:
isign       sign of isign is the sign of exponent in fourier kernel
idim        dimension to transform, which must be either 1 or 2 (see notes)
n1          1st (fast) dimension of array to be transformed (see notes)
n2          2nd (slow) dimension of array to be transformed (see notes)
rz          array of real values (may be equivalenced to cz)

Output:
cz          array of complex values (may be equivalenced to rz)

Notes:
Table of valid n and cost for prime factor fft.  For each n, cost
was estimated to be the inverse of the number of ffts done in 1 sec
on an IBM RISC System/6000 Model 320H, by Dave Hale, 08/04/91.
(Redone by Jack Cohen for 15 sec to rebuild NTAB table on advice of
David and Gregory Chudnovsky, 05/03/94).
Cost estimates are least accurate for very small n.  An alternative method
for estimating cost would be to count multiplies and adds, but this
method fails to account for the overlapping of multiplies and adds
that is possible on some computers, such as the IBM RS/6000 family.

npfa:
The returned n will be composed of mutually prime factors from
the set {2,3,4,5,7,8,9,11,13,16}.  Because n cannot exceed
720720 = 5*7*9*11*13*16, 720720 is returned if nmin exceeds 720720.

npfao:
The returned n will be composed of mutually prime factors from
the set {2,3,4,5,7,8,9,11,13,16}.  Because n cannot exceed
720720 = 5*7*9*11*13*16, 720720 is returned if nmin exceeds 720720.
If nmin does not exceed 720720, then the returned n will not be 
less than nmin.  The optimal n is chosen to minimize the estimated
cost of performing the fft, while satisfying the constraint, if
possible, that n not exceed nmax.

npfar and npfaro:
Current implemenations of real-to-complex and complex-to-real prime 
factor ffts require that the transform length n be even and that n/2 
be a valid length for a complex-to-complex prime factor fft.  The 
value returned by npfar satisfies these conditions.  Also, see notes 
for npfa.

pfacc:
n must be factorable into mutually prime factors taken 
from the set {2,3,4,5,7,8,9,11,13,16}.  in other words,
	n = 2**p * 3**q * 5**r * 7**s * 11**t * 13**u
where
	0 <= p <= 4,  0 <= q <= 2,  0 <= r,s,t,u <= 1
is required for pfa to yield meaningful results.  this
restriction implies that n is restricted to the range
	1 <= n <= 720720 (= 5*7*9*11*13*16)

pfacr:
Because pfacr uses pfacc to do most of the work, n must be even 
and n/2 must be a valid length for pfacc.  The simplest way to
obtain a valid n is via n = npfar(nmin).  A more optimal n can be 
obtained with npfaro.

pfarc:
Because pfarc uses pfacc to do most of the work, n must be even 
and n/2 must be a valid length for pfacc.  The simplest way to
obtain a valid n is via n = npfar(nmin).  A more optimal n can be 
obtained with npfaro.

pfamcc:
To perform a two-dimensional transform of an n1 by n2 complex array 
(assuming that both n1 and n2 are valid "n"), stored with n1 fast 
and n2 slow:
    pfamcc(isign,n1,n2,1,n1,z); (to transform 1st dimension)
    pfamcc(isign,n2,n1,n1,1,z); (to transform 2nd dimension)

pfa2cc:
Only one (either the 1st or 2nd) dimension of the 2-D array is transformed.

If idim equals 1, then n2 transforms of n1 complex elements are performed; 
else, if idim equals 2, then n1 transforms of n2 complex elements are 
performed.

Although z appears in the argument list as a one-dimensional array,
z may be viewed as an n1 by n2 two-dimensional array:  z[n2][n1].

Valid n is computed via the "np" subroutines.

To perform a two-dimensional transform of an n1 by n2 complex array 
(assuming that both n1 and n2 are valid "n"), stored with n1 fast 
and n2 slow:  pfa2cc(isign,1,n1,n2,z);  pfa2cc(isign,2,n1,n2,z);

pfa2cr:
If idim equals 1, then n2 transforms of n1/2+1 complex elements to n1 real 
elements are performed; else, if idim equals 2, then n1 transforms of n2/2+1 
complex elements to n2 real elements are performed.

Although rz appears in the argument list as a one-dimensional array,
rz may be viewed as an n1 by n2 two-dimensional array:  rz[n2][n1].  
Likewise, depending on idim, cz may be viewed as either an n1/2+1 by 
n2 or an n1 by n2/2+1 two-dimensional array of complex elements.

Let n denote the transform length, either n1 or n2, depending on idim.
Because pfa2rc uses pfa2cc to do most of the work, n must be even 
and n/2 must be a valid length for pfa2cc.  The simplest way to
obtain a valid n is via n = npfar(nmin).  A more optimal n can be 
obtained with npfaro.

pfa2rc:
If idim equals 1, then n2 transforms of n1 real elements to n1/2+1 complex 
elements are performed; else, if idim equals 2, then n1 transforms of n2 
real elements to n2/2+1 complex elements are performed.

Although rz appears in the argument list as a one-dimensional array,
rz may be viewed as an n1 by n2 two-dimensional array:  rz[n2][n1].  
Likewise, depending on idim, cz may be viewed as either an n1/2+1 by 
n2 or an n1 by n2/2+1 two-dimensional array of complex elements.

Let n denote the transform length, either n1 or n2, depending on idim.
Because pfa2rc uses pfa2cc to do most of the work, n must be even 
and n/2 must be a valid length for pfa2cc.  The simplest way to
obtain a valid n is via n = npfar(nmin).  A more optimal n can be 
obtained with npfaro.

References:  
Temperton, C., 1985, Implementation of a self-sorting
in-place prime factor fft algorithm:  Journal of
Computational Physics, v. 58, p. 283-299.

Temperton, C., 1988, A new set of minimum-add rotated
rotated dft modules: Journal of Computational Physics,
v. 75, p. 190-198.

Differ significantly from Press et al, 1988, Numerical Recipes in C, p. 417.

Author:  Dave Hale, Colorado School of Mines, 04/27/89
\end{verbatim}
\pagebreak
\begin{verbatim}
POLAR - Functions to map data in rectangular coordinates to polar and vise versa

recttopolar	convert a function p(x,y) to a function q(a,r)
polartorect	convert a function q(a,r) to a function p(x,y)

Function Prototypes:
void recttopolar ( int nx, float dx, float fx, int ny, float dy, float fy,
			float **p, int na, float da, float fa, int nr,
			float dr, float fr, float **q);
void polartorect ( int na, float da, float fa, int nr, float dr, float fr,
			float **q, int nx, float dx, float fx, int ny,
			float dy, float fy, float **p)

recttopolar:
Input:
nx		number of x samples
dx		x sampling interval
fx		first x sample
ny		number of y samples
dy		y sampling interval
fy		first y sample
p		array[ny][nx] containing samples of p(x,y)
na		number of a samples
da		a sampling interval
fa		first a sample
nr		number of r samples
dr		r sampling interval
fr		first r sample

Output:
q		array[nr][na] containing samples of q(a,r)

polartorect:
Input:
na		number of a samples
da		a sampling interval
fa		first a sample
nr		number of r samples
dr		r sampling interval
fr		first r sample
nx		number of x samples
dx		x sampling interval
fx		first x sample
ny		number of y samples
dy		y sampling interval
fy		first y sample
q		array[nr][na] containing samples of q(a,r)

Output:
p		array[ny][nx] containing samples of p(x,y)

Notes:
The polar angle a is measured in radians, 
x = r*cos(a) and y = r*sin(a).

recttopolar:
Linear extrapolation is used to determine the value of p(x,y) for
x and y coordinates not in the range corresponding to nx, dx, ....

polartorect:
Linear extrapolation is used to determine the value of q(a,r) for
a and r coordinates not in the range corresponding to na, da, ....

Author:  Dave Hale, Colorado School of Mines, 06/15/90
\end{verbatim}
\pagebreak
\begin{verbatim}
PRINTERPLOT - Functions to make a printer plot of a 1-dimensional array

pp1d		printer plot of 1d array 
pplot1		printer plot of 1d array 

Function Prototypes:
void pp1d (FILE *fp, char *title, int lx, int ifx, float x[]);
void pplot1 (FILE *fp, char *title, int nx, float ax[]);

pp1d:
Input:
fp		file pointer for output (e.g., stdout, stderr, etc.)
title		title of plot
lx		length of x
ifx		index of first x
x		array[lx] to be plotted

pplot1:
Input:
fp		file pointer for printed output (e.g., stdout, stderr, etc.)
title		title of plot
nx		number of x values to be plotted
ax		array[nx] of x values

Notes:
Are two subroutines to do this really necessary?

Author:  Dave Hale, Colorado School of Mines, 06/02/89
\end{verbatim}
\pagebreak
\begin{verbatim}
QUEST - Functions to ESTimate Quantiles:

quest		returns estimate - use when entire array is in memory
questalloc	returns quantile estimator - use before questupdate
questupdate	updates and returns current estimate - use for large
		numbers of floats, too big to fit in memory at one time
questfree	frees quantile estimator

quest:
Input:
p		quantile to be estimated (0.0<=p<=1.0 is required)
n		number of samples in array x (n>=5 is required)
x		array[n] of floats

Returned:	the estimate of a specified quantile

questalloc:
Input:
p		quantile to be estimated (0.0<=p<=1.0 is required)
n		number of samples in array x (n>=5 is required)
x		array[n] of floats

Returned:	pointer to a quantile estimator

questupdate:
Input:
q		pointer to quantile estimator (as returned by questalloc)
n		number of samples in array x
x		array[n] of floats

Returned:	quantile estimate

questfree:
q		pointer to quantile estimator (as returned by questalloc)

Notes:
quest:
The estimate should approach the sample quantile in the limit of large n.

The estimate is most accurate for cumulative distribution functions
that are smooth in the neighborhood of the quantile specified by p.

This function is an implementation of the algorithm published by
Jain and Chlamtac (1985).

questalloc:
This function must be called before calling function questupdate.
See also notes in questupdate.

questupdate:
The estimate should approach the sample quantile in the limit of large n.

The estimate is most accurate for cumulative distribution functions
that are smooth in the neighborhood of the quantile specified by p.

This function is an implementation of the algorithm published by
Jain, R. and Chlamtac, I., 1985, The PP algorithm for dynamic
calculation of quantiles and histograms without storing observations:
Comm. ACM, v. 28, n. 10.

Reference:
Jain, R. and Chlamtac, I., 1985, The PP algorithm for dynamic
calculation of quantiles and histograms without storing observations:
Comm. ACM, v. 28, n. 10.

Author:  Dave Hale, Colorado School of Mines, 05/07/89
\end{verbatim}
\pagebreak
\begin{verbatim}
RESSINC8 - Functions to resample uniformly-sampled data  via 8-coefficient sinc
		 approximations:

ress8c	resample a uniformly-sampled complex function via 8-coeff. sinc approx.
ress8r	resample a uniformly-sampled real function via 8-coeff. sinc approx.

Function Prototypes:
void ress8r (int nxin, float dxin, float fxin, float yin[], 
	float yinl, float yinr, 
	int nxout, float dxout, float fxout, float yout[]);
void ress8c (int nxin, float dxin, float fxin, complex yin[], 
	complex yinl, complex yinr, 
	int nxout, float dxout, float fxout, complex yout[]);

Input:
nxin		number of x values at which y(x) is input
dxin		x sampling interval for input y(x)
fxin		x value of first sample input
yin		array[nxin] of input y(x) values:  yin[0] = y(fxin), etc.
yinl		value used to extrapolate yin values to left of yin[0]
yinr		value used to extrapolate yin values to right of yin[nxin-1]
nxout		number of x values at which y(x) is output
dxout		x sampling interval for output y(x)
fxout		x value of first sample output

Output:
yout		array[nxout] of output y(x) values:  yout[0] = y(fxout), etc.

Notes:
Because extrapolation of the input function y(x) is defined by the
left and right values yinl and yinr, the output x values defined
by nxout, dxout, and fxout are not restricted to lie within the range 
of input x values defined by nxin, dxin, and fxin.
 
The maximum error for frequiencies less than 0.6 nyquist is 
less than one percent.

Author:  Dave Hale, Colorado School of Mines, 06/06/90
\end{verbatim}
\pagebreak
\begin{verbatim}
RFWTVA - Rasterize a Float array as Wiggle-Trace-Variable-Area.

rfwtva	rasterize a float array as wiggle-trace-variable-area.

Function Prototype:
void rfwtva (int n, float z[], float zmin, float zmax, float zbase,
	int yzmin, int yzmax, int xfirst, int xlast,
	int wiggle, int nbpr, unsigned char *bits, int endian);

Input:
n		number of samples in array to rasterize
z		array[n] to rasterize
zmin		z values below zmin will be clipped
zmax		z values above zmax will be clipped
zbase		z values between zbase and zmax will be filled (see notes)
yzmin		horizontal raster coordinate corresponding to zmin
yzmax		horizontal raster coordinate corresponding to zmax
xfirst		vertical raster coordinate of z[0] (see notes)
xlast		vertical raster coordinate of z[n-1] (see notes)
wiggle		=0 for no wiggle (VA only); =1 for wiggle (with VA)
		wiggle 2<=wiggle<=5 for solid/grey coloring of VA option
                shade of grey: wiggle=2 light grey, wiggle=5 black
nbpr		number of bytes per row of bits
bits		pointer to first (top,left) byte in image
endian		byte order  =1 big endian  =0 little endian 

Output:
bits		pointer to first (top,left) byte in image

Notes:
The raster coordinate of the (top,left) bit in the image is (0,0).
In other words, x increases downward and y increases to the right.
Raster scan lines run from left to right, and from top to bottom.
Therefore, xfirst, xlast, yzmin, and yzmax should not be less than 0.
Likewise, yzmin and yzmax should not be greater than nbpr*8-1, and 
care should be taken to ensure that xfirst and xlast do not cause bits 
to be set outside (off the bottom) of the image. 

Variable area fill is performed on the right-hand (increasing y) side
of the wiggle.  If yzmin is greater than yzmax, then z values between
zmin will be plotted to the right of zmax, and z values between zbase
and zmin are filled.  Swapping yzmin and yzmax is an easy way to 
reverse the polarity of a wiggle.

The variable "endian" must have a value of 1 or 0. If this is
not a case an error is returned.

Author:  Dave Hale, Colorado School of Mines, 07/01/89
MODIFIED:  Paul Michaels, Boise State University, 29 December 2000
           Added solid/grey shading scheme, wiggle>=2 option for peaks/troughs
MODIFIED:  Kris Vanneste, Royal Observatory of Belgium, July 2005
           Added missing wiggle trace bits in y (amplitude) direction
\end{verbatim}
\pagebreak
\begin{verbatim}
RFWTVAINT - Rasterize a Float array as Wiggle-Trace-Variable-Area, with
	    8 point sinc INTerpolation.

rfwtvaint	rasterize a float array as wiggle-trace-variable-area, and
		apply sinc interploation for display purposes.

Function Prototype:
void rfwtvaint (int n, float z[], float zmin, float zmax, float zbase,
	int yzmin, int yzmax, int xfirst, int xlast,
	int wiggle, int nbpr, unsigned char *bits, int endian);

Input:
n		number of samples in array to rasterize
z		array[n] to rasterize
zmin		z values below zmin will be clipped
zmax		z values above zmax will be clipped
zbase		z values between zbase and zmax will be filled (see notes)
yzmin		horizontal raster coordinate corresponding to zmin
yzmax		horizontal raster coordinate corresponding to zmax
xfirst		vertical raster coordinate of z[0] (see notes)
xlast		vertical raster coordinate of z[n-1] (see notes)
wiggle		=0 for no wiggle (VA only); =1 for wiggle (with VA)
                wiggle 2<=wiggle<=5 for solid/grey coloring of VA option
                shade of grey: wiggle=2 light grey, wiggle=5 black
nbpr		number of bytes per row of bits
bits		pointer to first (top,left) byte in image
endian		byte order  =1 big endian  =0 little endian 

Output:
bits		pointer to first (top,left) byte in image

Notes:
The raster coordinate of the (top,left) bit in the image is (0,0).
In other words, x increases downward and y increases to the right.
Raster scan lines run from left to right, and from top to bottom.
Therefore, xfirst, xlast, yzmin, and yzmax should not be less than 0.
Likewise, yzmin and yzmax should not be greater than nbpr*8-1, and 
care should be taken to ensure that xfirst and xlast do not cause bits 
to be set outside (off the bottom) of the image. 

Variable area fill is performed on the right-hand (increasing y) side
of the wiggle.  If yzmin is greater than yzmax, then z values between
zmin will be plotted to the right of zmax, and z values between zbase
and zmin are filled.  Swapping yzmin and yzmax is an easy way to 
reverse the polarity of a wiggle.

The variable "endian" must have a value of 1 or 0. If this is
not a case an error is returned.

The interpolation is by the 8 point sinc interpolation routine s8r.

Author:  Dave Hale, Colorado School of Mines, 07/01/89
	Memorial University of Newfoundland: Tony Kocurko, Sept 1995.
	 Added sinc interpolation.
MODIFIED: Paul Michaels, Boise State University, 29 December 2000
          added solid/grey color scheme for peaks/troughs  wiggle=2 option
MODIFIED:  Kris Vanneste, Royal Observatory of Belgium, July 2005
           Added missing wiggle trace bits in y (amplitude) direction
\end{verbatim}
\pagebreak
\begin{verbatim}
SBLAS - Single precision Basic Linear Algebra Subroutines
	(adapted from LINPACK FORTRAN):

isamax	return index of element with maximum absolute value
sasum	return sum of absolute values
saxpy	compute y[i] = a*x[i]+y[i]
scopy	copy x[i] to y[i] (i.e., set y[i] = x[i])
sdot	return sum of x[i]*y[i] (i.e., return the dot product of x and y)
snrm2	return square root of sum of squares of x[i]
sscal	compute x[i] = a*x[i]
sswap	swap x[i] and y[i]

Function Prototypes:
int isamax (int n, float *sx, int incx);
float sasum (int n, float *sx, int incx);
void saxpy (int n, float sa, float *sx, int incx, float *sy, int incy);
void scopy (int n, float *sx, int incx, float *sy, int incy);
float sdot (int n, float *sx, int incx, float *sy, int incy);
float snrm2 (int n, float *sx, int incx);
void sscal (int n, float sa, float *sx, int incx);
void sswap (int n, float *sx, int incx, float *sy, int incy);

isamax:
Input:
n		number of elements in array
sx		array[n] of elements
incx		increment between elements 

Returned:	index of element with maximum absolute value

sasum:
Input:
n		number of elements in array
sx		array[n] of elements
incx		increment between elements 

Returned:	sum of absolute values

saxpy:
Input:
n		number of elements in arrays
sa		the scalar multiplier
sx		array[n] of elements to be scaled and added
incx		increment between elements of sx
sy		array[n] of elements to be added
incy		increment between elements of sy

Output:
sy		array[n] of accumulated elements

scopy:
Input:
n		number of elements in arrays
sx		array[n] of elements to be copied
incx		increment between elements of sx
incy		increment between elements of sy

Output:
sy		array[n] of copied elements

sdot:
Input:
n		number of elements in arrays
sx		array[n] of elements
incx		increment between elements of sx
sy		array[n] of elements
incy		increment between elements of sy

Returned:	dot product of the two arrays

snrm2
Input:
n		number of elements in array
sx		array[n] of elements
incx		increment between elements 

Returned:	square root of sum of squares of x[i]

sscal:
Input:
n		number of elements in array
sa		the scalar multiplier
sx		array[n] of elements
incx		increment between elements 

Output:
sx		array[n] of scaled elements

sswap:
Input:
n		number of elements in arrays
sx		array[n] of elements
incx		increment between elements of sx
sy		array[n] of elements
incy		increment between elements of sy

Output:
sx		array[n] of swapped elements
sy		array[n] of swapped elements

Notes:
Adapted from Linpack Fortran.

snrm2:
This simple version may cause overflow or underflow! 

Author:  Dave Hale, Colorado School of Mines, 10/01/89
\end{verbatim}
\pagebreak
\begin{verbatim}
SCAXIS - compute a readable scale for use in plotting axes

scaxis		compute a readable scale for use in plotting axes

Function Prototype:
void scaxis (float x1, float x2, int *nxnum, float *dxnum, float *fxnum);

Input:
x1		first x value
x2		second x value
nxnum		desired number of numbered values

Output:
nxnum		number of numbered values
dxnum		increment between numbered values (dxnum>0.0)
fxnum		first numbered value

Notes:
scaxis attempts to honor the user-specified nxnum.  However, nxnum
will be modified if necessary for readability.  Also, fxnum and nxnum
will be adjusted to compensate for roundoff error; in particular, 
fxnum will not be less than xmin-eps, and fxnum+(nxnum-1)*dxnum 
will not be greater than xmax+eps, where eps = 0.0001*(xmax-xmin).
xmin is the minimum of x1 and x2.  xmax is the maximum of x1 and x2.

Author:  Dave Hale, Colorado School of Mines, 01/13/89
\end{verbatim}
\pagebreak
\begin{verbatim}
SGA - Single precision general matrix functions adapted from LINPACK FORTRAN:

sgefa	Gaussian elimination to obtain the LU factorization of a matrix.
sgeco	Gaussian elimination to obtain the LU factorization and 
	condition number of a matrix.
sgesl	Solve linear system Ax = b or A'x = b after LU factorization.

Function Prototypes:
void sgefa (float **a, int n, int *ipvt, int *info);
void sgeco (float **a, int n, int *ipvt, float *rcond, float *z);
void sgesl (float **a, int n, int *ipvt, float *b, int job);

sgefa:
Input:
a		matrix[n][n] to be factored (see notes below)
n		dimension of a

Output:
a		matrix[n][n] factored (see notes below)
ipvt		indices of pivot permutations (see notes below)
info		index of last zero pivot (or -1 if no zero pivots)

sgeco:
Input:
a		matrix[n][n] to be factored (see notes below)
n		dimension of a

Output:
a		matrix[n][n] factored (see notes below)
ipvt		indices of pivot permutations (see notes below)
rcond		reciprocal condition number (see notes below)

Workspace:
z		array[n]

sgesl
Input:
a		matrix[n][n] that has been LU factored (see notes below)
n		dimension of a
ipvt		indices of pivot permutations (see notes below)
b		right-hand-side vector[n]
job		=0 to solve Ax = b
		=1 to solve A'x = b

Output:
b		solution vector[n]

Notes:
These functions were adapted from LINPACK FORTRAN.  Because two-dimensional 
arrays cannot be declared with variable dimensions in C, the matrix a
is actually a pointer to an array of pointers to floats, as declared
above and used below.

Elements of a are stored as follows:
a[0][0]    a[1][0]    a[2][0]   ... a[n-1][0]
a[0][1]    a[1][1]    a[2][1]   ... a[n-1][1]
a[0][2]    a[1][2]    a[2][2]   ... a[n-1][2]
.                                       .
.             .                         .
.                        .              .
.                                       .
a[0][n-1]  a[1][n-1]  a[2][n-1] ... a[n-1][n-1]

Both the factored matrix a and the pivot indices ipvt are required
to solve linear systems of equations via sgesl.

sgeco:
Given the reciprocal of the condition number, rcond, and the float
epsilon, FLT_EPSILON, the number of significant decimal digits, nsdd,
in the solution of a linear system of equations may be estimated by:
	nsdd = (int)log10(rcond/FLT_EPSILON)

This function was adapted from LINPACK FORTRAN.  Because two-dimensional 
arrays cannot be declared with variable dimensions in C, the matrix a
is actually a pointer to an array of pointers to floats, as declared
above and used below.

Author:  Dave Hale, Colorado School of Mines, 10/01/89
\end{verbatim}
\pagebreak
\begin{verbatim}
SHFS8R - Shift a uniformly-sampled real-valued function y(x) via
	a table of 8-coefficient sinc approximations.

shfs8r  shift a uniformly-sampled real function via a table of 8-coeff. sinc
	approximations.

Function Prototypes:
void shfs8r (float dx, int nxin, float fxin, float yin[], 
	float yinl, float yinr, int nxout, float fxout, float yout[]);

Input:
dx	      x sampling interval for both input and output y(x)
nxin	    number of x values at which y(x) is input
fxin	    x value of first sample input
yin	     array[nxin] of input y(x) values:  yin[0] = y(fxin), etc.
yinl	    value used to extrapolate yin values to left of yin[0]
yinr	    value used to extrapolate yin values to right of yin[nxin-1]
nxout	   number of x values a which y(x) is output
fxout	   x value of first sample output

Output:
yout	    array[nxout] of output y(x) values:  yout[0] = y(fxout), etc.

Notes:
Because extrapolation of the input function y(x) is defined by the
left and right values yinl and yinr, the output samples defined by
dx, nxout, and fxout are not restricted to lie within the range of 
input sample locations defined by dx, nxin, and fxin.

The maximum error for frequencies less than 0.6*nyquist is less
than one percent.

Author:  Dave Hale, Colorado School of Mines, 06/02/89
\end{verbatim}
\pagebreak
\begin{verbatim}
SINC - Return SINC(x) for as floats or as doubles

fsinc		return float value of sinc(x) for x input as a float
dsinc		return double precision sinc(x) for double precision x

Function Prototype:
float fsinc (float x);
double dsinc (double x);

Input:
x		value at which to evaluate sinc(x)

Returned: 	sinc(x)

Notes:
    sinc(x) = sin(PI*x)/(PI*x) 

Author:  Dave Hale, Colorado School of Mines, 06/02/89
\end{verbatim}
\pagebreak
\begin{verbatim}
SORT - Functions to sort arrays of data or arrays of indices

hpsort		sort an array of floats by the heap sort method
qkisort		sort an array of indices i[] so that 
		a[i[0]] <= a[i[1]] <= ... <= a[i[n-1]]
		uses the "quick sort" method
qkifind		partially sort an array of indices i[] so that the 
		index i[m] has the value it would have if the entire
		array of indices were sorted such that 
		a[i[0]] <= a[i[1]] <= ... <= a[i[n-1]]
		uses the "quick sort" method
qksort		sort an array of floats such that a[0] <= a[1] <= ... <= a[n-1]
		uses the "quick sort" method
qkfind		partially sort an array of floats  so that the element a[m] has
		the value it would have if the entire array were sorted
		such that a[0] <= a[1] <= ... <= a[n-1]
		uses the "quick sort" method

Function Prototypes:
void hpsort (int n, float a[]);
void qkisort (int n, float a[], int i[]);
void qkifind (int m, int n, float a[], int i[]);
void qksort (int n, float a[]);
void qkfind (int m, int n, float a[]);

hpsort:
Input:
n		number of elements in a
a		array[n] to be sorted

Output:
a		array[n] sorted

qkisort:
Input:
n		number of elements in array a
a		array[n] elements
i		array[n] indices to be sorted

Output:
i		array[n] indices sorted

qkifind:
Input:
m		index of element to be found
n		number of elements in array a
a		array[n] elements
i		array[n] indices to be partially sorted

Output:
i		array[n] indices partially sorted sorted

qksort:
Input:
n		number of elements in array a
a		array[n] containing elements to be sorted

Output:
a		array[n] containing sorted elements

qkfind:
Input:
m		index of element to be found
n		number of elements in array a
a		array[n] to be partially sorted

Output:
a		array[n] partially sorted


Notes:
hpsort:
The heap sort algorithm is, at worst, N log_2 N, and in most cases
is 20% faster.  Adapted from Standish.

qkisort, qkifind, qksort, qkfind:
n must be less than 2^NSTACK, where NSTACK is defined above.

qkisort:
This function is adapted from procedure quicksort by
Hoare, C.A.R., 1961, Communications of the ACM, v. 4, p. 321;
the main difference is that recursion is accomplished
explicitly via a stack array for efficiency; also, a simple
insertion sort is used to sort subarrays too small to be
partitioned efficiently.

qkifind:
This function is adapted from procedure find by
Hoare, C.A.R., 1961, Communications of the ACM, v. 4, p. 321.

qksort:
This function is adapted from procedure quicksort by
Hoare, C.A.R., 1961, Communications of the ACM, v. 4, p. 321;
the main difference is that recursion is accomplished
explicitly via a stack array for efficiency; also, a simple
insertion sort is used to sort subarrays too small to be
partitioned efficiently.

qkfind:
This function is adapted from procedure find by Hoare 1961.

Reference:
hpsort:
Standish, T. A., Data Structure Techniques, p. 91.
See also Press, W. A., et al., Numerical Recipes in C.

quick sort:
Hoare, C.A.R., 1961, Communications of the ACM, v. 4, p. 321.

Author:  Dave Hale, Colorado School of Mines, 12/26/89
\end{verbatim}
\pagebreak
\begin{verbatim}
SQR - Single precision QR decomposition functions adapted from LINPACK FORTRAN:

sqrdc	Compute QR decomposition of a matrix.
sqrsl	Use QR decomposition to solve for coordinate transformations,
	projections, and least squares solutions.
sqrst	Solve under-determined or over-determined least squares problems,
	with a user-specified tolerance.

Function Prototypes:
void sqrdc (float **x, int n, int p, float *qraux, int *jpvt,
	float *work, int job);
void sqrsl (float **x, int n, int k, float *qraux,
	float *y, float *qy, float *qty,
	float *b, float *rsd, float *xb, int job, int *info);
void sqrst (float **x, int n, int p, float *y, float tol,
	float *b, float *rsd, int *k,
	int *jpvt, float *qraux, float *work);

sqrdc:
Input:
x		matrix[p][n] to decompose (see notes below)
n		number of rows in the matrix x
p		number of columns in the matrix x
jpvt		array[p] controlling the pivot columns (see notes below)
job		=0 for no pivoting;
		=1 for pivoting

Output:
x		matrix[p][n] decomposed (see notes below)
qraux		array[p] containing information required to recover the
		orthogonal part of the decomposition
jpvt		array[p] with jpvt[k] containing the index of the original
		matrix that has been interchanged into the k-th column, if
		pivoting is requested.

Workspace:
work		array[p] of workspace

sqrsl:
Input:
x		matrix[p][n] containing output of sqrdc.
n		number of rows in the matrix xk; same as in sqrdc.
k		number of columns in the matrix xk; k must not be greater
		than MIN(n,p), where p is the same as in sqrdc.
qraux		array[p] containing auxiliary output from sqrdc.
y		array[n] to be manipulated by sqrsl.
job		specifies what is to be computed.  job has the decimal
		expansion ABCDE, with the following meaning:
		if A != 0, compute qy.
		if B, C, D, or E != 0, compute qty.
		if C != 0, compute b.
		if D != 0, compute rsd.
		if E != 0, compute xb.
		Note that a request to compute b, rsd, or xb automatically
		triggers the computation of qty, for which an array must
		be provided.

Output:
qy		array[n] containing Qy, if its computation has been
		requested.
qty		array[n] containing Q'y, if its computation has
		been requested.  Here Q' denotes the transpose of Q.
b		array[k] containing solution of the least squares problem:
			minimize norm2(y - xk*b),
		if its computation has been requested.  (Note that if
		pivoting was requested in sqrdc, the j-th component of
		b will be associated with column jpvt[j] of the original
		matrix x that was input into sqrdc.)
rsd		array[n] containing the least squares residual y - xk*b,
		if its computation has been requested.  rsd is also the
		orthogonal projection of y onto the orthogonal complement
		of the column space of xk.
xb		array[n] containing the least squares approximation xk*b,
		if its computation has been requested.  xb is also the
		orthogonal projection of y onto the column space of x.
info		=0 unless the computation of b has been requested and R
		is exactly singular.  In this case, info is the index of
		the first zero diagonal element of R and b is left
		unaltered.

sqrst:
Input:
x		matrix[p][n] of coefficients (x is destroyed by sqrst.)
n		number of rows in the matrix x (number of observations)
p		number of columns in the matrix x (number of parameters)
y		array[n] containing right-hand-side (observations)
tol		relative tolerance used to determine the subset of
		columns of x included in the solution.  If tol is zero,
		a full complement of the MIN(n,p) columns is used.
		If tol is non-zero, the problem should be scaled so that
		all the elements of X have roughly the same absolute
		accuracy eps.  Then a reasonable value for tol is roughly
		eps divided by the magnitude of the largest element.

Output:
x		matrix[p][n] containing output of sqrdc
b		array[p] containing the solution (parameters); components
		corresponding to columns not used are set to zero.
rsd		array[n] of residuals y - Xb
k		number of columns of x used in the solution
jpvt		array[p] containing pivot information from sqrdc.
qraux		array[p] containing auxiliary information from sqrdc.

Workspace:
work		array[p] of workspace.	

Notes:
!!! WARNING !!!
These functions have many options, not all of which have been tested!
(Dave Hale, 12/31/89)


This function was adapted from LINPACK FORTRAN.  Because two-dimensional 
arrays cannot be declared with variable dimensions in C, the matrix x
is actually a pointer to an array of pointers to floats, as declared
above and used below.

Elements of x are stored as follows:
x[0][0]    x[1][0]    x[2][0]   ... x[p-1][0]
x[0][1]    x[1][1]    x[2][1]   ... x[p-1][1]
x[0][2]    x[1][2]    x[2][2]   ... x[p-1][2]
.                                       .
.             .                         .
.                        .              .
.                                       .
x[0][n-1]  x[1][n-1]  x[2][n-1] ... x[p-1][n-1]

sqrdc:
Uses Householder transformations to compute the QR decomposition of an n by p
matrix x.  Column pivoting based on the 2-norms of the reduced columns may be
performed at the user's option.

After decomposition, x contains in its upper triangular matrix R of the QR
decomposition.  Below its diagonal x contains information from which the
orthogonal part of the decomposition can be recovered.  Note that if
pivoting has been requested, the decomposition is not that of the original
matrix x but that of x with its columns permuted as described by jpvt.

The selection of pivot columns is controlled by jpvt as follows.
The k-th column x[k] of x is placed in one of three classes according to
the value of jpvt[k].
	if jpvt[k] >  0, then x[k] is an initial column.
	if jpvt[k] == 0, then x[k] is a free column.
	if jpvt[k] <  0, then x[k] is a final column.
Before the decomposition is computed, initial columns are moved to the
beginning of the array x and final columns to the end.  Both initial and
final columns are frozen in place during the computation and only free
columns are moved.  At the k-th stage of the reduction, if x[k] is occupied
by a free column it is interchanged with the free column of largest reduced
norm.  jpvt is not referenced if job == 0.

sqrsl:
Uses the output of sqrdc to compute coordinate transformations, projections,
and least squares solutions.  For k <= MIN(n,p), let xk be the matrix
	xk = (x[jpvt[0]], x[jpvt[1]], ..., x[jpvt[k-1]])
formed from columns jpvt[0], jpvt[1], ..., jpvt[k-1] of the original
n by p matrix x that was input to sqrdc.  (If no pivoting was done, xk
consists of the first k columns of x in their original order.)  sqrdc
produces a factored orthogonal matrix Q and an upper triangular matrix R
such that
	xk = Q * (R)
	         (0)
This information is contained in coded form in the arrays x and qraux.

The parameters qy, qty, b, rsd, and xb are not referenced if their
computation is not requested and in this case can be replaced by NULL
pointers in the calling program.  To save storage, the user may in
some cases use the same array for different parameters in the calling
sequence.  A frequently occuring example is when one wishes to compute
any of b, rsd, or xb and does not need y or qty.  In this case one may
equivalence y, qty, and one of b, rsd, or xb, while providing separate
arrays for anything else that is to be computed.  Thus the calling
sequence
	sqrsl(x,n,k,qraux,y,NULL,y,b,y,NULL,110,&info)
will result in the computation of b and rsd, with rsd overwriting y.
More generally, each item in the following list contains groups of
permissible equivalences for a single calling sequence.
	1. (y,qty,b) (rsd) (xb) (qy)
	2. (y,qty,rsd) (b) (xb) (qy)
	3. (y,qty,xb) (b) (rsd) (qy)
	4. (y,qy) (qty,b) (rsd) (xb)
	5. (y,qy) (qty,rsd) (b) (xb)
	6. (y,qy) (qty,xb) (b) (rsd)
In any group the value returned in the array allocated to the group
corresponds to the last member of the group.

sqrst:
Computes least squares solutions to the system
	Xb = y
which may be either under-determined or over-determined.  The user may
supply a tolerance to limit the columns of X used in computing the solution.
In effect, a set of columns with a condition number approximately bounded
by 1/tol is used, the other components of b being set to zero.

On return, the arrays x, jpvt, and qraux contain the usual output from
sqrdc, so that the QR decomposition of x with pivoting is fully available
to the user.  In particular, columns jpvt[0], jpvt[1], ..., jpvt[k-1]
were used in the solution, and the condition number associated with
those columns is estimated by ABS(x[0][0]/x[k-1][k-1]).

Author:  Dave Hale, Colorado School of Mines, 12/29/89
\end{verbatim}
\pagebreak
\begin{verbatim}
STOEP - Functions to solve a symmetric Toeplitz linear system of equations
	 Rf=g for f

stoepd		solve a symmetric Toeplitz system - doubles
stoepf		solve a symmetric Toeplitz system - floats

Function Prototypes:
void stoepd (int n, double r[], double g[], double f[], double a[]);
void stoepf (int n, float r[], float g[], float f[], float a[]);

Input:
n		dimension of system
r		array[n] of top row of Toeplitz matrix
g		array[n] of right-hand-side column vector

Output:
f		array[n] of solution (left-hand-side) column vector
a		array[n] of solution to Ra=v (Claerbout, FGDP, p. 57)

Notes:
These routines do NOT solve the case when the main diagonal is zero, it
just silently returns.

The left column of the Toeplitz matrix is assumed to be equal to the top
row (as specified in r); i.e., the Toeplitz matrix is assumed symmetric.

Author:  Dave Hale, Colorado School of Mines, 06/02/89
\end{verbatim}
\pagebreak
\begin{verbatim}
STRSTUFF -- STRing manuplation subs

cwp_strdup -  duplicate a string
strchop - chop off the tail end of a string "s" after a "," returning
	  the front part of "s" as "t".
cwp_strrev - reverse a string



Input:
char *str	input string	

Output:
none

Returns:
char *	  duplicated string

Notes:
This local definition of strdup is necessary because some systems
do not have it.

Author: John Stockwell, Spring 2000.
\end{verbatim}
\pagebreak
\begin{verbatim}
SWAPBYTE - Functions to SWAP the BYTE order of binary data 

swap_short_2		swap a short integer
swap_u_short_2		swap an unsigned short integer
swap_int_4		swap a 4 byte integer
swap_u_int_4		swap an unsigned integer
swap_long_4		swap a long integer
swap_u_long_4		swap an unsigned long integer
swap_float_4		swap a float
swap_double_8		swap a double

Function Prototypes:
void swap_short_2(short *tni2);
void swap_u_short_2(unsigned short *tni2);
void swap_int_4(int *tni4);
void swap_u_int_4(unsigned int *tni4);
void swap_long_4(long *tni4);
void swap_u_long_4(unsigned long *tni4);
void swap_float_4(float *tnf4);
void swap_double_8(double *tndd8);

Notes:
These routines are necessary for reversing the byte order of binary data
for transportation between big-endian and little-endian machines. Examples
of big-endian machines are IBM RS6000, SUN, NeXT. Examples of little
endian machines are PC's and DEC.

These routines have been tested with PC data and run on PC's running
several PC versions of UNIX, but have not been tested on DEC.

Also, the number appended to the name of the routine refers to the
number of bytes that the item is assumed to be.

Authors: Jens Hartmann,   Institut fur Geophysik, Hamburg, Jun 1993
	 John Stockwell, CWP, Colorado School of Mines, Jan 1994
\end{verbatim}
\pagebreak
\begin{verbatim}
SYMMEIGEN - Functions solving the eigenvalue problem for symmetric matrices
eig_jacobi - find eigenvalues and corresponding eigenvectors via 
         the jacobi algorithm for symmetric matrices
sort_eigenvalues -  sort eigenvalues and corresponding eigenvectors
			in descending order

Function Prototypes:
void eig_jacobi(float **a, float d[], float **v, int n);
void sort_eigenvalues(float d[], float **v, int n);
 (inspired by Press et. al., 1996)                     

 Macro used internally
define ROTATE(a,i,j,k,l) g=a[i][j];h=a[k][l];a[i][j]=g-s*(h+g*tau);\
    a[k][l]=h+s*(g-h*tau);

void eig_jacobi(float **a, float d[], float **v, int n)
eig_jacobi - find eigenvalues and corresponding eigenvectors via 
         the jacobi algorithm for symmetric matrices

Function Prototype:
void eig_jacobi(float **a, float d[], float **v, int n);
 (inspired by Press et. al., 1996)                     
{
    int j,iq,ip,i;
    float tresh,theta,tau,t,sm,s,h,g,c,*b,*z;

    /* allocate space temporarily
    b=alloc1float(n); b-=1;
    z=alloc1float(n); z-=1;
    
    /* initialize v to the identity matrix
    for (ip=1;ip<=n;ip++) {
        for (iq=1;iq<=n;iq++) v[ip][iq]=0.0;
        v[ip][ip]=1.0;
    }
    /* initialilize to the diagonal on matrix a
    for (ip=1;ip<=n;ip++) {
        b[ip]=d[ip]=a[ip][ip];
        z[ip]=0.0;
    }

    /* main iteration loop
    for (i=1;i<=50;i++) {

        sm=0.0;
        for (ip=1;ip<=n-1;ip++) {
            for (iq=ip+1;iq<=n;iq++)
                sm += fabs(a[ip][iq]);
        }
        /* normal return
        if (sm == 0.0) {
            z+=1; free1float(z);
            b+=1; free1float(b);
            return;
        }
        /* tresh values for first 3 sweeps and therafter
        if (i < 4)
            tresh=0.2*sm/(n*n);
        else
            tresh=0.0;
        for (ip=1;ip<=n-1;ip++) {
            for (iq=ip+1;iq<=n;iq++) {
                g=100.0*fabs(a[ip][iq]);
                if (i > 4 && (float)(fabs(d[ip])+g) == (float)fabs(d[ip])
                    && (float)(fabs(d[iq])+g) == (float)fabs(d[iq]))
                    a[ip][iq]=0.0;
                else if (fabs(a[ip][iq]) > tresh) {
                    h=d[iq]-d[ip];
                    if ((float)(fabs(h)+g) == (float)fabs(h))
                        t=(a[ip][iq])/h;
                    else {
                        theta=0.5*h/(a[ip][iq]);
                        t=1.0/(fabs(theta)+sqrt(1.0+theta*theta));
                        if (theta < 0.0) t = -t;
                    }
                    c=1.0/sqrt(1+t*t);
                    s=t*c;
                    tau=s/(1.0+c);
                    h=t*a[ip][iq];
                    z[ip] -= h;
                    z[iq] += h;
                    d[ip] -= h;
                    d[iq] += h;
                    a[ip][iq]=0.0;
                    
                    /* Jacobi rotations
                    for (j=1;j<=ip-1;j++) {
                        ROTATE(a,j,ip,j,iq)
                    }
                    for (j=ip+1;j<=iq-1;j++) {
                        ROTATE(a,ip,j,j,iq)
                    }
                    for (j=iq+1;j<=n;j++) {
                        ROTATE(a,ip,j,iq,j)
                    }
                    for (j=1;j<=n;j++) {
                        ROTATE(v,j,ip,j,iq)
                    }
                }
            }
        }
        for (ip=1;ip<=n;ip++) {
            b[ip] += z[ip];
            d[ip]=b[ip];
            z[ip]=0.0;
        }
    }
    /* this will not happen, hopefully
    fprintf(stderr,"jacobi iteration does not converge\n");
}


void sort_eigenvalues(float d[], float **v, int n)
sort_eigenvalues -  sort eigenvalues and corresponding eigenvectors
			in descending order

Function Prototypes:
void sort_eigenvalues(float d[], float **v, int n);
 (inspired by Press et. al., 1996)                     
{
    int k,j,i;
    float p;

    for (i=1;i<n;i++) {
        p=d[k=i];
        for (j=i+1;j<=n;j++)
            if (d[j] >= p) p=d[k=j];
        if (k != i) {
            d[k]=d[i];
            d[i]=p;
            for (j=1;j<=n;j++) {
                p=v[j][i];
                v[j][i]=v[j][k];
                v[j][k]=p;
            }
        }
    }
}
\end{verbatim}
\pagebreak
\begin{verbatim}

include "cwp.h"

TEMPORARY_FILENAME - Creates a file name in a user-specified directory.

Function prototypes:
FILE *temporary_stream (char *tempfile);
char *temporary_filename(char *tempfile);

temporary_stream:
Input:
tempfile	pointer to directory prefix string (eg. /usr/tmp/)

Output:
filestream	pointer to temporary file stream
temporary_filename:
Input:
tempfile	pointer to directory prefix string (eg. /usr/tmp/)

Output:
tempfile	pointer to filename string (eg. /usr/tmp/1206aaa)

Notes:
temporary_stream creates a file stream by appending a sequence of
numbers and letters (which is created by mkstemp) to the prefix string
passed as its argument. 

Author:  Andreas Klaedtke, 12/2/2009

temporary_filename creates a file name by appending a sequence of
numbers and letters (which is created by tmpnam) to the prefix string
passed as its argument.  On return the input argument points to the
(now augmented) prefix string.

It is duty of the calling program to provide room for the augmented
string.  The resulting string is typically used as a name for a
temporary file; in this case it is the calling program's job to make
sure that the supplied prefix ends with a slash.

This routine was written to supplement the ANSI C function tmpnam
which also creates a temporary filename, but within a fixed directory,
usually the /tmp directory.  Unfortunately, some /tmp directories are
too small to hold typical seismic data sets, so this routine allows
the user to specify a directory with sufficient capacity.  Also note
that on many systems, the tmpfile() call avoids this problem by
simulating a temporary file with a memory buffer.  However, this is
not a panacea as the file size might exceed available memory and on
some systems this call does actually create a file (again, usually in
tmp).
Author:  Jack K. Cohen, Colorado School of Mines, 12/12/95

\end{verbatim}
\pagebreak
\begin{verbatim}
TRIDIAGONAL - Functions to solve tridiagonal systems of equations Tu=r for u.

tridif		Solve a tridiagonal system of equations (float version)
tridid		Solve a tridiagonal system of equations (double version)
tripd		Solve a positive definite, symmetric tridiagonal system
tripp		Solve an unsymmetric tridiagonal system that uses 
		Gaussian elimination with partial pivoting

Function Prototypes:
void tridif (int n, float a[], float b[], float c[], float r[], float u[]);
void tridid (int n, double a[], double b[], double c[], double r[], double u[]);
void tripd (float *d, float *e, float *b, int n);
void tripp(int n, float *d, float *e, float *c, float *b);

tridif, tridid:
Input:
n		dimension of system
a		array[n] of lower sub-diagonal of T (a[0] ignored)
b		array[n] of diagonal of T
c		array[n] of upper super-diagonal of T (c[n-1] ignored)
r		array[n] of right-hand-side column vector

Output:
u		array[n] of solution (left-hand-side) column vector

tripd:
Input:
d	array[n], the diagonal of A 
e	array[n], the superdiagonal of A
b	array[n], the rhs column vector of Ax=b

Output:
b	b is overwritten with the solution to Ax=b
tripp:
Input:
d	diagonal vector of matrix
e       upper-diagonal vector of matrix
c       lower-diagonal vector of matrix
b       right-hand vector
n       dimension of matrix

Output:
b       solution vector

Notes:
For example, a tridiagonal system of dimension 4 is specified as:

    |b[0]    c[0]     0       0  | |u[0]|     |r[0]|
    |a[1]    b[1]    c[1]     0  | |u[1]|  =  |r[1]|
    | 0      a[2]    b[2]    c[2]| |u[2]|     |r[2]|
    | 0       0      a[3]    b[3]| |u[3]|     |r[3]|

The tridiagonal matrix is assumed to be non-singular.

tripd:
Given an n-by-n symmetric, tridiagonal, positive definite matrix A and
 n-vector b, following algorithm overwrites b with the solution to Ax = b

Authors:  tridif, tridid: Dave Hale, Colorado School of Mines, 10/03/89
tripd, tripp: Zhenyue Liu, Colorado School of Mines,  1992-1993
\end{verbatim}
\pagebreak
\begin{verbatim}
UNWRAP_PHASE - routines to UNWRAP phase of fourier transformed data

oppenheim_unwrap_phase - using the method of Oppenheim and Schafer (1975)
simple_unwrap_phase - by searching for phase jumps


Function Prototype:
void oppenheim_unwrap_phase(int n, int trend, int zeromean, 
		float df, float *xr, float *xi, float *phase);
void simple_unwrap_phase(int n, int trend, int zeromean, 
		float w,  float *phase);
oppenheim_unwrap_phase:
Input:
n		number of samples
df		frequency sampling interval
trend		remove linear trend from unwrapped phase
xr		real part of signal
xi		imaginary part of signal

Output:
phase		array[n] of output unwrapped phase values

simple_unwrap_phase:
Input:
n               number of samples
trend           remove linear trend from phase
zeromean        =0 assume phase(0)=0.0, else assume zero mean
w               unwrapping parameter; returns an error if w=0
phase           array[n] of inpu

Output:
phase           array[n] of output phase values

oppenheim_unwrap_phase:
Notes:
1) The phase unwrapping method proposed by Oppenheim and Schaffer 
   1975 calculates the unwrapped phase by integrating the derivative 
   with respect to frequency of the phase of a signal F(w) . 

   Let u(w) = Re F(w) and v(w) = Im F(w)

   phase(w) = arctan[v(w)/u(w)] 

   Taking the derivative with respect to w of both sides

d/dw [ phase(w) ] = d/dw ( arctan (v/u) )
=   [ 1/ (1 +(v/u)^2) ] ( v'/u - vu'/u^2 ) = ( v'u - vu' )/(u^2 +v^2) 

2) Then, the d/dw phase   is integrated with respect to w 
   to produce the phase function

   phase(w) = integral phase'(w)  dw

3) the user has the option of removing the linear trend in the phase

   The approach allows us to avoid the principle branch behavior of
   the arctangent function.

References:
Oppenheim A.V. and R.W. Schafer, Digital Signal Processing,
Prentice-Hall, Englewood Cliffs, New Jersey, 1975.

Tria M., M. Van Der Baan 2, A. Larue, J. Mars 1, 2007,
Wavelet estimation in homomorphic domain by spectral 
averaging, for deconvolution of seismic data
For the BLISS Project, Universit� de Leeds, 
ITF Consorsium collaboration(s) (2007)

simple_unwrap_phase:
Notes:
The strategy is to look at the change in phase (dphase) with each
time step. If |dphase| >  PI/w, then use the previous value of
dphase. No attempt is made at smoothing the dphase curve.

oppenheim_unwrap_phase:
Author: John Stockwell, CWP, 2010

simple_unwrap_phase
Author: John Stockwell, CWP, 2010
\end{verbatim}
\pagebreak
\begin{verbatim}
VANDERMONDE - Functions to solve Vandermonde system of equations Vx=b 

vanded		solve Vandermonde system of doubles
vandef		solve Vandermonde system of floats

Function Prototypes:
void vanded (int n, double v[], double b[], double x[]);
void vandef (int n, float v[], float b[], float x[]);

Input:
n		dimension of system
v		array[n] of 2nd row of Vandermonde matrix (1st row is all ones)
b		array[n] of right-hand-side column vector

Output:
x		array[n] of solution column vector

Notes:
The arrays b and x may be equivalenced.

Reference:
Adapted from Algorithm 5.6-2 in Golub, G. H., and Van Loan, C. F., 1983,
Matrix Computations, John-Hopkins University Press.

Author:  Dave Hale, Colorado School of Mines, 06/02/89
\end{verbatim}
\pagebreak
\begin{verbatim}
WAVEFORMS   Subroutines to define some wavelets for modeling of seismic
            data
    
ricker1_wavelet     Compute the time response of a source function as
            a Ricker wavelet with peak frequency "fpeak" Hz.    

ricker2_wavelet     Compute a Ricke wavelet with a given period, amplitude
            and distorsion factor

akb_wavelet         Compute the time response of a source function as
            a wavelet based on a wavelet used by Alford, Kelly, 
            and Boore.

spike_wavelet       Compute the time response of a source function as
            a spike.    

unit_wavelet        Compute the time response of a source function as
            a constant unit shift.  

zero_wavelet        Compute the time response of a source function as
            zero everywhere.    

berlage_wavelet     Compute the time response of a source function as a
            Berlage wavelet with peak frequency "fpeak" Hz, 
            exponential decay factor "decay", time exponent 
            "tn", and initial phase angle "ipa".

gaussian_wavelet    Compute the time response of a source function as a
            Gaussian wavelet with peak frequency "fpeak" in Hz.

gaussderiv_wavelet  Compute the time response of a source function as a
            Gaussian first derivative wavelet with peak frequency "fpeak" 
            in Hz.
deriv_n_gauss  Compute the n-th derivative of a gaussian in double precision

Function Prototypes:
void ricker1_wavelet (int nt, float dt, float fpeak, float *wavelet);
void ricker2_wavelet (int hlw, float dt, float period, float ampl, 
    float distort, float *wavelet);
void akb_wavelet (int nt, float dt, float fpeak, float *wavelet);
void spike_wavelet (int nt, int tindex, float *wavelet);
void unit_wavelet (int nt, float *wavelet);
void zero_wavelet (int nt, float *wavelet);
void berlage_wavelet (int nt, float dt, float fpeak, float ampl, float tn,
                       float decay, float ipa, float *wavelet);
void gaussian_wavelet (int nt, float dt, float fpeak, float *wavelet);
void gaussderiv_wavelet (int nt, float dt, float fpeak, float *wavelet);

Authors: Tong Fei, Ken Larner 
Author: Nils Maercklin, February 2007

\end{verbatim}
\pagebreak
\begin{verbatim}
WINDOW - windowing routines

hanningnWindow - returns an n element long hanning window 

Function prototypes:
void hanningnWindow(int n,float *w);
Author: Potash Corporation, Sascatchewan: Balasz Nemeth given to CWP 2008

\end{verbatim}
\pagebreak
\begin{verbatim}
wrapArray - wrap an array

Function prototype:
void wrapArray(void *base,size_t nmemb,size_t size,int f)
Author: Potash Corporation: Balasz Nemeth, given to CWP 2008

\end{verbatim}
\pagebreak
\begin{verbatim}
XCOR - Compute z = x cross-correlated with y

xcor	compute z= x cross-correlated with y

Function Prototype:
void xcor (int lx, int ifx, float *x, int ly, int ify, float *y ,
	int lz, int ifz, float *z);

Input:
lx		length of x array
ifx		sample index of first x
x		array[lx] to be cross-correlated with y
ly		length of y array
ify		sample index of first y
y		array[ly] with which x is to be cross-correlated
lz		length of z array
ifz		sample index of first z

Output:
z		array[lz] containing x cross-correlated with y

Notes:
See notes for convolution function convolve_cwp().

The operation "x cross correlated with y"  is defined to be:

           ifx+lx-1
    z[i] =   sum    x[j]*y[i+j]  ;  i = ifz,...,ifz+lz-1
            j=ifx

This function performs cross-correlation by
(1) reversing the samples in the x array while copying
    them to a temporary array, and
(2) calling function convolve_cwp() with ifx set to 1-ifx-lx.
Assuming that the overhead of reversing the samples in x is negligible,
this method enables cross-correlation to be performed as efficiently as
convolution, while reducing the amount of code that must be optimized
and maintained.

Author:  Dave Hale, Colorado School of Mines, 11/23/91
\end{verbatim}
\pagebreak
\begin{verbatim}
XINDEX - determine index of x with respect to an array of x values

xindex		determine index of x with respect to an array of x values

Input:
nx		number of x values in array ax
ax		array[nx] of monotonically increasing or decreasing x values
x		the value for which index is to be determined
index		index determined previously (used to begin search)

Output:
index		for monotonically increasing ax values, the largest index
		for which ax[index]<=x, except index=0 if ax[0]>x;
		for monotonically decreasing ax values, the largest index
		for which ax[index]>=x, except index=0 if ax[0]<x

Notes:
This function is designed to be particularly efficient when called
repeatedly for slightly changing x values; in such cases, the index 
returned from one call should be used in the next.

Author:  Dave Hale, Colorado School of Mines, 12/25/89
\end{verbatim}
\pagebreak
\begin{verbatim}
YCLIP - Clip a function y(x) defined by linear interpolation of the
uniformly sampled values:  y(fx), y(fx+dx), ..., y(fx+(nx-1)*dx).
Returns the number of samples in the clipped function.

yclip		clip a function y(x) defined by linear interplolation of
		uniformly sampled values

Function Prototype:
int yclip (int nx, float dx, float fx, float y[], float ymin, float ymax,
	float xc[], float yc[]);

Input:
nx		number of x (and y) values
dx		x sampling interval
fx		first x
y		array[nx] of uniformly sampled y(x) values
ymin		minimum y value; must not be greater than ymax
ymax		maximum y value; must not be less than ymin

Output:
xc		array[?] of x values for clipped y(x)
yc		array[?] of y values for clipped y(x)

Returned:	number of samples in output arrays xc and yc

Notes:
The output arrays xc and yc should contain space 2*nx values, which
is the maximum possible number (nc) of xc and yc returned.

Author:  Dave Hale, Colorado School of Mines, 07/03/89
\end{verbatim}
\pagebreak
\begin{verbatim}
YXTOXY - Compute a regularly-sampled, monotonically increasing function x(y)
	from a regularly-sampled, monotonically increasing function y(x) by
	inverse linear interpolation.

yxtoxy		compute a regularly sampled function x(y) from a regularly
		sampled, monotonically increasing function y(x)

Function Prototype:
void yxtoxy (int nx, float dx, float fx, float y[], 
	int ny, float dy, float fy, float xylo, float xyhi, float x[]);

Input:
nx		number of samples of y(x)
dx		x sampling interval; dx>0.0 is required
fx		first x
y		array[nx] of y(x) values; y[0] < y[1] < ... < y[nx-1] required
ny		number of samples of x(y)
dy		y sampling interval; dy>0.0 is required
fy		first y
xylo		x value assigned to x(y) when y is less than smallest y(x)
xyhi		x value assigned to x(y) when y is greater than largest y(x)

Output:
x		array[ny] of x(y) values

Notes:
User must ensure that:
(1) dx>0.0 && dy>0.0
(2) y[0] < y[1] < ... < y[nx-1]

Author:  Dave Hale, Colorado School of Mines, 06/02/89
\end{verbatim}
\pagebreak
\begin{verbatim}
ZASC - routine to translate ncharacters from ebcdic to ascii

	zasc - convert n characters from ebcdic to ascii format

Input:
nchar		number of characters to be translated
ainput		pointer to input characters

Output:
aoutput		pointer to output characters

Function Prototype:
int zasc(char *ainput, char *aoutput, integer nchar);
Notes:
 translated by f2c.  Horribly inefficient, but little used
 
Author: Stew Levin of Mobil, 1997
\end{verbatim}
\pagebreak
\begin{verbatim}
ZEBC - routine to translate ncharacters from ascii to ebcdic

	zebc - convert n characters from ascii to ebcdic format

Input:
nchar		number of characters to be translated
ainput		pointer to input characters

Output:
aoutput		pointer to output characters

Function Prototype:
int zebc(char *ainput, char *aoutput, integer nchar);
Notes:
 translated by f2c.  Horribly inefficient, but little used
 
Author: Stew Levin of Mobil, 1997
\end{verbatim}
\pagebreak
\begin{verbatim}
CHECK - CHECK triangulated models


badModel		bad Model flag
checkVertexUse		check Vertex Use
checkEdgeUse		check Edge Use
checkFace		check Face
checkModel		check Model

Function Prototypes:

void badModel(void);
void checkVertexUse (VertexUse *vu);
void checkEdgeUse (EdgeUse *eu);
void checkFace (Face *f);
void checkModel (Model *m);

checkVertexUse:
Input:
vu	Pointer to VertexUse

checkEdgeUse:
Input:
eu	pointer to EdgeUse

checkFace:
Input:
f	pointer to Face

checkModel:
Input:
m	pointer to Model

Notes: Routines for checking triangulated models.

Author:  Dave Hale, Colorado School of Mines, 07/09/90

\end{verbatim}
\pagebreak
\begin{verbatim}
CIRCUM - define CIRCUMcircles for Delaunay triangulation

circum - compute center and radius-squared of circumcircle of 3 (x,y)
          locations
circumTri - compute center and radius-squared of circumcircle of 
            triangular face

Function Prototypes:
void circum (float x1, float y1, float x2, float y2, float x3, float y3,
	float *xc, float *yc, float *rs);
void circumTri (Tri *t);
circum:
Input:
x1	x-coordinate of first point
y1	y-coordinate of first point
x2	x-coordinate of second point
y2	y-coordinate of second point
x3	x-coordinate of third point
y3	y-coordinate of third point

Output:
xc	pointer to x-coordinate of center of circumcircle
yc	pointer to y-coordinate of center of circumcircle
rs	pointer radius^2 of circumcircle

circumTri:
Input:
t	Pointer to Tri	

Returns:
xc	x-coordinate of circumcircle
yc      y-coordinate of circumcircle
rs      radius^2 of circumcircle

Author:  Dave Hale, Colorado School of Mines, Fall 1990.
\end{verbatim}
\pagebreak
\begin{verbatim}
COLINEAR - determine if edges or vertecies are COLINEAR in triangulated
           model
edgesColinear		see whether or not two edges are colinear
vertexBetweenVertices	determine whether or not a vertex is on a line
                          between two other vertices
Function Prototypes:
int edgesColinear (Edge *e1, Edge *e2);
int vertexBetweenVertices (Vertex *v, Vertex *v1, Vertex *v2);

edgesColinear:
Input:
e1	pointer to first Edge
e2	pointer to second Edge

Returns: (int)
1	if colinear

vertexBetweenVertices:
Input:
v		pointer to first Vertex in question
v1		pointer to first reference Vertex
v2		pointer to second reference Vertex

Returns: integer
1		if v=v1 or v=v2 or if v is between v1 and v2
0		otherwise

Author:  Dave Hale, Colorado School of Mines, Fall 1990.
\end{verbatim}
\pagebreak
\begin{verbatim}
CREATE - create model, boundary edge triangles, edge face, edge vertex, add
         a vertex

makeModel		Make and return a pointer to a new model
makeBoundaryEdgeTri	Create a boundary edge and triangle 
makeEdgeFace		Create an edge by connecting two vertices
makeEdgeVertex		Create an edge connecting an existing vertex (v1) to a
                        new vertex
addVertexToModel	Add a vertex to model, and return pointer to new vertex
insideTriInModel	return pointer to triangle in model containing
                   	specified (x,y) coordinates

Function Prototypes:
Model *makeModel (float xmin, float ymin, float xmax, float ymax);
void makeBoundaryEdgeTri (Vertex *v, Edge **enew, Tri **tnew);
void makeEdgeFace (Vertex *v1, Vertex *v2, Edge **enew, Face **fnew);
Vertex* addVertexToModel (Model *m, float x, float y);
Tri* insideTriInModel (Model *m, Tri *start, float x, float y);

makeModel:
Input:
xmin		minimum x-coordinate
ymin		minimum y-coordinate
xmax		maximum x-coordinate
ymax		maximum y-coordinate

Returns: pointer to a new Model

makeBoundaryEdgeTri:
Input:
v		specified boundary Vertex

Output:
enew		new boundary Edge
tnew		new boundary triangle

Notes:
The specified vertex and the adjacent vertices on the boundary
are assumed to be colinear.  Therefore, the resulting
boundary triangle has zero area, and is intended to enable
deletion of the specified vertex from the boundary.

makeEdgeFace:
Input:
v1		First Vertex
v2		second Vertex

Output:
enew		new Edge
fnew		new Face

Notes:
The vertices must be adjacent to a single common face.
This face is closed off by the new edge, and a new edge and
a new face are made and returned. 

addVertexToModel:
Input:
m		model
x		x-coordinate of new vertex
y		y-coordinate of new vertex

Notes:
If the new vertex is close to an existing vertex, this function returns NULL.

insideTriInModel:
Input:
m		Model
start		triangle to look at first (NULL to begin looking anywhere)
x		x-coordinate
y		y-coordinate

Notes:
Points on an edge of a triangle are assumed to be inside that triangle.
An edge may be used by two triangles, so two triangles may "contain"
a point that lies on an edge.  The first triangle found to contain
the specified point is returned.

Author:  Dave Hale, Colorado School of Mines, Fall 1990.
\end{verbatim}
\pagebreak
\begin{verbatim}
DELETE - DELETE vertex, model, edge, or boundary edge from triangulated model

deleteVertexFromModel		Delete a vertex from model
killModel			Delete a model along with everything in it
killEdge			Delete an edge
killBoundaryEdge		Kill a boundary edge

Function Prototypes:
void deleteVertexFromModel (Model *m, Vertex *v);
void killModel (Model *m);
void killEdge (Edge *e, Face **fs);
void killBoundaryEdge (Edge *e);

deleteVertexFromModel:
Input:
m	 	pointer to Model	
v		pointer to Vertex to be deleted

killModel:
Input:
m		pointer to Model

killEdge:
Input:
e		Edge to delete

Output:
fs		surviving Face

killBoundaryEdge:
Input:
e 	boundary Edge

Notes:
Killing a boundary edge is typically done after a new boundary vertex
is inserted on an existing boundary edge.

Author:  Dave Hale, Colorado School of Mines, Fall 1990.
\end{verbatim}
\pagebreak
\begin{verbatim}
DTE - Distance to Edge

distanceToEdge - return distance to edge from specified (x,y) coordinates

Function Prototype:
float distanceToEdge (Edge *e, float x, float y);
distanceToEdge:
Input:
e		edge to which distance is to be computed
x		x-coordinate
y		y-coordinate

Author:  Dave Hale, Colorado School of Mines, 09/11/90
\end{verbatim}
\pagebreak
\begin{verbatim}
FIXEDGES - FIX or unFIX EDGES between verticies

fixEdgeBetweenVertices		Fix edge(s) between vertices, creating new
                          	  colinear edges as necessary.
unfixEdge			unfix edge 

fixEdgesBetweenVertices:
Input:
v1		pointer to first Vertex
v2		pointer to second Vertex

Returns:
0		if unable to fix edges
1		otherwise

unfixEdge:
Input:
e		edge to be unfixed

Returns:
0		if unable to unfix edge
1		otherwise
Author:  Dave Hale, Colorado School of Mines, 06/04/91
\end{verbatim}
\pagebreak
\begin{verbatim}
INSIDE -  Is a vertex or point inside a circum circle, etc. of a triangulated
          model

inCircum	determine whether or not a vertex is inside a circum circle
inCircumTri	determine whether or not a vertex is inside a circum circle of
                 a triangle
in3Vertices	determine whether or not a vertex is inside triangle (v1,v2,v3)
inTri		determine whether or not a vertex is inside a triangle 

Function Prototypes:
int inCircum (float x, float y, float xc, float yc, float rs);
int inCircumTri (float x, float y, Tri *t);
int in3Vertices (float x, float y, Vertex *v1, Vertex *v2, Vertex *v3);
int inTri (float x, float y, Tri *t);
inCircum:
Input:
x	x-coordinate of vertex
y	y-coordinate of vertex
xc	x-coordinate of center of circumcircle
yc	y-coordinate of center of circumcircle
rs	radius^2 of circumcircle

Returns:
1	if x,y inside of circumcircle
0	otherwise

Notes:
A vertex exactly on the edge of a circumcircle is taken as being outside

inCircumTri:
Input:
x	x-coordinate of vertex
y	y-coordinate of vertex
t	pointer to Tri

Returns:
1	if x,y inside of circumcircle of a triangle
0	otherwise

Notes:
A vertex exactly on the edge of a circumcircle is taken as being outside

in3Vertices:
Input:
x	x-coordinate of vertex
y	y-coordinate of vertex
v1	pointer to first Vertex
v2	pointer to second Vertex
v3	pointer to third Vertex

Returns:
1	if x,y inside of v1,v2,v3
0	otherwise

Notes:
A vertex exactly on an edge of the triangle is taken as being inside

inTri:
Input:
x	x-coordinate of vertex
y	y-coordinate of vertex
t	pointer to Tri

Returns:
1	if x,y inside a triangle
0	otherwise
Notes:
A vertex exactly on the edge of a triangle is inside

Author:  Dave Hale, Colorado School of Mines, 06/04/91
\end{verbatim}
\pagebreak
\begin{verbatim}
NEAREST - NEAREST edge or vertex in triangulated model

nearestEdgeInModel	Return pointer to edge in model nearest to
                    	specified (x,y) coordinates

nearestVertexInModel	Return pointer to vertex in model nearest
                   	to specified (x,y) coordinates

Function Prototypes:
Vertex* nearestVertexInModel (Model *m, Vertex *start, float x, float y);
Edge* nearestEdgeInModel (Model *m, Edge *start, float x, float y);

nearestEdgeInModel:
Input:
m		model
start		edge to look at first (NULL to begin looking anywhere)
x		x-coordinate
y		y-coordinate

Returns: pointer to nearest Edge

nearestVertexInModel:
Input:
m		model
start		vertex to look at first (NULL to begin looking anywhere)
x		x-coordinate
y		y-coordinate

Returns: pointer to nearest Vertex

Author:  Dave Hale, Colorado School of Mines, Fall 1990
\end{verbatim}
\pagebreak
\begin{verbatim}
PROJECT - project to edge in triangulated model

projectToEdge - Project point with specified (x,y) coordinates to specified
                edge
Function Prototype:
void projectToEdge (Edge *e, float *x, float *y)

Input:
e		edge to which point is to be projected
x		x-coordinate before projection
y		y-coordinate before projection

Output:
x		x-coordinate after projection
y		y-coordinate after projection

Author:  Dave Hale, Colorado School of Mines, 09/11/90
\end{verbatim}
\pagebreak
\begin{verbatim}
READWRITE - READ or WRITE a triangulated model

readModel		Read a model in the form produced by writeModel
writeModel		Write a model to a file

Function Prototypes:
Model *readModel (FILE *fp);
void writeModel (Model *m, FILE *fp);

readModel:
Input:
fp		file pointer to file containing model

writeModel:
Input:
m		pointer to Model
fp		file pointer

Author:  Jack K. Cohen, Center for Wave Phenomena, 09/21/90
Modified:  Dave Hale, Center for Wave Phenomena, 11/30/90
	Converted representation of model from ascii to binary for speed.
	Added code to read attributes.
Modified (writeModel):  Craig Artley, Center for Wave Phenomena, 04/08/94
	Corrected bug;  previously the edgeuses and vertextuses of the
	exterior face were not written.

\end{verbatim}
\pagebreak
\end{document}
